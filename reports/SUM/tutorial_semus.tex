% !TEX root = MAIN.tex

\chapter{SEMUS - Tutorial}
\label{chapter:semus:tutorial}

\section{Introduction}

This tutorial instructs on how to use \SEMUS on a typical case using a simple example. 

The example concerns the use of \SEMUS on the ASN.1 case study, provided by ESA.

\section{Running \SEMUS}
\label{sec:semus_running}

Firstly, the user needs to set up the \SEMUS toolset, please refer to Section~\ref{sec:install:semus}.

The ASN.1 (or ASN1SCC) case study is an open source ASN.1 compiler that generates C/C++ and SPARK/Ada code suitable for low resource environments such as space systems. Moreover, the compiler can produce a test harness that provides full statement coverage in the generated code, and therefore significantly improves its quality.

In the context of FAQAS, we focus our analysis on the auto-generated source code by the ASN.1 compiler. For the definition of the ASN.1 compiler case study, we introduce the example of a simple grammar, more details on ASN1SCC and the grammar used, please refer to Section 3.7 of the deliverable \texttt{ITT-1-9873-ESA-FAQAS-D2}.

For the given grammar, the size of the auto-generated source code is 4\,338 LOC. While the unit test suite consists of 107 auto-generated test cases.

The objective of this tutorial is to generate test inputs that kill the mutants non detected by the ASN.1 test suite. Consequently, we consider a precondition of this tutorial to have applied \MASS to the ASN.1 case study, and thus to have the list of live mutants.

Given that the engineer has already properly setup and initialized the workspace for \SEMUS (see Section~\ref{sec:semus:initialization}). The steps for test generation on the ASN.1 case study consists of:

\begin{enumerate}
    \item Configure the file faqas\_semus\_config.sh
    \item Generate the test templates for the SUT functions
    \item Launch the test generation process for the case study
\end{enumerate}



