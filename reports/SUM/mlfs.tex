% !TEX root = MAIN.tex

\subsection{Mathematical Library for Flight Software Example}
\label{sec:single_machine}

The first step regards installing the \MASS framework, please refer to Section~\ref{sec:install}.

The second step consists of creating and installing a workspace folder for running \MASS on the MLFS example. For this case, the workspace folder will be created on \texttt{/opt/MLFS}. Note that variable \texttt{\$FAQAS} represents the installation folder of the \FAQAS.

\begin{lstlisting}[language=bash]
  $ cd $FAQAS/MASS/FAQAS-Setup
  $ export INSTALL_DIR=/opt/MLFS
  $ ./install.sh
\end{lstlisting}

The third step consists of configuring the \MASS configuration file \texttt{mass\_conf.sh}. In the following, we provide all the excerpts that require manual editing. Listing~\ref{mass_conf_single} contains the necessary configuration for the MLFS case study.

\begin{lstlisting}[language=bash, label=mass_conf_single, caption=\MASS variables. Excerpt of mass\_conf.sh file.]
# set FAQAS path
export SRCIROR=/opt/srcirorfaqas

...

# set directory path where MASS files can be stored
export APP_RUN_DIR=/opt/MLFS

# specifies the building system, available options are "Makefile" and "waf"
export BUILD_SYSTEM="Makefile"

# directory root path of the software under test
export PROJ=$HOME/mlfs

# directory src path of the SUT
export PROJ_SRC=$PROJ/libm

# directory test path of the SUT
export PROJ_TST=$HOME/unit-test-suite

# directory coverage path of the SUT
export PROJ_COV=$HOME/blts_workspace

# directory path of the compiled binary
export PROJ_BUILD=$PROJ/build-host/bin

# filename of the compiled file/library
export COMPILED=libmlfs.a

# path to original Makefile
export ORIGINAL_MAKEFILE=$PROJ/Makefile

# compilation command of the SUT
export COMPILATION_CMD=(make all ARCH=host EXTRA_CFLAGS="-DNDEBUG" \&\& make all COVERAGE="true" ARCH=host_cov EXTRA_CFLAGS="-DNDEBUG")

# compilation additional commands of the SUT (e.g., setup of workspace)
export ADDITIONAL_CMD=(cd $HOME/blts/BLTSConfig \&\& make clean install INSTALL_PATH="$HOME/blts_install" \&\& cd $HOME/blts_workspace \&\& $HOME/blts_install/bin/blts_app --init)

# command to be executed after each test case (optional)
export ADDITIONAL_CMD_AFTER=(rm -rf $HOME/blts_workspace/*)

# compilation command for TCE analysis
export TCE_COMPILE_CMD=(make all ARCH=host EXTRA_CFLAGS="-DNDEBUG")

# command to clean installation of the SUT
export CLEAN_CMD=(make cleanall)

# relative path to location of gcov files (i.e., gcda and gcno files)
export GC_FILES_RELATIVE_PATH=Reports/Coverage/Data
\end{lstlisting}

Also \MASS variables shall be configured within the same file. Particularly, we will run \MASS with the setup contained in Listing~\ref{mass_conf_specific}.

\begin{lstlisting}[language=bash, label=mass_conf_specific, caption=\MASS specific variables. Excerpt of mass\_conf.sh file.]
### MASS variables

# TCE flags to be tested
export FLAGS=("-O0" "-O1" "-O2" "-O3" "-Ofast" "-Os")

# specify if MASS will be executed on a HPC, possible values are "true" or "false"
export HPC="false"

# set if MASS should be executed with a prioritized and reduced test suite
export PRIORITIZED="true"

# set sampling technique, possible values are "uniform", "stratified", "fsci", and "no"
# note: if "uniform" or "stratified" is set, $PRIORITIZED must be "false"
export SAMPLING="fsci"

# set sampling rate if whether "uniform" or "stratified" sampling has been selected
export RATE=""
\end{lstlisting}

The fourth step consists of configuring the \texttt{PrepareSUT} configuration file \\(\texttt{MASS\_STEPS\_LAUNCHERS/PrepareSUT.sh}, the actions provided in this file enables \MASS to collect the code coverage; within this file the following actions must be performed by the engineer (see~\ref{preparesut_single}), notice that we use the tool \texttt{bear} for the generation of the \texttt{compile\_commands.json}:

\begin{lstlisting}[language=bash, label=preparesut_single, caption=PrepareSUT.sh file.]
#!/bin/bash

cd /opt/MLFS
. ./mass_conf.sh

# 1. Compile SUT

cd $PROJ

# generate compile_commands.json and delete build
bear make all && rm -rf build* && sed -i 's: libm: /home/mlfs/mlfs/libm:' compile_commands.json && mv compile_commands.json $MUTANTS_DIR
eval "${COMPILATION_CMD[@]}"

# 2. Prepare test scripts
# example

cd $HOME/blts/BLTSConfig
make clean install INSTALL_PATH="$HOME/blts_install"

# Preparing MLFS workspace (e.g., where test cases data is stored)
cd $HOME/blts_workspace
$HOME/blts_install/bin/blts_app --init

# 3. Execute test cases
# Note: execution time for each test case should be measured and passed as argument to FAQAS-CollectCodeCoverage.sh

# example
for tst in $(find $HOME/unit-test-suite -name '*.xml');do
    cd $HOME/blts_workspace

    tst_filename_wo_xml=$(basename -- $tst .xml)

    start=$(date +%s)
    $HOME/blts_install/bin/blts_app -gcrx $tst_filename_wo_xml -b coverage --nocsv -s $tst
    end=$(date +%s)

    # call to FAQAS-CollectCodeCoverage.sh
    # parameter should be test case name and the execution time
    source $MASS/FAQAS-GenerateCodeCoverageMatrixes/FAQAS-CollectCodeCoverage.sh $tst_filename_wo_xml "$(($end-$start))"
done
\end{lstlisting}

The fifth step consists of defining the function \texttt{run\_tst\_case} within the file \\\texttt{mutation\_additional\_functions.sh}. An example of its implementation is provided in Listing~\ref{mutation_additional}.

\begin{lstlisting}[language=bash, label=mutation_additional, caption='run\_tst\_case' Bash function for the MLFS. Excerpt of mutation\_additional\_functions.sh file.]
run_tst_case() {

    tst_name=$1
    tst=$PROJ_TST/$tst_name.xml

    echo $tst_name $tst

    # run the test case
    cd $PROJ_COV
    $HOME/blts_install/bin/blts_app -gcrx $tst_name -b coverage --nocsv -s $tst

    # define if test case execution passed or failed
    summaryreport=$tst_name/Reports/SessionSummaryReport.xml
    originalreport=$HOME/unit-reports/$summaryreport

    test_cases_failed=`xmllint --xpath "//report_summary/test_set_summary/test_cases_failed/text()" $summaryreport`
    o_test_cases_failed=`xmllint --xpath "//report_summary/test_set_summary/test_cases_failed/text()" $originalreport`

    echo "comparing with original execution"
    echo $test_cases_failed $o_test_cases_failed

    if [ "$test_cases_failed" != "$o_test_cases_failed" ]; then
        return 1
    else
        return 0
    fi
}
\end{lstlisting}

The sixth step consists of providing a template for the build script for the trivial compiler optimizations step. In particular, we replaced the optimization flag in the original build script:

\begin{lstlisting}[language=bash, caption=Excerpt from Makefile.]
CFLAGS = -c -Wall -std=gnu99 -pedantic -Wextra -frounding-math -fsignaling-nans -g O2 -fno-builtin $(EXTRA_CFLAGS)
\end{lstlisting}

with \texttt{TCE}, creating a new template for the build script:

\begin{lstlisting}[language=bash, caption=Excerpt from Makefile.template.]
CFLAGS = -c -Wall -std=gnu99 -pedantic -Wextra -frounding-math -fsignaling-nans TCE -fno-builtin $(EXTRA_CFLAGS)
\end{lstlisting}

The seventh step consists of launching the one step launcher (see Section~\ref{sec:singlelaunch}):

\begin{lstlisting}[language=bash]
 $ /opt/MLFS/Launcher.sh
\end{lstlisting}

The following results shall be reported at the end of the execution:

\begin{lstlisting}[language=bash, label=mass_output, caption=\MASS output.]
##### MASS Output #####
## Total mutants generated: 28071
## Total mutants filtered by TCE: 6918
## Sampling type: fsci
## Total mutants analyzed: 461
## Total killed mutants: 369
## Total live mutants: 92
## Total likely equivalent mutants: 53
## MASS mutation score (%): 90.44
## List A of useful undetected mutants: /opt/MLFS/RESULTS/useful_list_a
## List B of useful undetected mutants: /opt/MLFS/RESULTS/useful_list_b
## Number of statements covered: 1973
## Statement coverage (%): 100
## Minimum lines covered per source file: 2
## Maximum lines covered per source file: 138
\end{lstlisting}
