% !TEX root = MAIN.tex

\chapter{DAMAt - Tutorial}
\label{chapter:dama:tutorial}

\section{Introduction}

This tutorial instructs on how to use \DAMA on a typical case using a simple example.

\section{Running \DAMA}
\label{sec:dama_single_machine}

Firstly, the user needs to set up the \DAMA pipeline, please refer to Section~\ref{sec:dama_setup}.

The directory chosen for \DAMA in this case will be \texttt{/opt/DAMAt}.

This is a very simple example of SUT.

\begin{lstlisting}[language=c++]
#include <iostream>
#include <vector>
#include <stdlib.h>

int main()
{
    // Create a vector containing doubles
    std::vector<double> v;

    v.push_back(1.1);
    v.push_back(2.2);
    v.push_back(3.3);
    v.push_back(4.4);
    v.push_back(5.5);

    for(std::vector<double>::iterator it = v.begin(); it != v.end(); ++it) {
    	std::cout << *it << '\n';
    }

    return 0;
}
\end{lstlisting}

\DAMA works in six steps:

\begin{enumerate}
	\item The user prepares a fault model specification tailored to the SUT.
	\item \DAMA generates a mutation API with the functions that modify the data according to the provided fault model.
	\item The user manually modifies the SUT by introducing mutation probes (i.e., invocations to the mutation API) into its source code.
	\item \DAMA generates and compiles mutants.
	\item \DAMA executes the test suite against all the mutants.
	\item \DAMA generates mutation analysis results.
\end{enumerate}


\subsection{Step 1}

This is a simple example of a fault model for the SUT.
Specific details about the operators' configuration can be found in the D2.

\begin{lstlisting}
FaultModel,DataItem,Span,Type,FaultClass,Min,Max,Threshold,Delta,State,Value
IfHK,1,1,DOUBLE,VAT,NA,NA,10.3,15.2,NA,NA
IfHK,2,1,DOUBLE,VOR,-5.5,5.5,NA,2,NA,NA
IfHK,3,1,DOUBLE,IV,NA,NA,NA,NA,NA,69.69
IfHK,4,1,DOUBLE,VBT,NA,NA,0,15.5,NA,NA
\end{lstlisting}


\subsection{Step 2}

The engineer shall modify the \texttt{DAMAt\_configure.sh} script according to the specifics of the SUT.
In this case the buffer elements are of type "double", the list of the tests (\texttt{test\_list.csv}) has been copied in the \texttt{/opt/DAMAt} directory along with the fault model (\texttt{fault\_model.csv}).
The \texttt{singleton} option is activated and there is no need to skip the initial elements, so \texttt{padding} is set to 0.

\begin{lstlisting}[language=bash]

tests_list=$DAMAt_FOLDER/tests.csv

fault_model=$DAMAt_FOLDER/fault_model.csv

buffer_type="double"

singleton="TRUE"

padding=0

\end{lstlisting}

The user shall go to the \texttt{/opt/DAMAt} directory and generate the probes with the following commands.

\begin{lstlisting}[language=bash]

cd /opt/DAMAt
bash DAMAt_probe_generation.sh

\end{lstlisting}


\subsection{Step 3}

The user shall manually instrument the SUT as shown below.
\begin{lstlisting}[language=c++]

#include <iostream>
#include <vector>
#include <stdlib.h>
// INCLUDING THE FAQAS API
#include "FAQAS_dataDrivenMutator.h"

int main()
{
    // Create a vector containing doubles
    std::vector<double> v;

    v.push_back(1.1);
    v.push_back(2.2);
    v.push_back(3.3);
    v.push_back(4.4);
    v.push_back(5.5);

    //MANUALLY ADDED PROBE
    mutate_FM_IfHK( &v );
    //MANUALLY ADDED PROBE END

    for(std::vector<double>::iterator it = v.begin(); it != v.end(); ++it) {
    	std::cout << *it << '\n';
    }

    return 0;
}

\end{lstlisting}


\subsection{Step 4, Step 5 and Step 6}

The user shall execute the rest of the pipeline with
the following command.

\begin{lstlisting}[language=bash]

bash DAMAt_mutants_launcher.sh

\end{lstlisting}

More details on the results of the \DAMA procedure and how to interpret them are available in Section~\ref{sec:dama_singlelaunch}.
