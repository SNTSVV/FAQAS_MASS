% !TEX root = MAIN.tex

\chapter{DAMAt -  libParam Tutorial}

\label{chapter:dama:tutorial_param}

\section{Introduction}

This tutorial instructs on how to use \DAMA on the \emph{\emph{libParam}} case study.


\section{Setting up and running \DAMA}
\label{sec:param_single_machine}

The directory chosen for \DAMA, in this case, will be \texttt{/opt/DAMAt}.
The content of the DAMAt toolset shall be copied into this folder.
For more details, please refer to Section~\ref{sec:dama_setup}.

\DAMA works in six steps:

\begin{enumerate}
	\item The user prepares a fault model specification tailored to the SUT.
	\item \DAMA generates a mutation API with the functions that modify the data according to the provided fault model.
	\item The user manually modifies the SUT by introducing mutation probes (i.e., invocations to the mutation API) into its source code.
	\item \DAMA generates and compiles mutants.
	\item \DAMA executes the test suite against all the mutants.
	\item \DAMA generates mutation analysis results.
\end{enumerate}


\subsection{Step 1}

Here is reported the fault model used for the \emph{libParam} test case.
There are six fault model, each for a different type of exchanged message:
\begin{itemize}
  \item General
  \item GET
  \item SET
  \item SAVE
  \item LOAD
  \item REPLY
\end{itemize}
Specific details about the operators' configuration can be found in the D2.

\begin{lstlisting}
FaultModel,DataItem,Span,Type,FaultClass,Min,Max,Threshold,Delta,State,Value
General,0,1,LONG,IV,NA,NA,NA,NA,NA,0x00
General,0,1,LONG,IV,NA,NA,NA,NA,NA,0x55
General,0,1,LONG,IV,NA,NA,NA,NA,NA,0xff
General,0,1,LONG,IV,NA,NA,NA,NA,NA,0x44
General,0,1,LONG,IV,NA,NA,NA,NA,NA,0x77
General,0,1,LONG,IV,NA,NA,NA,NA,NA,0x88
General,0,1,LONG,IV,NA,NA,NA,NA,NA,0x89
General,0,1,LONG,IV,NA,NA,NA,NA,NA,0x99
General,0,1,LONG,IV,NA,NA,NA,NA,NA,0x9a
General,1,1,LONG,VAT,NA,NA,20,1,NA,NA
General,1,1,LONG,FVAT,NA,NA,20,1,NA,NA
General,1,1,LONG,SS,NA,NA,NA,1,NA,NA
General,1,1,LONG,SS,NA,NA,NA,-1,NA,NA
General,1,1,LONG,HV,NA,NA,NA,NA,NA,10
General,2,1,LONG,SS,NA,NA,NA,1,NA,NA
General,2,1,LONG,SS,NA,NA,NA,-1,NA,NA
General,2,1,LONG,IV,NA,NA,NA,NA,NA,0
General,2,1,LONG,VAT,NA,NA,180,1,NA,NA
General,2,1,LONG,FVAT,NA,NA,180,1,NA,NA
General,4,1,LONG,SS,NA,NA,NA,1,NA,NA
General,4,1,LONG,SS,NA,NA,NA,-1,NA,NA
General,5,1,LONG,SS,NA,NA,NA,1,NA,NA
General,5,1,LONG,SS,NA,NA,NA,-1,NA,NA
GET,6,1,LONG,SS,NA,NA,NA,1,NA,NA
GET,6,1,LONG,SS,NA,NA,NA,-1,NA,NA
SET,6,1,BIN,BF,0,7,NA,NA,-1,1
SAVE,6,1,LONG,SS,NA,NA,NA,1,NA,NA
SAVE,6,1,LONG,SS,NA,NA,NA,-1,NA,NA
SAVE,7,1,LONG,SS,NA,NA,NA,1,NA,NA
SAVE,7,1,LONG,SS,NA,NA,NA,-1,NA,NA
LOAD,6,1,LONG,SS,NA,NA,NA,1,NA,NA
LOAD,6,1,LONG,SS,NA,NA,NA,-1,NA,NA
LOAD,7,1,LONG,SS,NA,NA,NA,1,NA,NA
LOAD,7,1,LONG,SS,NA,NA,NA,-1,NA,NA
REPLY,2,1,LONG,SS,NA,NA,NA,1,NA,NA
REPLY,2,1,LONG,SS,NA,NA,NA,-1,NA,NA
REPLY,2,1,LONG,IV,NA,NA,NA,NA,NA,0
REPLY,2,1,LONG,VAT,NA,NA,180,1,NA,NA
REPLY,2,1,LONG,FVAT,NA,NA,180,1,NA,NA
REPLY,4,1,LONG,SS,NA,NA,NA,1,NA,NA
REPLY,4,1,LONG,SS,NA,NA,NA,-1,NA,NA
REPLY,5,1,LONG,SS,NA,NA,NA,1,NA,NA
REPLY,5,1,LONG,SS,NA,NA,NA,-1,NA,NA
REPLY,6,1,BIN,BF,0,7,NA,NA,-1,1
\end{lstlisting}

The user shall also provide a list of test cases with the nominal execution time in the csv format.

\begin{lstlisting}
bindings,4427.200000
example,4145.400000
file_store,2954.000000
i2c,2793.400000
log,2881.200000
param3,3007.800000
param4,3273.600000
rparam3,5099.400000
rparam4,8003.200000
serialize,2656.400000
spi,2825.800000
store,2774.400000
store_load,3182.000000
vmem_store_checksum_first,3005.800000
vmem_store_checksum_last,2963.400000
\end{lstlisting}

\subsection{Step 2}

The engineer shall modify the \texttt{DAMAt\_configure.sh} script according to the specifics of the SUT.
In this case the buffer elements are of type "unsigned long long int", the list of the tests (\texttt{tests\_param.csv}) has been copied in the \texttt{/opt/DAMAt} directory along with the fault model (\texttt{fault\_model\_param.csv}).
The \texttt{singleton} option is activated to avoid memory issues and there is no need to skip the initial elements, so \texttt{padding} is set to 0.

\begin{lstlisting}[language=bash]

# the location of the csv with all the test identifiers and the execution time
tests_list=$DAMAt_FOLDER/tests_param.csv

# the location of the csv containing the definitions of the mutation operators
fault_model=$DAMAt_FOLDER/fault_model_param.csv

# the datatype of the elements of the target buffer
buffer_type="unsigned long long int"

# padding: can be used to skip the first n bit of a buffer, normally set to 0
padding=0

# singleton: can set to true to load the fault model into a singleton variable, normally set to "TRUE", can also  be set to "FALSE"
singleton="TRUE"

\end{lstlisting}

The user shall go to the \texttt{/opt/DAMAt} directory and generate the probes with the following commands.

\begin{lstlisting}[language=bash]

cd /opt/DAMAt
bash DAMAt_probe_generation.sh

\end{lstlisting}


\subsection{Step 3}

The user shall manually instrument the SUT as shown below.
The probes were inserted in the \texttt{gs\_rparam\_process\_packet} function, contained in the \texttt{libparam/src/csp\_service\_handler.c} file.
\begin{lstlisting}[language=c]

static void gs_rparam_process_packet(csp_conn_t * conn, csp_packet_t * request_packet)
{
    csp_packet_t * reply_packet = NULL;
    gs_rparam_query_t * reply;


    /* Handle endian */
    gs_rparam_query_t * request = (gs_rparam_query_t *) request_packet->data;


    request->length = csp_ntoh16(request->length);
    request->checksum = csp_ntoh16(request->checksum);

//start of the mutation probe
    unsigned long long int v_General[6];

    v_General[0] = (unsigned long long int) request->action;
    v_General[1] = (unsigned long long int) request->table_id;
    v_General[2] = (unsigned long long int) request->length;
    v_General[3] = (unsigned long long int) request->checksum;
    v_General[4] = (unsigned long long int) request->seq;
    v_General[5] = (unsigned long long int) request->total;


    mutate_FM_General(v_General);

    request->action = (uint8_t) v_General[0];
    request->table_id = (uint8_t) v_General[1];
    request->length = (uint16_t) v_General[2];
    request->checksum = (uint16_t) v_General[3];
    request->seq = (uint16_t) v_General[4];
    request->total = (uint16_t) v_General[5];
// end of the mutation probe


    /* Check valid mem */
    gs_param_table_instance_t * tinst = gs_param_table_by_id(request->table_id);
    if (tinst == NULL) {
        csp_buffer_free(request_packet);
        return;
    }


    switch(request->action) {
	case RPARAM_GET: {

    //start of the  mutation probe
    unsigned long long int v_GET[8];

    v_GET[0] = (unsigned long long int) request->action;
    v_GET[1] = (unsigned long long int) request->table_id;
    v_GET[2] = (unsigned long long int) request->length;
    v_GET[3] = (unsigned long long int) request->checksum;
    v_GET[4] = (unsigned long long int) request->seq;
    v_GET[5] = (unsigned long long int) request->total;
    v_GET[6] = (unsigned long long int)request->payload.addr[0];
    v_GET[7] = 0;

    mutate_FM_GET(v_GET);

    request->action = (uint8_t) v_GET[0];
    request->table_id = (uint8_t) v_GET[1];
    request->length = (uint16_t) v_GET[2];
    request->checksum = (uint16_t) v_GET[3];
    request->seq = (uint16_t) v_GET[4];
    request->total = (uint16_t) v_GET[5];
    request->payload.addr[0]=(uint16_t)v_GET[6];

    //end of the mutation probe

            if (request->length > GS_RPARAM_QUERY_MAX_PAYLOAD) {
                log_error("Request too long %u", request->length);
                goto out_free;
            }

          										[...]

                /* Set sequence */
                reply->seq = csp_hton16(param_pos);
                reply->total = csp_hton16(tinst->row_count);

                /* Set lengths */
                reply->length = csp_hton16(length);
                reply_packet->length = length + offsetof(gs_rparam_query_t, payload.packed);

                //start of the  mutation probe
                unsigned long long int v_REPLY[8];

                v_REPLY[0] = (unsigned long long int) request->action;
                v_REPLY[1] = (unsigned long long int) request->table_id;
                v_REPLY[2] = (unsigned long long int) request->length;
                v_REPLY[3] = (unsigned long long int) request->checksum;
                v_REPLY[4] = (unsigned long long int) request->seq;
                v_REPLY[5] = (unsigned long long int) request->total;
                v_REPLY[6]= (unsigned long long int)request->payload.packed[0];
                v_REPLY[7] = 0;

                mutate_FM_REPLY(v_REPLY);

                request->action = (uint8_t) v_REPLY[0];
                request->table_id = (uint8_t) v_REPLY[1];
                request->length = (uint16_t) v_REPLY[2];
                request->checksum = (uint16_t) v_REPLY[3];
                request->seq = (uint16_t) v_REPLY[4];
                request->total = (uint16_t) v_REPLY[5];
                request->payload.packed[0]=(uint8_t)v_REPLY[6];

                //end of the mutation probe

                if (!csp_send(conn, reply_packet, 1000)) {
                    csp_buffer_free(reply_packet);
                    goto out_free;
                }
            }

            goto out_free;
	}
	case RPARAM_SET: {

    //start of the  mutation probe
    unsigned long long int v_SET[8];

    v_SET[0] = (unsigned long long int) request->action;
    v_SET[1] = (unsigned long long int) request->table_id;
    v_SET[2] = (unsigned long long int) request->length;
    v_SET[3] = (unsigned long long int) request->checksum;
    v_SET[4] = (unsigned long long int) request->seq;
    v_SET[5] = (unsigned long long int) request->total;
    v_SET[6] = (unsigned long long int)request->payload.packed[0];
    v_SET[7] = 0;

    mutate_FM_SET(v_SET);

    request->action = (uint8_t) v_SET[0];
    request->table_id = (uint8_t) v_SET[1];
    request->length = (uint16_t) v_SET[2];
    request->checksum = (uint16_t) v_SET[3];
    request->seq = (uint16_t) v_SET[4];
    request->total = (uint16_t) v_SET[5];
    request->payload.packed[0]=(uint8_t)v_SET[6];

    //end of the mutation probe

            if (!validate_checksum(tinst, request->checksum)) {
                request_packet->data[0] = RPARAM_ERROR;
                goto out_send_status;
            }


            if (gs_param_deserialize(tinst, request->payload.packed, request->length, F_FROM_BIG_ENDIAN) == GS_OK) {
                request_packet->data[0] = RPARAM_SET_OK;
            } else {
                request_packet->data[0] = RPARAM_ERROR;
            }
            goto out_send_status;


														[...]


	case RPARAM_LOAD: {

    //start of the  mutation probe
    unsigned long long int v_LOAD[8];

    v_LOAD[0] = (unsigned long long int) request->action;
    v_LOAD[1] = (unsigned long long int) request->table_id;
    v_LOAD[2] = (unsigned long long int) request->length;
    v_LOAD[3] = (unsigned long long int) request->checksum;
    v_LOAD[4] = (unsigned long long int) request->seq;
    v_LOAD[5] = (unsigned long long int) request->total;
    v_LOAD[6]= (unsigned long long int)request->payload.copy.from;
    v_LOAD[7]= (unsigned long long int)request->payload.copy.to;

    mutate_FM_LOAD(v_LOAD);

    request->action = (uint8_t) v_LOAD[0];
    request->table_id = (uint8_t) v_LOAD[1];
    request->length = (uint16_t) v_LOAD[2];
    request->checksum = (uint16_t) v_LOAD[3];
    request->seq = (uint16_t) v_LOAD[4];
    request->total = (uint16_t) v_LOAD[5];
    request->payload.copy.from=(uint8_t)v_LOAD[6];
    request->payload.copy.to=(uint8_t)v_LOAD[7];

    //end of the mutation probe

														[...]

	}
	case RPARAM_SAVE: {

    //start of the  mutation probe

    unsigned long long int v_SAVE[8];

    v_SAVE[0] = (unsigned long long int) request->action;
    v_SAVE[1] = (unsigned long long int) request->table_id;
    v_SAVE[2] = (unsigned long long int) request->length;
    v_SAVE[3] = (unsigned long long int) request->checksum;
    v_SAVE[4] = (unsigned long long int) request->seq;
    v_SAVE[5] = (unsigned long long int) request->total;
    v_SAVE[6]= (unsigned long long int)request->payload.copy.from;
    v_SAVE[7]= (unsigned long long int)request->payload.copy.to;

    mutate_FM_SAVE(v_SAVE);

    request->action = (uint8_t) v_SAVE[0];
    request->table_id = (uint8_t) v_SAVE[1];
    request->length = (uint16_t) v_SAVE[2];
    request->checksum = (uint16_t) v_SAVE[3];
    request->seq = (uint16_t) v_SAVE[4];
    request->total = (uint16_t) v_SAVE[5];
    request->payload.copy.from=(uint8_t)v_SAVE[6];
    request->payload.copy.to=(uint8_t)v_SAVE[7];

    //end of the mutation probe

          												[...]

	}

																	[...]

	default: {
            request_packet->data[0] = RPARAM_ERROR;
            goto out_send_status;
        }
    }

out_send_status:
    request_packet->length = 1;
    if (csp_send(conn, request_packet, 0)) {
        return;
    }
out_free:
    if (request_packet != NULL) {
        csp_buffer_free(request_packet);
    }
}
\end{lstlisting}

The user shall also modify the file \texttt{libparam/wscript} as presented below to compile with the correct macros enabled.

\begin{lstlisting}[language=python]
import gs_gcc
import gs_doc
import gs_dist
from waflib.Build import BuildContext

APPNAME = 'param'


def options(ctx):
    ctx.load('gs_gcc gs_doc')
    gs_gcc.gs_recurse(ctx)

    gr = ctx.add_option_group('libparam options')
    gr.add_option('--param-max-tables', action='store', default='20', help='Set max table instances')
    gr.add_option('--param-disable-lock', action='store_true', help='Disable table locking')
#####################################################################################################
    gr.add_option('--mutation-opt', type=str, help='set mutation') #added by FAQAS
    gr.add_option('--singleton', type=str, help='singleton?') #added by FAQAS
#####################################################################################################

def configure(ctx):
    ctx.load('gs_gcc gs_doc')

    ctx.env.append_unique('USE_PARAM', ['param_client', 'gscsp', 'util'])

    ctx.env.append_unique('FILES_PARAM', ['src/**/*.c'])
#####################################################################################################
    ctx.env.append_unique('CFLAGS', ["-DMUTATIONOPT=" + ctx.options.mutation_opt]) #added by FAQAS

    ctx.env.append_unique('CFLAGS', ["-D_FAQAS_SINGLETON_FM=" + ctx.options.singleton]) #added by FAQAS
####################################################################################################
    ctx.gs_add_doxygen(example=['tst'], exclude=['*/include/deprecated/param/*'])

    # Max number of tables
    ctx.env.append_unique('DEFINES_PARAM', 'GS_PARAM_MAX_TABLES=' + str(int(ctx.options.param_max_tables)))
    ctx.env.append_unique('DEFINES_PARAM',
                          'GS_PARAM_DISABLE_LOCK=' + str(int(1 if ctx.options.param_disable_lock else 0)))


														[...]

\end{lstlisting}


\subsection{Step 4, Step 5 and Step 6}

Before executing the rest of the step the user shall modify \texttt{DAMAT\_compile.sh} and \texttt{DAMAT\_run\_tests.sh} to include the correct commands in the indicated sections.

This is the setup for \texttt{DAMAT\_compile.sh}:

\begin{lstlisting}[language=bash]


									[...]


# here the user must invoke the compilation of the SUT, we provided a simple example.


TEST_FOLDER="/home/csp/libparam/tst"

pushd $TEST_FOLDER

for f in *; do
    if [ -d "$f" ] && [ "$f" != "include" ]; then

        pushd $f
        echo "cleaning..."
        ./waf clean

        echo "configuring..."
        if [ $singleton == "TRUE" ]; then
        ./waf configure --mutation-opt $mutant_id --singleton $singleton
        else
        ./waf configure --mutation-opt $mutant_id
        fi

        if [ $? -eq 0 ]; then
            echo $x " configuration OK"
        else
            echo $x " configuration FAILED"
        fi
        popd
    fi
done

popd


									[...]


\end{lstlisting}

And the set up for \texttt{DAMAT\_run\_tests.sh} is reported below:

\begin{lstlisting}[language=bash]

											[...]

# here the user shall call the execution of the current test case,

    tmp_log="$results_dir"/tmp_log

    TEST_FOLDER="/home/csp/libparam/tst"

    pushd $TEST_FOLDER

    pushd $tst

    touch $tmp_log

    timeout $TIMEOUT ./waf --mutation-opt=$mutant_id  --singleton=TRUE 2>&1 | tee $tmp_log

    EXEC_RET_CODE=$?

    mutant_end_time=$(($(date +%s%N)/1000000))
    mutant_elapsed="$(($mutant_end_time-$mutant_start_time))"

    if [ $EXEC_RET_CODE -ne 124 ]; then
        if grep "successfully" $tmp_log
        then
            EXEC_RET_CODE=0
            echo "PASSED"
        else
            EXEC_RET_CODE=1
            echo "FAILED"
        fi
    fi
    popd
    rm $tmp_log
    popd

											[...]

\end{lstlisting}

The user shall then execute the rest of the pipeline with the following command.

\begin{lstlisting}[language=bash]

bash DAMAt_mutants_launcher.sh

\end{lstlisting}

The results for \emph{libParam} shall be contained in the \texttt{DAMAt/results} folder.
A summary of the results can be found in the \texttt{mutation\_sum\_up.csv} file.

\begin{lstlisting}[language=bash]
all_fault_models,covered_fault_models,fault_model_coverage
6,6,1.0
covered_mutants,applied_mutants,operator_coverage
44,41,0.932
applied_mutants,killed_mutants,mutation_score
41,28,0.683
\end{lstlisting}

More details on the results of the \DAMA procedure and how to interpret them are available in Section~\ref{sec:dama_singlelaunch}.
