% !TEX root = MAIN.tex

\newpage
\subsection{Libparam Example}
\label{sec:single_machine_libparam}

This tutorial concerns the example of the Libparam Case Study provided by GomSpace Luxembourg.

The first step regards installing the \MASS framework, please refer to Section~\ref{sec:install}.

The second step, consists of creating and installing a workspace folder for running \MASS on the PARAM example. For this case, the workspace folder will be created on \texttt{/opt/PARAM}. Note that variable \texttt{\$FAQAS} represents the installation folder of the \FAQAS.

\begin{lstlisting}[language=bash]
  $ cd $FAQAS/MASS/FAQAS-Setup
  $ export INSTALL_DIR=/opt/PARAM
  $ ./install.sh
\end{lstlisting}

The third step consists of configuring the \MASS configuration file \texttt{mass\_conf.sh}. In the following, we provide excerpts of the file that require intervention from the engineer. Listing~\ref{mass_conf_single} contains the necessary configuration for the PARAM case study.

\begin{lstlisting}[language=bash, label=mass_conf_single, caption=\MASS variables. Excerpt of mass\_conf.sh file.]
# set FAQAS path
export SRCIROR=/opt/srcirorfaqas
                                                     
...

# set directory path where MASS files can be stored
export APP_RUN_DIR=/opt/PARAM

# specifies the building system, available options are "Makefile" and "waf"
export BUILD_SYSTEM="waf"

# directory root path of the software under test
export PROJ=/home/csp/libparam

# directory src path of the SUT
export PROJ_SRC=/home/csp/libparam/src

# directory test path of the SUT
export PROJ_TST=/home/csp/libparam/tst

# directory coverage path of the SUT
export PROJ_COV=$PROJ_TST

# relative path to location of gcov files (i.e., gcda and gcno files)
export GC_FILES_RELATIVE_PATH=build

# list of folders not be included during the analysis
export COVERAGE_NOT_INCLUDE="tst\|libutil\|libgscsp\|libparam_client"

# directory path of the compiled binary
export PROJ_BUILD=$PROJ/build

# filename of the compiled file/library
export COMPILED=libgsparam.so

# path to original Makefile
export ORIGINAL_MAKEFILE=$PROJ/tools/buildtools/gs/buildtools/compiler_settings.json

# compilation command of the SUT
export COMPILATION_CMD=(./waf build --notests)

# compilation additional commands of the SUT (e.g., setup of workspace) (optional)
export ADDITIONAL_CMD=""

# command to be executed after each test case (optional)
export ADDITIONAL_CMD_AFTER=""

# compilation command for TCE analysis
export TCE_COMPILE_CMD=(./waf configure \&\& ./waf build)

# command to clean installation of the SUT
export CLEAN_CMD=(./waf clean \&\& rm -rf build)
\end{lstlisting}

Also \MASS variables shall be configured within the same file. Particularly, we will run \MASS with the setup contained in Listing~\ref{mass_conf_specific}. 

\begin{lstlisting}[language=bash, label=mass_conf_specific, caption=\MASS specific variables. Excerpt of mass\_conf.sh file.]
### MASS variables

# TCE flags to be tested 
export FLAGS=("-O0" "-O1" "-O2" "-O3" "-Ofast" "-Os")

# specify if MASS will be executed on a HPC, possible values are "true" or "false"
export HPC="false"

# set if MASS should be executed with a prioritized and reduced test suite
export PRIORITIZED="true"

# set sampling technique, possible values are "uniform", "stratified", and "fsci"
# note: if "uniform" or "stratified" is set, $PRIORITIZED must be "false"
export SAMPLING="fsci"

# set sampling rate if whether "uniform" or "stratified" sampling has been selected
export RATE=""
\end{lstlisting}

The fourth step consists of configuring the prepare SUT configuration file \\(\texttt{MASS\_STEPS\_LAUNCHERS/PrepareSUT.sh}; within this file the following actions must be provided by the engineer (see~\ref{preparesut_single}):

\begin{lstlisting}[language=bash, label=preparesut_single, caption=PrepareSUT.sh file.]
#!/bin/bash

# This file should be prepared by the engineer!
cd /opt/PARAM
. ./mass_conf.sh

# 1. Compile SUT

cd $PROJ
./tools/buildtools/gsbuildtools_bootstrap.py

# generate compile_commands.json and delete build
./waf configure && ./waf build && sed -i 's:\.\./:/home/csp/libparam/:' build/compile_commands.json && mv build/compile_commands.json $MUTANTS_DIR

# 2.  Execute test cases

shopt -s extglob
for tst in $PROJ_TST/!(include|example|bindings)/;do
    echo $tst

    cd $tst
    ./waf configure

    start=$(date +%s)
    ./waf
    end=$(date +%s)

    tst_name=$(basename -- $tst)
    # call to FAQAS-CollectCodeCoverage.sh
    source $MASS/FAQAS-GenerateCodeCoverageMatrixes/FAQAS-CollectCodeCoverage.sh $tst_name "$(($end-$start))"

    cd ../
done
\end{lstlisting}

The fifth step consists of defining the function \texttt{run\_tst\_case} within the file \\\texttt{mutation\_additional\_functions.sh}. An example of its implementation is provided in Listing~\ref{mutation_additional}.

\begin{lstlisting}[language=bash, label=mutation_additional, caption='run\_tst\_case' Bash function for PARAM. Excerpt of mutation\_additional\_functions.sh file.]
run_tst_case() {
    tst_name=$1

    echo "Running test case $tst_name"

    ./waf --alltests -v
    RET=$?

    return $RET
}
\end{lstlisting}

The sixth step consists of providing a template for the build script for the trivial compiler optimizations step. In particular, we replaced the \texttt{CFLAGS\_RELEASE} value of the original build script: 

\begin{lstlisting}[language=bash, caption=libparam/tools/buildtools/gs/buildtools/compiler\_settings.json]
...
"name": "Linux x86 gcc-7, 64 bit",
"gs_part": "x86_64-gcc-7",
"gs_arch": "x86_64",
"gs_os": "linux",
"expected_cc_version": "7.4.0",
"CFLAGS": ["-std=gnu99", "-m64"],
"CFLAGS_RELEASE": ["03"],
... 
\end{lstlisting}

with \texttt{TCE}, creating a new template for the build script:

\begin{lstlisting}[language=bash, caption=libparam/tools/buildtools/gs/buildtools/compiler\_settings.json.template]
...
"name": "Linux x86 gcc-7, 64 bit",
"gs_part": "x86_64-gcc-7",
"gs_arch": "x86_64",
"gs_os": "linux",
"expected_cc_version": "7.4.0",
"CFLAGS": ["-std=gnu99", "-m64"],
"CFLAGS_RELEASE": ["TCE"],
...
\end{lstlisting}

The seventh step consists of launching the one step launcher (see Section~\ref{sec:singlelaunch}):

\begin{lstlisting}[language=bash]
 $ /opt/PARAM/Launcher.sh
\end{lstlisting}

The following results shall be reported at the end of the execution:

\begin{lstlisting}[language=bash, label=mass_output, caption=\MASS output.]
##### MASS Output #####
##
## Total mutants generated: 6450
## Total mutants filtered by TCE: 2545
## Sampling type: fsci
## Total mutants analyzed: 603
## Total killed mutants: 419
## Total live mutants: 184
## Total likely equivalent mutants: 101
## MASS mutation score (%): 83.46
## List A of useful undetected mutants: ./DETECTION/test_runs/useful_list_a
## List B of useful undetected mutants: ./DETECTION/test_runs/useful_list_b
## Number of statements covered: 1239
## Statement coverage (%): 77.29
## Minimum lines covered per source file: 0
## Maximum lines covered per source file: 179
\end{lstlisting}