% !TEX root =  ../MAIN.tex
\begin{lstlisting}[style=CStyle, caption=Example of extended data-driven mutation probe for libParam, label=GSLcover]

/**
   Get string.
   @note If the returned string is max length, the value buffer will not be 0 terminated.
   @param[in] node CSP address
   @param[in] table_id remote table id.
   @param[in] addr parameter address (remote table).
   @param[in] checksum checksum
   @param[in] timeout_ms timeout
   @param[out] value returned value (user allocated)
   @param[in] value_size size of \a value, i.e. size of parameter type in bytes.
   @return_gs_error_t
*/
static inline gs_error_t gs_rparam_get_string(uint8_t node, gs_param_table_id_t table_id, uint16_t addr,
                                              uint16_t checksum, uint32_t timeout_ms, char * value, size_t value_size)
{
    return gs_rparam_get(node, table_id, addr, GS_PARAM_STRING, checksum, timeout_ms, value, value_size);
}


gs_error_t gs_rparam_get(uint8_t node,
                         gs_param_table_id_t table_id,
                         uint16_t addr,
                         gs_param_type_t type,
                         uint16_t checksum,
                         uint32_t timeout_ms,
                         void * value,
                         size_t value_element_size)
{
    return gs_rparam_get_array(node, table_id, addr, type, checksum, timeout_ms, value, value_element_size, 1);
}


gs_error_t gs_rparam_get_array(uint8_t node,
                               gs_param_table_id_t table_id,
                               uint16_t addr,
                               gs_param_type_t type,
                               uint16_t checksum,
                               uint32_t timeout_ms,
                               void * value,
                               size_t value_element_size,
                               size_t array_size)
{
    /* Calculate length */
    gs_rparam_query_t * query;
    const size_t query_payload_size = sizeof(query->payload.addr[0]) * array_size;
    const size_t query_size = RPARAM_QUERY_LENGTH(query, query_payload_size);
    const size_t reply_payload_element_size = value_element_size + sizeof(query->payload.addr[0]);
    const size_t reply_payload_size = reply_payload_element_size * array_size;
    const size_t reply_size = RPARAM_QUERY_LENGTH(query, reply_payload_size);

    query = alloca(reply_size);
    query->action = RPARAM_GET;
    query->table_id = table_id;
    query->checksum = csp_hton16(checksum);
    query->seq = 0;
    query->total = 0;
    for(unsigned int i = 0; i < array_size; i++) {
        query->payload.addr[i] = csp_hton16(addr + (value_element_size * i));
    }
    query->length = csp_hton16(query_payload_size);

    FaultModel *fm_General = _FAQAS_General_FM();
    unsigned long long int v_General[6];

    v_General[0] = (unsigned long long int) query->action;
    v_General[1] = (unsigned long long int) query->table_id;
    v_General[2] = (unsigned long long int) query->length;
    v_General[3] = (unsigned long long int) query->checksum;
    v_General[4] = (unsigned long long int) query->seq;
    v_General[5] = (unsigned long long int) query->total;


    _FAQAS_cover(v_General,fm_General);


    /* Run single packet transaction */
    if (csp_transaction2(CSP_PRIO_HIGH, node, GS_CSP_PORT_RPARAM, timeout_ms, query, query_size, query, reply_size, CSP_O_CRC32) <= 0) {
        return GS_ERROR_IO;
    }
 ... 
 
 }

\end{lstlisting}



%flag E_Decode(E* pVal, BitStream* pBitStrm, int* pErrCode)
%{
%    flag ret = TRUE;
%    *pErrCode = 0;
%    (void)pVal;
%    (void)pBitStrm;
%
%
%    (*(pVal))=5; ret = TRUE; *pErrCode = 0;
%
%    // Manually added probe 
%    E_mutate(pVal);
%    // Manually added probe END
%    return ret  && E_IsConstraintValid(pVal, pErrCode);
%}
