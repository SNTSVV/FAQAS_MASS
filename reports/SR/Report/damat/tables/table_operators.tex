% !TEX root = ../MAIN-DataDrivenMutationAnalysis.tex


\begin{table*}[tb]
\caption{Data-driven mutation operators}
\label{table:operators}
\scriptsize
\begin{tabular}{|p{40mm}|p{90mm}|}
\hline
\textbf{Fault Class}&\textbf{Description}\\
\hline
Value above threshold (VAT)&
Replaces the current value with a value above the threshold T for a delta (\D). 
\\
\hline
Value below threshold (VBT)&
Replaces the current value with a value below the threshold T for a delta (\D). 
\\
\hline
Value out of range (VOR)&
Replaces the current value with a value out of the range $[MIN;MAX]$.\\
\hline
Bit flip (BF)&
A number of bits randomly chosen in the positions between MIN and MAX are flipped.
\\
\hline
Invalid numeric value (INV)&
Replace the current value with a mutated value that is legal (i.e., in the specified range) but different than current value. 
\\
\hline
Illegal Value (IV)
&
Replace the current value with a value that is equal to the parameter \emph{VALUE}. 
\\
\hline
Anomalous Signal Amplitude (ASA)
&
The mutated value is derived by amplifying the observed value by a factor \emph{V} and by adding/removing a constant value \D from it. 
\\
\hline
Signal Shift (SS)
&
The mutated value is derived by adding a value \D to the observed value. 
\\
\hline
Hold Value (HV)
&
This operator keeps repeating an observed value for $V$ times. It emulates a constant signal replacing a signal supposed to vary.
\\
\hline
Fix value above threshold (FVAT)&
In the presence of a value above the threshold, it replaces the current value with a value below the threshold T for a delta \D. 
\\
\hline
Fix value below threshold (FVBT)&
It is the counterpart of FVAT for the operator VBT.
\\
\hline
Fix value out of range (FVOR)&
In the presence of a value out of the range  $[MIN;MAX]$ it replaces the current value with a random value within the range. 
\\
\hline
\end{tabular}
\end{table*}%