% !TEX root = MAIN.tex




\section{Data-driven Mutation Testing: DAMTE} % (fold)
\label{sec:data:test_suite_augmentation}

\STARTCHANGEDWPT

The \INDEX{test suite augmentation process} concerns the definition of additional test cases to increase the mutation score.
It consists of four activities \INDEX{Identify Test Inputs}, \INDEX{Generate Test Oracles}, \INDEX{Execute the SUT}, \INDEX{Fix the SUT}. 
In the presence of mutants not killed by test cases (i.e., when the  \INDEX{mutation score} is not equal to 100\%), engineers are expected to manually investigate the underlying problems. Indeed, as reported in Section~\ref{sec:mutationAnalysisResults}, two might be the reasons for a low MS: poor oracle quality and missing test input sequences (i.e., the software does not reach the state in which it could kill the mutant).
If the low mutation score is due to poor oracle quality, manual work is needed because automated approaches to automatically generate test oracles in the presence of system or integration test suites are not available. 
If the low mutation score is due to missing test input sequences (i.e., the software does not reach the state in which it could kill the mutant), manual work is required because existing test generation approaches (e.g., KLEE) might suffer from scalability problem that prevent bringing the system into a desired state; also, they cannot deal with systems whose components communicate through channels. 

FAQAS thus focussed on a methodology (i.e., data-driven mutation testing, \INDEX{DAMTE}) that specifies how to rely on KLEE to generate test inputs that increase the fault model coverage and the mutation operation coverage.
When mutation operators are not applied because of the lack of appropriate data to mutate (i.e., in the presence of fault model coverage and mutation operation coverage below 100\%), engineers are expected to generate new test inputs for the SUT that enable the application of all the mutation operators. 
We thus rely on an  \INDEX{extended data mutation probe} that
invokes a version of the data mutation API that instead of mutating the data targeted by the mutation operator not covered by the fault model, includes a reachability assertion that is used to request KLEE to find a test input that reaches the mutant code. The test input shall then be inspected by the engineer, who will need then to integrate it into his test suite.
%Figure~\ref{fig:dataDrivenTestSuiteAugmentationB} exemplifies how data driven mutation testing works, for the producer consumer and client-server cases.
%
%
%
%
%\begin{figure}[h]
%  \centering
%    \includegraphics[width=14cm]{images/dataDrivenTestSuiteAugmentationB}
%      \caption{Data-driven mutation analysis for different architectures.}
%      \label{fig:dataDrivenTestSuiteAugmentationB}
%\end{figure}







