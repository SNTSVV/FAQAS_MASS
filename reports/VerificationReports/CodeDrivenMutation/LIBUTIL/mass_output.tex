% !TEX root = MAIN.tex

\chapter{Mutation Analysis Process}
\label{mass_output}

\TODO{Title should be about 'Results' not 'Process'}

\TODO{Re-read the paper to fix third person singular/plural errors}

In this chapter, we report the outcome of the mutation analysis process via MASS. Mutation analysis assess the quality of test suites by injecting artificial faults in the source code (i.e., a mutant), and then measuring the number of mutated versions that are detected by the test suite. The result is reported in terms of a mutation score, that is, the number of killed mutants (i.e., artificial faults detected by the test suite) divided by the total number of mutants.

\TODO{This is a verification report, which means that we need to provide action items for the company. What is the suggestions form below, should they reach 100\% code coverage?}

The code-driven mutation analysis process in Libutil will target all the components covered by the Libutil unit test suite. 
The Libutil unit test suite do not cover hardware-specific functions (e.g., drivers), which are covered at the system level.

\begin{table}[h]
\centering
\begin{tabular}{|l|l|}
\hline
\textbf{Coverage Type} & \textbf{Coverage Rate} \\
\hline
Statement     & 83.2\% (8\,817 of 10\,596 statements)\\
Functions     & 82.1\% (725 of 883 functions)\\
Branches      & 56.6\% (2\,618 of 4\,627 branches)\\
\hline
\end{tabular}
\caption{libutil code coverage.}
\label{table:gslibutil_coverage}
\end{table}

Table~\ref{table:gslibutil_coverage} provides the code coverage information of the unit test suite for the Libutil library. 

\TODO{We know that a subset of the files are third party and not of interest from the test suite; we should exclude them, no?}

Below, we report the files covered by the test suite and targeted by mutation analysis; they correspond to 82\% of the source files. 

\begin{itemize}
	\item src/base16.c
	\item src/bytebuffer.c
	\item src/byteorder.c
	\item src/clock.c
	\item src/crc32.c
	\item src/crc8.c
	\item src/error.c
	\item src/fletcher.c
	\item src/function\_scheduler.c
	\item src/hexdump.c
	\item src/lock.c
	\item src/rtc.c
	\item src/string.c
	\item src/strtoint.c
	\item src/time.c
	\item src/timestamp.c
	\item src/cmd/command.c
	\item src/cmd/log.c
	\item src/cmd/vmem.c
	\item src/drivers/sys/memory.c
	\item src/gosh/command.c
	\item src/gosh/console.c
	\item src/gosh/default\_commands.c
	\item src/linux/clock.c
	\item src/linux/command\_line.c
	\item src/linux/cwd.c
	\item src/linux/delay.c
	\item src/linux/function.c
	\item src/linux/mutex.c
	\item src/linux/queue.c
	\item src/linux/rtc.c
	\item src/linux/sem.c
	\item src/linux/signal.c
	\item src/linux/stdio.c
	\item src/linux/thread.c
	\item src/linux/time.c
	\item src/linux/drivers/gpio/gpio.c
	\item src/linux/drivers/gpio/gpio\_sysfs.c
	\item src/linux/drivers/gpio/gpio\_virtual.c
	\item src/linux/drivers/i2c/i2c.c
	\item src/linux/drivers/spi/spi.c
	\item src/linux/drivers/sys/memory.c
	\item src/log/commands.c
	\item src/log/log.c
	\item src/log/appender/console.c
	\item src/log/appender/simple\_file.c
	\item src/test/cmocka.c
	\item src/test/command.c
	\item src/test/log.c
	\item src/vmem/commands.c
	\item src/vmem/vmem.c
	\item src/watchdog/monitor\_task.c
	\item src/watchdog/watchdog.c
	\item src/zip/zip.c
	\item src/zip/miniz/miniz.c
\end{itemize}


\section{MASS Output}

After applying MASS to the artifact, we report in the following the metrics produced by the tool:

\begin{itemize}
	\item MASS generates a total of 20\,268 mutants, this set of mutants was generated using the operators ABS, AOR, ICR, LCR, ROR, SDL, UOI, AOD, LOD, ROD, BOD, SOD, and LVR.

	\item 5\,694 mutants are filtered by trivial compiler optimisations, that is, mutants that are equivalent with respect to the original program or redundant with respect to other mutants.

	\item In total, MASS analyzed 14\,574 mutants, the process took 3\,567 minutes (59 hours) to complete.

	\item The Libutil test suite identified a total of 10\,376 mutants (i.e., killed mutants).

	\item The Libutil test suite did not identify a total of 4\,198 mutants (i.e., live mutants).

	\item The partial mutation score is 71.20\%.

	\item The statement coverage is 83.2\%.
	
	\item The total number of statements covered is 8\,817.
	
	\item The minimum statements covered per source file is 2, and the maximum is 817 statements per source file.
\end{itemize}

\TODO{What do we conclude form the above? Are you excluding source files that are not of interest for LXS?}
