% !TEX root = MAIN.tex

\chapter{Results}

% \section{Metrics}

Starting from the mutation operators represented in Table~\ref{tab:fault_model}, \DAMA generated \textbf{29} mutants. Every mutated version of the program was executed against the \case test suite.

The results were expressed with the following three metrics:
\begin{enumerate}
\item \textbf{FMC}: Fault model coverage, the percentage of fault models covered by the test suite.
\item \textbf{MOC}: Mutation operation coverage, the percentage of data items that have been mutated at least once, considering only those that belong to the data buffers covered by the test suite.
\item \textbf{MS}: Mutation Score, the percentage of mutants killed by the test suite (i.e., leading to at least one test case failure) among the mutants that target a fault model and for which at least one mutation operation was successfully performed.
\end{enumerate}

% \TODO{Why "low score" ? A score below 100\% indicates the presence of test suite shortcomings}
A score below 100\% in one of the metrics indicate one of following scenarios, respectively:
\begin{enumerate}
% \item \TODO{I think it should be 'one or more message types ...'}
\item \textbf{FMC below 100\%}: one or more \textbf{message types} targeted by a fault model is \textbf{never exercised}.
% \item \TODO{I think it should be 'for the message types covered by the test suite, not all the input part......'}
\item \textbf{MOC below 100\%}: for the message types covered by the test suite, \textbf{not all the input partitions} \textbf{are exercised} by the test suite.
% \item \TODO{'The test suite does not always fail in the presence of mutated data. Precisely, none of the test cases can detect the presence of mutated data even if the data is systematically mutated (i.e., the expected data is never observed).'}.
\item \textbf{MS below 100\%}: The test suite does not always fail in the presence of mutated data. Precisely, none of the test cases can detect the presence of mutated data even if the data is systematically mutated (i.e., the expected data is never observed) It may depend on two reasons: (1) \textbf{the oracles are imprecise} (e.g., they do not verify all the state variables), (2) the system is not brought into a state where the effect of the mutation is noticeable: \textbf{the scenarios exercised are insufficient}.
\end{enumerate}

The aforementioned metrics for the \case are reported in Table~\ref{tab:metrics}.



% !TEX root = ../MAIN.tex

\begin{table}[H]

\begin{tabular}{|l|l|}
\hline
Fault Models & 1 \\
Covered Fault Models & 1 \\
\hline
\textbf{FMC} & \textbf{100.00\%} \\
\hline
Covered Mutants & 33 \\
Applied Mutants & 21 \\
\hline
\textbf{MOC} & \textbf{63.64\%} \\
\hline
Applied Mutants & 21 \\
Killed mutants & NA \\
\hline
\textbf{MS} & \textbf{NA} \\
\hline
\end{tabular}
\caption{\DAMA metrics for the \case}
\label{tab:metrics}
\end{table}


\action The metrics in Table~\ref{tab:metrics}. show that the test suite presents some potential shortcomings, please verify them according to the suggestions presented in the following Sections.

\section{Fault Model Coverage}

The \textbf{FMC} is \textbf{100\%}. There is no fault model that is not covered by the test suite of the \case.

\section{Mutation Operation Coverage}

The \emph{mutation operation coverage} was \textbf{82.76\%}. A total of \textbf{5} mutants were not applied: the function implementing the mutation probe was called, so the message type was covered by the test suite, but it was not possible to perform the corresponding mutation operation.
The conditions for this are summarized in Table~\ref{tab:operators} for all the operators.
The mutants are presented in Table~\ref{tab:not_applied}, which contains the definition of the mutation operator that generated them and a suggestion on how to improve the test suite.

\action Apply the suggestions reported in Table~\ref{tab:not_applied}.

 {
\scriptsize
% Please add the following required packages to your document preamble:
% \usepackage{longtable}
% Note: It may be necessary to compile the document several times to get a multi-page table to line up properly
\begin{longtable}{|l|l|l|l|l|l|l|l|l|l|l|l|p{0.1 \textwidth}|}
\caption{Mutants covered by the \case test suite, but unable to apply the mutation.}
\label{tab:not_applied}\\
\hline
\textbf{\#} &
\textbf{FaultModel} &
 \textbf{D.Item} &
 \textbf{Span} &
 \textbf{Type} &
 \textbf{F.Class} &
 \textbf{Min} &
 \textbf{Max} &
 \textbf{Thresh.} &
 \textbf{Delta} &
 \textbf{State} &
 \textbf{Value} &
 \textbf{Suggestion}
 \\ \hline
\endfirsthead
%
\endhead
% 10 & IfHK & 12 & 2 & DOUBLE & FVAT & NA & NA & 3.6 & 0.1 & NA & NA & Check if value of VCCb > \emph{Threshold} is ever tested \\ \hline
2 & Identifier & 1 & 1 & INT & FVAT & NA & NA & 3 & 1 & NA & NA & Check if value of pri > \emph{Threshold} is ever tested\\ \hline
3 & Identifier & 1 & 1 & INT & BF & 0 & 0 & NA & NA & 1 & 1 & Check if bit at position \emph{Value} of data item pri is ever tested with value = \emph{State}\\ \hline
5 & Identifier & 1 & 1 & INT & BF & 1 & 1 & NA & NA & 1 & 1 & Check if bit at position \emph{Value} of data item pri is ever tested with value = \emph{State}\\ \hline
9 & Identifier & 2 & 1 & INT & FVAT & NA & NA & 31 & 1 & NA & NA & Check if value of src > \emph{Threshold} is ever tested\\ \hline
12 & Identifier & 3 & 1 & INT & FVAT & NA & NA & 31 & 1 & NA & NA & Check if value of dst > \emph{Threshold} is ever tested\\ \hline
14 & Identifier & 4 & 1 & INT & VOR & 8 & 30 & NA & 1 & NA & NA & Check if value of dport in range \emph{Min-Max} is ever tested\\ \hline
15 & Identifier & 4 & 1 & INT & VOR & 8 & 30 & NA & 1 & NA & NA & Check if value of dport in range \emph{Min-Max} is ever tested\\ \hline
20 & Identifier & 5 & 1 & INT & VOR & 8 & 30 & NA & 1 & NA & NA &  Check if value of sport in range \emph{Min-Max} is ever tested\\ \hline
21 & Identifier & 5 & 1 & INT & VOR & 8 & 30 & NA & 1 & NA & NA &  Check if value of sport in range \emph{Min-Max} is ever tested\\ \hline
27 & Identifier & 6 & 1 & BIN & BF & 1 & 1 & NA & NA & 1 & 1 & Check if bit at position \emph{Value} of data item flags is ever tested with value = \emph{State}\\ \hline
29 & Identifier & 6 & 1 & BIN & BF & 2 & 0 & NA & NA & 1 & 1 & Check if bit at position \emph{Value} of data item flags is ever tested with value = \emph{State}\\ \hline
31 & Identifier & 6 & 1 & BIN & BF & 3 & 1 & NA & NA & 1 & 1 & Check if bit at position \emph{Value} of data item flags is ever tested with value = \emph{State}\\ \hline

\end{longtable}}



\section{Mutation Score}

 The \emph{mutation score} was \textbf{100\%}. A total of \textbf{0} mutants were applied but not killed.
