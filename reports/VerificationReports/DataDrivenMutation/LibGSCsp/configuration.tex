% !TEX root = MAIN.tex

\chapter{Configuration of the tool-set}

\section{Probe insertion}

In the \case the mutation probes were inserted into the inserted in the function \emph{csp\_service\_handler}, contained in the file \emph{libgscsp/lib/libcsp/src/csp\_service\_handler.c}

This function handles an incoming connection and depending on the destination port, handles the packet to the corresponding service handler. The function takes as a parameter a connection structure of type \emph{csp\_conn\_}t and a \emph{LibGSCSP} packet of type \emph{csp\_packet\_t}. Both are represented by custom data structures.

We were interested in mutating \emph{idin} field, which contain the header of the LibGSCSP packet.

The packet header is represented by the data structure \emph{csp\_id\_t}. To enable the application of \DAMA, the data is converted into an array before mutation and converted back afterwards.

In both cases, a \emph{Mutation Probe} has been inserted to mutate the data contained in the buffer.
An example of this probe insertion strategy is reported in Listing~\ref{lst:probe_insertion}.

\begin{lstlisting}[language=C++, caption=Probe insertion Strategy, label={lst:probe_insertion}]

// manually inserted probe

int damat_buffer[7];

damat_buffer[0] = 0;
damat_buffer[1] = csp_conn_pri_faqas(conn);
damat_buffer[2] = csp_conn_src(conn);//conn->idin.src;
damat_buffer[3] = csp_conn_dst(conn);//conn->idin.dst;
damat_buffer[4] = csp_conn_dport(conn);//conn->idin.dport;
damat_buffer[5] = csp_conn_sport(conn);//conn->idin.sport;
damat_buffer[6] = csp_conn_flags(conn);//conn->idin.flags;

mutate_FM_Identifier(damat_buffer);

set_conn_idin_pri(conn,damat_buffer[1]);
set_conn_idin_src(conn,damat_buffer[2]);
set_conn_idin_dst(conn,damat_buffer[3]);
set_conn_idin_dport(conn,damat_buffer[4]);
set_conn_idin_sport(conn,damat_buffer[5]);
set_conn_idin_flags(conn,damat_buffer[6]);

// end of probe

\end{lstlisting}

\section{Fault Model}

The target of the mutation was the header of the LibGSCSP packet on so we defined a single specific Fault Model, called \emph{Identifier}; the relative mutation operators are reported in Table~\ref{tab:fault_model}.

{
\scriptsize
% Please add the following required packages to your document preamble:
% \usepackage{longtable}
% Note: It may be necessary to compile the document several times to get a multi-page table to line up properly
\begin{longtable}{|l|l|l|l|l|l|l|l|l|l|l|}
\caption{Fault model for the \case}
\label{tab:fault_model}\\
\hline
\textbf{FaultModel} &
 \textbf{DataItem} &
 \textbf{Span} &
 \textbf{Type} &
 \textbf{FaultClass} &
 \textbf{Min} &
 \textbf{Max} &
 \textbf{Threshold} &
 \textbf{Delta} &
 \textbf{State} &
 \textbf{Value} \\ \hline
\endfirsthead
%
\endhead
%
%IfStatus & 0 & 1 & BIN & BF & 3 & 3 & NA & NA & -1 & 1 \\ \hline
General & 0 & 1 & LONG & IV & NA & NA & NA & NA & NA & 0x00\\ \hline
General & 0 & 1 & LONG & IV & NA & NA & NA & NA & NA & 0x55\\ \hline
General & 0 & 1 & LONG & IV & NA & NA & NA & NA & NA & 0xff\\ \hline
General & 0 & 1 & LONG & IV & NA & NA & NA & NA & NA & 0x44\\ \hline
General & 0 & 1 & LONG & IV & NA & NA & NA & NA & NA & 0x77\\ \hline
General & 0 & 1 & LONG & IV & NA & NA & NA & NA & NA & 0x88\\ \hline
General & 0 & 1 & LONG & IV & NA & NA & NA & NA & NA & 0x89\\ \hline
General & 0 & 1 & LONG & IV & NA & NA & NA & NA & NA & 0x99\\ \hline
General & 0 & 1 & LONG & IV & NA & NA & NA & NA & NA & 0x9a\\ \hline
General & 1 & 1 & LONG & VAT & NA & NA & 20 & 1 & NA & NA\\ \hline
General & 1 & 1 & LONG & FVAT & NA & NA & 20 & 1 & NA & NA\\ \hline
General & 1 & 1 & LONG & SS & NA & NA & NA & 1 & NA & NA\\ \hline
General & 1 & 1 & LONG & SS & NA & NA & NA & -1 & NA & NA\\ \hline
General & 1 & 1 & LONG & HV & NA & NA & NA & NA & NA & 10\\ \hline
General & 2 & 1 & LONG & SS & NA & NA & NA & 1 & NA & NA\\ \hline
General & 2 & 1 & LONG & SS & NA & NA & NA & -1 & NA & NA\\ \hline
General & 2 & 1 & LONG & IV & NA & NA & NA & NA & NA & 0\\ \hline
General & 2 & 1 & LONG & VAT & NA & NA & 180 & 1 & NA & NA\\ \hline
General & 2 & 1 & LONG & FVAT & NA & NA & 180 & 1 & NA & NA\\ \hline
General & 4 & 1 & LONG & SS & NA & NA & NA & 1 & NA & NA\\ \hline
General & 4 & 1 & LONG & SS & NA & NA & NA & -1 & NA & NA\\ \hline
General & 5 & 1 & LONG & SS & NA & NA & NA & 1 & NA & NA\\ \hline
General & 5 & 1 & LONG & SS & NA & NA & NA & -1 & NA & NA\\ \hline
GET & 6 & 1 & LONG & SS & NA & NA & NA & 1 & NA & NA\\ \hline
GET & 6 & 1 & LONG & SS & NA & NA & NA & -1 & NA & NA\\ \hline
SET & 6 & 1 & BIN & BF & 0 & 7 & NA & NA & -1 & 1\\ \hline
SAVE & 6 & 1 & LONG & SS & NA & NA & NA & 1 & NA & NA\\ \hline
SAVE & 6 & 1 & LONG & SS & NA & NA & NA & -1 & NA & NA\\ \hline
SAVE & 7 & 1 & LONG & SS & NA & NA & NA & 1 & NA & NA\\ \hline
SAVE & 7 & 1 & LONG & SS & NA & NA & NA & -1 & NA & NA\\ \hline
LOAD & 6 & 1 & LONG & SS & NA & NA & NA & 1 & NA & NA\\ \hline
LOAD & 6 & 1 & LONG & SS & NA & NA & NA & -1 & NA & NA\\ \hline
LOAD & 7 & 1 & LONG & SS & NA & NA & NA & 1 & NA & NA\\ \hline
LOAD & 7 & 1 & LONG & SS & NA & NA & NA & -1 & NA & NA\\ \hline
REPLY & 2 & 1 & LONG & SS & NA & NA & NA & 1 & NA & NA\\ \hline
REPLY & 2 & 1 & LONG & SS & NA & NA & NA & -1 & NA & NA\\ \hline
REPLY & 2 & 1 & LONG & IV & NA & NA & NA & NA & NA & 0\\ \hline
REPLY & 2 & 1 & LONG & VAT & NA & NA & 180 & 1 & NA & NA\\ \hline
REPLY & 2 & 1 & LONG & FVAT & NA & NA & 180 & 1 & NA & NA\\ \hline
REPLY & 4 & 1 & LONG & SS & NA & NA & NA & 1 & NA & NA\\ \hline
REPLY & 4 & 1 & LONG & SS & NA & NA & NA & -1 & NA & NA\\ \hline
REPLY & 5 & 1 & LONG & SS & NA & NA & NA & 1 & NA & NA\\ \hline
REPLY & 5 & 1 & LONG & SS & NA & NA & NA & -1 & NA & NA\\ \hline
REPLY & 6 & 1 & BIN & BF & 0 & 7 & NA & NA & -1 & 1\\ \hline
\end{longtable}}


Every line of the fault model file represents a mutation operator, while every column represents a configuration parameter for that operator.

\begin{itemize}
  \item Column \emph{FaultModel} contains the name of the Fault Model containing the operator. Typically the user shall define a fault model for every different kind of message exchanged through the buffer.

  \item Column \emph{DataItem} refers to the index of the first element of the targeted data item in the array representing the buffer.

  \item Column \emph{Span} reports the number of array elements that make up the data item target by the mutation.

  \item Column \emph{Type} reports about the type of data targeted by the mutation: \texttt{INT}, \texttt{LONG}, \texttt{FLOAT}, \texttt{DOUBLE}, \texttt{BIN} or \texttt{HEX}.

  \item Column \emph{FaultClass} contains the type of fault that the mutation will emulate, depending on the chosen mutation operator. A summary of the mutation operators can be found in Table~\ref{tab:operators}.

  \item The other columns represent configuration parameters and assume different meanings depending on the mutation operator they refer to. More details on the data-driven mutation operators and their configuration can be found in Table~\ref{tab:operators}.

\end{itemize}

% !TEX root =  ../Main.tex

\newcommand{\op}{\mathit{op}}
\newcommand{\ArithmeticSet}{ \texttt{+}, \texttt{-}, \texttt{*}, \texttt{/}, \texttt{\%} }
\newcommand{\LogicalSet}{ \texttt{&&}, \texttt{||} }
\newcommand{\RelationalSet}{ \texttt{>}, \texttt{>=}, \texttt{<}, \texttt{<=}, \texttt{==}, \texttt{!=} }
\newcommand{\BitWiseSet}{ \texttt{\&}, \texttt{|}, \land }
\newcommand{\ShiftSet}{ \texttt{>>}, \texttt{<<} }


\begin{table}[h]
\caption{Implemented set of mutation operators.}
\label{table:operators} 
\centering
\scriptsize
\begin{tabular}{|@{}p{4mm}@{}|@{}p{2cm}@{\hspace{1pt}}|@{}p{11.1cm}@{}|}
\hline
&\textbf{Operator} & \textbf{Description$^{*}$} \\
\hline
\multirow{7}{*}{\rotatebox{90}{\emph{Sufficient Set}}}&ABS               & $\{(v, -v)\}$	\\
\cline{2-3}
&AOR               & $\{(\op_1, op_2) \,|\, \op_1, \op_2 \in \{ \ArithmeticSet \} \land \op_1 \neq \op_2 \} $       \\
&    			  & $\{(\op_1, \op_2) \,|\, \op_1, \op_2 \in \{\texttt{+=}, \texttt{-=}, \texttt{*=}, \texttt{/=}, \texttt{\%} \texttt{=}\} \land \op_1 \neq \op_2 \} $       \\
\cline{2-3}
&ICR               & $\{i, x) \,|\, x \in \{1, -1, 0, i + 1, i - 1, -i\}\}$           \\
\cline{2-3}
&LCR               & $\{(\op_1, \op_2) \,|\, \op_1, \op_2 \in \{ \texttt{\&\&}, || \} \land \op_1 \neq \op_2 \}$            \\
&				  & $\{(\op_1, \op_2) \,|\, \op_1, \op_2 \in \{ \texttt{\&=}, \texttt{|=}, \texttt{\&=}\} \land \op_1 \neq \op_2 \}$            \\
&				  & $\{(\op_1, \op_2) \,|\, \op_1, \op_2 \in \{ \texttt{\&}, \texttt{|}, \texttt{\&\&}\} \land \op_1 \neq \op_2 \}$            \\
\cline{2-3}
&ROR               & $\{(\op_1, \op_2) \,|\, \op_1, \op_2 \in \{ \RelationalSet \}\}$            \\
&				  & $\{ (e, !(e)) \,|\, e \in \{\texttt{if(e)}, \texttt{while(e)}\} \}$ \\
\cline{2-3}
&SDL               & $\{(s, \texttt{remove}(s))\}$            \\
\cline{2-3}
&UOI               & $\{ (v, \texttt{--}v), (v, v\texttt{--}), (v, \texttt{++}v), (v, v\texttt{++}) \}$            \\   
\hline
\hline
\multirow{5}{*}{\rotatebox{90}{\emph{OODL}}}&AOD               & $\{((t_1\,op\,t_2), t_1), ((t_1\,op\,t_2), t_2) \,|\, op \in \{ \ArithmeticSet \} $       \\ 
\cline{2-3}
&LOD               & $\{((t_1\,op\,t_2), t_1), ((t_1\,op\,t_2), t_2) \,|\, op \in \{  \} \}$       \\ 
\cline{2-3}
&ROD               & $\{((t_1\,op\,t_2), t_1), ((t_1\,op\,t_2), t_2) \,|\, op \in \{ \RelationalSet \} \}$       \\ 
\cline{2-3}
&BOD               & $\{((t_1\,op\,t_2), t_1), ((t_1\,op\,t_2), t_2) \,|\, op \in \{ \BitWiseSet \} \}$       \\ 
\cline{2-3}
&SOD               & $\{((t_1\,op\,t_2), t_1), ((t_1\,op\,t_2), t_2) \,|\, op \in \{ \ShiftSet \} \}$       \\ 
%\hline
%COR               & $\{(\op_1, \op_2) \,|\, \op_1, \op_2 \in \{ \texttt{\&\&}, \texttt{||}, \land \} \land \op_1 \neq \op_2 \}$            \\
\hline
\hline
\multirow{3}{*}{\rotatebox{90}{\emph{Other}}}&LVR			& $\{(l_1, l_2) \,|\, (l_1, l_2) \in \{(0,-1), (l_1,-l_1), (l_1, 0), (\mathit{true}, \mathit{false}), (\mathit{false}, \mathit{true})\}\}$\\
&&\\
&&\\
\hline
\end{tabular}

$^{*}$Each pair in parenthesis shows how a program element is modified by the mutation operator. Th eleft element of the pair is replaced with the right element. We follow standard syntax~\cite{kintis2018effective}. Program elements are literals ($l$), integer literals ($i$), boolean expressions ($e$), operators ($\op$), statements ($s$), variables ($v$), and terms ( $t_i$, which might be either variables or literals).
\end{table}

A mutation operator can generate one or more mutants performing a \emph{Mutation Operation}.
