% !TEX root =  ../Main.tex

\subsection{Applicability of state-of-the-art solutions to space software}
\label{sec:background}

In this section, we discuss the applicability of state-of-the-art mutation testing optimizations in the context of space software. A detailed overview of mutation testing solutions and optimizations can be found in deliverable D1.

\subsubsection{Mutation adequacy and mutation score computation}

A mutant is said to be killed if at least one test case of the test suite fails when executed against the mutant.
Mutants that do not lead to the failure of any test case are said to be live.
Three conditions should hold for a test case to kill a mutant: reachability (i.e, the test case should execute the mutated statement), necessity (i.e., the test case should reach an incorrect intermediate state after executing the mutated statement), and sufficiency (i.e., the final state of the mutated program should differ from that of the original program)~\cite{offutt1997automatically}.

%The identification of killed and live mutants enables the definition of a mutation adequacy criterion as follow, a test suite is mutation-adequate if all mutants are killed by at least one test case of the test suite. Also, 
The mutation score, i.e., the percentage of killed mutants, measures the quality of a test suite quantitatively. Recent studies have shown that achieving a high mutation score improves significantly the fault detection capability of a test suite
~\cite{papadakis2018mutation}. 
%However, 
%to ensure better fault detection than a randomly selected subset of test cases, a test suite should achieve a very high mutation score~\cite{Chekam:17}.
%More precisely, they show that among randomly selected test suites, ranked based on mutation score and structural coverage, 
%only  the test suites in the top 5\% rank according to mutation score achieve a better fault detection rate than the ones ranked according to other criteria (e.g., branch coverage)~\cite{Chekam:17}. 
%The literature lacks studies on the identification of a mutation score threshold that guarantees achieving a fault detection rate higher than the one achieved by other adequacy criteria. 
However, a very high mutation score (i.e., above 0.75) is required to achieve a higher fault detection rate than the one obtained with other coverage criteria, such as statement and branch coverage~\cite{Chekam:17}.

The capability of a test case to kill a mutant also depends on the observability of the program state. To overcome the limitations due to observability, different strategies to identify killed mutants can be adopted; they are known as strong, weak, firm, and flexible mutation coverage~\cite{ammann2016introduction}. For space software, we suggest to rely on strong mutation because it is the only criterion that assesses the actual test suite's capability of detecting a fault; indeed, it relies on a mutation score that reflects the percentage of mutants identified by test failures. With the other mutation coverage criteria, a mutant is killed if the state of the mutant after the execution of the mutated statement differs from the one observed with the original code without any guarantee that either the erroneous values in state variables propagate or test oracles detect them. 




\subsubsection{Mutation Operators}
\label{sec:related:operators}

%% !TEX root =  ../Main.tex

\newcommand{\op}{\mathit{op}}
\newcommand{\ArithmeticSet}{ \texttt{+}, \texttt{-}, \texttt{*}, \texttt{/}, \texttt{\%} }
\newcommand{\LogicalSet}{ \texttt{&&}, \texttt{||} }
\newcommand{\RelationalSet}{ \texttt{>}, \texttt{>=}, \texttt{<}, \texttt{<=}, \texttt{==}, \texttt{!=} }
\newcommand{\BitWiseSet}{ \texttt{\&}, \texttt{|}, \land }
\newcommand{\ShiftSet}{ \texttt{>>}, \texttt{<<} }


\begin{table}[h]
\caption{Implemented set of mutation operators.}
\label{table:operators} 
\centering
\scriptsize
\begin{tabular}{|@{}p{4mm}@{}|@{}p{2cm}@{\hspace{1pt}}|@{}p{11.1cm}@{}|}
\hline
&\textbf{Operator} & \textbf{Description$^{*}$} \\
\hline
\multirow{7}{*}{\rotatebox{90}{\emph{Sufficient Set}}}&ABS               & $\{(v, -v)\}$	\\
\cline{2-3}
&AOR               & $\{(\op_1, op_2) \,|\, \op_1, \op_2 \in \{ \ArithmeticSet \} \land \op_1 \neq \op_2 \} $       \\
&    			  & $\{(\op_1, \op_2) \,|\, \op_1, \op_2 \in \{\texttt{+=}, \texttt{-=}, \texttt{*=}, \texttt{/=}, \texttt{\%} \texttt{=}\} \land \op_1 \neq \op_2 \} $       \\
\cline{2-3}
&ICR               & $\{i, x) \,|\, x \in \{1, -1, 0, i + 1, i - 1, -i\}\}$           \\
\cline{2-3}
&LCR               & $\{(\op_1, \op_2) \,|\, \op_1, \op_2 \in \{ \texttt{\&\&}, || \} \land \op_1 \neq \op_2 \}$            \\
&				  & $\{(\op_1, \op_2) \,|\, \op_1, \op_2 \in \{ \texttt{\&=}, \texttt{|=}, \texttt{\&=}\} \land \op_1 \neq \op_2 \}$            \\
&				  & $\{(\op_1, \op_2) \,|\, \op_1, \op_2 \in \{ \texttt{\&}, \texttt{|}, \texttt{\&\&}\} \land \op_1 \neq \op_2 \}$            \\
\cline{2-3}
&ROR               & $\{(\op_1, \op_2) \,|\, \op_1, \op_2 \in \{ \RelationalSet \}\}$            \\
&				  & $\{ (e, !(e)) \,|\, e \in \{\texttt{if(e)}, \texttt{while(e)}\} \}$ \\
\cline{2-3}
&SDL               & $\{(s, \texttt{remove}(s))\}$            \\
\cline{2-3}
&UOI               & $\{ (v, \texttt{--}v), (v, v\texttt{--}), (v, \texttt{++}v), (v, v\texttt{++}) \}$            \\   
\hline
\hline
\multirow{5}{*}{\rotatebox{90}{\emph{OODL}}}&AOD               & $\{((t_1\,op\,t_2), t_1), ((t_1\,op\,t_2), t_2) \,|\, op \in \{ \ArithmeticSet \} $       \\ 
\cline{2-3}
&LOD               & $\{((t_1\,op\,t_2), t_1), ((t_1\,op\,t_2), t_2) \,|\, op \in \{  \} \}$       \\ 
\cline{2-3}
&ROD               & $\{((t_1\,op\,t_2), t_1), ((t_1\,op\,t_2), t_2) \,|\, op \in \{ \RelationalSet \} \}$       \\ 
\cline{2-3}
&BOD               & $\{((t_1\,op\,t_2), t_1), ((t_1\,op\,t_2), t_2) \,|\, op \in \{ \BitWiseSet \} \}$       \\ 
\cline{2-3}
&SOD               & $\{((t_1\,op\,t_2), t_1), ((t_1\,op\,t_2), t_2) \,|\, op \in \{ \ShiftSet \} \}$       \\ 
%\hline
%COR               & $\{(\op_1, \op_2) \,|\, \op_1, \op_2 \in \{ \texttt{\&\&}, \texttt{||}, \land \} \land \op_1 \neq \op_2 \}$            \\
\hline
\hline
\multirow{3}{*}{\rotatebox{90}{\emph{Other}}}&LVR			& $\{(l_1, l_2) \,|\, (l_1, l_2) \in \{(0,-1), (l_1,-l_1), (l_1, 0), (\mathit{true}, \mathit{false}), (\mathit{false}, \mathit{true})\}\}$\\
&&\\
&&\\
\hline
\end{tabular}

$^{*}$Each pair in parenthesis shows how a program element is modified by the mutation operator. Th eleft element of the pair is replaced with the right element. We follow standard syntax~\cite{kintis2018effective}. Program elements are literals ($l$), integer literals ($i$), boolean expressions ($e$), operators ($\op$), statements ($s$), variables ($v$), and terms ( $t_i$, which might be either variables or literals).
\end{table}


Mutation testing introduces small syntactical changes into the code (source code or machine code) of a program through a set of mutation operators that simulate programming mistakes. 

The  \emph{sufficient set of operators} is widely used for conducting empirical evaluations ~\cite{offutt1996experimental,rothermel1996experimental,andrews2005mutation,kintis2017detecting}. 
%Initially defined by Offutt et al., the set has been extended to include newly defined operators.
The original sufficient set, defined by Offutt et al., is composed of the following operators: Absolute Value Insertion (ABS), Arithmetic Operator Replacement (AOR), Integer Constraint Replacement (ICR), Logical Connector Replacement (LCR), Relational Operator Replacement (ROR), and Unary Operator Insertion (UOI)~\cite{offutt1996experimental}.
% operator and the \INDEX{statement deletion operator} (SDL).
Andrews et al.~\cite{andrews2005mutation} have included the 
\emph{statement deletion operator} (SDL)~\cite{delamaro2014designing}, which ensures that every pointer-manipulation and field-assignment statement is tested. 
%Table~\ref{table:sufficient_operators} provides an overview of the operators belonging to the sufficient set.
%, thus targeting faults not simulated by the rest of the sufficient operators. In addition, recent research results show that the SDL operator is the most effective for fault detection~\cite{delamaro2014designing}. 
\MREVISION{C-P-06}{Deletion operators produce significantly less equivalent mutants~
\cite{delamaro2014designing,delamaro2014experimental}; also, 
test suites that kill mutants generated with the SDL and the \emph{OODL operators} kill a very high percentage of all mutants (i.e., 97\%)~\cite{delamaro2014experimental}. 
The OODL mutation operators delete Arithmetic (AOD), Bitwise (BOD), Logical (LOD), Relational (ROD), and Shift (SOD) operators in the program source code.}

The sufficient set of operators enables an accurate estimation of the mutation score of a test suite~\cite{siami2008sufficient}; furthermore, the mutation score computed with the sufficient set is a good estimate the fault detection rate (i.e., the portion of real faults discovered) of a test suite~\cite{andrews2005mutation}. 

\MREVISION{C-P-06}{Recent work has shown that, to maximize the detection of real faults, a set of operators should be used in addition to the sufficient set and includes: Conditional Operator Replacement (COR),
Literal Value Replacement (LVR), and Arithmetic Operator Deletion (AOD)~\cite{Kintis2018}. }






An alternative to the sufficient set of operators is the generation of higher order mutants, which result from the application of multiple mutation operators for each mutation~\cite{jia2009higher,kintis2010evaluating,offutt1992investigations,papadakis2010empirical}. However, higher order mutants are of \emph{relatively lower strength than the first order ones}~\cite{papadakis2010mutation,papadakis2019mutation}, also there is \emph{lack of clear theory on which mutants are of some value}, and they may lead to redundant mutants~\cite{papadakis2019mutation}. For this reason, we focus our study on first order mutations.
% generated with the sufficient set.

%decreases consistently. 
%For instance, Papadakis and Malevris~\cite{papadakis2010empirical}, worked on a approach for higher order mutants for the C programming language that lead to a reduction of approximately 80-90\% of the generated equivalent mutants, with a fault detection ability loss only of 11-15\%. 
%
%
%that higher order mutants are of relatively lower strength than the first order ones
%
%Since modern space software is implemented in C and C++, thus sharing a degree of similarity with the case studies considered in the empirical evaluations for the sufficient set, we believe that the empirical findings concerning the sufficient set also hold for space software.

%rely on the sufficient set of operators.
%, to determine if results reported in the literature may generalize to the case of space software.
%For these reasons, we believe the sufficient set of operators to be necessary to be considered also for space software.

%% !TEX root =  ../Main.tex

\begin{table}
\scriptsize
\centering
\label{table:codeoperatorssummary}
\caption{Categories of mutation operators applicable to space software.}
\begin{tabular}{p{1.8cm}|p{5.4cm}}
\toprule
Operator Category & Description\\
\midrule
arithmetic & Includes operators that performs mutations on arithmetic expressions and assignments, usually these mutations replaces an element by an operator $op \in \{+, -, *, /, \%\}$. \\
bitwise & Includes operators that performs bitwise mutations on expressions and assignments, these mutations aims for replacing an element of an expression by an operator $op \in \{|, \&, \wedge, \sim\}$. \\
casts & Includes operators that work by mutating casting expressions.\\
constants & Includes operators that mutate constants in the code. \\
control-flow & Includes operators that modify and alter the execution flow of the running application.\\
coverage & Includes operators that mutate the code in such a way that the static structure of the code (e.g., branches) is likely covered during execution.\\
deletion & Includes operators that mutate the code by deleting statements, operators, constants and variables.  \\
floating points & Includes operators working on floating point comparison (FPC).\\
function calls & Includes operators that mutate function calls and their components (e.g., signature and parameters).\\
logical & Includes operators that performs logical mutations on expressions and assignments; these mutations replace an element of an expression by an operator $op \in \{\&\&, \|, !\}$.\\
variable references & Includes operators mutating different types of variable references in the code (e.g., scalars, arrays and pointers).\\
shift & Includes operators that performs shift mutations on expressions and assignments; usually, these mutations replace an element by an operator $op \in \{<<, <<<, >>, >>>\}$.\\
SQL & Includes operators that modify SQL clauses, SQL \texttt{NULL} operators, and SQL \texttt{WHERE} operators.\\
statements & Includes operators performing syntactical changes at a statement-level granularity.\\
strings & Includes operators that performs mutations on variables of string type.\\
structures & Includes operators that mutate C/C++ structures.\\
higher-order & Includes operators for generating second- and higher-order mutants. These strategies combine first order mutants to simulate more complex faults, motivated by a desire to capture subtle faults~\cite{jia2009higher}. Usually, each operator of this category introduce a strategy on how to combine first order operators to form a higher-order one. For example, the \emph{RMix} operator combines any two randomly selected first order operator to generate a higher-order mutant. \\
system-configuration & Includes operators that perform mutations at a system-configuration level; usually, these mutation operators aim to cause faults concerning the configuration of the system itself.\\
system-level & Includes operators that performs mutations at a system-level; usually, these mutation operators aim to cause faults concerning the integration of system components. \\
ADA expressions & Includes operators that work by modifying expressions in the ADA programming language. \\
ADA operands & Includes operators that perform mutations by replacing operands in ADA expressions (e.g., variables, constants, records, arrays, pointers).\\
ADA tasks & Includes operators that mutate ADA code at the granularity of tasks and aim to trigger problems related to concurrency in the program.\\
\bottomrule
\end{tabular}
\end{table}
%
%In addition to the sufficient set of operators, we can group the mutation operators targeting space software (i.e., software implemented in C/C++) into XX categories. Table~\ref{} provides an overview of these additional categories of operators along with references and a discussion of the reasons why they should be selected or avoided when applying mutation testing to space software.

%% !TEX root =  ../Main.tex

\newcommand{\op}{\mathit{op}}
\newcommand{\ArithmeticSet}{ \texttt{+}, \texttt{-}, \texttt{*}, \texttt{/}, \texttt{\%} }
\newcommand{\LogicalSet}{ \texttt{&&}, \texttt{||} }
\newcommand{\RelationalSet}{ \texttt{>}, \texttt{>=}, \texttt{<}, \texttt{<=}, \texttt{==}, \texttt{!=} }
\newcommand{\BitWiseSet}{ \texttt{\&}, \texttt{|}, \land }
\newcommand{\ShiftSet}{ \texttt{>>}, \texttt{<<} }


\begin{table}[h]
\caption{Implemented set of mutation operators.}
\label{table:operators} 
\centering
\scriptsize
\begin{tabular}{|@{}p{4mm}@{}|@{}p{2cm}@{\hspace{1pt}}|@{}p{11.1cm}@{}|}
\hline
&\textbf{Operator} & \textbf{Description$^{*}$} \\
\hline
\multirow{7}{*}{\rotatebox{90}{\emph{Sufficient Set}}}&ABS               & $\{(v, -v)\}$	\\
\cline{2-3}
&AOR               & $\{(\op_1, op_2) \,|\, \op_1, \op_2 \in \{ \ArithmeticSet \} \land \op_1 \neq \op_2 \} $       \\
&    			  & $\{(\op_1, \op_2) \,|\, \op_1, \op_2 \in \{\texttt{+=}, \texttt{-=}, \texttt{*=}, \texttt{/=}, \texttt{\%} \texttt{=}\} \land \op_1 \neq \op_2 \} $       \\
\cline{2-3}
&ICR               & $\{i, x) \,|\, x \in \{1, -1, 0, i + 1, i - 1, -i\}\}$           \\
\cline{2-3}
&LCR               & $\{(\op_1, \op_2) \,|\, \op_1, \op_2 \in \{ \texttt{\&\&}, || \} \land \op_1 \neq \op_2 \}$            \\
&				  & $\{(\op_1, \op_2) \,|\, \op_1, \op_2 \in \{ \texttt{\&=}, \texttt{|=}, \texttt{\&=}\} \land \op_1 \neq \op_2 \}$            \\
&				  & $\{(\op_1, \op_2) \,|\, \op_1, \op_2 \in \{ \texttt{\&}, \texttt{|}, \texttt{\&\&}\} \land \op_1 \neq \op_2 \}$            \\
\cline{2-3}
&ROR               & $\{(\op_1, \op_2) \,|\, \op_1, \op_2 \in \{ \RelationalSet \}\}$            \\
&				  & $\{ (e, !(e)) \,|\, e \in \{\texttt{if(e)}, \texttt{while(e)}\} \}$ \\
\cline{2-3}
&SDL               & $\{(s, \texttt{remove}(s))\}$            \\
\cline{2-3}
&UOI               & $\{ (v, \texttt{--}v), (v, v\texttt{--}), (v, \texttt{++}v), (v, v\texttt{++}) \}$            \\   
\hline
\hline
\multirow{5}{*}{\rotatebox{90}{\emph{OODL}}}&AOD               & $\{((t_1\,op\,t_2), t_1), ((t_1\,op\,t_2), t_2) \,|\, op \in \{ \ArithmeticSet \} $       \\ 
\cline{2-3}
&LOD               & $\{((t_1\,op\,t_2), t_1), ((t_1\,op\,t_2), t_2) \,|\, op \in \{  \} \}$       \\ 
\cline{2-3}
&ROD               & $\{((t_1\,op\,t_2), t_1), ((t_1\,op\,t_2), t_2) \,|\, op \in \{ \RelationalSet \} \}$       \\ 
\cline{2-3}
&BOD               & $\{((t_1\,op\,t_2), t_1), ((t_1\,op\,t_2), t_2) \,|\, op \in \{ \BitWiseSet \} \}$       \\ 
\cline{2-3}
&SOD               & $\{((t_1\,op\,t_2), t_1), ((t_1\,op\,t_2), t_2) \,|\, op \in \{ \ShiftSet \} \}$       \\ 
%\hline
%COR               & $\{(\op_1, \op_2) \,|\, \op_1, \op_2 \in \{ \texttt{\&\&}, \texttt{||}, \land \} \land \op_1 \neq \op_2 \}$            \\
\hline
\hline
\multirow{3}{*}{\rotatebox{90}{\emph{Other}}}&LVR			& $\{(l_1, l_2) \,|\, (l_1, l_2) \in \{(0,-1), (l_1,-l_1), (l_1, 0), (\mathit{true}, \mathit{false}), (\mathit{false}, \mathit{true})\}\}$\\
&&\\
&&\\
\hline
\end{tabular}

$^{*}$Each pair in parenthesis shows how a program element is modified by the mutation operator. Th eleft element of the pair is replaced with the right element. We follow standard syntax~\cite{kintis2018effective}. Program elements are literals ($l$), integer literals ($i$), boolean expressions ($e$), operators ($\op$), statements ($s$), variables ($v$), and terms ( $t_i$, which might be either variables or literals).
\end{table}

\subsubsection{Compile-time Scalability}
\label{sec:compile:time}

%The time spend in compiling mutants depend on the number of mutants to be compiled. 
%The time spent in compiling mutants depends on the number of invocations of the compiler.
%For this reason, \emph{mutant schemata} include all the mutations into a single executable~\cite{untch1993mutation} thus needing only a single  compiler pass. 

To reduce the number of invocations of the compiler to one, \emph{mutant schemata} include all the mutations into a single executable~\cite{untch1993mutation}. 
With mutant schemata, the mutations to be tested are selected at run-time through configuration parameters. They may lead to a compilation speed-up of 300\% \cite{papadakis2010automatic}. 


Another solution to address compile-time scalability issues consists of mutating machine code  (e.g., binary code~\cite{becker2012xemu}, assembly language~\cite{crouzet2006sesame},
Java~\cite{ma2006mujava}, 
 and
.NET~\cite{derezinska2011object} bytecode) thus avoiding the execution of the compilation process after creating a mutant. 
%Empirical results show that the generation of mutants for compiled code requires only 50\% of the time required by a traditional mutation testing process applied to source code~\cite{derezinska2011object,becker2012xemu}.
A common solution consists of mutating the
 LLVM Intermediate Representation (IR) \cite{hariri2016evaluating}, 
which enables the development of mutants that work with multiple programming languages~\cite{hariri2019comparing} and facilitates the integration of optimizations based on dynamic program analysis~\cite{denisov2018mull}.


Unfortunately, the mutation of machine code 
may lead to mutants that are not representative of real faults because impossible to generate with the source code~\cite{schuler2009efficient}.
In the case of IR mutation, a part of these mutants can be automatically identified~\cite{denisov2018mull}; however,
the number of generated mutants tend to be higher at the IR level than at the source code level, which may reduce scalability~\cite{hariri2019comparing}.
 In addition, we have encountered three problems that prevented the application of 
 mutation testing tools based on  LLVM IR to our case study systems.
First, space software relies on compiler pipelines (e.g., RTEMS~\cite{RTEMS}) that include architecture-specific optimizations not supported by LLVM. 
Second, there is no guarantee that the executables generated by LLVM are equivalent to those produced by other compilers.
%, which may 
%undermine the validity of mutation testing assessment 
% (e.g., mutants may fail because of errors introduced by the compiler). 
 Third, efficient toolsets based on LLVM often  perform mutations dynamically~\cite{denisov2018mull}, which is infeasible when the software under test needs to be executed within a dedicated simulator (this is the case of the ESAIL case study system).




\subsubsection{Runtime Scalability}
\label{sec:scalability}

A straightforward mutation testing process consists of executing the full test suite against every mutant, which may lead to scalability problems in the case of a large SUT.
Simple optimizations that can be applied to space software consist of (S1) stopping the execution of the test suite when the mutant has been killed, (S2) executing only those test cases that cover the mutated statements~\cite{delamaro1996proteum}, and (S3) rely on time thresholds to automatically detect infinite loops introduced by mutation~\cite{papadakis2019mutation}. 

\emph{Split-stream} execution consists of generating a modified version of the SUT that creates multiple processes (one for each mutant) only when the mutated code is reached \cite{king1991fortran,tokumoto2016muvm}, thus saving time and resources. Unfortunately, it cannot be applied in the case of space software that needs to be run on simulators because, in general, the hosting simulator cannot be forked by the hosted SUT.

A practical solution consists of  randomly selecting only a portion of the generated mutants~\cite{zhang2010operator,gopinath2015hard,zhang2013operator}. 
%
%Zhang et al. \cite{zhang2010operator} demonstrated that a test suite that is capable of killing 50\% of the mutants results in killing more than 99\% of all mutants.
%
%Gopinath et al \cite{gopinath2015hard} show that the mutation score obtained for a subset of the mutants is representative of the mutation score obtained by considering all the mutants. Best results were obtained by considering 1 000 mutants. The number of mutants to consider is independent of the total number of available mutants. With 1 000 mutants, the error in the prediction of the mutation score was 7\% with a probability of 95\%.
% 
Zhang et al. \cite{zhang2013operator} empirically demonstrated that a random selection of 5\% of the mutants is sufficient for 
%correctly predicting the mutation score. 
estimating, with high confidence, the mutation score obtained with the complete mutants set.
Also,
they show that sampling mutants uniformly across different program elements (e.g., functions) %i.e., to have a same percentage of mutants selected for every function/methods) 
leads to a more accurate mutation score prediction than sampling mutants globally in a random fashion. It also fares better than uniformly distributing the sampled mutants across different mutation operators. In the presence of large software systems that lead to hundred thousand mutants, random mutation testing is the only viable solution. However, for large systems, a percentage of mutants below 5\% might need to be selected.

Other solutions aim to sort test cases to augment the likelihood of executing first the ones that kill the mutants thus preventing the execution of most of the test suite and save time.
Solutions that simply prioritize faster test cases~\cite{just2012using}
 may not be useful with system-level test suites whose test cases have homogeneous execution time.
Approaches that rely on data-flow analysis to identify and prioritize the test cases that likely satisfy the killing conditions~\cite{papadakis2011automatically} may not scale with large space systems.
Solutions that combine multiple coverage measures need to be adapted in order 
to be feasible in the space context~\cite{zhang2013faster}.
Current work~\cite{zhang2013faster} combines three criteria: (1) the number of times the mutated statement is exercised by the test case (multiple iterations over a same statement are likely performed with a diverse set of variable values and thus have higher probability to kill the mutant), (2) the proximity of the mutated statement to the end of the test case (closer ones have higher chances of satisfying the sufficiency condition), and (3) the percentage of mutants belonging to the same class file of the mutated statement that were already killed by the test case (test cases that kill multiple mutants likely exercise the SUT with a diverse set of inputs). 
Criterion (3) is also used to reduce the test suite size (i.e., to select only the test cases above a given threshold). 
Unfortunately, only criterion (1) might be applicable for space software; indeed, criterion (2) might be ineffective with system test cases whose results are checked after long executions, while criterion (3) might be inefficient when only a random selection of mutants is executed. 






%For \INDEX{test case reduction}, the idea is to remove those test cases that are somehow redundant (e.g., test cases that when removed from the test suites do not change the mutation score).
%Usaola et al. \cite{usaola2012reduction} proposed a greedy algorithm that iteratively selects  the test cases that kill most of the mutants that were not killed by the previously selected test cases. 
%%\DONE{No change to do here. However please keep them in mind for the current work.}
%Shi et al. \cite{shi2014balancing} assessed the effects of reducing the size of test suites with an experiment on 18 projects with a total of 261\,235 test cases. Their results show that \emph{it is possible to maintain constant the mutation score and reduce test suite size without loss in the \emph{fault detection rate}}. 
%On the same line, Zhang et al. \cite{zhang2013faster} suggest to define a subset of tests of the original test suite and to run the mutants against the subset, their assumption is that if the mutants cannot be killed by the subset also the original test suite will not be able to kill the mutants.
%
%
%




\subsubsection{Detection of Equivalent Mutants}

Despite identifying equivalent mutants is an undecidable problem~\cite{madeyski2013overcoming,Bugg:Correctness:82}, several heuristics to partially address the problem had been developed. 

The simplest solution consists of relying on \emph{trivial compiler optimisations}~\cite{papadakis2015trivial, kintis2017detecting,papadakis2019mutation}, i.e., compile both the mutants and the original program with compiler optimisations enabled and determine that the mutant is equivalent when their executable code match. In the case of programs written in C, compiler optimisations can reduce the total number of mutants by 28\%~\cite{kintis2017detecting}.


%Solutions that identify equivalent mutants based on symbolic execution~\cite{holling2016nequivack}, bounded model checking~\cite{riener2011test}, and program slicing ~\cite{harman2001relationship} are unlikely to scale with large systems because of the limitations of static analysis. Also, their implementation often relies on LLVM, which may prevent their applicability to space software (see Section~\ref{sec:compile:time}).

\CHANGED{Solutions that identify equivalent mutants based on \emph{static program analysis} (e.g., concolic execution~\cite{holling2016nequivack}, and bounded model checking~\cite{riener2011test}) require the manual identification of the state variables and program outputs that contain the results of the computation performed by the mutated function.
% and thus should differ after the execution of the original and mutated code. 
Such manual analysis is infeasible in the presence of a large set of mutants, most particularly for large systems. 
%Despite some form of automation might be envisioned in the presence of unit test cases, such solution is not applicable when test cases target the whole system. 
Alternatively, reachability analysis might be employed to determine if two mutants are not equivalent according to weak mutation. 
However, weak mutation does not guarantee that the sufficiency condition holds; consequently, the generated results still need to be manually inspected to determine if a difference in the final state of the system is observable. In addition, existing static analysis solutions cannot work with system-level test cases that require hardware and environment simulators.
Indeed, (1) simulation results cannot be predicted by pure static analysis, and (2) concolic execution tools, which rely on LLVM, cannot be run if the SUT executable should be generated with a specific compiler (see Section~\ref{sec:compile:time}).}


Alternative solutions rely on comparing the data collected at runtime when testing the original software and the mutants~\cite{grun2009impact,schuler2010covering,schuler2013covering}.
%A large number of program classes presenting different statement coverage in test suites executions with the original and the mutated code is an indicator for  mutants to be non-equivalent~\cite{grun2009impact}.
The most extensive empirical study on the topic shows that non-equivalent mutants can be detected by counting the number of methods (mutated method excluded) that, for at least one test case, either (1) have statements that are executed at a different frequency with the mutant, (2) generate at least one different return value, or (3) are invoked at a different frequency~\cite{schuler2013covering}. To determine if a mutant is non-equivalent, it is possible to define a threshold indicating the smallest number of methods with such characteristics. A threshold of one identifies non-equivalent mutants with an average precision above 70\% and an average recall above 60\%. It outperforms more sophisticated methods relying on dynamic invariants~\cite{schuler2009efficient}.
Values above one slightly improve precision but make recall drop (e.g., recall is below 50\% for a threshold of five); also, the proper threshold value may depend on the size of the test suite. 

Concerning the applicability of coverage-based methods to space software, it is worth noting that, because of real-time constraint, it might be feasible to collect only coverage frequency data, which, however, lead to results close to the ones achieved by including all the three criteria~\cite{schuler2013covering}.
Unfortunately, the  results reported in the literature concern a small number of mutants (i.e., 140) for Java software; further empirical evaluations might be needed to determine if these findings hold for C/C++ space software. 
In particular, the identification of coverage differences across the whole software might be ineffective when the SUT is exercised with system test cases that may lead to non-deterministic internal software behaviours (e.g., because of interrupt handlers). Finally, we notice that although coverage-based approaches might be an effective solutions to detect mutants that are not-equivalent (i.e., two executions showing different coverage are likely semantically different) they might be inappropriate to identify equivalent mutants. Indeed, not-equivalent mutants that are not exercised with an appropriate set of inputs may lead to the same coverage. For example, the condition $(x >= 0)$ leads to the same coverage of $(x > 0)$ if not tested with $(x=0)$. For this reason, a mutation score computed by including only such non-equivalent mutants might be higher than the real mutation score, which might be dangerous if mutation testing is used to assess the quality of test suites for safety critical software.

\subsubsection{Detection of Redundant Mutants}

Redundant mutants are either \emph{duplicate}, i.e., mutants that are equivalent with each other but not equivalent to the original program, or \emph{subsumed}, i.e., mutants that are not equivalent with each other but are killed by the same test cases. 

We observe that duplicate mutants can be detected by relying on the same approaches adopted for equivalent mutants. 

To identify subsumed mutants, Shin et al. suggest augmenting the test suite with additional test cases so that each mutant can trigger a test failure that cannot be observed with other mutants~\cite{Shin:TSE:DCriterion:2018}. 
The augmented test suites have a higher
 fault detection rate than test suites that simply satisfy mutation coverage; however, with large software systems the approach might be infeasible because of the lack of scalable test inputs generation approaches.


\subsubsection{Summary}

We aim to rely on the sufficient set of operators because it has been successfully used to generate a mutation score that accurately estimate the fault detection rate for software written in C and C++, typical languages of space software systems.
Based on recent results, we should extend the sufficient set with COR, and AOD.

To speed up mutation testing by reducing the number of mutants, the SSDL operator alone might be considered. However, the confidence of the results generated with the SSDL operator should be evaluated.

Among compile time optimizations, only mutant schemata appear to be feasible.
\MREVISION{C-P-10}{However, based on preliminary evaluation with the case studies for FAQAS, compile time optimizations appear not to be necessary if a smart compilation process is in place. Basically, the compilation routine that we propose (i.e., copy the mutated file inside a compiled source tree) enables fast compilation of mutants.}

Simple runtime scalability optimizations (i.e., S1, S2, and S3 in Section~\ref{sec:scalability}) are feasible in the case of space software. Other feasible solutions are the ones relying on mutant sampling and coverage metrics. However, the confidence of the results generated for mutant sampling rates below 5\% should be evaluated. Furthermore, code coverage metrics that are feasible for space software need to be defined.

Equivalent mutants can be identified through trivial compiler optimizations and the analysis of coverage differences; however, coverage metrics that are robust with respect to non-deterministic internal software behaviours should be developed. \MREVISION{C-P-08}{The same solution can be adopted to identify duplicate mutants. The identification of scalable solutions for the generation of test cases that distinguish subsumed mutants is out of the scope of this project. However, the percentage of subsumed mutants should be reported by the FAQAS framework.}

