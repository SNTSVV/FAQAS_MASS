%%
%% This is file `sample-authordraft.tex',
%% generated with the docstrip utility.
%%
%% The original source files were:
%%
%% samples.dtx  (with options: `authordraft')
%%
%% IMPORTANT NOTICE:
%%
%% For the copyright see the source file.
%%
%% Any modified versions of this file must be renamed
%% with new filenames distinct from sample-authordraft.tex.
%%
%% For distribution of the original source see the terms
%% for copying and modification in the file samples.dtx.
%%
%% This generated file may be distributed as long as the
%% original source files, as listed above, are part of the
%% same distribution. (The sources need not necessarily be
%% in the same archive or directory.)
%%
%% The first command in your LaTeX source must be the \documentclass command.
%,authordraft
%\documentclass[sigconf]{acmart}

%\documentclass[sigconf,review]{acmart}

\documentclass[acmsmall,nonacm]{acmart}

\acmConference[ICSE 2022]{The 44th International Conference on Software Engineering}{May 21–29, 2022}{Pittsburgh, PA, USA}


%% NOTE that a single column version may required for
%% submission and peer review. This can be done by changing
%% the \doucmentclass[...]{acmart} in this template to
%% \documentclass[manuscript,screen]{acmart}
%%
%% To ensure 100% compatibility, please check the white list of
%% approved LaTeX packages to be used with the Master Article Template at
%% https://www.acm.org/publications/taps/whitelist-of-latex-packages
%% before creating your document. The white list page provides
%% information on how to submit additional LaTeX packages for
%% review and adoption.
%% Fonts used in the template cannot be substituted; margin
%% adjustments are not allowed.


\usepackage[utf8]{inputenc}
\usepackage[T1]{fontenc}
\usepackage{booktabs}
\usepackage{longtable}
\usepackage{amsmath}
\usepackage{array}
\usepackage{lipsum}
\usepackage{listings}
\definecolor{mGreen}{rgb}{0,0.6,0}
\definecolor{mGray}{rgb}{0.5,0.5,0.5}
\definecolor{mPurple}{rgb}{0.58,0,0.82}
\definecolor{backgroundColour}{rgb}{0.95,0.95,0.92}

\lstdefinestyle{CStyle}{
    backgroundcolor=\color{backgroundColour},
    commentstyle=\color{mGreen},
    keywordstyle=\color{magenta},
    numberstyle=\tiny\color{mGray},
    stringstyle=\color{mPurple},
    basicstyle=\scriptsize\ttfamily,
    breakatwhitespace=false,
    breaklines=true,
    captionpos=b,
    keepspaces=true,
    numbers=left,
    numbersep=5pt,
    showspaces=false,
    showstringspaces=false,
    showtabs=false,
    tabsize=2,
    language=C
}


\lstset{
  backgroundcolor=\color{white},   % choose the background color; you must add \usepackage{color} or \usepackage{xcolor}; should come as last argument
  basicstyle=\footnotesize\ttfamily,        % the size of the fonts that are used for the code
  breakatwhitespace=false,         % sets if automatic breaks should only happen at whitespace
  breaklines=true,                 % sets automatic line breaking
  captionpos=b,                    % sets the caption-position to bottom
  commentstyle=\color{gray},    % comment style
  deletekeywords={...},            % if you want to delete keywords from the given language
  %escapeinside={\%*}{*)},          % if you want to add LaTeX within your code
  %extendedchars=true,              % lets you use non-ASCII characters; for 8-bits encodings only, does not work with UTF-8
  %firstnumber=1000,                % start line enumeration with line 1000
  frame=single,                    % adds a frame around the code
  keepspaces=true,                 % keeps spaces in text, useful for keeping indentation of code (possibly needs columns=flexible)
  keywordstyle=\color{blue},       % keyword style
  language=C,                 % the language of the code
  %morekeywords={*,...},            % if you want to add more keywords to the set
  numbers=left,                    % where to put the line-numbers; possible values are (none, left, right)
  numbersep=5pt,                   % how far the line-numbers are from the code
  numberstyle=\tiny\color{gray}, % the style that is used for the line-numbers
  rulecolor=\color{black},         % if not set, the frame-color may be changed on line-breaks within not-black text (e.g. comments (green here))
  showspaces=false,                % show spaces everywhere adding particular underscores; it overrides 'showstringspaces'
  showstringspaces=false,          % underline spaces within strings only
  showtabs=false,                  % show tabs within strings adding particular underscores
  stepnumber=1,                    % the step between two line-numbers. If it's 1, each line will be numbered
  stringstyle=\color{black},     % string literal style
  tabsize=2,                     % sets default tabsize to 2 spaces
}








%\usepackage{algorithmic}

\usepackage{xspace}
\usepackage{blindtext}
\usepackage{hyperref}

%\usepackage{longtable}
%\usepackage{cite}
%\usepackage{array}
%\usepackage{lipsum}
%\usepackage{algorithm}
%\usepackage{algpseudocode}
%
%\usepackage{listings}
%
\usepackage{multirow}
%\usepackage{subcaption}
%\usepackage{graphicx}
%\usepackage{booktabs}
%\usepackage{textcomp}
%\usepackage{xcolor}

\newcommand{\EMPH}[1]{\emph{#1}}

\newcommand{\D}{$\Delta$\xspace}

\newcommand{\DAMAT}{\emph{DAMAt}\xspace{}}
\newcommand{\GomSpace}{GomSpace\xspace}
\newcommand{\LuxSpace}{LuxSpace\xspace}
\newcommand{\ONE}{GSL\xspace}
\newcommand{\TWO}{LXS\xspace}
\newcommand{\CITONE}{~\cite{GSL}}
\newcommand{\CITTWO}{~\cite{LXS}}
\newcommand{\ESA}{ESA\xspace}

\newcommand{\ESAIL}{ESAIL\xspace}

\newcommand{\LAUNCH}{on September 2020~\cite{ESAILlaunch}}
\newcommand{\OPENCSP}{the open source CubeSat Space Protocol (\CSP) library~\cite{CSP}}
\newcommand{\SAIL}{\emph{\ESAIL}\xspace}

\newcommand{\GCSP}{\emph{LIBGCSP}\xspace}
\newcommand{\CSP}{\emph{LIBGCSP}\xspace}
\newcommand{\PARAM}{\emph{LIBParam}\xspace}
\newcommand{\UTIL}{\emph{LIBUTIL}\xspace}
\newcommand{\CITSAIL}{~\cite{ESAIL}}
\newcommand{\MLFS}{\emph{MLFS}\xspace}


\newcommand{\FIXME}[1]{\textcolor{red}{#1}}
\newcommand{\UPDATED}[1]{\textcolor{black}{#1}}

\newcommand{\CHANGED}[1]{\textcolor{black}{#1}}

\newcommand{\YAGO}{Yago Isasi Parache}
\newcommand{\ExaE}{ExactEarth}

\newcommand{\EduardoSpace}{GomSpace Luxembourg\xspace}
\newcommand{\YagoSpace}{LuxSpace\xspace}

\newcommand{\ADCS}{\emph{\ESAIL-ADCS}\xspace}
\newcommand{\GPS}{\emph{\ESAIL-GPS}\xspace}
\newcommand{\PDHU}{\emph{\ESAIL-PDHU}\xspace}
\newcommand{\SVF}{\emph{\ESAIL-SVF}\xspace}

%% \BibTeX command to typeset BibTeX logo in the docs
\AtBeginDocument{%
  \providecommand\BibTeX{{%
    \normalfont B\kern-0.5em{\scshape i\kern-0.25em b}\kern-0.8em\TeX}}}

%% Rights management information.  This information is sent to you
%% when you complete the rights form.  These commands have SAMPLE
%% values in them; it is your responsibility as an author to replace
%% the commands and values with those provided to you when you
%% complete the rights form.
%\setcopyright{acmcopyright}
\setcopyright{rightsretained}
\copyrightyear{2021}
\acmYear{2021}
\acmDOI{10.1145/1122445.1122456}


%% These commands are for a PROCEEDINGS abstract or paper.
%\acmConference[ASE '21]{ASE '21: IEEE/ACM International Conference on Automated Software Engineering}{September 21--25, 2021}{Melbourne, AU}
%\acmBooktitle{IEEE/ACM International Conference on Automated Software Engineering, September 21--25, 2021, Melbourne, AU}
\acmPrice{15.00}
\acmISBN{978-1-4503-XXXX-X/18/06}


%%
%% Submission ID.
%% Use this when submitting an article to a sponsored event. You'll
%% receive a unique submission ID from the organizers
%% of the event, and this ID should be used as the parameter to this command.
%%\acmSubmissionID{123-A56-BU3}

%%
%% The majority of ACM publications use numbered citations and
%% references.  The command \citestyle{authoryear} switches to the
%% "author year" style.
%%
%% If you are preparing content for an event
%% sponsored by ACM SIGGRAPH, you must use the "author year" style of
%% citations and references.
%% Uncommenting
%% the next command will enable that style.
%%\citestyle{acmauthoryear}

\usepackage{draftwatermark}
\SetWatermarkText{}
\SetWatermarkScale{1.1}



%%%% Graphics
\usepackage{graphicx}
\usepackage{tikz}
\definecolor{unilublue}{RGB}{55,149,218}
\definecolor{sntred}{RGB}{219,46,27}
\definecolor{sntpurple}{RGB}{86,30,130}
%\definecolor{sntblue}{RGB}{50,130,207}
\definecolor{sntblue}{RGB}{55,149,218}

\pgfdeclareimage[width=30mm]{logo-snt}{logos/logo-snt}
\pgfdeclareimage[width=30mm]{logo-uni-lu}{logos/logo-uni-lu}

%%%% Fancy
%\usepackage{fancyhdr}

%%%% Increase page length
%\addtolength{\textheight}{1in}

%%%% Sections
\usepackage{xspace}
%\usepackage{sectsty}
%\allsectionsfont{\sffamily}

%\setlength{\parskip}{1em}

%\renewcommand{\,}{$^{\cdot}$}

%%%% NEW TEMPLATE %%%%
\newcommand{\DOCUMENTID}{ITT-1-9873-ESA-FAQAS-AS}
\newcommand{\titleone}{Applicability of Mutation Testing to Flight Software}
\newcommand{\titletwo}{\textsf{(FAQAS)}}
\newcommand{\titlethree}{\textsf{Abstract} (AS)}
%%%% NEW TEMPLATE END %%%%

%%%% Title
% \newcommand{\titleone}{\textsf{FR}}
% \newcommand{\titletwo}{\textsf{Final Report}}
% \newcommand{\titlethree}{\textsf{}}

\newcommand{\todoinline}[1]{\todo[color=orange,inline]{ \textbf{TODO}: #1 }}

\newcommand{\TODO}[1]{\todo[color=orange,inline]{ \textbf{TODO}: #1 }}
\newcommand{\DONE}[1]{\todo[color=green,inline]{ \textbf{DONE}: #1 }}


\usepackage{amsmath}
\usepackage{array}
\usepackage{lipsum}

\usepackage{algorithm}
\usepackage{algpseudocode}

\newcommand{\DAMTE}{\emph{DAMTE}\xspace}
\newcommand{\MASS}{\emph{MASS}\xspace}
\newcommand{\SEMUS}{\emph{SEMuS}\xspace}
\newcommand{\LXS}{LXS\xspace{}}
\newcommand{\GSL}{GSL\xspace{} }
%\newcommand{\ESA}{ESA\xspace{}}
%\newcommand{\ESAIL}{ESAIL\xspace}
%\newcommand{\PARAM}{LIBPARAM\xspace}
%\newcommand{\UTIL}{LIBUTIL\xspace}

%\newcommand{\SAIL}{\emph{ESAIL}\xspace}
%\newcommand{\MLFS}{\emph{MLFS}\xspace}
%\newcommand{\GCSP}{\emph{LIBGCSP}\xspace}

\newcommand{\MPTS}{\emph{MASS-reduced} test suite\xspace}
\newcommand{\MPTSs}{\emph{MASS-reduced} test suites\xspace}
%\newcommand{\ExaE}{ExactEarth}

%\newcommand{\GomSpace}{GomSpace\xspace}
%\newcommand{\LuxSpace}{LuxSpace\xspace}
%\newcommand{\ONE}{GSL\xspace}
%\newcommand{\TWO}{LXS\xspace}
%\newcommand{\CITONE}{~\cite{GSL}}
%\newcommand{\CITTWO}{~\cite{LXS}}
%\newcommand{\LAUNCH}{on September 2020~\cite{ESAILlaunch}}
%\newcommand{\OPENCSP}{the open source CubeSat Space Protocol (\CSP) library~\cite{CSP}}


%\newcommand{\ADCS}{\emph{\ESAIL-ADCS}\xspace}
%\newcommand{\GPS}{\emph{\ESAIL-GPS}\xspace}
%\newcommand{\PDHU}{\emph{\ESAIL-PDHU}\xspace}
%\newcommand{\SVF}{\emph{\ESAIL-SVF}\xspace}

%\newcommand{\CHANGED}[1]{{#1}}
\newcommand{\CHANGEDTWO}[1]{{#1}}
\newcommand{\CHANGEDOCT}[1]{{#1}}
\newcommand{\CHANGEDNOV}[1]{{#1}}

\newcommand{\TRFOUR}[1]{{#1}}

\newcommand{\STARTCHANGEDNOV}{\color{black}}
\newcommand{\ENDCHANGEDNOV}{\color{black}}


\newcommand{\STARTCHANGEDWPT}{\color{blue}}
\newcommand{\ENDCHANGEDWPT}{\color{black}}

%\newcommand{\UPDATED}[1]{#1}

%\newcommand{\D}[0]{$\Delta$}


%\newcommand{\MREVISION}[2]{\todo{\tiny{#1}}\textcolor{blue}{#2}}
\newcommand{\MREVISION}[2]{#2}

%\newcommand{\REVTWO}[2]{\todo[color=red]{\tiny{#1}}\textcolor{red}{#2}}
\newcommand{\REVTWO}[2]{#2}

\newcommand{\REVNOV}[2]{\textcolor{black}{#2}}

\newcommand{\REVTOOL}[2]{{#2}}

\newcommand{\JMR}[2]{\textcolor{black}{#2}}
\newcommand{\REVOCT}[2]{#2}
%\newcommand{\FIXME}[2]{#2}
\newcommand{\NEWFSCI}[1]{\textcolor{black}{#1}}

\newcommand{\JMRCHANGE}[1]{\textcolor{black}{#1}}
\newcommand{\UPDATE}[1]{\textcolor{black}{#1}}

%\newcommand{\EMPH}[1]{\textbf{\emph{#1}}}
\newcommand{\INDEX}[1]{\index{\MakeLowercase{#1}}\EMPH{#1}}

%\newcommand{\APPR}{\emph{MASS}\xspace}



%%
%% end of the preamble, start of the body of the document source.
\begin{document}

%%
%% The "title" command has an optional parameter,
%% allowing the author to define a "short title" to be used in page headers.
\title[]{Applicability of Mutation Testing Method for Flight Software: Fault-based, Automated Quality Assurance Assessment for Space Software (FAQAS)}
\renewcommand{\shortauthors}{}

%%%% NEW TEMPLATE %%%%
%\pagenumbering{gobble}
\pagenumbering{arabic}


\fancyhf{}


\renewcommand{\sectionmark}[1]{\markright{\textit{#1}}}
\lfoot{\fancyplain{}{\textit{\DOCUMENTID}}}
\rhead{\fancyplain{}{\rightmark }}


\rfoot{Page \thepage \hspace{1pt} of \ref{TotPages}}
%%%% NEW TEMPLATE END %%%%

%%
%% The "author" command and its associated commands are used to define
%% the authors and their affiliations.
%% Of note is the shared affiliation of the first two authors, and the
%% "authornote" and "authornotemark" commands
%% used to denote shared contribution to the research.
\author{Enrico Viganò, Oscar Cornejo, Fabrizio Pastore}
\affiliation{%
  \institution{University of Luxembourg}
  \streetaddress{JFK 29}
  \city{Luxembourg }
  \country{Luxembourg}}
\email{{enrico.vigano,oscar.cornejo,fabrizio.pastore}@uni.lu}

%\author{Enrico Viganò}
%\affiliation{%
%  \institution{SnT Centre, University of Luxembourg}
%  \streetaddress{JFK 29}
%  \city{Luxembourg}
%  \country{Luxembourg}}
%\email{enrico.vigano@uni.lu}
%
%\author{Oscar Cornejo}
%\affiliation{%
%  \institution{SnT Centre, University of Luxembourg}
%  \streetaddress{JFK 29}
%  \city{Luxembourg}
%  \country{Luxembourg}}
%\email{oscar.cornejo@uni.lu}
%
%\author{Fabrizio Pastore}
%\affiliation{%
%  \institution{SnT Centre, University of Luxembourg}
%  \streetaddress{JFK 29}
%  \city{Luxembourg}
%  \country{Luxembourg}}
%\email{fabrizio.pastore@uni.lu}
%
%\author{Lionel Briand}
%\affiliation{%
%  \institution{SnT Centre, University of Luxembourg}
%  \streetaddress{JFK 29}
%  \city{Luxembourg}
%  \country{Luxembourg}}
%  \affiliation{%
%  \institution{School of EECS, University of Ottawa}
%%  \streetaddress{JFK 29}
%  \city{Ottawa}
%  \country{Canada}}
%\email{lionel.briand@uni.lu}

%%
%% By default, the full list of authors will be used in the page
%% headers. Often, this list is too long, and will overlap
%% other information printed in the page headers. This command allows
%% the author to define a more concise list
%% of authors' names for this purpose.


%%
%% The abstract is a short summary of the work to be presented in the
%% article.
%\begin{abstract}
%\textbf{Abstract.} This document is the final report of the ESA activity \EMPH{ITT-1-9873-ESA (Applicability of Mutation Testing Method for Flight Software)}, which concerns the development of a framework for the automated assessment and the automated improvement of test suites for embedded software deployed on spacecrafts.
%
%The activity led to the development of a toolset, the \EMPH{FAQAS toolset}, which includes three tools \EMPH{MASS}, \EMPH{SEMuS}, and \EMPH{DAMAt}. MASS (Mutation Analysis for Space Software) is a tool for the assessment of test suites based on mutation analysis. Mutation analysis evaluates the quality of a test suite by generating faulty software versions called mutants and by reporting the percentage of mutants detected by the test suite. MASS scales to large software systems because it relies on mutants sampling based on confidence interval estimation.  SEMuS (Symbolic Execution Mutation testing for Space software) is a tool for the automated improvement of test suites; it automatically generates unit test cases that detect the presence of mutants. DAMAt (DAta-driven Mutation Analysis with Tables) is a tool that assesses the quality of test suites by simulating errors in the data exchanged by software components, different from MASS it simulates faults concerning components interoperability.
%
%The \EMPH{scalability} and \EMPH{effectiveness} of the FAQAS toolset has been demonstrated through the application of the toolset to case study systems provided by GomSpace, LuxSpace, and ESA. Both MASS and SEMuS enabled the identification of faults affecting the software under analysis. Both MASS and DAMAt enabled the identification of major pitfalls in the test suite. Mutation analysis has been demonstrated to be feasible (i.e., executable in few days even for large systems); however, adequate computational resources (e.g., multiple computation nodes) are necessary. The automated generation of unit test cases, instead, can produce useful results in few minutes. Our results show that the FAQAS toolset enables ensuring high-quality in space software.
%
%\end{abstract}

%%
%% The code below is generated by the tool at http://dl.acm.org/ccs.cfm.
%% Please copy and paste the code instead of the example below.
%%
\begin{CCSXML}
<ccs2012>
<concept>
<concept_id>10011007.10011074.10011099</concept_id>
<concept_desc>Software and its engineering~Software verification and validation</concept_desc>
<concept_significance>500</concept_significance>
</concept>
</ccs2012>
\end{CCSXML}

\ccsdesc[500]{Software and its engineering~Software verification and validation}

%%
%% Keywords. The author(s) should pick words that accurately describe
%% the work being presented. Separate the keywords with commas.
\keywords{Mutation analysis, CPS, CPS Interoperability, Integration testing}

%% A "teaser" image appears between the author and affiliation
%% information and the body of the document, and typically spans the
%% page.


%\usepackage{fontspec,xltxtra,xunicode}
%\defaultfontfeatures{Mapping=tex-text}


%\newcommand{\MREVISION}[2]{\todo{\tiny{#1}}\textcolor{blue}{#2}}


%\newcommand{\REVTWO}[2]{\todo[color=red]{\tiny{#1}}\textcolor{red}{#2}}









%% This command processes the author and affiliation and title
%% information and builds the first part of the formatted document.

%%%% NEW TEMPLATE %%%%
%\maketitle

\begin{tikzpicture}[remember picture,overlay]
\path [fill=sntred]    ([xshift=30pt,yshift=20pt]current page.south west) rectangle
                       ([xshift=170pt,yshift=50pt] current page.south west);
\path [fill=sntpurple] ([xshift=170pt,yshift=20pt] current page.south west) rectangle
                       ([xshift=310pt,yshift=50pt] current page.south west);
\path [fill=sntblue]   ([xshift=310pt,yshift=20pt] current page.south west) rectangle
                       ([xshift=460pt,yshift=51pt] current page.south west);
\path [fill=unilublue] ([xshift=30pt,yshift= 50pt] current page.south west) --
                       ([xshift=460pt,yshift= 50pt] current page.south west)
                       [rounded corners=20pt] --
                       ([xshift=460pt,yshift=630pt] current page.south west)
                       [sharp corners] --
                       ([xshift=30pt,yshift=630pt] current page.south west);
\node [fill=white,rounded corners=0pt,inner xsep=6pt,inner ysep=3pt]
      at ([xshift=415pt,yshift=120pt] current page.south west)
      {\pgfuseimage{logo-uni-lu}};

\node [fill=white,rounded corners=2pt,inner xsep=6pt,inner ysep=3pt]
      at ([xshift=410pt,yshift=670pt] current page.south west)
      {\pgfuseimage{logo-snt}};

%\node [circle, fill=white, draw=sntblue] at ([xshift=550pt,yshift=35pt] current page.south west) {a};

\node[draw=none,fill=none,right] at (-0.5, -3){\color{white}\LARGE\bf\titleone};
\node[draw=none,fill=none,right] at (-0.5, -4){\color{white}\LARGE\bf\titletwo};
\node[draw=none,fill=none,right] at (-0.5, -5){\color{white}\LARGE\bf\titlethree};

\node[draw=none,fill=none,right] at (-0.5, -6){\color{white}\Large\textsf{F. Pastore, O. Cornejo, E. Viganò}};
\node[draw=none,fill=none,right] at (-0.5, -7){\color{white}\Large\textsf{Interdisciplinary Centre for Security, Reliability and Trust}};
\node[draw=none,fill=none,right] at (-0.5, -8){\color{white}\Large\textsf{University of Luxembourg}};
\node[draw=none,fill=none,right] at (9.8, -10){\color{white}\textsf{\DOCUMENTID}};
\node[draw=none,fill=none,right] at (9.8, -11){\color{white}\Large\textsf{Issue 1, Rev. 2}};
\node[draw=none,fill=none,right] at (9.8, -12){\color{white}\Large\textsf{\today}};

\node[draw=none,fill=none,right] at (-0.5, -16.8){\color{white}\tiny\textsf{EUROPEAN SPACE AGENCY. CONTRACT REPORT.}}; 
\node[draw=none,fill=none,right] at (-0.5, -17.1){\color{white}\tiny\textsf{The work described in this report was done under ESA contract.}}; 
\node[draw=none,fill=none,right] at (-0.5, -17.4){\color{white}\tiny\textsf{Responsibility for the contents resides in the author or organisation that prepared it.}};
\node[draw=none,fill=none,right] at (-0.5, -17.7){\color{white}\tiny\textsf{The copyright in this document is vested in the University of Luxembourg.}};
\node[draw=none,fill=none,right] at (-0.5, -18.0){\color{white}\tiny\textsf{This document may only be reproduced in whole or in part, or stored in a retrieval system, or transmitted}};
\node[draw=none,fill=none,right] at (-0.5, -18.3){\color{white}\tiny\textsf{in any form,~\  or by any means electronic,~\  mechanical, photocopying or otherwise,~\ either with the prior }}; 
\node[draw=none,fill=none,right] at (-0.5, -18.6){\color{white}\tiny\textsf{permission of ~\ the University of Luxembourg ~\ or in accordance with the terms of ESTEC ~\ Contract No.}};
\node[draw=none,fill=none,right] at (-0.5, -18.9){\color{white}\tiny\textsf{4000128969/19/NL/AS.}};


\end{tikzpicture}

\newpage
%%%% NEW TEMPLATE END %%%%

%%

% \maketitle

% !TEX root = MAIN.tex

\section{Introduction}
\label{sec:introduction}
\addcontentsline{toc}{chapter}{Introduction}

%This document is the summary report of the ESA activity ITT-1-9873-ESA, which concerns the development of a framework for the automated assessment and the automated improvement of test suites for space software\footnote{In this report, we use the term space software to indicate software to be deployed on hardware that runs on-orbit.}.
 
From spacecrafts to ground stations, software has a prominent role in space systems; for this reason, the success of space missions depends on the quality of the system hardware as much on the dependability of its software. Mission failures due to insufficient software sanity checks~\cite{Schiaparelli} are unfortunate examples, pointing to the necessity for systematic and predictable quality assurance procedures in space software. 


Since one of the primary objectives of software testing is to identify the presence of software faults, an effective way to assess the quality of a test suite consists of artificially injecting faults in the software under test and verifying the extent to which the test suite can detect them. 
This approach is known as \emph{mutation analysis}~\cite{DeMillo78}. 
In mutation analysis, faults are automatically injected in the program through automated procedures referred to as mutation operators. Mutation operators enable the generation of faulty software versions that are referred to as \emph{mutants}.  
Mutation analysis helps evaluate the effectiveness of a test suite, \JMRCHANGE{for a specific software system,} based on its mutation score, which is the percentage of mutants leading to test failures. Also, mutation analysis enables \emph{mutation testing}, which concerns the automated generation of test cases that discover mutants.

Despite its potential, mutation analysis is not widely adopted by industry. The main reasons include its limited scalability and the pertinence of the mutation score as an adequacy criterion~\cite{papadakis2016threats}. Indeed, for a large software system, the number of generated mutants might prevent the execution of the test suite against all the mutated versions. Also, the generated mutants might be either 
semantically equivalent to the original software~\cite{madeyski2013overcoming} or redundant with each other~\cite{Shin:TSE:DCriterion:2018}. Equivalent and redundant mutants may bias the mutation score as an adequacy criterion. 
Finally, test generation approaches are preliminary and cannot be applied in industrial space context. For example, they can generate test inputs only for batch programs that can be compiled with the LLVM infrastructure~\cite{chekam2021killing}.

%The mutation analysis literature has proposed several optimizations to address problems related to scalability and mutation score pertinence~\cite{zhang2013operator,gopinath2015hard,zhang2013faster,grun2009impact,schuler2010covering,schuler2013covering,schuler2009efficient}. 
%However, these approaches 
%have not been evaluated on industrial, embedded systems
%and there are no feasibility studies concerning the integration of such optimizations and their resulting, combined benefits.
%Also, existing mutation analysis approaches cannot identify problems related to the interoperability of integrated components (integration testing), which is a major problem in 
%Cyber-physical Systems~\cite{Givehchi:2017,Jirkovsk:2017} and, consequently, space software --- mainly due to the wide variety and heterogeneity of the technologies and standards adopted.

The FAQAS activity addresses the problems above. It is a joint work between the SnT Centre of the University of Luxembourg\footnote{https://wwwen.uni.lu/snt}, Gomspace Luxembourg\footnote{https://gomspace.com/} (GSL) and OHB Luxspace\footnote{https://luxspace.lu/} (LXS).
FAQAS led to the development of a toolset that addresses the challenges above. It includes four tools:
\EMPH{MASS} (Mutation Analysis for Space Software), 
\EMPH{DAMAt} (DAta-driven Mutation Analysis with Tables), 
\EMPH{SEMuS} (Symbolic Execution-based MUtant analysis for Space software),
and \EMPH{DAMTE} (DAta-driven Mutation TEsting).



% of code-driven mutation analysis in the space context. The evaluation has shown that the most effective solutions to improve scalability and mutation score accuracy are mutants sampling and equivalence metrics based on compiler optimizations, respectively. To guarantee a scalable mutation testing process and the accurate computation of the mutation score, mutants sampling should be based on sequential analysis relying on fixed-width sequential confidence interval, a research discovery done within FAQAS.
%
%•	An empirical evaluation demonstrating the feasibility of data-driven mutation analysis with space software.
%•	The definition of an approach for code-driven mutation testing that relies on symbolic execution to identify test inputs that enable killing mutants not killed by the test suite under analysis.
%•	Demonstrating the feasibility of automated test generation for mutation testing based on symbolic execution. More precisely, symbolic execution can be successfully used to select test inputs that kill live mutants within unit test cases. However, unsurprisingly, it cannot be adopted when, to kill a mutant, it is necessary to rely on external components (e.g., networks or simulators), in such cases, which are common for integration and system test suites, symbolic execution alone is insufficient to generate test cases (e.g., because it cannot translate the simulator logic into an SMT formula to derive test cases from).
%\item The definition of guidelines for the adoption of mutation analysis and testing strategies within ECSS activities. The proposed guidelines support both quality assurance activities described in ECSS standards and Independent Software Verification and Validation (ISVV) practices.
%\end{itemize}

\begin{figure*}[tb]
\begin{center}
\includegraphics[width=0.7\textwidth]{images/FAQAS-overview.pdf}
\caption{Overview of the FAQAS toolset}
\label{fig:FAQAS:toolset}
\end{center}
\end{figure*}

Figure~\ref{fig:FAQAS:toolset} provides an overview of the input and outputs of the FAQAS toolset. It relies on the idea of generating multiple modified versions of the software system under test (SUT), some are derived by modifying the implementation of the software (code-driven mutants) other by integrating a mutation API that alters the messages exchanged by the software components of the SUT (data-driven mutants). 
The SUT test suite shall be executed with all the mutants, if it is effective then it shall fail with each of them. The mutants for which a failure is not observed are said to be \EMPH{live} and indicate a pitfall in the test suite.
All the FAQAS tools take as input the software under test (SUT), its test suite, and a set of configuration files. 

\EMPH{MASS} generates code-driven mutants. It integrates a pipeline of solutions that make mutation analysis feasible wit large SUT. The three main contributions of MASS are (1) the automated identification of trivially equivalent mutants using an ensemble of compiler optimization options, (2) the computation of the mutation score based on mutant sampling with fixed size confidence interval approach (FSCI), (3) the automated identification of equivalent mutants based on coverage. 
MASS reports the set of live mutants, the set of killed mutants (i.e., mutants that are discovered by the test suite), and information useful to draft a verification report, which includes the statement coverage of the SUT test suite and the mutation score (i.e., the percentage of mutants discovered by the test suite).

\EMPH{DAMAt} generates mutants for data-driven mutation analysis. Data-driven mutation analysis is a research contribution of FAQAS. Instead of mutating the implementation of the SUT, it consists of altering the data exchanged by software components. 
DAMAt relies on fault models that specify how to mutate the data exchanged by software components through data-driven mutation operators. DAMAt can automatically alter data that is stored in data buffers (e.g., before serialization on the communication channel).
DAMAt enables the simulation of faults that affect simulated components (e.g., sensors), which is not feasible with traditional, code-driven mutation analysis. 
DAMAt generates as output a set of killed mutants (i.e., mutants that, during testing, successfully alter the data, and lead to test case failures), a set of live mutants (i.e., mutants that, during testing, successfully alter the data, but do not lead to test case failures), and a set of mutants not applied (i.e., mutants that, during testing, could not alter any data because the data they target is never exercised by the SUT); also, it provides information useful to draft a verification report, which includes the fault model coverage (i.e., percentage of fault models with at least one mutant applied), the mutation operation coverage (i.e., percentage of mutants applied), and the mutation score.

\EMPH{SEMuS} automatically generates executable unit test cases based on code-driven mutation analysis results. The generated unit test cases detect mutants not detected by the original test suite. The generated test cases include test oracles that shall be manually validated by engineers, which enables detecting faults. The generated test cases can be integrated into regression test suites.

SEMuS takes as input the list of live mutants detected by MASS. It generates a set of additional test cases that can be integrated into the SUT test suite. Also, it reports the list of killed mutants and the list of mutants that remain live (i.e., for which SEMuS did not generate a test case that kill them). Live mutants shall be manually inspected by engineers to either determine if they are equivalent or to manually derive a test case capable of killing them.

\EMPH{DAMTE} is a manual procedure supported by an automated symbolic execution toolset; it automatically identifies the test inputs that make software components exchange the data targeted by data-driven mutation operators. The derived test inputs can then be manually integrated into the SUT test suite.
 
The activity also included an extensive empirical evaluation demonstrating the feasibility, effectiveness, and scalability of the proposed toolsets in the space context, as described in the following sections.
 
%FAQAS.drawio.pdf

%Sections~\ref{ch:mass:approach} to~\ref{sec:data:test_suite_augmentation} provide an overview of the FAQAS tools: MASS, SEMuS, DAMAt, and DAMTE.
%Section~\ref{chapter:caseStudies} introduces the case study subjects considered for empirical evaluation.
%Section~\ref{sec:summary:results} provides an overview of the empirical results obtained.
%Section~\ref{sec:conclusion} concludes this report.



%% !TEX root =  Main.tex
\section{Code-driven Mutation Analysis: MASS}
\label{ch:mass:approach}

\subsection{Overview}
\label{sec:approach}

\begin{figure}[tb]
\begin{center}
\includegraphics[width=0.6\textwidth]{images/Approach}
\caption{Overview of MASS}
\label{fig:approach}
\end{center}
\end{figure}

Figure~\ref{fig:approach} provides an overview of the mutation analysis process that we propose, \emph{Mutation Analysis for Space Software (\APPR)}. Its goal is to propose a comprehensive solution for making mutation analysis applicable to embedded software in industrial cyber-physical systems. \JMR{3.4}{The ultimate goal of \APPR is to assess the effectiveness of test suites with respect to detecting violations of functional requirements.}

\APPR consists of eight steps: (Step 1) Collect SUT Test Suite Data, (Step 2) Create Mutants, (Step 3) Compile Mutants, (Step 4) Remove Equivalent and Duplicate Mutants Based on Compiled Code, (Step 5) Sample Mutants,  (Step 6) Execute Prioritized Subset of Test Cases,
(Step 7) Identify Likely Equivalent / Duplicate mutants Based on Coverage, and
(Step 8) Compute the Mutation Score. Different from related work, \APPR enables FSCI-based sampling by iterating between mutants sampling (Step 5) and test cases execution (Step 6).
Also, it integrates test suite prioritization and reduction (Step 6) before the computation of the mutation score.
Finally, it includes methods to identify likely equivalent and duplicate mutants based on code coverage (Step 7).
We describe each step in the following paragraphs.


\subsection{Step 1: Collect SUT Test Data}

In Step 1, the test suite is executed against the SUT
%software under test (SUT)
and code coverage information is collected.
More precisely, we rely on the combination of gcov~\cite{GCOV}
and GDB~\cite{GDB}, enabling the collection of coverage information for embedded systems without a file system~\cite{THANASSIS}.
% and Vector CAST~\cite{VectorCAST}
%to record the number of times each line of code of the SUT has been exercised by a test case.

\subsection{Step 2: Create Mutants}

In Step 2, we automatically generate mutants for the SUT by relying on a set of selected mutation operators.
In \APPR, we rely on an extended sufficient set of mutation operators, which are listed in Table~\ref{table:operators}.
%In addition, in our experiments, we also evaluate the feasibility of relying only on the SDL operator, combined or not with OODL operators, instead of the entire sufficient set of operators.

% !TEX root =  ../Main.tex

\newcommand{\op}{\mathit{op}}
\newcommand{\ArithmeticSet}{ \texttt{+}, \texttt{-}, \texttt{*}, \texttt{/}, \texttt{\%} }
\newcommand{\LogicalSet}{ \texttt{&&}, \texttt{||} }
\newcommand{\RelationalSet}{ \texttt{>}, \texttt{>=}, \texttt{<}, \texttt{<=}, \texttt{==}, \texttt{!=} }
\newcommand{\BitWiseSet}{ \texttt{\&}, \texttt{|}, \land }
\newcommand{\ShiftSet}{ \texttt{>>}, \texttt{<<} }


\begin{table}[h]
\caption{Implemented set of mutation operators.}
\label{table:operators} 
\centering
\scriptsize
\begin{tabular}{|@{}p{4mm}@{}|@{}p{2cm}@{\hspace{1pt}}|@{}p{11.1cm}@{}|}
\hline
&\textbf{Operator} & \textbf{Description$^{*}$} \\
\hline
\multirow{7}{*}{\rotatebox{90}{\emph{Sufficient Set}}}&ABS               & $\{(v, -v)\}$	\\
\cline{2-3}
&AOR               & $\{(\op_1, op_2) \,|\, \op_1, \op_2 \in \{ \ArithmeticSet \} \land \op_1 \neq \op_2 \} $       \\
&    			  & $\{(\op_1, \op_2) \,|\, \op_1, \op_2 \in \{\texttt{+=}, \texttt{-=}, \texttt{*=}, \texttt{/=}, \texttt{\%} \texttt{=}\} \land \op_1 \neq \op_2 \} $       \\
\cline{2-3}
&ICR               & $\{i, x) \,|\, x \in \{1, -1, 0, i + 1, i - 1, -i\}\}$           \\
\cline{2-3}
&LCR               & $\{(\op_1, \op_2) \,|\, \op_1, \op_2 \in \{ \texttt{\&\&}, || \} \land \op_1 \neq \op_2 \}$            \\
&				  & $\{(\op_1, \op_2) \,|\, \op_1, \op_2 \in \{ \texttt{\&=}, \texttt{|=}, \texttt{\&=}\} \land \op_1 \neq \op_2 \}$            \\
&				  & $\{(\op_1, \op_2) \,|\, \op_1, \op_2 \in \{ \texttt{\&}, \texttt{|}, \texttt{\&\&}\} \land \op_1 \neq \op_2 \}$            \\
\cline{2-3}
&ROR               & $\{(\op_1, \op_2) \,|\, \op_1, \op_2 \in \{ \RelationalSet \}\}$            \\
&				  & $\{ (e, !(e)) \,|\, e \in \{\texttt{if(e)}, \texttt{while(e)}\} \}$ \\
\cline{2-3}
&SDL               & $\{(s, \texttt{remove}(s))\}$            \\
\cline{2-3}
&UOI               & $\{ (v, \texttt{--}v), (v, v\texttt{--}), (v, \texttt{++}v), (v, v\texttt{++}) \}$            \\   
\hline
\hline
\multirow{5}{*}{\rotatebox{90}{\emph{OODL}}}&AOD               & $\{((t_1\,op\,t_2), t_1), ((t_1\,op\,t_2), t_2) \,|\, op \in \{ \ArithmeticSet \} $       \\ 
\cline{2-3}
&LOD               & $\{((t_1\,op\,t_2), t_1), ((t_1\,op\,t_2), t_2) \,|\, op \in \{  \} \}$       \\ 
\cline{2-3}
&ROD               & $\{((t_1\,op\,t_2), t_1), ((t_1\,op\,t_2), t_2) \,|\, op \in \{ \RelationalSet \} \}$       \\ 
\cline{2-3}
&BOD               & $\{((t_1\,op\,t_2), t_1), ((t_1\,op\,t_2), t_2) \,|\, op \in \{ \BitWiseSet \} \}$       \\ 
\cline{2-3}
&SOD               & $\{((t_1\,op\,t_2), t_1), ((t_1\,op\,t_2), t_2) \,|\, op \in \{ \ShiftSet \} \}$       \\ 
%\hline
%COR               & $\{(\op_1, \op_2) \,|\, \op_1, \op_2 \in \{ \texttt{\&\&}, \texttt{||}, \land \} \land \op_1 \neq \op_2 \}$            \\
\hline
\hline
\multirow{3}{*}{\rotatebox{90}{\emph{Other}}}&LVR			& $\{(l_1, l_2) \,|\, (l_1, l_2) \in \{(0,-1), (l_1,-l_1), (l_1, 0), (\mathit{true}, \mathit{false}), (\mathit{false}, \mathit{true})\}\}$\\
&&\\
&&\\
\hline
\end{tabular}

$^{*}$Each pair in parenthesis shows how a program element is modified by the mutation operator. Th eleft element of the pair is replaced with the right element. We follow standard syntax~\cite{kintis2018effective}. Program elements are literals ($l$), integer literals ($i$), boolean expressions ($e$), operators ($\op$), statements ($s$), variables ($v$), and terms ( $t_i$, which might be either variables or literals).
\end{table}

%To automatically generate mutants, we have extended SRCIRor~\cite{hariri2018srciror} to include all the
%operators in Table~\ref{table:operators}.
%After mutating the original source file, our extension saves the mutated source file and keeps track of the mutation applied. Our toolset is available under the ESA Software Community Licence Permissive~\cite{ESAlicence} at the following URL \textbf{https://faqas.uni.lu/}.

\subsection{Step 3: Compile mutants}
\label{sec:appr:compile}

In Step 3, we
compile mutants by relying on an optimized compilation procedure that leverages the build system of the SUT. To this end, we have developed a toolset that, for each mutated source file: (1) backs-up the original source file, (2) renames the mutated source file as the original source file, (3) runs the build system (e.g., executes the command \texttt{make}), (4) copies the generated executable mutant in a dedicated folder, (5) restores the original source file. Mutants that lead to compilation errors are discarded.

%Build systems \JMR{1.15}{(e.g., GNU make~\cite{MAKE} driving the GCC~\cite{GCC} compiler)} create one object file for each source file to be compiled and then link these object files together into the final executable.
%\CHANGED{After the first build, in subsequent builds,
%build systems
%%they
%recompile only the modified files and link them to the rest.}
%For this reason, our optimized compilation procedure, which modifies at most two source files for each mutant (i.e., the mutated file and the file restored to eliminate the previous mutation), can reuse almost all the compiled object files in subsequent compilation runs, thus speeding up the compilation of multiple mutants. The experiments conducted with our subjects have shown that
%%additional optimizations are not necessary to make the compilation of mutants feasible.
%\CHANGED{our optimization is sufficient to make the compilation of mutants feasible for large projects. Other state-of-the-art solutions introduce additional complexity (e.g., change the structure of the software under test~\cite{untch1993mutation}) that does not appear to be justified by scalability needs.}

% Concerning compilation warnings, we assume the build system of the SUT has been properly configured; more precisely, if the system should compile without warnings, the compiler is expected to be configured to treat warnings as errors otherwise mutants that lead to warning are retained.}

\subsection{Step 4: Remove equivalent and redundant mutants based on compiled code}

In Step 4, we rely on trivial compiler optimizations to identify and remove equivalent and redundant mutants.
\REVNOV{C-P-15}{We compile the original software and every mutant multiple times once for each every available optimization option (i.e., \texttt{-O0}, \texttt{-O1}, \texttt{-O2}, \texttt{-O3}, \texttt{-Os}, \texttt{-Ofast} in GCC) or a subset of them.
After each execution of the compiler, we compute the SHA-512 hash summary of the generated executable.}To detect equivalent mutants, \APPR compares the hash summaries of the mutants with that of the original executable. To detect duplicate mutants but avoid combinatorial explosion, \APPR focuses its comparison of hash summaries on pairs of mutants belonging to the same source file (restricting the scope of the comparison is common practice~\cite{kintis2017detecting}).
Hash comparison allows us to (1) determine the presence of equivalent mutants (i.e., mutants having the same hash as the original executable), and (2) identify duplicate mutants (i.e., mutants with the same hash). %Mutants that are identified as being either equivalent and duplicate mutants are ignored in the following steps of \APPR.
Equivalent and duplicate mutants are then discarded.

%We compare hash summaries rather than executable files because it is much faster, an important consideration when dealing with a large number of mutants.
%The outcome of Step 4 is a set of \INDEX{unique mutants}, i.e., mutants with compiled code that differs from the original software and any other mutant.


\subsection{Step 5: Sample Mutants}
\label{sec:codeDriven:samplingStep}
\STARTCHANGEDNOV


In Step 5, \APPR samples the mutants to be executed to compute the mutation score.
\JMR{1.8 3.3}{\APPR does not selectively generate mutants but samples them from the whole set of successfully compiled, nonequivalent, and nonduplicated mutants (result of Steps 2 to 4). This choice aims to avoid sampling bias which may result from the presence of such mutants; indeed, there is no guarantee that these mutants, if they were discarded after being sampled, would be uniformly distributed across program statements. Our choice does not affect the feasibility of \APPR since Steps 2 to 4 have negligible cost.}

Our pipeline supports different sampling strategies: \INDEX{proportional uniform sampling}, \INDEX{proportional method-based sampling},  \INDEX{uniform fixed-size sampling}, and \INDEX{uniform FSCI sampling}, which is an innovative contribution of the FAQAS actvity~\cite{Oscar:MASS:TSE}.

%The strategies \INDEX{proportional uniform sampling} and \INDEX{proportional method-based sampling} were selected based on the results of Zhang et al.~\cite{zhang2013operator}, who compared eight strategies for sampling mutants.
%The former was the best performing strategy and consists of sampling mutants evenly across all functions of the SUT, i.e., sampling $r\%$ mutants from each set of mutants generated inside the same function.
%The latter consists of randomly selecting $r\%$ mutants from the complete mutants set. This is included in our study because it is simpler to implement and showed to be equivalent to stratified sampling strategies, based on recent work~\cite{gopinath2015hard}.
%
%The \INDEX{uniform fixed-size sampling} strategy stems from the work of Gopinath et al.~\cite{gopinath2015hard} and consists of selecting a fixed number $N_M$ of mutants for the computation of the mutation score. Based their work, with 1,000 mutants, one can guarantee an accurate estimation of the mutation score.
%% In our empirical evaluation, we assess the accuracy obtained for $N_M$ values across the range [100;1000].}
%
%In this paper, we introduce the \INDEX{uniform FSCI sampling} strategy that determines the sample size dynamically, while exercising mutants, based on a fixed-width sequential confidence interval approach.
%With \INDEX{uniform FSCI sampling}, we introduce a cycle between Step 6 and Step 5, such that a new mutant is sampled only if deemed necessary.
%% of the mutation testing results collected so far.
% More precisely, \APPR iteratively selects a random mutant from the set of unique mutants and exercises it using the SUT test suite.
%% The result of each mutant execution (i.e., killed or live) is treated as a Bernoulli trial that is used to compute the confidence interval according to the FSCI method.
%Based on related work, we assume that the mutation score computed with a sample of mutants follows a binomial distribution (see Section~\ref{sec:scalability}).
%For this reason, to compute the confidence interval for the FSCI analysis, we rely on the Clopper-Pearson method since it is reported to provide the best results (see Section~\ref{sec:scalability}).
%Mutation analysis (i.e., sampling and testing a mutant) stops when the confidence interval is below a given threshold $T_{\mathit{CI}}$ (we use $T_{\mathit{CI}}=0.10$ in our experiments). More formally, given a confidence interval
%$[\mathit{L}_{S};\mathit{U}_{S}]$, with $\mathit{L}_{S}$ and $\mathit{U}_{S}$ indicating the lower and upper bound of the interval, mutation analysis stops when the following condition holds:
%\begin{equation}
%\label{eq:CI:T}
%(\mathit{U}_{S}-\mathit{L}_{S})<T_{\mathit{CI}}.
%\end{equation}
%
%Unfortunately, the assumption about the estimated mutation score following a binomial distribution may not hold when a subset of the test suite is executed for every mutant (which could happen in Step 6). Without going into the details behind the implementation of Step 6, which is described in Section~\ref{sec:step:prioritize},
%we can expect that a reduced test suite may not be able to kill all the mutants killed by the entire test suite, i.e., the estimated mutation score may be affected by negative bias. Consequently, over multiple runs, the mean of the estimated mutation score may not be close to the \INDEX{actual mutation score} (i.e., the mutation score computed with the entire test suite exercising all the mutants for the SUT)
% but may converge to a lower value.
%To compute a correct confidence interval that includes the actual mutation score of the SUT, we thus need to take into account this negative bias.
%
%To study the effect of negative bias on the confidence interval, we address first the relation between the actual mutation score and the mutation score computed with the reduced test suite when the entire set of mutants for the SUT is executed.
%A mutant killed by the entire test suite has a probability $P_{\mathit{KErr}}$ of not being killed by the reduced test suite.
%The probability $P_{\mathit{KErr}}$  can be estimated as the proportion of mutants (erroneously) not killed by the reduced test suite
%\begin{equation}
%P_{\mathit{KErr}} = \frac{|E_R|}{|M|}
%\end{equation}
%with
%$E_R$ being the subset of mutants that are killed by the entire test suite but not by the reduced test suite, and $M$ being the full set of mutants for the SUT.
%
%The mutation score for the reduced test suite ($\mathit{MS}_R$) can be computed as
%
%\begin{equation}
%\small
%\mathit{MS}_R=\frac{|K|-|E_R|}{|M|}=\frac{|K|}{|M|}-\frac{|E_R|}{|M|}=\mathit{MS}-\frac{|E_R|}{|M|}=\mathit{MS}-P_{\mathit{KErr}}
%\end{equation}
%
%where $K$ is the set of mutants killed by the whole test suite, $M$ is the set of all the mutants of the SUT,  and $\mathit{MS}$ is the actual mutation score. Consequently, the actual mutation score can be computed as
%\begin{equation}
%\label{eq:MS}
%\mathit{MS}=\mathit{MS}_R+P_{\mathit{Err}_R}
%\end{equation}
%
%We now discuss the effect of a reduced test suite on the confidence interval for a mutation score estimated with mutants sampling. When mutants are sampled and tested with the entire test suite, the actual mutation score is expected to lie in the confidence interval $[\mathit{L}_{S};\mathit{U}_{S}]$.
%\CHANGED{In the presence of a reduced test suite, we can still rely on the Clopper-Pearson method to compute the confidence interval $\mathit{CI}_R=[\mathit{L}_{R};\mathit{U}_{R}]$.
%However, }
%%in the presence of a reduced test suite,
%we have to take into account the probability of an error in the computation of the mutation score $\mathit{MS}_R$;  $\mathit{MS}_R$ can be lower than $\mathit{MS}$ and, based on Equation~\ref{eq:MS}, we expect the actual mutation score to lie in
%%the interval.
%\JMR{NEW}{an interval that is shifted with respect to the interval for $\mathit{MS}_R$:}
%
%\begin{equation}
%\label{eq:CI}
%\mathit{CI}=[\mathit{L}_{R}+P_\mathit{KErr};\mathit{U}_{R}+P_{\mathit{KErr}}]
%\end{equation}
%
%We can only estimate  $P_{\mathit{KErr}}$ since computing it would require the execution of all the mutants with the complete test suite, thus undermining our objective of reducing test executions.
%To do so, we can randomly select a subset $M_R$ of mutants, on which to execute the entire test suite and identify the mutants killed by the reduced test suite. %\footnote{In our implementation, we record the outcome of each test case of the whole suite and then simulate the execution of the reduced test suite, thus saving time.}
%The size of the set $M_R$ should be lower than the number of mutants we expect FSCI sampling to return,
%%(e.g., we use $M_R=100$ in our experiments\footnote{Though it would be possible to also estimate $P_{\mathit{KErr}}$ with FSCI, we leave it for future work.}),
%otherwise sampling would not provide any cost reduction benefit.
%Since, for every mutant in $M_R$, we can determine if it is erroneously reported as not killed by the reduced test suite R,
%we can
%%end up with a vector of boolean evaluations $E=e_1, ..., e_n$ where each evaluation $e_i$ is equal to \emph{true} if the $i^{th}$ mutant has been erroneously indicated as not killed by the reduced test suite.
%%This population of evaluations enable us to
%estimate the probability $P_{\mathit{KErr}}$ as the percentage of such mutants.
%As for the case of the mutation score,
%%these mutants tend to be positively  correlated\footnote{If a line of code contains a mutant that is not killed by the reduced test suite, it may contain another such mutant.} and, for this reason,
%we assume that the binomial distribution provides a conservative estimate of the variance for $P_{\mathit{KErr}}$.
%
%We can estimate the confidence interval for $P_{\mathit{KErr}}$ using one of the methods for binomial distributions.
%We rely on the Wilson score method because it is known to perform well with small samples~\cite{Newcombe:Wilson:1998}.
%The value of $P_{\mathit{KErr}}$ will thus lie within $\mathit{CI}_E=[\mathit{L}_{E};\mathit{U}_{E}]$,  with $\mathit{L}_{E}$ and $\mathit{U}_{E}$ indicating the lower and upper bounds of the interval.
%
%
%
%\NEWFSCI{Based on Equation~\ref{eq:CI},
%%to ensure that the actual mutation score lies in the computed confidence interval, we should assume the worst case, i.e., $P_{\mathit{KErr}}=\mathit{U}_{E}$.
%the confidence interval to be used with FSCI sampling in the presence of a reduced test suite should thus be }
%\begin{equation}
%\label{eq:CI:FSCI}
%\mathit{CI}=[\mathit{L}_{R}+\mathit{L}_{E};\mathit{U}_{R}+\mathit{U}_{E}]
%\end{equation}
%
%\JMRCHANGE{The estimated mutation score is the value lying in the middle of the interval.}
%
%\UPDATE{Since the width of the confidence interval CI (hereafter, $|CI|$) results from the sum of $|\mathit{CI}_R|$ and $|\mathit{CI}_E|$,
%%Also, from a practical perspective, \JMRCHANGE{the terms $+\mathit{L}_{E}$ and} $+\mathit{U}_{E}$ may augment the size of the interval returned by the Clopper-Pearson method; consequently,
%mutation sampling with a reduced test suite may lead to the execution of a larger set of mutants.}
%
%\UPDATE{Based on Equations~\ref{eq:CI:T} and~\ref{eq:CI:FSCI}, $|\mathit{CI}_R| \le T_{\mathit{CI}} - |\mathit{CI}_E|$.
%Consequently, when $|\mathit{CI}_E|>T_{\mathit{CI}}$, the reduced test suite cannot lead to sufficiently accurate results.
%Also, a large $|\mathit{CI}_E|$ may prevent the identification of accurate results with a feasible number of mutants. For example, Clopper-pearson may require up to 1568 samples for a confidence interval below 0.05~\cite{Goncalves2012}}.
%%the interval is 3.5, you will need a $[\mathit{L}_{R};\mathit{U}_{R}]$ of 7.5, which may require XXXX samples in the worst case).
%%https://select-statistics.co.uk/calculators/sample-size-calculator-population-proportion/
%%Basic Business Statistics, Global Edition, 13th Edition
%\UPDATE{
%We shall thus identify a threshold ($T_{\mathit{CE}}$) for the confidence interval $|\mathit{CI}_E|$ that enables
%%$|\mathit{CI}_R| \le (T_{\mathit{CI}} - |\mathit{CI}_E|)$
%accurate estimates
%with a small sample size (e.g., in the worst case, with less than 1000 samples, the sample size for related work).
%For this reason, starting from a minimal number of samples to estimate $P_{\mathit{KErr}}$ (150 in our experiments), \APPR keeps estimating $P_{\mathit{KErr}}$ until it yields $|\mathit{CI}_E| \le T_{\mathit{CE}}$.
%In our experiments we set $T_{\mathit{CE}} = 0.035$.
%To select $T_{\mathit{CE}}$, we have identified a reasonable minimal mutation score to be expected in space software (i.e., 65\%) and identified, based on confidence interval estimation methods with finite population correction factor~\cite{BasicBusinessStatistics},
%the minimal value for $|\mathit{CI}_E|$ that requires a number of samples below 850 (i.e., $1000-150$).
%%Indeed, for a binomial proportion with a probability of success above 65\% (the minimal mutation score), a confidence interval width of 0.065 (i.e., $0.1-0.035$) shall be achieved with a sample size of 821.}
%}
%
%When it is not possible to estimate $|\mathit{CI}_E| \le T_{\mathit{CE}}$ \UPDATE{or when the number of samples required to estimate $|\mathit{CI}_E| \le T_{\mathit{CE}}$ is sufficient to accurately estimate the mutation score,} the test suite can be prioritized but not reduced and the confidence interval is computed using the traditional Clopper-Pearson method, i.e., $[\mathit{L}_{S};\mathit{U}_{S}]$.
%
%
%\ENDCHANGEDNOV

\subsection{Step 6: Execute prioritized subset of test cases}
\label{sec:step:prioritize}

In Step 6, we execute a prioritized subset of test cases.
We select only the test cases that satisfy
the reachability condition (i.e., cover the mutated statement) and  execute them in sequence.
Similarly to the approach of Zhang et al. \cite{zhang2013faster}, we define the order of execution of test cases based on their estimated likelihood of killing a mutant.
\CHANGED{However, in our work, this likelihood is estimated differently since, as discussed above, the measurements they rely on are not applicable in the context of system-level testing and complex cyber-physical systems.} \CHANGEDNOV{In contrast, to minimize the impact of measurements on real-time constraints, we only collect code coverage information for a small part of the system.}
%However, we redefined the criteria for the prioritization and selection of test cases because of the inapplicability of the ones proposed by Zhang et al. (see Section~\ref{sec:scalability}).

%However, we notice that such optimization may not be sufficient when test suites are particularly large; indeed, prioritizing test cases may not be sufficient to reduce execution time. For example, live mutants may lead to the execution of a large number of test cases when almost all the test cases of the test suite exercise the mutated statement.
\REVNOV{C-P-46}{We execute only covered statements assuming that the test suite is optimal with respect to code coverage. More precisely, we addume that if a statement is not covered there is a good reason for it (e.g., it depends on hardware).
If a statement is not covered by the test suite, there is no chance that a mutant generated in the non-covered statement can be possibly detected by any test case.
If the test suite does not reach the required coverage there is no reason to perform mutation testing, because is already known that the test suite is not good.}

%To reduce the number of test cases to be executed with a mutant,
%we should first execute the ones that more likely satisfy the necessity condition.
%This might be achieved by executing a test case that exercises the mutated statement with variable values not observed before.
%Unfortunately, in our context, the size of the SUT and its real-time constraints prevent us from recording all the variable values processed during testing.
%
%Therefore, we rely on code coverage to determine if two test case executions exercise the mutated statement with diverse variable values. Such coverage is collected by efficient procedures provided by compilers, thus having lower impact on execution performance than other types of dynamic analysis solutions (e.g., tracing variable values).
%% as a surrogate indicator of  variable values diversity.
%%diversity in values assigned to the variables used in a statement.
%Since, because of control- and data-flow dependencies, a different set of input values may lead to differences in code coverage,
%the latter helps determine if two or more test cases likely exercise a mutated statement with different variable values. %Indeed, a difference in the set of
%%statements covered by two test cases depends on the values reaching some of the program statements, including the mutated one.
%%Indeed, a difference in the set of
%%statements covered by two or more test cases that exercise a same mutated statement may depend on the values used in such statement.
%%For example, the definition of a variable may lead to the execution of different branches when different values are assigned across distinct executions.
%To increase the likelihood that the observed differences in code coverage are due to the use of different variable values to exercise the mutated statement, we restrict the scope of code coverage analysis
%to the functions belonging to the component (i.e., the source file) that contains the mutated statement.
%%However, two test case executions may also present coverage differences
%%because of a diverse set of variable values used by statements other than the mutated one.
%%For this reason, we restrict the scope of code coverage analysis
%%to the functions belonging to a same component (i.e., a same source file).
%Indeed, such functions typically present several control- and data-flow dependencies, thus
%augmenting the likelihood that a coverage difference is due to the execution of the mutated statement with a diverse set of values. Also, collecting code coverage for a small part of the system further reduces the impact of our analysis on system performance.
%
%%to the mutated function, its callers, and its callees.
%%to maximize the chances that a change in the behaviour of the software depends on the values used in a mutated statement, we determine that two executions likely exercise a mutated statement with diverse values by focussing on the coverage of
%%the mutated function, its callers, and its callees.
%%A reduced scope is effective in determining behavioural differences based on the analysis of variable valuations~\cite{Pastore:VART:2014}.
%
%%Since related work focuses on either statement coverage or the frequency of execution of a statement,
%Based on related work, we have identified two possible strategies to characterize test case executions based on code coverage:
%\begin{itemize}
%\item[S1] Compare the sets of source code statements that have been covered by test cases~\cite{grun2009impact}.
%%\item[C2] Identify the set of unique pairs $\langle\mathit{statement},\mathit{arity}\rangle$, where $\mathit{statement}$ is a unique identifier for the source code statement, and $\mathit{arity}$ is a symbol (i.e., $1$ or $*$) indicating if the statement has been covered one or more times.
%\item[S2] Compare the number of times each statement has been covered by test cases~\cite{schuler2013covering}.
%\end{itemize}
%
%
%To determine how dissimilar two test cases are and, consequently, how likely they exercise the mutated statement with different values, we rely on widely adopted distance metrics.
%In the case of S1, we rely on the Jaccard and Ochiai index, which are two similarity indices for binary data and have successfully been used to compare program executions based on code coverage~\cite{Zou:Ochiai:2019,Keller:Jaccard:2017,Briand:2019}.
%\REVNOV{C-P-18}{The Jaccard index is also known as \INDEX{intersection over union}
%it measures similarity between sets, and is defined as the size of the intersection divided by the size of the union.
%The Jaccard distance measures dissimilarity between sample sets and results from subtracting the Jaccard coefficient from 1.
%The Ochiai index calculates cosine similarity with binary data, it is used in molecular biology and software engineering.}
%Given two test cases $T_A$ and $T_B$, the Jaccard  ($D_J$) and Ochiai ($D_O$) distances are computed as follows:
%
%$D_J(T_a,T_b)=1-\frac{|C_a \cap C_b|}{|C_a \cup C_b|}$ \hspace{2mm} $D_O(T_a,T_b)=1-\frac{|C_a \cap C_b|}{\sqrt{|C_a| * |C_b|}}$,
%where $C_a$ and $C_b$ are the set of covered statements exercised by $T_a$ and $T_b$, respectively.
%
%In the case of S2, we compute the distance between two test cases by relying on the euclidean distance ($D_E$) and the cosine similarity distance ($D_C$), two popular distance metrics used in machine learning.
%\REVNOV{C-P-18}{Euclidean distance is the straight-line distance between two points in Euclidean space; precisely, the Euclidean distance between two points is the length of the line segment connecting them. In our context, the two vectors consist of the number of times the program statements had been exercised by a test. Cosine similarity measures similarity between two vectors of an inner product space. It results from the inner product of the same vectors normalized to both have length 1, which matches the the cosine of the angle between them. It is widely adopted to measure cohesion within clusters in data mining.}
%Given two vectors $V_A$ and $V_B$, whose elements capture the number of times a statement has been covered by test cases $T_A$ and $T_B$, the distances $D_E$ and $D_C$ can be computed as follows:
%
%$D_E=\sqrt{\sum_{i=1}^{n}(A_i-B_i)^2}$
%
%$D_C= 1-\frac{\sum_{i=1}^{n}A_i*B_i}{\sqrt{\sum_{i=1}^{n}{A_i}^2}*\sqrt{\sum_{i=1}^{n}{B_i}^2}}$,
%%Their main difference is that cosine similarity is used when the magnitude of the vectors should not matter.
%where $A_i$ and $B_i$ refer to the number of times the i-th statement had been covered by $T_A$ and $T_B$, respectively.
%
%Figure~\ref{alg:prioritize} shows the pseudocode of our algorithm for selecting and prioritizing test cases. It generates as output
%a prioritized test suite (\INDEX{PTS}).
%Based on the findings of Zhang et al. \cite{zhang2013faster}, we first select the test case that exercises the mutated statement the highest number of times (Line~\ref{alg:prioritize:first}) \CHANGED{ and add it to the prioritized test suite (Line~\ref{alg:prioritize:add}).}
%Then, in the next iterations, the test case selected is the one with the largest distance from the closest test case already selected (Lines~\ref{alg:prioritize:selectStart} to~\ref{alg:prioritize:selectEnd}).
%When two or more test cases have the same distance, we select randomly among the test cases that exercise the mutated statement the most.
%%is most different than any other test case already included in the prioritized test suite.
%
%%Then, since we aim to maximize test cases diversity, the next selected test case should be the one that is most different than any other test case already included in the prioritized test suite.
%%For this reason, for each test case $n$ not selected yet (Line~\ref{alg:prioritize:notSel}), we identify the test case $t$ showing the most similar coverage (i.e., the one with the minimal distance $d$, Line~\ref{alg:prioritize:minD}). We then select the test case $n$ with the highest distance from its closest test case (Lines~\ref{alg:prioritize:selectStart} to~\ref{alg:prioritize:selectEnd}).
%
%The algorithm iterates as long as it identifies a test case
%showing a difference in code coverage from the
%%that exercises
%%the program statements differently than
%already selected test cases (Line~\ref{alg:prioritize:until}).
%
%Test cases are then executed in the selected order. During  execution, we collect code coverage information and identify killed and live mutants.
%
%
%% !TEX root =  ../Main.tex

\newcommand{\INDA}{10}
\newcommand{\INDB}{15}
\newcommand{\INDC}{5}

%\vspace{-3mm}
\begin{figure}[tb]

\begin{algorithmic}[1]

%\footnotesize
\scriptsize
\Require \emph{TS}, the test suite of the software under test
\Require \emph{Cov}, coverage information, for each test case
\Require \emph{ms}, the mutated statement
\Ensure \emph{PTS}, a list of test cases to be executed, sorted by priority
% (source inputs, follow-up inputs, output data).


\State from \emph{Cov}, identify the test case $t$ that exercises $\mathit{ms}$ more times \label{alg:prioritize:first}
\State $\mathit{PTS} \gets \mathit{PTS} \cup t$

\State \textbf{repeat} \label{alg:prioritize:repeat}
\State \hspace{\INDC mm} \textbf{for each} $n$ in the set ($\mathit{TS}$ - $\mathit{PTS}$) \label{alg:prioritize:notSel}
\State \hspace{\INDA mm} \textbf{for each} $t$ in $\mathit{PTS}$
\State \hspace{\INDB mm} compute the distance between $t$ and $n$
\State \hspace{\INDA mm} identify $t_n$ i.e., the test case $t$ with the minimal $d$ \label{alg:prioritize:minD}
\State \hspace{\INDC mm} among all the $t_n$ identified, select the one with the highest distance $d$ \label{alg:prioritize:selectStart}
\State \hspace{\INDC mm} \textbf{if} $d > 0$ \textcolor{darkgray}{//there is at least a test case with a different coverage}
\State \hspace{\INDA mm} \textcolor{darkgray}{//note: $n$ is the test case in the set ($\mathit{TS}$ - $\mathit{PTS}$) closer to $t_n$}
\State \hspace{\INDA mm} $\mathit{PTS} \gets \mathit{PTS} \cup n$ \label{alg:prioritize:selectEnd}
\State \textbf{until} $d > 0$ \label{alg:prioritize:until}


\end{algorithmic}
\vspace{-3mm}
\caption{Algorithm for prioritizing test cases}
%\vspace{-0.2cm}
\label{alg:prioritize}
\end{figure}




\subsection{Step 7: Discard Mutants}
\label{sec:algostepSeven}


In this step, we identify likely nonequivalent and likely nonduplicate mutants by relying on code coverage information \CHANGED{collected in the previous step}.

Similarly to related work~\cite{schuler2013covering},
%since the size of a program may determine the degree of non-determinism in statement coverage,
we identify nonequivalent and nonduplicate mutants based on a threshold.

In our case, consistently with previous steps of \APPR,
we compute normalized distances based on the distance metrics $D_J$, $D_O$, $D_E$, and $D_C$. A mutant is considered nonequivalent when the distance from the original program is above the threshold $T_E$, for at least one test case.
Similarly, a mutant is considered nonduplicate when the distance from every other mutant is above the threshold $T_D$, for at least one test case. For the identification of nonequivalent mutants, we consider live mutants only. To identify nonduplicate mutants, we consider both live and killed mutants; however, to avoid combinatorial explosion, we compare only mutants belonging to the same source file (indeed, mutants belonging to different files are unlikely to be redundant).
Killed mutants that lead to the failure of different test cases are not duplicate, regardless of their distance.

Thresholds $T_E$ and $T_D$ should enable the identification of mutants that are guaranteed to be nonequivalent and nonduplicate. In particular, we are interested in the set of \emph{live, nonequivalent, nonduplicate mutants} (hereafter, $\mathit{LNEND}$) and the set of \emph{killed, nonduplicate mutants} (hereafter, $\mathit{KND}$). With such guarantees, the mutation score can be adopted as an adequacy criterion in safety certification processes. For example, certification agencies may require safety-critical software to reach a mutation score of 100\%, which is feasible in the presence of nonequivalent mutants.
%This will enable the adoption of mutation score as an adequacy criterion,

%\REVNOV{C-P-19}{Figure~\ref{alg:nonEquivalent:nonRedeundat} shows the algorithm for detecting nonequivalent and nonduplicate mutants.
%It first identify among the list of killed mutants all the non-duplicate ones (Line~\ref{alg:equivalent:KND}).
%Then it identifies the non-equivalent mutants among the list of live mutants (Line~\ref{alg:equivalent:LNE}).
%Finally, it further filters the list of non-equivalent mutants to keep only the ones that appear to be nonduplicate (Line~\ref{alg:equivalent:LNEND}).}
%
%% !TEX root =  ../Main.tex

\renewcommand{\INDA}{5}
\renewcommand{\INDB}{10}
\renewcommand{\INDC}{15}
\newcommand{\INDD}{20}
\newcommand{\INDE}{25}

\renewcommand{\Comment}[1]{\textcolor{darkgray}{\textit{//#1}}}

%\vspace{-3mm}
\begin{figure}[tb]

\begin{algorithmic}[1]

%\footnotesize
\scriptsize
\Require \emph{D}, the distance function to use to identify equivalent/duplicate mutants
\Require \emph{KM}, list of killed mutants
\Require \emph{LM}, list of live mutants
\Require $\mathit{Cov}_O$, coverage information for all the test cases, for the original program
\Require $\mathit{Cov}_M$, coverage information for all the executed test cases, for every mutant
\Require \emph{TS}, list of test cases
\Require \emph{TR}, test results, for all the executions
\Ensure \emph{KNR}, a list of killed, non-duplicate  mutants
\Ensure \emph{LNENR}, a list of live, non-equivalent, non-duplicate mutants
% (source inputs, follow-up inputs, output data).

%\State $\mathit{TS}_m \gets$ subset of $\mathit{TS}$ that cover the mutated statement $\mathit{ms}$, based on \emph{Cov} \label{alg:equivalent:select}
%\State $\mathit{PTS} \gets \mathit{new} \mathit{list}$ \textcolor{darkgray}{//this list is initially empty}
%\State $\mathit{PTS} \gets$ based on \emph{Cov} select from $\mathit{TS_m}$ the test case $t$ that exercises $\mathit{ms}$ more times \label{alg:equivalent:first}
\State $\mathit{KNR} \gets \mathit{identifyNonDuplicateMutants(} \mathit{KM}, \mathit{TS}, \mathit{TR}, \mathit{Cov}_M)$ \label{alg:equivalent:KNR}
\State $\mathit{LNE} \gets \mathit{identifyNonEquivalentMutants(} \mathit{LNR}, \mathit{TS}, \mathit{Cov}_M, \mathit{Cov}_O)$ \label{alg:equivalent:LNE}
\State $\mathit{LNENR} \gets \mathit{identifyDuplicateMutants(} \mathit{LNE}, \mathit{TS}, \mathit{Cov}_M)$ \label{alg:equivalent:LNENR}


\Procedure{identifyNonDuplicateMutants}{$\mathit{M},\mathit{TS},\mathit{TR},\mathit{Cov}_M$}\Comment{$M$ is a  list of mutants, $\mathit{TS}$, $\mathit{TR}$, and $\mathit{Cov}_M$ are defined above}
\State $\mathit{NR} \gets \mathit{empty} \mathit{set}$
\State $\mathit{k1} \gets $ extract and remove first element of M
\State $\mathit{NR} \gets \mathit{NR} \cup \mathit{k1}$
\While {$\mathit{M}$ not empty}
\State $\mathit{k2} \gets $ extract and remove first element of M
\For {mutant $k1$ in $\mathit{NR}$}
\State $\mathit{duplicate}=\mathit{TRUE}$
\For {test case $t$ in $\mathit{TS}$}
\If {$t$ has different result in $k1$ and $k2$ }
\State $\mathit{duplicate}=\mathit{FALSE}$
\State \textbf{break}
\Else
\State $\mathit{cov}_{k1t} \gets $ extract coverage information for test case $t$ executed with mutant $k1$
\State $\mathit{cov}_{\mathit{k2}t} \gets $ extract coverage information for test case $t$ executed with mutant $k2$
\If {$D(\mathit{cov}_{k1t},\mathit{cov}_{\mathit{k2}t}) > T_R$ }
\State $\mathit{duplicate}=\mathit{FALSE}$
\State \textbf{break}
\EndIf
\EndIf
\EndFor
\If {$\mathit{duplicate}==\mathit{FALSE}$}
\State \textbf{break} \Comment{No need to compare with all the mutants if we know that it is not duplicate}
\EndIf
\EndFor
\If {$\mathit{duplicate}==\mathit{FALSE}$}
\State $\mathit{NR} \gets \mathit{NR} \cup \mathit{k2}$
\EndIf
\EndWhile
\State \textbf{return} $\mathit{NR}$
\EndProcedure


\Procedure{identifyNonEquivalentMutants}{$\mathit{M},\mathit{TS},\mathit{Cov}_M,\mathit{Cov}_O$}\Comment{$M$ is a  list of mutants, $\mathit{TS}$, and $\mathit{Cov}_M$, and $\mathit{Cov}_O$ are defined above}
\State $\mathit{NE} \gets \mathit{empty} \mathit{set}$
\While {$\mathit{M}$ not empty}
\State $\mathit{m} \gets $ extract and remove first element of M

\For {test case $t$ in $\mathit{TS}$}
\State $\mathit{cov}_{m} \gets $ extract coverage information for test case $t$ executed with mutant $m$
\State $\mathit{cov}_{o} \gets $ extract coverage information for test case $t$ executed with original program

\If {$D(\mathit{cov}_{m},\mathit{cov}_{o}) > T_E$ }
\State $\mathit{equivalent}=\mathit{FALSE}$
\State \textbf{break}
\EndIf

\EndFor

\If {$\mathit{equivalent}==\mathit{FALSE}$}
\State $\mathit{NE} \gets \mathit{NE} \cup \mathit{m}$
\EndIf

\EndWhile

\State \textbf{return} $\mathit{NE}$
\EndProcedure



%
%
%\State \hspace{\INDA mm} \textbf{if} $\mathit{KM}$ not empty
%\State \hspace{\INDB mm} compute the distance between $t$ and $n$
%\State \hspace{\INDA mm} identify $t_n$ i.e., the test case $t$ with the minimal $d$ \label{alg:equivalent:minD}
%\State \hspace{\INDC mm} among all the $t_n$ identified, select the one with the highest distance $d$ \label{alg:equivalent:selectStart}
%\State \hspace{\INDC mm} \textbf{if} $d > 0$ \textcolor{darkgray}{//there is at least a test case with a different coverage}
%\State \hspace{\INDA mm} \textcolor{darkgray}{//note: $n$ is the test case in the set ($\mathit{TS}_m$ - $\mathit{PTS}$) closer to $t_n$}
%\State \hspace{\INDA mm} $\mathit{PTS} \gets \mathit{PTS} \cup n$ \label{alg:equivalent:selectEnd}
%\State \textbf{until} $d > 0$ \label{alg:equivalent:until}


\end{algorithmic}
\vspace{-3mm}
\caption{Algorithm for identifying non-equivalent and non-duplicate mutants}
%\vspace{-0.2cm}
\label{alg:nonEquivalent:nonRedeundat}
\end{figure}



\subsection{Step 8: Compute Mutation Score}
\label{sec:appr:score}


The \INDEX{mutation score} (MS) is computed as the percentage of killed nonduplicate mutants
%(hereafter, \emph{KND})
over the number of nonequivalent, nonduplicate mutants identified in Step 7):

\begin{equation}
\label{equation:ms}
\mathit{MS} = \frac{|\mathit{KND}|}{|\mathit{LNEND}|+|\mathit{KND}|}
\end{equation}
%Similarly,
%
%obof at lest one test case with respect
%
%
%The code coverage difference between the mutant and the original program is represented by a \textit{threshold T\%}, a difference of code coverage over a certain T\% indicates that both versions are not equivalent.

%
%% !TEX root =  MAIN.tex

\newpage

\section{Code-driven Mutation Testing: SEMuS}

\subsection{Overview}
\label{sec:semus}





To achieve \INDEX{test suite augmentation} (i.e., automatically generate test cases that kill mutants), in FAQAS, we have implemented an extension of SEMu that we call SEMu for Space Software (\INDEX{SEMuS}).

%Symbolic Execution-based Mutant analysis for Space software (FAQAS-SEMuS), is an extension of SEMu that implements test generation for space software. 

% \TODO{I've extended the following paragraph; please check}

In the FAQAS context, we cannot use SEMu as it is, since it requires mutants to be compiled with the \INDEX{LLVM compiler} in LLVM-IR format. As demonstrated with our preliminary evaluation of existing mutation analysis tools, any analysis requiring compilation into LLVM format is unlikely applicable to space software because of two reasons:  (1) our case studies rely on compiler pipelines (e.g., RTEMS) that include architecture-specific optimizations that are not supported by LLVM, and (2) there is no guarantee that the compiled objects produced by LLVM are equivalent to those produced by the original compiler (e.g., memory allocation). 

To overcome the limitations above and \EMPH{apply the test generation approach of SEMu in FAQAS} we have implemented three solutions. 
First, to avoid mutants to behave differently than the original software because of LLVM specificities (case 2 above), we rely on  MASS to identify killed and live mutants, and then compile only the live mutants into LLVM-IR format. 
Second, to increase the likelihood of successfully compiling the SUT with LLVM (case 1 above),  instead of compiling the whole software, we compile only the mutated function and its dependencies, which enables the generation of unit test cases and shall be sufficient to ensure the quality of the system under test in this context (e.g., unit test cases are normally used to ensure high code coverage). Third, to enable the generation of the meta-mutants, which is necessary to apply SEMu, we have extended MASS to (1) generate, for each source file of the SUT, both a meta-mutant processed by SEMu and the mutants processed by MASS, and (2) trace each mutant, after mutation analysis, to each mutant contained into the meta-mutant. 





% \TODO{Please add an example of a meta-mutant and single mutants generated by MASS; also, add an sentence above to refer to such listings}





Figure~\ref{fig:semus_architecture} shows the architecture of SEMuS and how it interacts with MASS. SEMuS consists of five components, which are \INDEX{Test Template Generator},  \INDEX{Pre-SEMu},  \INDEX{KLEE-SEMu},  \INDEX{KTest to Unit Test}, and \INDEX{LLVM}.
They are detailed in the following paragraphs.

\begin{figure}[h]
\begin{center}
\includegraphics[width=0.8\textwidth]{images/semus-architecture2}
\caption{FAQAS-SEMuS Architecture and Workflow}
\label{fig:semus_architecture}
\end{center}
\end{figure}

%enable the adoption of SEMu in the space context. In particular, SEMuS (1) automates the generation of a test template including symbolic variables that guides the generation of test inputs, and (2) compiles the test template and the mutant into the format required by SEMu (i.e., LLVM-IR). 

\subsection{Test Template Generator}

The \INDEX{Test Template Generator} (TTG) component automates the generation of templates for the symbolic execution search. The component receives as inputs the SUT source code and the list of SUT functions. 

Listing~\ref{test_template} shows an example of a test template generated by the TTG. The TTG generates a template for every SUT function. The TTG parses the function arguments and declares them symbolic through use of the KLEE function \texttt{klee\_make\_symbolic}. Then, it adds a call to the function under analysis with symbolic values, and it saves the return value into a support variable (i.e., \texttt{result\_faqas\_semu} in Listing~\ref{test_template}). Finally, it generates a number of invocations of the \emph{printf} function that print the value of the software outputs and adds a return statement with the value returned by the function under test (e.g., \texttt{result\_faqas\_semu} in Listing~\ref{test_template}). 
Such \emph{printf} invocations are necessary because of the way SEMu determines that a mutant is killed; two are the cases in which the mutant is considered killed: (1) the main function returns a different return value, (2) different values are printed to the standard output. Consequently, it is necessary to print out all the values of the software outputs.
To select which variables to print, for every source file under test, the engineer can specify, in the SEMuS configuration file, the arguments (typically pointers) that should be either considered output or considered both input and outputs. By default, the TTG considers all the function arguments as inputs, in case some argument (e.g., a pointer to a memory buffer) is used both as input and as output the engineer shall specify it as such; similarly, in case some some argument (e.g., a pointer to a struct) is used to store outputs, the engineer shall indicate that it is an output. Moreover, since, in  the C language, the function \emph{printf} cannot automatically determine how to print the different fields specified in a data structure (i.e., it prints only the memory address of the pointer), the engineer can also specify how to printout the different fields of a data structure.

Listing~\ref{test_config} shows an example configuration file for SEMuS. In addition to output arguments (i.e., \emph{OUT\_\-ARGS\_NAMES}), input/output arguments (i.e., \emph{IN\_OUT\_ARGS\_NAMES}), and customized printf instructions (i.e., \emph{TYPES\_TO\_PRINTCODE}), the SEMuS configuration file enables the engineer to customize the generation of the test template further. Indeed, engineers can specify input argument types that should not be treated symbolically but that shall be initialized using a specific function of the SUT (i.e., \emph{TYPE\_TO\_INITIALIZATION\-CODE}); also, engineers can specify the fields, within such types (e.g., the attributes of a struct), that shall be treated symbolically.
Moreover, since the template returns the value of the function under test, the engineer can specify how to convert the returned value to int, if necessary (see parameter \emph{TYPES\_TO\_INTCONVERT}). Finally, in case the function under test receives a pointer to an array (which is typically passed as a pointer), the engineer can specify the size of such array (see parameter \emph{ARG\_TYPE\_TO\_ITS\_POINTER\_ELEM\_NUM}); by default, SEMuS assumes that a pointer refers to a single element (i.e., it is a pointer to a variable not an array).  


% !TEX root =  ../MAIN.tex

\begin{lstlisting}[style=CStyle, caption=SEMuS test template., label=test_template]
int main(int argc, char** argv) {
    // Declare variable to hold function returned value
    _Bool result_faqas_semu; 
    // Declare arguments and make input ones symbolic
    unsigned long pVal;
    int pErrCode;
    klee_make_symbolic(&pVal, sizeof(pVal), "pVal"); // Call function under test
    result_faqas_semu = T_INT_IsConstraintValid(&pVal, &pErrCode); // Make some output
    printf("FAQAS-SEMU-TEST_OUTPUT: %d\n", pErrCode);
    printf("FAQAS-SEMU-TEST_OUTPUT: %d\n", result_faqas_semu);
    return (int)result_faqas_semu;
}

\end{lstlisting}


\begin{lstlisting}[language={}, caption=Klee-test output, label=ktest]
ktest file : 'test000001.ktest'
args       : ['/MakeSym-TestGen-Input/direct/T_INT_IsConstraintValid/test.MetaMu.bc']
num objects: 2
object    0: name: b'model_version'
object    0: size: 4
object    0: data: b'\x01\x00\x00\x00'
object    1: name: b'pVal'
object    1: size: 8
object    1: data: b'\x00\x00\x00\x00\x00\x00\x00\x00'
\end{lstlisting}

% !TEX root =  ../MAIN.tex

\begin{lstlisting}[float=h, style=CStyle, caption=SEMuS configuration example., label=test_config]
{
"TYPES_TO_INTCONVERT": {},
"TYPES_TO_PRINTCODE": {"gs_timestamp_t": "printf(\"FAQAS-SEMU-TEST-OUTPUT: result_faqas_semu = tv_sec: %u, tv_nsec: %u\\n\", {}.tv_sec, {}.tv_nsec);"},
"OUT_ARGS_NAMES": ["pErrCode"],
"IN_OUT_ARGS_NAMES": ["base"],
"TYPE_TO_INITIALIZATIONCODE": {},
"TYPE_TO_SYMBOLIC_FIELDS_ACCESS": {},
"VOID_ARG_SUBSTITUTE_TYPE": "",
"ARG_TYPE_TO_ITS_POINTER_ELEM_NUM": {"char *": 6}
}
\end{lstlisting}


\subsection{Pre-SEMu}

The \INDEX{Pre-SEMu} component generates \INDEX{mutant schemata}; specifically, the component includes and compiles all the live mutants (i.e., MASS output) into a single bytecode file named the \emph{Meta Mutant}. SEMu will select which mutant to consider for test generation based on a parameter. The compilation of the Meta Mutant into LLVM bitcode is supported by the \emph{LLVM} compiler infrastructure. 

% !TEX root =  ../MAIN.tex

\begin{lstlisting}[style=CStyle, float=t, caption=Meta-Mutant of function T\_INT\_IsConstraintValid., label=meta_mutant_example]
flag T_INT_IsConstraintValid(const T_INT* pVal, int* pErrCode)
{
    flag ret = TRUE;
    (void)pVal;

    ret = ((*(pVal)) <= 50UL);

    klee_semu_GenMu_Mutant_ID_Selector_Func(1,2);
    *pErrCode = ret ? (klee_semu_GenMu_Mutant_ID_Selector==2?
    ((-1)):
    (klee_semu_GenMu_Mutant_ID_Selector==1?
    (1):
    (0))) :  ERR_T_INT;
    klee_semu_GenMu_Post_Mutation_Point_Func(0,0);
    klee_semu_GenMu_Post_Mutation_Point_Func(1,2);

    return ret;
}
\end{lstlisting}

\begin{lstlisting}[style=CStyle, float=t, caption=Mutant 1 of function T\_INT\_IsConstraintValid., label=meta_mutant_1]
flag T_INT_IsConstraintValid(const T_INT* pVal, int* pErrCode)
{
    flag ret = TRUE;
    (void)pVal;

    ret = ((*(pVal)) <= 50UL);
    *pErrCode = ret ? 1 :  ERR_T_INT;

    return ret;
}
\end{lstlisting}

\begin{lstlisting}[style=CStyle, float=t, caption=Mutant 2 of function T\_INT\_IsConstraintValid., label=meta_mutant_2]
flag T_INT_IsConstraintValid(const T_INT* pVal, int* pErrCode)
{
    flag ret = TRUE;
    (void)pVal;

    ret = ((*(pVal)) <= 50UL);
    *pErrCode = ret ? (-1) :  ERR_T_INT;

    return ret;
}
\end{lstlisting}

Listings~\ref{original_meta} provides the source code of function \emph{T\_INT\_IsConstraintValid}, while Listings~\ref{meta_mutant_1} and~\ref{meta_mutant_2} provide two example mutants generated by MASS.
Listing~\ref{meta_mutant_example} provides an example \INDEX{meta-mutant} including the same two mutants of Listings~\ref{meta_mutant_1} and~\ref{meta_mutant_2}. 
To select the mutants to analyze at runtime, SEMu relies on three support functions that shall be invoked within the eta-mutant:

\begin{itemize}
	\item \texttt{klee\_semu\_GenMu\_Mutant\_ID\_Selector\_Func}: function that takes two mutant IDs as arguments, representing a range of mutant IDs. It specifies where a portion of code containing mutants start.
    \item \texttt{klee\_semu\_GenMu\_Mutant\_ID\_Selector}: global variable that contains the ID of the mutant to be activated durng the analysis with SEMu.
	\item \texttt{klee\_semu\_GenMu\_Post\_Mutation\_Poin\_Func}: 
	function that takes two mutant IDs as arguments, it specifies where a portion of code containing mutants ends.
	It is used by SEMu to identify the portion of the code where to compare the state of the original and the mutated program and determine if the mutation has affected the program state (i.e., the \INDEX{necessity} condition to kill a mutant). In other words, it enables SEMu to perform conservative pruning and remove the mutant states that are not infected.
\end{itemize}

In Listing~\ref{meta_mutant_example}, mutant 2 from Listing~\ref{meta_mutant_2} appears on line 11 (indeed, the value \emph{-1} is selected when the mutant ID is equal to 2). Mutant 1 from Listing~\ref{meta_mutant_1} appears on line 12 (indeed, the value \emph{1} is selected when the mutant ID is equal to 1). The original software is represented by the mutant ID zero; indeed the value \emph{0} (i.e., what appears in line 7 of the Listing~\ref{original_meta}) is selected on line 14 (i.e., when the mutant ID is neither \emph{2} nor \emph{1}).

\subsection{KLEE-SEMu}



\INDEX{KLEE-SEMu} is the underlying test generation component, previously described in Section~\ref{sec:testGen:CP}. This component receives as inputs the \emph{LLVM bitcode} of the \emph{Meta Mutant} and the \emph{Test Template} for the function under test, and proceeds to apply dynamic symbolic execution to generate test inputs to kill the mutants. The output of this component are the \emph{KLEE tests}.

% \TODO{you need to list what are "the parameters of the execution"}
A \INDEX{KLEE test} is a binary file that contains information about the execution of KLEE such as the entry point of the analysis, and the generated test inputs.


An example of a KLEE test is presented in Listing~\ref{ktest}. The field \emph{args} report the entry point of the analysis; in this case, the test generation was performed for live mutants present in the function \texttt{T\_INT\_Is\-Constraint\-Valid}, which SEMuS stores in a dedicated folder. The fields named \emph{object} provide information about the outputs generated by KLEE (e.g., the generated test inputs). 
For each object, the KLEE test provides a \emph{name} (usually the name of the symbolic variable), its \emph{size}, and the actual \emph{value} generated by KLEE through constraint solving (usually this is reported in binary form).
Objects are numbered. Object number \emph{0} reports information about the data structure used by KLEE, that is, the version of the structure. The other objects report information about the generated test inputs.
Our example shows that one value of size 8 was generated for the variable \texttt{pVal}, the data field shows the binary representation of the \texttt{pVal} variable, in this case \texttt{pVal=0}.

% !TEX root =  ../MAIN.tex
\begin{lstlisting}[float=h, language={}, caption=Klee-test output, label=ktest_2]
ktest file : 'test000001.ktest'
args       : ['/MakeSym-TestGen-Input/direct/T_INT_IsConstraintValid/test.MetaMu.bc']
num objects: 2
object    0: name: b'model_version'
object    0: size: 4
object    0: data: b'\x01\x00\x00\x00'
object    1: name: b'pVal'
object    1: size: 8
object    1: data: b'\x00\x00\x00\x00\x00\x00\x00\x00'
\end{lstlisting}


\subsection{KTest to Unit Test}

% !TEX root =  ../MAIN.tex

\begin{lstlisting}[float=t, style=CStyle,  caption=Generated test case, label=gen_test_case]
#include <stdio.h>
#include <string.h>

#include "asn1crt.c"
#include "asn1crt_encoding.c"
#include "asn1crt_encoding_uper.c"


int main(int argc, char** argv)
{
    (void)argc;
    (void)argv;

    // Declare variable to hold function returned value
    _Bool result_faqas_semu;

    // Declare arguments and make input ones symbolic
    unsigned long pVal;
    int pErrCode;
    memset(&pVal, 0, sizeof(pVal));
    const unsigned char pVal_faqas_semu_test_data[] = {0x00, 0x00, 0x00, 0x00, 0x00, 0x00, 0x00, 0x00};
    memcpy(&pVal, pVal_faqas_semu_test_data, sizeof(pVal)); // Unsigned val is 0

    // Call function under test
    result_faqas_semu = T_INT_IsConstraintValid(&pVal, &pErrCode);

    // Make some output
    printf("FAQAS-SEMU-TEST_OUTPUT: pErrCode = %d\n", pErrCode);
    printf("FAQAS-SEMU-TEST_OUTPUT: result_faqas_semu = %d\n", result_faqas_semu);
    return (int)result_faqas_semu;
}
\end{lstlisting}



The component \INDEX{KTest to Unit Test} (KTU) converts a KLEE test into a human readable, compilable, and executable C test case. The unit test case generated by KTU, match the test template generated by TTG except for the declaration of variables where symbolic variables are replaced with concrete variables initialized with the values stored in the KTest file.
 
Listing~\ref{gen_test_case} shows an example of a test case generated for a mutant present in the function \texttt{T\_INT\_Is\-ConstraintValid}. For instance, line 20 shows that the variable \texttt{pVal} is initially filled with zeros, then in line 21, it is filled with the value stored in the variable \texttt{pVal\_faqas\_semu\_test\_data}, which holds the binary output produced by KLEE. In line 25, the function under test is invoked with the concrete value of \texttt{pVal}. 

The test case generated by the KTU prints the function return value and the value of every variable passed by reference, using the same instructions of the test template.
KTU cannot generate test assertions because only engineers can know, based on specifications, what are the values to be expected at the end of the test case execution.
However, the generated printf invocations still play the role of an oracle for regression testing,as explained in the next sections.

\subsection{Test suite augmentation} 

The procedure for testing with SEMuS is shown in Figure~\ref{fig:semus:test:example}.
In Step 1, the engineer executes SEMuS, which generates an output folder for every live mutant that is killed by the generated test cases.
Every folder contains: a script (i.e., \emph{runTest.sh}) that can be used to execute the generated test case, (2) the test case itself (i.e., test1.c), and (3) a text file with extension \emph{.expected} that contains the output that is observed when executing the test case with the SUT.
In Step 2, the engineer visually inspects every file with extension \emph{.expected} to determine if the observed output matches the specifications; if not, the software is faulty and needs to be fixed. In this case mutation testing enables detecting a fault. 
After verifying all the generated files with extension \emph{.expected} the engineer can reuse the test cases generated by SEMuS for regression testing in future versions of the SUT. Basically, the output folders generated by SEMuS become part of the test suite of the SUT.

When there is a new version of the SUT (Step 3, in Figure~\ref{fig:semus:test:example}), the folder with the source code of the SUT is replaced with the new version of the SUT (this can be done automatically with version control software).
The engineer can then trigger test execution by simply re-executing all the scripts \emph{runTest.sh} generated by SEMuS.
The script \emph{runTest.sh} first executes the test case (Step 4.1), then it stores the test outputs into a text file with extension \emph{.got}, finally if compares the observed output with the output generated for the previous version. If the function under test was not modified, differences may indicate that the test case FAILED.

\begin{figure}[h]
\begin{center}
\includegraphics[width=0.8\textwidth]{images/semus-out}
\caption{Workflow for test suite augmentation with SEMuS}
\label{fig:semus:test:example}
\end{center}
\end{figure}

\subsection{Live mutants}

SEMuS may not be capable of selecting test inputs that kill the mutant under analysis. We may distinguish two cases:
\begin{itemize}
\item SEMuS execution terminates and no test case is generated.
\item SEMuS execution does terminate (i.e., it is killed after a timeout configured by the end-user, usually 15 minutes are sufficient to generate test cases).
\end{itemize}

In the first case, SEMuS has successfully exercised all the execution paths covering the mutated statements but did not identify inputs that satisfy the killing conditions. This may case indicate that the mutant is equivalent and may be discarded (however, the engineer shall verify that the test template is configured correctly). Also, we may be in such situations when some of the functions under test belong to libraries not compiled with LLVM that, consequently are not correctly processed by KLEE-SEMu; to detect these cases the engineer shall look for errors in the output generated by KLEE.

In the second case, SEMuS did not complete the analysis of the possible execution paths covering the mutated statement. Such cases may indicate that test generation is complex and probably an engineer may more efficiently select test inputs than the underlying constraint solving solution implemented by KLEE-SEMu.
%
%% !TEX root =  ../MAIN.tex

\section{Data-driven Mutation Analysis: DAMAt}

\renewcommand{\APPR}{DAMAt\xspace{} }

\INDEX{Data-driven mutation analysis} is
a new mutation analysis paradigm
that alters the data exchanged by software components to evaluate the capability of a test suite to detect interoperability faults. 
Data-driven mutation analysis aims to evaluate the effectiveness of a test suite in detecting \INDEX{semantic interoperability \UPDATED{faults}}. 
It is achieved by modifying (i.e., mutating) the data exchanged by CPS components. It generates \INDEX{mutated data} that are representative of data that might be observed at runtime in the presence of a component that behaves differently than expected in the test case; also, it mutates  data that are not automatically corrected by the software 
(e.g., through cyclic redundancy check codes)
%(e.g., through CRC mechanisms, which aim to correct technical interoperability problems) 
and thus causes software failures (i.e., the mutated data shall have a different semantic than the original data). For these reasons, data mutation is driven by a fault model specified by the engineers based on domain knowledge.

Although different types of fault models might be envisioned,
%see background
we propose a technique (\INDEX{data-driven mutation analysis with tables}, \APPR),
which automates data-driven mutation analysis by relying on
a tabular \CHANGED{block model}, itself tailored to the \UPDATED{SUT} through predefined mutation operators.
To concretely perform data mutation at runtime, \APPR relies on a set of \INDEX{mutation probes} that shall be integrated by software engineers into the software layer that handles the communication between components. The runtime behaviour of mutation probes (i.e, what data shall be mutated and how) is driven by the fault model. Thus, \APPR can automatically generate the implementation of mutation probes from the provided fault model.
Depending on the CPS, probes might be inserted either into the \UPDATED{SUT}, into the simulator infrastructure, or both.
For example, Figure~\ref{fig:appr:mutateProbesInserted} shows the architecture of the \ESAIL satellite system with mutation probes integrated into the SVF
%\footnote{Software Validation Facility~\cite{Isasi2019}; it usually includes one or more simulators, an emulator to run the code compiled for the target hardware, and test harnesses.} 
functions that handle communication with external components (PDHU, GPS, and ADCS in this case). 






\APPR works in six steps, which are shown in Figure~\ref{fig:appr:approach}. 
In Step 1, based on the provided methodology and predefined mutation operators, the engineer prepares a fault model specification tailored to the SUT.
% capturing the data format and the types of faults that shall be injected for every data item exchanged by the system components.
In Step 2, \APPR generates a mutation API with the functions that modify the data according to the provided fault model.
In Step 3, the engineer modifies the \UPDATED{SUT} by introducing mutation probes (i.e., invocations to the mutation API) into it.
\REVTOOL{P-2}{Instead of modifying the SUT the engineer may modify the test harness (e.g., the SVF simulator); such choice depends on the software under test, if the test cases are executed through a simulator, such choice prevents introducing damaging changes into the SUT (e.g., delay task execution and break strict real-time requirements).}
In Step 4, \APPR generates and compiles mutants. 
Since the \APPR mutation operators may generate mutated data by applying multiple mutation procedures, \APPR may generate several mutants, one for each \UPDATED{mutation operation (i.e., a mutation procedure configured for a data item, according to our terminology)}.
In Step 5, \APPR executes the test suite with all the mutants including a mutant (i.e., the coverage mutant) which does not  modify the data but traces the coverage of the fault model.
In Step 6, \APPR generates mutation analysis results.

%\UPDATED{
%In our context, the \emph{software under test (SUT)} is the CPS embedded software that is verified by a test suite, which is the target of data-driven mutation analysis. Therefore, we refer to the software developed by the engineers as the \emph{original SUT}. An \emph{SUT mutant} (simply, a \emph{mutant}) is a version of the SUT that integrates a \emph{mutation probe} and shall make test cases fail. 
%%A mutation operator simulates one specific interoperability error (a specific type of that automatically alters data by applying one specific \emph{mutation operation}.
%}



\begin{figure}[h]
	\centering
		\includegraphics[width=8.4cm]{damat/images/dataMutationExample}
		\caption{\CHANGED{Data mutation probes integrated into \ESAIL.}}
		\label{fig:appr:mutateProbesInserted}
	\end{figure}

\begin{figure}[h]
	\centering
		\includegraphics[width=8cm]{damat/images/dataDrivenBufferProcess}
		\caption{The \APPR process.}
		\label{fig:appr:approach}
	\end{figure}


\subsubsection{Fault Model Structure}
\label{sec:faultModelStructure}





The \APPR fault model enables the specification of the format of the data exchanged between components along with the type of faults that may affect such data. 
We refer to the data exchanged by two components as \INDEX{message}; also, each CPS component may generate or receive different \INDEX{message types}.
For a single CPS, more than one fault model can be specified. 

The \APPR fault model enables the modelling of data that is exchanged through a specific data structure: the data buffer. 
The \APPR fault model enables engineers to specify (1) the \emph{position} of each data item in the buffer, (2) their \emph{span}, and (3) their \emph{representation type}. Our current implementation supports six data representation types: int, long int, float, double, bin (i.e., data that should be treated in its binary form), hex (i.e., data that should be treated as hexadecimal).
Further, for each data item, \APPR enables engineers to specify one or more data faults using the mutation operator identifiers. For each operator, the engineer 
shall provide values for the required configuration parameters.
Table~\ref{table:operators} provides the list of mutation operators included in \APPR.

	
	% !TEX root = ../MAIN-DataDrivenMutationAnalysis.tex


\begin{table*}[tb]
\caption{Data-driven mutation operators}
\label{table:operators}
\scriptsize
\begin{tabular}{|p{40mm}|p{90mm}|}
\hline
\textbf{Fault Class}&\textbf{Description}\\
\hline
Value above threshold (VAT)&
Replaces the current value with a value above the threshold T for a delta (\D). 
\\
\hline
Value below threshold (VBT)&
Replaces the current value with a value below the threshold T for a delta (\D). 
\\
\hline
Value out of range (VOR)&
Replaces the current value with a value out of the range $[MIN;MAX]$.\\
\hline
Bit flip (BF)&
A number of bits randomly chosen in the positions between MIN and MAX are flipped.
\\
\hline
Invalid numeric value (INV)&
Replace the current value with a mutated value that is legal (i.e., in the specified range) but different than current value. 
\\
\hline
Illegal Value (IV)
&
Replace the current value with a value that is equal to the parameter \emph{VALUE}. 
\\
\hline
Anomalous Signal Amplitude (ASA)
&
The mutated value is derived by amplifying the observed value by a factor \emph{V} and by adding/removing a constant value \D from it. 
\\
\hline
Signal Shift (SS)
&
The mutated value is derived by adding a value \D to the observed value. 
\\
\hline
Hold Value (HV)
&
This operator keeps repeating an observed value for $V$ times. It emulates a constant signal replacing a signal supposed to vary.
\\
\hline
Fix value above threshold (FVAT)&
In the presence of a value above the threshold, it replaces the current value with a value below the threshold T for a delta \D. 
\\
\hline
Fix value below threshold (FVBT)&
It is the counterpart of FVAT for the operator VBT.
\\
\hline
Fix value out of range (FVOR)&
In the presence of a value out of the range  $[MIN;MAX]$ it replaces the current value with a random value within the range. 
\\
\hline
\end{tabular}
\end{table*}%



Inspired by work on \INDEX{abstract mutation analysis}~\cite{Offutt2006}, we have defined three metrics to evaluate test suites with data-driven mutation analysis: \INDEX{fault model coverage}, \INDEX{mutation operation coverage}, and \INDEX{mutation score}. 
%The first two provide information about the quality of test inputs, whereas the latter provides information about the quality of test oracles \CHANGED{and the test process}. Different from code-driven mutation analysis, data-driven mutation analysis thus enables engineers to distinguish between these two distinct problems affecting test suite effectiveness.
These metrics measure the frequency of the following scenarios: (case 1) the message type targeted by a mutant is never exercised, (case 2) the message type is covered by the test suite but it is not possible to perform some of the mutation operations (e.g., because the test suite does not exercise out-of-range cases), (case 3) the mutation is performed but the test suite does not fail.
\CHANGED{Different from code-driven mutation analysis, these three metrics enable engineers to distinguish between possible test suite shortcomings, including untested message types, uncovered input partitions, poor oracle quality, 
%faulty software, 
and lack of test inputs.}

\INDEX{Fault model coverage (FMC)} is the percentage of fault models covered by the test suite. It provides information about the extent to which the message types actually exchanged by the SUT are exercised and verified by the test suites. 
\CHANGED{Low fault model coverage may indicate that only a small portion of the integrated functionalities have been tested.}

\INDEX{Mutation operation coverage (MOC)} is the percentage of data items that have been mutated at least once, considering only those that belong to the data buffers covered by the test suite. It provides information about the input partitions covered for each data item.
The \INDEX{mutation score (MS)} is the percentage of mutants killed by the test suite \UPDATED{(i.e., leading to at least one test case failure)} among the mutants that target a fault model and for which at least one mutation operation was successfully performed. It provides information about the quality of test oracles; indeed, a mutant that performs a mutation operation and is not killed (i.e., is \emph{live}) indicates that the test suite cannot detect the effect of the mutation (e.g., the presence of warnings in logs).
% or an unexpected output from the system). 
\CHANGED{Also, a low mutation score may indicate missing test input sequences. Indeed, live mutants may be due to either software faults (e.g., the SUT does not provide the correct output for the mutated data item instance) or the software not being in the required state (e.g., input partitions for data items are covered when the software is paused); in such cases, with appropriate input sequences, the  test suite would have discovered the fault or brought the SUT into the required state. Both poor oracles and lack of inputs indicate flaws in the test case definition process (e.g., the stateful nature of the software was ignored).}


%% !TEX root = MAIN.tex



\clearpage
\section{Data-driven Mutation Testing: DAMTE} % (fold)
\label{sec:data:test_suite_augmentation}

\STARTCHANGEDWPT

\subsection{Overview}

In this section we describe a methodology (i.e., data-driven mutation testing, \INDEX{DAMTE}) that specifies how to rely on KLEE to generate test inputs that increase the fault model coverage and the mutation operation coverage.

The \INDEX{test suite augmentation process} concerns the definition of additional test cases to increase the mutation score.
It consists of four activities \INDEX{Identify Test Inputs}, \INDEX{Generate Test Oracles}, \INDEX{Execute the SUT}, \INDEX{Fix the SUT}. 
Despite these activities match the ones performed in the case of code-driven mutation testing, they are triggered and implemented in a different manner, as described below.

In the presence of mutants not killed by test cases (i.e., when the  \INDEX{mutation score} is not equal to 100\%), engineers are expected to manually investigate the underlying problems. Indeed, as reported in Section~\ref{sec:mutationAnalysisResults}, two might be the reasons for a low MS: poor oracle quality and missing test input sequences (i.e., the software does not reach the state in which it could kill the mutant).
For the first case (poor oracle quality), manual work is needed because automated approaches to automatically generate test oracles in the presence of system or integration test suites are not available. For the second case, existing test generation approaches (e.g., KLEE) might suffer from scalability problem that prevent bringing the system into a desired state ; also, they cannot deal with systems whose components communicate through channels. For this reason, generating test oracles and fixing the SUT (in case a fault is discovered after test suite augmentation) shall be performed manually.

When mutation operators are not applied because of the lack of appropriate data to mutate (i.e., in the presence of fault model coverage and mutation operation coverage below 100\%), engineers are expected to generate new test inputs for the SUT that enable the application of all the mutation operators. 
However, the methodology to adopt may vary based on the test objective and the system architecture. 
We discuss the case of the producer-consumer and client-server architecture, two common software architectures. We leave the discussion of other architectures (e.g., broker architecture and event-bus architecture) to future work.

In Figures~\ref{fig:dataDrivenTestSuiteAugmentationC} to~\ref{fig:dataDrivenTestSuiteAugmentationE}, we exemplify the two architectures. In both the two cases, data-driven mutation may concern the generated data and occur either on the component that generates the data (Figure~\ref{fig:dataDrivenTestSuiteAugmentationC}), or on the component that receives the data (Figure~\ref{fig:dataDrivenTestSuiteAugmentationD}).
For the client-server case, instead, data mutation may concern also the request for data and be performed either on the client or the server (Figure~\ref{fig:dataDrivenTestSuiteAugmentationE}). For the producer-consumer case, static program analysis may be employed to automatically generate the missing data; to this end, we aim to rely on an \INDEX{extended data mutation probe}. For the client-server case, the \INDEX{extended data mutation probe} may still be used but only to generate message requests; therefore, it would be useful only when data-driven analysis is performed on the  request message. Since the steps required to perform test generation is the same in both the two cases, we provide an example based on the client-server case.

\begin{figure}[h]
  \centering
    \includegraphics[width=14cm]{images/dataDrivenTestSuiteAugmentationC}
      \caption{Data-driven mutation analysis for different architectures.}
      \label{fig:dataDrivenTestSuiteAugmentationC}
\end{figure}

\begin{figure}[h]
  \centering
    \includegraphics[width=14cm]{images/dataDrivenTestSuiteAugmentationD}
      \caption{Data-driven mutation analysis for different architectures.}
      \label{fig:dataDrivenTestSuiteAugmentationD}
\end{figure}

\begin{figure}[h]
  \centering
    \includegraphics[width=14cm]{images/dataDrivenTestSuiteAugmentationE}
      \caption{Data-driven mutation analysis for different architectures.}
      \label{fig:dataDrivenTestSuiteAugmentationE}
\end{figure}

\begin{figure}[h]
  \centering
    \includegraphics[width=14cm]{images/dataDrivenTestSuiteAugmentationB}
      \caption{Data-driven mutation analysis for different architectures.}
      \label{fig:dataDrivenTestSuiteAugmentationB}
\end{figure}

\ENDCHANGEDWPT



\clearpage
\subsection{Test generation with a client-server system}

\STARTCHANGEDWPT

For our example, we rely on the libParam case study provided by GSL. Listing~\ref{GSLmutate} shows the mutation probe, which is inserted into function \emph{gs\_rparam\_process\_packet}, on the server side. The probe mutates the buffer \emph{v\_General}, which contains a copy of a message request (i.e., \emph{request}). In the case of GSL, the FVAT operator configured to mutate \emph{request-$\>$table\_id} cannot be applied (i.e., MOC is not equal to 100\%); this indicates that the test cases do not cover a scenario in which the client passes a \emph{table\_id} above the threshold. To generate such a test case we may rely on the extended probe combined with \INDEX{static program analysis}. 

Listing~\ref{GSLcover} shows how the \INDEX{extended mutation probe} might be inserted into the code of libParam. In practice, it requires the engineer to know the portion of code that handles the generation of a request message. Unfortunately, injecting the mutation probe is not sufficient to enable test generation but engineers need also to prepare a test template to enable test generation with KLEE. Listing~\ref{GSLtest} shows an example of such template based on existing libParam test cases; such test case requires the initialization of a number of state variables, which limits the possibility to automate its definition. For this reason, within FAQAS we did not find it feasible to automate data-driven mutation analysis with a tool but we aim to evaluate its manual feasibility in WP4.

Finally, when data-driven mutation is applied to the data generated by the server, test automation is made unfeasible by the fact that KLEE cannot work in the presence of a communication channel within the code to be analyzed. Such shortcoming is not observed when we mutate request data because the extended mutation probe is installed only on the client; the producer-consumer case is not affected by such shortcoming because, in this case, the probe is installed on the producer. Alternative test generation tools or extensions of KLEE shall be considered to overcome such limitations.

% !TEX root =  ../MAIN.tex
\begin{lstlisting}[style=CStyle, caption=Example of data-driven mutation probe for libParam, label=GSLmutate]

static void gs_rparam_process_packet(csp_conn_t * conn, csp_packet_t * request_packet)
{
    csp_packet_t * reply_packet = NULL;
    gs_rparam_query_t * reply;


    /* Handle endian */
    gs_rparam_query_t * request = (gs_rparam_query_t *) request_packet->data;


    request->length = csp_ntoh16(request->length);
    request->checksum = csp_ntoh16(request->checksum);


    FaultModel *fm_General = _FAQAS_General_FM();
    unsigned long long int v_General[6];

    v_General[0] = (unsigned long long int) request->action;
    v_General[1] = (unsigned long long int) request->table_id;
    v_General[2] = (unsigned long long int) request->length;
    v_General[3] = (unsigned long long int) request->checksum;
    v_General[4] = (unsigned long long int) request->seq;
    v_General[5] = (unsigned long long int) request->total;


    _FAQAS_mutate(v_General,fm_General);
    
\end{lstlisting}



%flag E_Decode(E* pVal, BitStream* pBitStrm, int* pErrCode)
%{
%    flag ret = TRUE;
%    *pErrCode = 0;
%    (void)pVal;
%    (void)pBitStrm;
%
%
%    (*(pVal))=5; ret = TRUE; *pErrCode = 0;
%
%    // Manually added probe 
%    E_mutate(pVal);
%    // Manually added probe END
%    return ret  && E_IsConstraintValid(pVal, pErrCode);
%}


% !TEX root =  ../MAIN.tex
\begin{lstlisting}[style=CStyle, caption=Example of extended data-driven mutation probe for libParam, label=GSLcover]

/**
   Get string.
   @note If the returned string is max length, the value buffer will not be 0 terminated.
   @param[in] node CSP address
   @param[in] table_id remote table id.
   @param[in] addr parameter address (remote table).
   @param[in] checksum checksum
   @param[in] timeout_ms timeout
   @param[out] value returned value (user allocated)
   @param[in] value_size size of \a value, i.e. size of parameter type in bytes.
   @return_gs_error_t
*/
static inline gs_error_t gs_rparam_get_string(uint8_t node, gs_param_table_id_t table_id, uint16_t addr,
                                              uint16_t checksum, uint32_t timeout_ms, char * value, size_t value_size)
{
    return gs_rparam_get(node, table_id, addr, GS_PARAM_STRING, checksum, timeout_ms, value, value_size);
}


gs_error_t gs_rparam_get(uint8_t node,
                         gs_param_table_id_t table_id,
                         uint16_t addr,
                         gs_param_type_t type,
                         uint16_t checksum,
                         uint32_t timeout_ms,
                         void * value,
                         size_t value_element_size)
{
    return gs_rparam_get_array(node, table_id, addr, type, checksum, timeout_ms, value, value_element_size, 1);
}


gs_error_t gs_rparam_get_array(uint8_t node,
                               gs_param_table_id_t table_id,
                               uint16_t addr,
                               gs_param_type_t type,
                               uint16_t checksum,
                               uint32_t timeout_ms,
                               void * value,
                               size_t value_element_size,
                               size_t array_size)
{
    /* Calculate length */
    gs_rparam_query_t * query;
    const size_t query_payload_size = sizeof(query->payload.addr[0]) * array_size;
    const size_t query_size = RPARAM_QUERY_LENGTH(query, query_payload_size);
    const size_t reply_payload_element_size = value_element_size + sizeof(query->payload.addr[0]);
    const size_t reply_payload_size = reply_payload_element_size * array_size;
    const size_t reply_size = RPARAM_QUERY_LENGTH(query, reply_payload_size);

    query = alloca(reply_size);
    query->action = RPARAM_GET;
    query->table_id = table_id;
    query->checksum = csp_hton16(checksum);
    query->seq = 0;
    query->total = 0;
    for(unsigned int i = 0; i < array_size; i++) {
        query->payload.addr[i] = csp_hton16(addr + (value_element_size * i));
    }
    query->length = csp_hton16(query_payload_size);

    FaultModel *fm_General = _FAQAS_General_FM();
    unsigned long long int v_General[6];

    v_General[0] = (unsigned long long int) query->action;
    v_General[1] = (unsigned long long int) query->table_id;
    v_General[2] = (unsigned long long int) query->length;
    v_General[3] = (unsigned long long int) query->checksum;
    v_General[4] = (unsigned long long int) query->seq;
    v_General[5] = (unsigned long long int) query->total;


    _FAQAS_cover(v_General,fm_General);


    /* Run single packet transaction */
    if (csp_transaction2(CSP_PRIO_HIGH, node, GS_CSP_PORT_RPARAM, timeout_ms, query, query_size, query, reply_size, CSP_O_CRC32) <= 0) {
        return GS_ERROR_IO;
    }
 ... 
 
 }

\end{lstlisting}



%flag E_Decode(E* pVal, BitStream* pBitStrm, int* pErrCode)
%{
%    flag ret = TRUE;
%    *pErrCode = 0;
%    (void)pVal;
%    (void)pBitStrm;
%
%
%    (*(pVal))=5; ret = TRUE; *pErrCode = 0;
%
%    // Manually added probe 
%    E_mutate(pVal);
%    // Manually added probe END
%    return ret  && E_IsConstraintValid(pVal, pErrCode);
%}


% !TEX root =  ../MAIN.tex
\begin{lstlisting}[style=CStyle, caption=Test template to enable data-driven mutation testing for libParam, label=GSLtest]

    // a little hack - this is next element, we use it check for overwrite and missing 0 termiation
    memset(alltypes_mem.string_A, 'Z', sizeof(alltypes_mem.string_A));
    alltypes_mem.string_A[0][1] = 0;

    char buf[GS_TEST_ALLTYPES_STRING_LENGTH + 10];

    // get max size - no 0 termination
    memset(alltypes_mem.string, 'B', sizeof(alltypes_mem.string));
    memset(buf, 'A', sizeof(buf));
    buf[GS_TEST_ALLTYPES_STRING_LENGTH + 1] = 0;
    
    csp_node CSP_NODE;
    unsigned long long int tableID;
    klee_make_symbolic(&CSP_NODE, sizeof(CSP_NODE), ”CSP_NODE”);
    klee_make_symbolic(&tableID, sizeof(tableID), ”tableID”);
    gs_rparam_get_string(&CSP_NODE, tableID, GS_TEST_ALLTYPES_STRING, GS_RPARAM_MAGIC_CHECKSUM, 1000, buf, GS_TEST_ALLTYPES_STRING_LENGTH);

    
\end{lstlisting}



%flag E_Decode(E* pVal, BitStream* pBitStrm, int* pErrCode)
%{
%    flag ret = TRUE;
%    *pErrCode = 0;
%    (void)pVal;
%    (void)pBitStrm;
%
%
%    (*(pVal))=5; ret = TRUE; *pErrCode = 0;
%
%    // Manually added probe 
%    E_mutate(pVal);
%    // Manually added probe END
%    return ret  && E_IsConstraintValid(pVal, pErrCode);
%}


\ENDCHANGEDWPT





%% !TEX root = MAIN.tex

\section{FAQAS case studies}
\label{chapter:caseStudies}

The FAQAS toolset has been applied to \MREVISION{C-P-40}{six} case study systems. Table~\ref{tab:caseStudies} provides the list of case studies along with an indication of the type of mutation analysis/testing (i.e., code-driven or data-driven) and the tools they are targeted for. 

\STARTCHANGEDWPT
For the validation of each tool, we selected case studies presenting characteristics compatible with the requirements of the tool under test. \emph{Code-driven mutation analysis} (implemented by MASS) does not present any specific requirement except the availability of source code; for this reason, it has been validated with all the available case studies systems.
\emph{Code-driven mutation testing} (implemented by SEMuS), instead, requires the software under test to be comprised of components communicating through function calls (e.g., not network channels); for this reason, for SEMuS, we selected unit test suites.
\emph{Data-driven mutation analysis} (implemented by DAMAt) targets components communicating through channels, which are usually tested with integration and system test suites. For this reason we selected systems tested with such types of test suites.
\emph{Data-driven mutation testing} (implemented by DAMTE) aims to generate test cases that improve data-driven mutation analysis; however, since it is performed with symbolic execution tools, it presents the same limitations of SEMuS, that is, the analyzed part of the software under test should be comprised of components communicating through function calls. For this reason, it can target only libGCSP and libParam.

The following sections describe each case study subject. 

\begin{table}[htp]
\caption{Case studies for the FAQAS activity.}
\label{tab:caseStudies}
\begin{center}
\begin{tabular}{|p{1.2cm}|p{6cm}|p{1.5cm}|p{1.5cm}|p{1.5cm}|p{1.5cm}|}
\hline
\textbf{}&\textbf{}&\multicolumn{2}{c}{\textbf{Code-driven}}&\multicolumn{2}{c|}{\textbf{Data-driven}}\\
\textbf{Partner}&\textbf{Case study}&\textbf{MASS}&\textbf{SEMuS}&\textbf{DAMAt}&\textbf{DaMTe}\\
\hline
LXS&System Test Suite for ESAIL&Y&N&Y&N\\
LXS&Unit Test Suite for ESAIL&Y&N&N&N\\
GSL&Unit Test Suite for libUtil&Y&Y*&N&N\\
GSL&Integration Test Suite for libgscsp&Y&N&Y*&N\\
GSL&System Test Suite for libparam&Y&N&Y&Y*\\
ESA&MLFS mathematical library&Y&Y&N&N\\
ESA&ASN1 Compiler&Y&Y&N&N\\
\hline
\end{tabular}
\end{center}
An asterisk (*) is used to indicate validation cases that will be finalized in WP4.
\end{table}

\ENDCHANGEDWPT

%\clearpage

%% !TEX root = MAIN.tex

\section{LXS - ESAIL System Test Suite}
\label{chapter:caseStudies:LXS}

\subsection{Overview of the case study}

ESAIL is a microsatellite developed by LXS in a PPP with ESA and ExactEarth. 
The Payload is an AIS Receiver for ship- and vessel-detection from space, and the satellite weight at launch will be approximately 115kg. The satellite payload also enables advanced raw data handling and RF-Spectrum sampling for Ground processing.

\begin{figure}[h]
	\centering
    \includegraphics[width=0.7\textwidth]{images/esail}
    \caption{ESAIL system testing environment.}
    \label{fig:esail_case_study}
\end{figure}
 
The SVF simulator has been used for functional validation of the ESAIL CSW (see Figure~\ref{fig:esail_case_study}). The SVF is indeed one of the main testing tools used in satellite projects. The SVF simulator can be seen as a testing facility that presents also its own dedicated test suite to ensure the correctness of the SVF models and assembly to avoid later misunderstanding of the expected behaviours of the satellite CSW. 
In the context of the FAQAS project we consider the system test suite for the validation of the CSW, which are implemented using the SVF as the driving tool.
%Therefore, the SVF Simulator enables the evaluation of the FAQAS framework against two test suites: (1) the Test Suite of the SVF Simulator that validates the Simulator itself, (2) the system tests for the validation of CSW, which are implemented using the SVF as the driving tool.

Details about ESAIL are provided in the document \emph{FAQAS-LXS-MAN-001\_1- SVF Software Installation and User Manual} uploaded on Alfresco.

ESAIL is the largest case study system in FAQAS, the software consists of 924 source files with a total size of 187\,116 LOC. The system test suite consists of 121 python test scripts with a total of 384 sub-test cases. The system test suite takes up to 10 hours to finish its execution.

\subsection{Code-driven mutation testing}

The code-driven mutation testing process in ESAIL will target all the components of the ESAIL on-board software, these components are:

\begin{itemize}
	\item ADCS
	\item CAN
	\item EPS
	\item FDIR
	\item OPSE
	\item SERVICES
	\item TCS
	\item TMTC
\end{itemize}

\subsection{Data-driven mutation testing}

A detailed description of the application of data-driven mutation testing to ESAIL is provided in APPENDIX~\ref{appendix:esailFM}.


%
%% !TEX root = MAIN.tex

\clearpage

\section{GSL - libgcsp}
\label{sec:caseStudies:GSL:libgcsp}

\subsection{Overview of the case study}

The GomSpace CSP library (libgscsp) is a GomSpace extension to the open source CubeSat Space Protocol library.
The GomSpace CSP library provides:
\begin{itemize}
\item convenience wrapping of CSP functionality, primarily initialization.
\item definition of standard CSP ports (used by other GomSpace products).
\item connecting low-level drivers (e.g. CAN, I2C from Embed library) with CSP interfaces 
\item generic CSP service dispatcher, forwards incoming connections to service handlers.
\end{itemize}

The libgscsp contains a GomSpace branch (https://github.com/GomSpace/libcsp) of the open source libcsp (https://github.com/libcsp/libcsp), located in the subfolder lib/libcsp, and an extension of the CSP library. The two libcsp branches are kept as identical as possible, as features specific to GomSpace are placed in libgscsp. The extension of the CSP library provides utility functions for GSL-specific products.

Details about libgscsp are provided in the document \emph{gs-man-nanosoft-ms100-command-and-management-sdk-3.6.2-1-g67fe6e1.pdf} uploaded on Alfresco.


The size of libgscsp is 1\,497 LOC, %include 306, src 1776+15 = total = 1497
while libcsp (GSL branch) is 8\,339 LOC. % 6789 + 1550

\MREVISION{C-P-45}{According to the definition described in ECSS-Q-ST-80C~\cite{ecss80C}, we consider this test suite an integration test suite. Indeed, every test case verifies the integration of multiple components. If we assume that every component is implemented in a separate file, libgscsp test suite matches this definition. For example, the test case \texttt{TEST\_csp\_port\_bind} test capabilities from components csp\_conn (e.g., csp\_close) and csp\_port (e.g., csp\_port\_get\_socket)} 

\MREVISION{C-P-44}{The libgscsp integration test suite consists of 89 test cases, the test infrastructure is based on the \emph{Google C++ Testing Framework}~\cite{googletest}. The validation environment is compiled and executed through the WAF meta-build system~\cite{waf}.
To perform the experiments in the \INDEX{UL HPC}, we configured an Ubuntu 16.04 Singularity container to be executed with two processors and 8 GB of memory.}


\subsection{Code-driven mutation testing}



In FAQAS, under the assumption that test suites are of high quality standards, we mutate only the statements covered by the test suite. This is also due to the fact that a mutant can be killed only if it is covered by at least one test case. 

In the case of libgscsp, the integration test suite covers basic algorithms such as error handling and routing while hardware specific functions such as the handling of CAN protocol is verified by means of a test suite for hardware in the loop. 

GomSpace relies on a risk-driven approach to integration testing; for this reason GSL code does not reach 100\% code coverage in the integration tests. The code that has not been covered is judged to be low risk, but is covered by other techniques like code inspection and manual testing. The integration tests run are covering the normal use cases for our satellites. libCSP is not specific to GomSpace, although GSL uses it heavily, and not all features are used in GomSpace satellites.

%For this reason, the code-driven mutation testing process in libgscsp will target all the components covered by the libgscsp unit test suite.

% !TEX root = ../MAIN.tex

\begin{table}[h]

\footnotesize
\parbox{.45\linewidth}{
\centering
\begin{tabular}{|l|l|}
\hline
\textbf{Coverage Type} & \textbf{Coverage Rate} \\
\hline
Statement     & 58.4\% (390 of 668 statements)\\
Functions     & 71.4\% (50 of 70 functions)\\
Branches      & 41.2\% (165 of 400 branches)\\
\hline
\end{tabular}
\caption{libgscsp code coverage.}
\label{table:libgscsp_coverage}
}
\hfill
\parbox{.45\linewidth}{
\centering
\begin{tabular}{|l|l|}
\hline
\textbf{Coverage Type} & \textbf{Coverage Rate} \\
\hline
Statement     & 64.1\% (2\,112 of 3\,297 statements)\\
Functions     & 72.5\% (248 of 342 functions)\\
Branches      & 44.9\% (989 of 2\,201 branches)\\
\hline
\end{tabular}
\caption{libcsp code coverage.}
\label{table:libcsp_coverage}
}
\end{table}	

\begin{enumerate}
	\item \textbf{libgscsp extension}

	Table~\ref{table:libgscsp_coverage} provides code coverage information of the libgscsp integration test suite for the GSL extension to the CSP library. Below, we report the files covered by the test suite and targeted by mutation testing; they correspond to 56\% of the source files. %(10/18)
	%we focus our analysis to the following subset of components (i.e., components with code coverage greater than 0\%):
	
	\begin{itemize}
	 	\item src/clock.c
	 	\item src/commands.c
	 	\item src/conn.c
	 	\item src/csp.c
	 	\item src/error.c
	 	\item src/log.c
	 	\item src/router.c
	 	\item src/service\_dispatcher.c
	 	\item src/service\_handler.c
	 	\item src/linux/command\_line.c

	 \end{itemize} 

	\item \textbf{libcsp (GSL branch)}

	Table~\ref{table:libcsp_coverage} presents the code coverage of the libgscsp integration test suite for the GSL branch of the CSP library. 
	% Given the code coverage, we focus our mutation analysis on the following subset of components (i.e., components with code coverage greater than 0\%):
Below, we report the files covered by the test suite and targeted by mutation testing; they correspond to 45\% of the source files. %(32/71)

	\begin{itemize}
		\item src/arch/csp\_time.c
		\item src/arch/csp\_system.c
		\item src/arch/posix/csp\_thread.c
		\item src/arch/posix/csp\_semaphore.c
		\item src/arch/posix/csp\_malloc.c
		\item src/arch/posix/csp\_queue.c
		\item src/arch/posix/csp\_time.c
		\item src/arch/posix/pthread\_queue.c
		\item src/arch/posix/csp\_system.c
		\item src/crypto/csp\_sha1.c
		\item src/crypto/csp\_hmac.c
		\item src/crypto/csp\_xtea.c
		\item src/interfaces/csp\_if\_lo.c
		\item src/rtable/csp\_rtable.c
		\item src/rtable/csp\_rtable\_cidr.c
		\item src/rtable/csp\_rtable\_static.c
		\item src/transport/csp\_rdp.c
		\item src/transport/csp\_udp.c
		\item src/csp\_sfp.c
		\item src/csp\_debug.c
		\item src/csp\_service\_handler.c
		\item src/csp\_crc32.c
		\item src/csp\_io.c
		\item src/csp\_qfifo.c
		\item src/csp\_iflist.c
		\item src/csp\_endian.c
		\item src/csp\_route.c
		\item src/csp\_buffer.c
		\item src/csp\_port.c
		\item src/csp\_conn.c
		\item src/csp\_init.c
		\item src/csp\_services.c
	\end{itemize}


\end{enumerate}

\subsubsection{Mutation Testing Preliminary Results}


\input{tables/libgscsp_preliminary}

In order to evaluate the feasibility of code-driven mutation testing for libgscsp, we conducted a preliminary experiment using the mutation operators AOR, ROR, ICR, LCR, ABS, UOI and SDL we generated 6\,196 mutants. For the experimentation we targeted only the libcsp (GSL branch) source code. Preliminary results can be found in Table~\ref{table:libgscsp_preliminary}.
Particularly, we observe that from the 6\,196 generated mutants, we had 1\,700 mutants that were not compiled by the compilation toolset of libgscsp, most probably because the mutation introduced a syntactical error that was detected by the toolset.
Then, we identified 1\,708 live mutants that were not detected by the test suite. Instead, we had 2\,495 mutants detected by the test suite, and 277 mutants killed by timeout (they led to infinite loop). The final mutation score was of 61.88\%.

The identification of equivalent mutants still needs to be assessed.


\subsection{Data-driven mutation testing}

\begin{figure}[h]
  \centering
    \includegraphics[width=0.9\textwidth]{images/csp_packet}
      \caption{CSP protocol header.}
      \label{fig:csp_packet}
\end{figure}

The data-driven mutation testing process in libgscsp will target the data packet transferred between the server and the client using the CSP protocol. Specifically, the mutations will affect the header of the packet, which contains routing information (see Figure~\ref{fig:csp_packet}) and the payload 
%the mutations will affect the content itself of the packet 
(i.e., the data being transferred).

%\DONE{Does the header contains any infor about the payload? Because the only mutation we can perform on the payload is to cut it.}

%\DONE{No, it seems not. The six fields in the packet header regards: pri (Priority), src (Source address), dst (Destination address)
%dport (Destination port), sport (Source port) and flags (specifies specific byte instructions such as use of fragmentation, HMAC verification, etc)}


\begin{figure}[h]
  \centering
    \includegraphics[width=0.9\textwidth]{images/FaultModelCSP_temp}
      \caption{CSP fault model.}
      \label{fig:csp_faultmodel}
\end{figure}

Figure~\ref{fig:csp_faultmodel} introduces an example of the possible fault models to be applied to the  header of the packet.
% and \emph{Payload} targets the data of the packet.
The \emph{Header} fault model include mutation operators such as bit-flips (BF), insertion of invalid values (INV), insertion of values out of range (VOR), and insertion of values above threshold (VAT).
%Instead, the \emph{Payload} fault model consists of applying a bit-flip to the data packet; it is used to verify that the test suite has at least one test case that controls the correct transmission of the content of the packet.

Concerning the flags data item, it has a size of 8 bits, i.e., 8 flags are available. Only 4 flags are currently used:
\begin{itemize}
\item 00000001              CRC32
\item 00000010             RDP
\item 00000100             XTEA
\item 00001000             HMAC
\end{itemize}
We force each flag to be set to zero by defining a different BF mutation operator with \emph{State=1} for each of the four bits.

% !TEX root =  ../MAIN.tex

\begin{minipage}{16cm}
\begin{lstlisting}[style=CStyle, caption=Data-driven mutation example on libgscsp (csp\_io.c excerpt)., label=csp_integration]
unsigned int v[6] = {packet->id.flags, packet->id.sport, packet->id.dport, 
                    packet->id.dst, packet->id.src, packet->id.pri};

FaultModel *fm = _FAQAS_Identifier_FM();                                                                                          
mutate(v, fm);
_FAQAS_delete_FM(fm);

packet->id.flags = v[0];
packet->id.sport = v[1];
packet->id.dport = v[2]; 
packet->id.dst = v[3]; 
packet->id.src = v[4]; 
packet->id.pri = v[5]; 
\end{lstlisting}
\end{minipage}

Both mutations (i.e, header and payload) will be performed before the message is serialized. 
For example, the mutations to the header will affect the \emph{csp\_send\_direct} function of the \emph{csp\_io component}. 
Listing~\ref{csp_integration} shows an example of our data-driven mutator prototype within the \emph{csp\_io component}. 
Particularly, in Line 1 the packet header is translated into an unsigned int vector. 
Then, in Line 4 an instance of the packet identifier fault model is created, 
the instance of the fault model and the unsigned int vector is then passed to the FAQAS mutate function in line 5.
%In this case, the fault model can target all six items of the header (i.e., Flags, Source Port, Destination Port, Destination Address, Source Address, and Priority). 
After the mutation has been applied, from Line 8 to Line 13, the vector is then translated back to the original representation.
 


\section{GSL - libparam}
\label{sec:caseStudies:GSL:libparam}

\subsection{Overview of the case study}

The Parameter System (i.e., libparam) is a light-weight parameter system designed for GomSpace satellite subsystems. It is based around a logical memory architecture, where every parameter is referenced directly by its logical address. A backend system takes care of translating addresses into physical addresses.
The features of this system include:
\begin{itemize}
\item Direct memory access for quick parameter reads.
\item Simple data types: uint, int, float, double, string.
\item Arrays of simple data types.
\item Supports multiple stores per table, e.g. FRAM, MCU flash, file (binary or text).
\item Remote client with support for most features (rparam).
\item Packed GET, SET queries, supporting multiple parameter set/get in a single request. Data serialization and deserialization.
\item Supports both little and big-endian systems.
\item Commands for both local (param) and remote access (rparam).
\item Parameter server for remote access over CSP.
\item Compile-time configuration of parameter system
\end{itemize}

Details about libparam are provided in the document \emph{gs-man-nanosoft-ms100-command-and-management-sdk-3.6.2-1-g67fe6e1.pdf} uploaded on Alfresco.

Details about code coverage, mutation  score, and fault model for data-driven mutation testing will be provided at the end of WP2.

%\subsection{Code-driven mutation testing}
%
%\TODO{Here we will simply add code coverage information for libparam}
%
%\subsection{Data-driven mutation testing}
%
%\TODO{To populate this section we need the following from GSL: Action 3. (deadline: end of this week is better) GSL should provide a description of the structure of the data exchanged by libparam along with a name of the function that loads the data received from the libcsp layer. GSL suggested looking for documentation inside the libparam folder, but we did not find such doc folder, actually, there is only one doc folder and it concerns the documentation for libcsp not libparam.}


\section{GSL - libutil}
\label{sec:caseStudies:GSL:libutil}

\subsection{Overview of the case study}

The Utility library provides cross-platform APIs for common functionality, for use in both embedded systems and standard PCs running Linux. 

Details about libutil are provided in the document \emph{gs-man-nanosoft-ms100-command-and-management-sdk-3.6.2-1-g67fe6e1.pdf} uploaded on Alfresco.

The size of libutil is 10\,576 LOC, while the unit test suite consists of 201 test cases written in C.

\subsection{Code-driven mutation testing}

The code-driven mutation testing process in libutil will target all the components covered by the libutil unit test suite. 
The libutil unit test suite do not cover hardware-specific functions (e.g., drivers), which are covered at the system level.

% !TEX root = ../MAIN.tex

\begin{table}[h]

\centering
\begin{tabular}{|l|l|}
\hline
\textbf{Coverage Type} & \textbf{Coverage Rate} \\
\hline
Statement     & 83.2\% (8\,817 of 10\,596 statements)\\
Functions     & 82.1\% (725 of 883 functions)\\
Branches      & 56.6\% (2\,618 of 4\,627 branches)\\
\hline
\end{tabular}
\caption{libutil code coverage.}
\label{table:gslibutil_coverage}

\end{table}

Table~\ref{table:gslibutil_coverage} provides the code coverage information of the unit test suite for the GSL libutil library. 
%Given the code coverage, we focus our analysis to the following subset of components (i.e., components with code coverage greater than 0\%):
Below, we report the files covered by the test suite and targeted by mutation testing; they correspond to 82\% of the source files. %(55/67)

\begin{itemize}
	\item src/base16.c
	\item src/bytebuffer.c
	\item src/byteorder.c
	\item src/clock.c
	\item src/crc32.c
	\item src/crc8.c
	\item src/error.c
	\item src/fletcher.c
	\item src/function\_scheduler.c
	\item src/hexdump.c
	\item src/lock.c
	\item src/rtc.c
	\item src/string.c
	\item src/strtoint.c
	\item src/time.c
	\item src/timestamp.c
	\item src/cmd/command.c
	\item src/cmd/log.c
	\item src/cmd/vmem.c
	\item src/drivers/sys/memory.c
	\item src/gosh/command.c
	\item src/gosh/console.c
	\item src/gosh/default\_commands.c
	\item src/linux/clock.c
	\item src/linux/command\_line.c
	\item src/linux/cwd.c
	\item src/linux/delay.c
	\item src/linux/function.c
	\item src/linux/mutex.c
	\item src/linux/queue.c
	\item src/linux/rtc.c
	\item src/linux/sem.c
	\item src/linux/signal.c
	\item src/linux/stdio.c
	\item src/linux/thread.c
	\item src/linux/time.c
	\item src/linux/drivers/gpio/gpio.c
	\item src/linux/drivers/gpio/gpio\_sysfs.c
	\item src/linux/drivers/gpio/gpio\_virtual.c
	\item src/linux/drivers/i2c/i2c.c
	\item src/linux/drivers/spi/spi.c
	\item src/linux/drivers/sys/memory.c
	\item src/log/commands.c
	\item src/log/log.c
	\item src/log/appender/console.c
	\item src/log/appender/simple\_file.c
	\item src/test/cmocka.c
	\item src/test/command.c
	\item src/test/log.c
	\item src/vmem/commands.c
	\item src/vmem/vmem.c
	\item src/watchdog/monitor\_task.c
	\item src/watchdog/watchdog.c
	\item src/zip/zip.c
	\item src/zip/miniz/miniz.c
\end{itemize}

\subsubsection{Mutation Testing Preliminary Results}

% !TEX root = ../MAIN.tex

\begin{table}[h]
\centering
\caption{Code-driven mutation testing preliminary results for the libutil case study.}
\label{table:libutil_preliminary}
\begin{tabular}{|l|l|l|l|l|l|l|}
\hline
        & \multicolumn{5}{c|}{Mutants}                                                                      & \multirow{3}{*}{\begin{tabular}[c]{@{}l@{}}Mutation Score\\ (K/K+L)\end{tabular}} \\ \cline{1-6}
        &     &                                                        &      & \multicolumn{2}{c|}{Killed} &                                                                                   \\ \cline{1-6}
Mutants & All & \begin{tabular}[c]{@{}l@{}}Not\\ Compiled\end{tabular} & Live & Test Failure    & Timeout   &                                                                                   \\ \hline
Total   &  16\,886   &  2\,561                                                      & 3\,402      & 10\,634                & 289          & 76.25\%                                                                           \\ \hline
\end{tabular}
\end{table}    
             

In order to analyze the feasibility of the code-driven mutation testing, we conducted a preliminary experimentation using the mutation operators AOR, ROR, ICR, LCR, and SDL we generated 16\,886 mutants. Preliminary results can be found in Table~\ref{table:libutil_preliminary}.
Particularly, we observe that from the 16\,886 generated mutants, we had 2\,561 mutants that were not compiled by the compilation toolset of libutil, most probably because the mutation introduced a syntactical error that was detected by the toolset.
Then, we identified 3\,402 live mutants that were not detected by the test suite. Instead, we had 10\,634 killed mutants detected by the test suite, and 289 mutants that produced libutil to go into an infinite loop, and thus were killed by timeout. The final mutation score was of 76.25\%.

The identification of equivalent mutants still needs to be performed.


\subsection{Data-driven mutation testing}

We do not plan to apply data drive mutation testing to this case study because is a standalone library; it does not integrate communicating components.

%
%
%% !TEX root = MAIN.tex
\clearpage

\section{MLFS}
\label{sec:caseStudies:GSL:MLSF}

\subsection{Overview of the case study}

The \INDEX{Mathematical Library for Flight Software} (MLFS) implements mathematical functions ready for qualification. 
%\DONE{Please rewrite the following sentence}
MLFS is born from the need of having a mathematical library ready for qualification for flight software. Well known mathematical libraries such as \texttt{libm}~\cite{libm} and \texttt{newlib}~\cite{newlib} are not completely validated with respect to specific input ranges, errors and performance, and so, they do not comply with ECSS criticality category B.
The set of functions provided by MLFS are limited to the functions typically needed in flight software. 

%\DONE{What is doc?}
Detailed information about the MLFS library and the test suite is provided in the following documents shared on Alfresco.

\begin{itemize}
	\item \emph{E1356-GTD-SUM-01\_I1\_R2.pdf}, installation and execution instructions of the Mathematical Library for Flight Software. 
	\item \emph{E1356-CS-SUM-01\_I1\_R5.pdf}, installation and execution instructions of the MLFS test suite.
\end{itemize}

The source code size is 5\,402 LOC, while the unit test suite consists of 4\,042 tests for 92 functions.

\subsection{Code-driven mutation analysis}

% !TEX root = ../MAIN.tex

\begin{table}[h]

\centering
\begin{tabular}{|l|l|}
\hline
\textbf{Coverage Type} & \textbf{Coverage Rate} \\
\hline
Statement     & 100\% (1\,978 of 1\,978 statements)\\
Functions     & 100\% (90 of 90 functions)\\
Branches      & 100\% (1\,302 of 1\,302 branches)\\
\hline
\end{tabular}
\caption{MLFS code coverage.}
\label{table:mlfs_coverage}

\end{table}

Table~\ref{table:mlfs_coverage} provides the code coverage information of the MLFS unit test suite. The test suite achieves 100\% code coverage (i.e., statement, function and branch coverage).

Given the code coverage, we focus our analysis on all components of the MLFS:

\begin{itemize}
	\item MLFS: implementation of 4 mathematical functions.
	\item Machine: dedicated implementation of 2 mathematical functions for the sparc v8 architecture.
	\item Math: implementation of 66 mathematical functions.
	\item Common: implementation of 20 mathematical functions.
\end{itemize}

%\subsubsection{Mutation Analysis Preliminary Results}
%
%% !TEX root = ../MAIN.tex

\begin{table}[h]
\small
\centering
\caption{Code-driven mutation testing preliminary results for the MLFS case study.}
\label{table:mlfs_preliminary}
\begin{tabular}{|l|l|l|l|l|l|l|}
\hline
        & \multicolumn{5}{c|}{Mutants}                                                                      & \multirow{3}{*}{\begin{tabular}[c]{@{}l@{}}Mutation Score\\ (K/K+L)\end{tabular}} \\ \cline{1-6}
        &     &                                                        &      & \multicolumn{2}{c|}{Killed} &                                                                                   \\ \cline{1-6}
Mutants & All & \begin{tabular}[c]{@{}l@{}}Not\\ Compiled\end{tabular} & Live & Test Failure    & Timeout   &                                                                                   \\ \hline
Total   &  17\,696   &  2\,153   & 22      & 15\,332                & 189          & 99.86\%                                                                           \\ \hline
\end{tabular}
\end{table}    
             
%
%In order to analyze the feasibility of the code-driven mutation analysis for MLFS, we conducted a preliminary experimentation using the mutation operators AOR, ROR, ICR, LCR, and SDL we generated 17\,696 mutants. Preliminary results can be found in Table~\ref{table:mlfs_preliminary}.
%Particularly, we observe that from the 17\,696 generated mutants, we had 2\,153 mutants that were not compiled by the compilation toolset of MLFS, most probably because the mutation introduced a syntactical error that was detected by the toolset.
%Then, we only identified 22 live mutants that were not detected by the test suite. Instead, we had 15\,332 killed mutants detected by the test suite, and 189 mutants that produced libutil to go into an infinite loop, and thus were killed by timeout. The final mutation score was of 99.86\%.
%
%The identification of equivalent mutants still needs to be performed.


\subsection{Data-driven mutation analysis}

We do not plan to apply data drive mutation analysis to this case study because is a standalone library; it does not integrate communicating components.




%
%% !TEX root = MAIN.tex

\clearpage

\section{ASN1SCC}
\label{sec:caseStudies:GSL:ASN1}

\subsection{Overview of the case study}

\INDEX{ASN1SCC} is an open source ASN.1 compiler that generates C/C++ and SPARK/Ada code suitable for low resource environments such as space systems. Moreover, the compiler can produce a test harness that provides full statement coverage in the generated code, and therefore significantly improves its quality.


In the context of FAQAS, we focus our analysis on the auto-generated source code by the ASN.1 compiler, rather than in ASN1SCC software itself.

For the definition of the ASN.1 compiler case study, we introduce the example of a specific grammar. An excerpt of such grammar is shown in Listing~\ref{asn_excerpt}. 
The excerpt of the grammar introduces the definition of six data types, each data type also specifies the expected constraint, for example, the data type \texttt{MyInt} is an INTEGER which can have values from 0 to 20. The full source code of the grammar can be found in the file \emph{test.asn} uploaded on Alfresco.

For the given grammar, the size of the auto-generated source code is 4\,338 LOC. While the unit test suite consists of 107 auto-generated test cases.

% !TEX root =  ../MAIN.tex

\begin{minipage}{15cm}
\begin{lstlisting}[style=CStyle, caption=Excerpt of the tested ASN1 grammar., label=asn_excerpt, mathescape=true]
MyInt ::= INTEGER (0 .. 20)

My2ndInt ::= MyInt ( 1 .. 18)

AType ::= SEQUENCE {
    blArray SEQUENCE (SIZE(10)) OF BOOLEAN
}

My2ndAType ::= AType

TypeNested ::= SEQUENCE {
    intVal  INTEGER(0..10),
    int2Val INTEGER(-10..10),
    int3Val MyInt (10..12),
    intArray    SEQUENCE (SIZE (10)) OF INTEGER (0..3),
    realArray   SEQUENCE (SIZE (10)) OF REAL (0.1 .. 3.14),
    octStrArray SEQUENCE (SIZE (10)) OF OCTET STRING (SIZE(1..10)),
    boolArray   SEQUENCE (SIZE (10)) OF T-BOOL,
    label   OCTET STRING (SIZE(10..40)),
    bAlpha  T-BOOL,
    bBeta   BOOLEAN,
    sString T-STRING,
    arr     T-ARR,
    arr2    T-ARR2
}

E ::= INTEGER (0..255|1299)(5)
\end{lstlisting}
\end{minipage}









% !TEX root = MAIN.tex


\section{Empirical evaluation}
\label{sec:summary:results}

The FAQAS activity has ben evaluated through an extended empirical evaluation; below we summarize our findings.


\subsection{MASS}

%\begin{table}[htb]
%\caption{Code-driven mutation analysis results: MASS mutation score.}
%\label{table:results:mass}
%\small
%\centering
%\begin{tabular}{|
%>{\arraybackslash}p{54mm}@{\hspace{1pt}}|
%>{\raggedleft\arraybackslash}p{40mm}@{\hspace{1pt}}|
%}
%\hline
%\textbf{Subject}&\textbf{MASS Mutation Score (\%)}\\
%\hline
%
%\SAIL{}$_{S}$ (System test suite)&65.95\\
%
%\SAIL{}$_{S}$ (Unit+System test suite)&70.56\\
%
%\GCSP{}&70.92\\
%\PARAM{}&85.95\\
%
%\UTIL{}&84.41\\
%\MLFS{}{}&93.49\\
%% K: 3104 L: 2219-1480=739 T: 5323-1480=3843
%ASN1SCC&80.77\\
%\hline
%$\textbf{Average}$&78.86\\
%\hline
%\end{tabular}
%
%\end{table}
%
%
%Table~\ref{table:results:mass} provides the code-driven mutation analysis results.
%The mutation score computed by \MASS for the case study subjects considered in our experiments was in line with the expectations of engineers.

Both GSL and LXS have manually inspected a subset of the live mutants identified by \MASS. The inspection enabled industry partners to identify relevant shortcomings in their test suites:
\begin{itemize}
\item 57\% of the live mutants are due to missing inputs. Of particular relevance are exceptional cases not being exercised by the test suite, which shows that engineers are not be able to determine all the unexpected execution conditions that the SUT shall take care of.
\item 23\% of the live mutants were due to missing oracles. Such result is particularly relevant because it indicates that although engineers believe to have tested a relevant scenario, the absence of an appropriate oracle prevents them from automatically detecting failures that might be observed when running the test cases.
\item The few remaining live mutants had ben reported as either equivalent or not relevant (e.g., because concerning third party software).
\item One fault was detected. More precisely, the implementation of a test case that detects the mutant has shown that the SUT provides an erroneous result.
\end{itemize}

%In addition, based on an independent evaluation performed on a case study subject not shared with the FAQAS team, LXS has reported that 36\% of the 34 live mutants detected by \MASS spot major limitations of the test suite.
%In their independent evaluation with libraries, industry partners did not negatively comment about the scalability of the process. However, they reported the need for a strategy to further prioritize the generated mutants for inspection.


Finally, our results show that MASS helps addressing scalability problems to a significant extent by reducing mutation analysis time by more than 70\% across subjects.
In practice, for large software systems like \SAIL{}, such reduction can make mutation analysis practically feasible; indeed, with 100 HPC nodes available for computation, \MASS can perform the mutation analysis of \SAIL{}\emph{-CSW} in half a day. In contrast, a traditional mutation analysis approach would take more than 100 days, thus largely delaying the development and quality assurance processes.
Also, we demonstrated that the FSCI sampling approach implemented by MASS leads to an accurate estimation of the mutation score.




\subsection{SEMuS}

%\begin{table}[htb]
%\caption{Test suite augmentation results.}
%\label{table:results:test-gen}
%\centering
%\footnotesize
%\begin{tabular}{|
%@{\hspace{1pt}}p{10mm}|
%@{\hspace{1pt}}>{\raggedleft\arraybackslash}p{18mm}@{\hspace{1pt}}|
%>{\raggedleft\arraybackslash}p{35mm}@{\hspace{1pt}}|
%>{\raggedleft\arraybackslash}p{25mm}@{\hspace{1pt}}|
% >{\raggedleft\arraybackslash}p{25mm}@{\hspace{1pt}}|
%}
%\hline
%\textbf{Subject}&\textbf{Live Mutants}&\textbf{Additionally Killed Mutants}&\textbf{Original MS (\%)}&\textbf{Updated MS (\%)}\\
%\hline
%$\mathit{MLFS}$&3\,891&697&81.80&85.06\\
%$\mathit{ASN.1}$&2\,219&1\,729&58.31&90.79\\
%$\mathit{ESAIL_S}$&1\,041&NA&70.56&NA\\
%% additionally killed: clock 2 error 4 timestamp 6 memory 21
%$\mathit{Libutil}$&4\,198&35&81.80&81.96\\
%\hline
%\end{tabular}
%
%\end{table}

The empirical evaluation demonstrated the scalability of \SEMUS for the case study subjects in which it can be successfully applied (i.e., ASN1CC, MLFS, and \UTIL).
Our results also demonstrated the usefulness of \SEMUS. Indeed, \SEMUS enabled the identification of two faults in our case studies. Also, the generated test cases concerned inputs that are relevant (according to specifications) but not tested by the test suites.

%The main limitation of \SEMUS derives from the choice of , it is currently limited by the choice of compiling with LLVM only the source file under test (to limit the probability of compilation errors). Table~\ref{table:results:test-gen} provides an overview of some of our results.
%



\subsection{DAMAt}

%\begin{table}[htb]
%\caption{Data-driven mutation analysis results.}
%\label{table:results:data-driven}
%\center
%\footnotesize
%\begin{tabular}{|
%@{\hspace{0pt}}>{\raggedleft\arraybackslash}p{24mm}@{\hspace{1pt}}|
%@{\hspace{0pt}}>{\raggedleft\arraybackslash}p{12mm}@{\hspace{1pt}}|
%@{\hspace{0pt}}>{\raggedleft\arraybackslash}p{12mm}@{\hspace{1pt}}|
%@{\hspace{0pt}}>{\raggedleft\arraybackslash}p{18mm}@{\hspace{1pt}}|
%@{\hspace{0pt}}>{\raggedleft\arraybackslash}p{12mm}@{\hspace{1pt}}|
%@{\hspace{0pt}}>{\raggedleft\arraybackslash}p{12mm}@{\hspace{1pt}}|
%@{\hspace{0pt}}>{\raggedleft\arraybackslash}p{12mm}@{\hspace{1pt}}|
%@{\hspace{0pt}}>{\raggedleft\arraybackslash}p{12mm}@{\hspace{1pt}}|
%@{\hspace{0pt}}>{\raggedleft\arraybackslash}p{12mm}@{\hspace{1pt}}|
%}
%\hline
%\textbf{Subject} &
%\textbf{\# FMs} &
%\textbf{FMC} &
%\textbf{\#MOs-CFM} &
%\textbf{\#CMOs} &
%\textbf{MOC}
%&\textbf{Killed}&\textbf{Live}&\textbf{MS}
%\\
%\hline
%
%\ADCS &10 &90.00\%   & 135 & 100 & 74.00\%   &    45&55&45.00\%\\
%\GPS &1 &100.00\%    &  23  &  22 & 95.65\%    &      21&1&95.45\%\\
%\PDHU &3 &100.00\%  &   29 & 24 & 82.76\%   &     24&0&100.00\%\\
%\PARAM &6 &100.00\%  &   44 & 41 & 93.20\%  &        37&4&90.24\%\\
%\GCSP &1 &100.00\%  &   33 & 21 & 63.64\%  &        NA&NA&NA\\
%
%
%\hline
%
%\end{tabular}
%
%FM=Fault Model, FMC=Fault Model Coverage, MOs-CFM=Mutation Operations in covered FMs,
%CMO=Covered Mutation Operation, MOC=Mutation Operation Coverage, Killed=Number of mutants killed by the test suite, Live=Number of mutants not killed by the test suite, MS=Mutation Score. The mutation score for \GCSP is not available because of nondeterminism observed while running the experiments.
%
%\end{table}
%
%
%Table~\ref{table:results:data-driven} shows the mutation analysis results of DAMAT.

The empirical evaluation of DAMAt has demonstrated the effectiveness of the approach. Indeed, LXS has indicated that 57\% out of the overall amount of 102 test suite problems detected by DAMAt were spotting major limitations of the test suite. Also, GSL has confirmed that the approach enabled the detection of relevant test suite shortcomings.
One possible limitation of the approach is that it may introduce slow-downs that lead to non-deterministic failures when the test suite exercises brief interaction scenarios in which most of the operations performed concern the encapsulation of data into the network.

Our results confirm that (1) uncovered fault models (i.e., low \INDEX{FMC}) indicate lack of coverage for certain message types (\INDEX{UMT}) and, in turn, the lack of coverage of a specific functionality (i.e., setting the pulse-width modulation in \ADCS); (2) uncovered mutation operations (i.e., low \INDEX{MOC}) highlight the lack of testing of input partitions (\INDEX{UIP}); (3) live mutants (i.e., low \INDEX{MS}) suggest poor oracle quality (\INDEX{POQ}).

Based on our evaluation, we observed that live mutants can be killed by introducing oracles that (1) verify additional entries in the log files (39 instances for \ADCS, 1 instance for \GPS), (2) verify additional observable state variables (14 instances for \ADCS, 4 instances for \PARAM), and (3) verify not only the presence of error messages but also their content (2 instances for \ADCS).
\REVOCT{C-P-18}{Such oracles may consist of additional assertions that verify data values already produced by the software under test (i.e., no modification of the SUT is needed).}

%Finally we also identified the following set of test suite shortcomings: (1) the test case does not distinguish failures across data items (e.g., temperature values collected by different sensors),
%(2) the test case does not distinguish errors across different messages (e.g., in \ADCS, the IfHK message reporting a broken sensor or the message sent by a sensor reporting malfunction), (3) the test case does not distinguish between errors in nominal and non-nominal data (e.g., it does not distinguish between VOR and FVOR), (4) the test case does not distinguish between upper and lower bounds (e.g., the mutants for VOR lead to the same assertion failures), and {(5) the test case does not distinguish between different error codes (i.e, it simply verifies that an error code is generated)}.

\subsection{DAMTE}

\DAMTE aims to address a task (i.e., test generation at system and integration level) that is particularly difficult to address with state-of-the-art technology (e.g., test generation toolsets based on symbolic execution).
%In particular, the test generation tool adopted in FAQAS (i.e., KLEE) requires manual intervention to specify which are the inputs to select thus preventing the automated generation of a large number of test cases.
For this reason, FAQAS only concerned the evaluation of the feasibility of DAMTE.

We relied on DAMTE to generate inputs for the \PARAM client API functions. Such inputs enable the exchange of messages between the \PARAM client and the \PARAM server. Overall, we conclude that the DAMTE approach may be feasible; however, it requires some manual effort for the configuration and execution of test cases which may limit its usefulness. The first step towards its large scale applicability is the improvement of underlying test generation tools and compiler procedures, such changes will facilitate DAMTE application to large projects without the need for manually creating test template files with dependencies.


\STARTCHANGEDWPT

\section{Industrial Validation Summary}
\label{sec:validationsum}

The developed toolset has demonstrated to be useful in industrial contexts. The common limitation cross the different tools is the usability; indeed, all the tools require relevant effort to be set-up (however, LXS has reported that if at least 6\% of the reported problems spot major limitations the benefits surmount costs). The need for manual effort mostly depends on the lack of a common development environment for different case study subjects. The identification of a reference platform for software development in industry context may facilitate the adoption of the FAQAS toolset.

Other limitations that need further research effort to simplify the adoption of the FAQAS toolset are the prioritization of mutants to be inspected, the need for a solution to compile whole SUTs with LLVM, the need for a solution to enable test generation in the presence of floating point variables, the need for a working solution to enable test generation based on data-driven mutation analysis results.

Below we report verbatim the positive and negative comments provided in the validation deliverables of the project by our industry partners.


\subsection{Overall Comments}\ \\

\textbf{POSITIVE COMMENTS}

\begin{itemize}
  \item \emph{Using the FAQAS toolset is for sure cost-effective when more than the 6\% of the detected major problems of the testsuite could have determined a software error would have gone undetected. }

  \item \emph{Both these fields of research are considered from LuxSpace worth to be explored in a possible extension of the FAQAS project:
  (*) the capability of the toolset to decrease the time of analysis of the results (*) the capability of the toolset to generate automatically additional/updated test cases that may patch the current failing testsuite.}
\end{itemize}

\subsection{MASS}\ \\

\textbf{POSITIVE COMMENTS}

\begin{itemize}

  \item \emph{The SUM contains all necessary information in a well written style. This includes an explanation of the library structure and the purpose of contained files, a description of all configuration variables within those files and instructions for running the toolset. Each important file is provided with its own subsection, allowing the end-user to get a clear understanding of the framework configuration.}

  \item \emph{The MASS toolset offers many configuration options to the end-user. All these configuration options and parameters are well described in the SUM. }

  \item \emph{The MASS tool is effective at identifying potential defects that are unanticipated by our current test suites. It is also worth mentioning that closer examination of the code inspired by this approach did seem to reveal actual defects in the software that were previously unknown.}

  \item \emph{The FAQAS team used Singularity as a container system, however it is also possible to create a Docker image that allows running MASS mutation tests in a docker container.}

  \item \emph{Configuration files can be stored together with source code and mounted as volumes. In principle this makes it trivial to spawn a new instance (horizontal scaling).}

  \item \emph{The evaluation showed that the MASS tool would be considered as a useful addition to GomSpace’s testing processes.}

  \item \emph{Applying this approach to other libraries maintained by GomSpace could allow managers to direct efforts towards improving test suites identified as lower quality. This would lead to an overall improvement in both test suite and code quality.}

\end{itemize}

\textbf{NEGATIVE COMMENTS}

\begin{itemize}

  \item \emph{The evaluation did identify that many abbreviations and acronyms are used within the SUM, some of them without explanation in the document. An expansion of the list of abbreviations and acronyms is recommended to improve the overall user experience.}

  \item \textbf{Action taken:} To address the comment above, SnT has improved the SUM accordingly.

  \item \emph{The large degree of possibility is challenging for a new user to learn and can be overwhelming for first use. GomSpace recommends providing an interactive script with default values to ease this process.}

  \item \emph{The initial (first-time) configuration process should be simplified to reduce the steep learning curve}

  \item \textbf{Action taken:} To address the two comments above, SnT has ....

\end{itemize}

\subsection{DAMAt} \ \\

\textbf{POSITIVE COMMENTS}

\begin{itemize}

  \item \emph{The SUM contains all necessary information in a well written style. This includes an explanation of the library structure and the purpose of contained files, a description of all configuration variables within those files and instructions for running the toolset. Each important file is provided with its own subsection, allowing the end-user to get a clear understanding of the framework configuration.}

  \item \emph{The DAMAt toolset offers many configuration options to the end-user. All these configuration options and parameters are well described in the SUM. The ease of configuring these parameters has increased compared to the previous MASS tool. For example: \texttt{DAMAt\_FOLDER=\$(pwd)} and using this variable further for configuration significantly saves time.}

  \item \emph{Analysis of the surviving mutants shows the toolset does identify valid (potential) test cases that the test suites currently miss. This implies the presence of missing test cases, often in areas that can be considered as challenging edge cases that are difficult for a developer or dedicated software tester to anticipate, and in some cases poorly written test cases. Based on the results generated, the DAMAt tool is effective at identifying potential defects and missing test cases that are unanticipated by our current test suites.}

  \item \emph{DAMAt can be containerized. The FAQAS team used Singularity as a container system, however it is also possible to create a Docker image that allows running DAMAt mutation tests in a docker container. Configuration files can be stored together with source code and mounted as volumes. In principle this makes it trivial to spawn a new instance (horizontal scaling).}

  \item \emph{The evaluation showed that the DAMAt would be considered as a useful addition to GomSpace’s testing processes.}

  \item \emph{After checking the output report, GomSpace realised that the test suite of libparam does not address certain side effects of functions. This already demonstrates the ability of the toolset to identify improvements that would raise the quality of existing test suites.}

  \item \emph{Applying this approach to other libraries maintained by GomSpace could allow managers to direct efforts towards improving test suites identified as lower quality (i.e., those with more surviving mutants) or to set certain gateway thresholds (i.e., a certain proportion of mutants must be caught before a test suite is considered high enough quality to proceed). This would lead to an overall improvement in both test suite and code quality.}

\end{itemize}

\textbf{NEGATIVE COMMENTS}

\begin{itemize}

  \item \emph{The evaluation did identify that some of the missing steps in the SUM were present in the readme of the project. Without which it was quite difficult to configure some of the steps, as there is no explanation in the document.}

  \item \emph{Finally, the SUM doesn’t have sufficient information about fault models and how they are important to the process, analysis, and results. Only an example is present in the SUM. It would be useful to have some information explaining fault models, their significance, and some background of its correlation to probes, testcases etc.}

  \item \textbf{Action taken:} To address the two comments above, SnT has improved the SUM accordingly.

  \item \emph{To apply DAMAt, we need to manually inject probes into the code, which requires prior additional knowledge/understanding of inner working of the software under test (SUT). Moreover, the injected probes should always be used only for tests. In production, they shouldn’t be present and there is necessity of automated process that handles probes management.}

  \item \emph{There is manual intervention of adding lots of probes and building a fault model which can be a time-consuming process.}

  \item \emph{The addition of probes as manual step would make it more difficult to scale especially when it is the case of microservices when there are large number of interacting microservices.}

  \item \textbf{Action taken:} To address the three comments above, SnT has introduced an integration to the DAMAt pipeline that enables the user to leave simple comments in the source code that will be substituted with the mutation probes during the procedure. The integration will also reinstate the unmodified code once the execution of DAMAt is conluded.

\end{itemize}

\subsection{SEMUS} \ \\

\textbf{POSITIVE COMMENTS}

\begin{itemize}

  \item \emph{The SEMUS offers many configuration options to the end-user. All these configuration options and parameters are well described in the SUM. There are also scripts that are used for automatic generation of JSON files and Test templates.}

  \item \emph{Analysis of the generated testcases for the mutants identified valid bug for timestamp.c as well as missing Test cases. This implies the presence of missing test cases, often in areas that can be considered as challenging edge cases that are difficult for a developer or dedicated software tester to anticipate and in some cases poorly written test cases.}

  \item \emph{SEMUS can be containerized. The FAQAS team used Singularity as a container system, however it is also possible to create a Docker image that allows running SEMUS test generation in a docker container. Configuration files can be stored together with source code and mounted as volumes. In principle this makes it trivial to spawn a new instance (horizontal scaling).}

  \item \emph{However, when it is fully executed it does help with identifying the missing test cases in test suite and as a side-effect points out to potential bugs.}

  \item \emph{After checking the output report, GomSpace realised that the test suite of libutil does not address certain tests. This already demonstrates the ability of the SEMUS to identify improvements that would raise the quality of existing test suites }

  \item \emph{Nevertheless, beyond the current limitations to the usability, the SEMuS tool seems promising for its effectiveness: all the generated test cases were correctly designed, in a way that they were able to detect software errors in the parts of the code whereas it was meant too}

\end{itemize}

\textbf{NEGATIVE  COMMENTS}
\begin{itemize}
  \item \emph{However, there are some typos in commands in SUM which can be corrected for error free configuration.}

  \item \textbf{Action taken:} SnT has improved the SUM.

  \item \emph{If you run it on Windows machine with WSL or vagrant, there are chances you might run in few troubles with respect to versioning of different libraries/container/virtual env etc.}

  \item \textbf{Action taken:} SnT had not been able to replicate the problems; further investigation will be taken care of during the maintenance period.

  \item \emph{Also creation of JSON file and test template can be it more difficult to scale especially when it is the case of microservices when there are large number of interacting microservices.}

  \item \emph{But keep in mind that there is manual intervention of generating large amount of JSON and test templates which can be a time-consuming process in case of large software libraries.}

  \item \textbf{Action:} As detailed in Section~\ref{sec:limitations}, the two comments above can be targeted only by a dedicated follow-on activity.

  \item \emph{However the tool is still in a very prototypical form:}

  \item \emph{Due to the complexity of dependencies of E-SAIL software, the tool was able to working on a limited number of functions (the tool cannot compile files than depends on other files).}

  \item \emph{Due to the previous limitation, the tool was able to generate the test cases only from the killed mutants, but not from the live mutants.}

  \item \emph{The tool is not working with floating point variables.}
  \emph{All these limitations make the SEMuS tool, in this preliminary version, inadequate to be deployed in any real development environment. }

  \item \textbf{Action:} As detailed in Section~\ref{sec:limitations}, such improvements can be targeted only by a dedicated follow-on activity.

\end{itemize}

\ENDCHANGEDWPT

% !TEX root = MAIN.tex

\chapter*{Conclusion}
\label{sec:conclusion}
\addcontentsline{toc}{chapter}{Conclusion}

The FAQAS activity had been motivated by the need for high-quality software in space systems; indeed, the success of space missions depends on the quality of the system hardware as much on the dependability of its software. Before FAQAS there was no work on identifying and assessing feasible and effective mutation analysis and testing approaches for space software. Space software is different from other types of software (e.g., Java graphical libraries or Unix utility programs); its characteristics prevent the adoption of well-known solutions to enhance mutation analysis scalability, identify mutants that are semantically equivalent to the original software or redundant, and automatically generate test cases. First, space software normally contains many functions to deal with signals and data transformation, which may diminish the effectiveness of both compiler-based and coverage-based approaches to identify equivalent and redundant mutants. Second, space software is thoroughly tested with large test suites thus exacerbating scalability problems. Third, it requires dedicated hardware, software emulators, or simulators, which affect the applicability of scalability optimizations that use multi-threading or other OS functions. The reliance on dedicated hardware, emulators, and simulators also prevents the use of static program analysis to detect equivalent mutants and automatically generate test cases. 

The main output of the FAQAS activity had been a toolset 

The empirical evaluation conducted with the aid of industrial case study providers have highlighted the practical usefulness of the FAQAS toolset, which lead to the identification of relevant test suite shortcomings (test input partitions not exercised and missing test oracles) and bugs in the case study subjects.

%%
%% The acknowledgments section is defined using the "acks" environment
%% (and NOT an unnumbered section). This ensures the proper
%% identification of the section in the article metadata, and the
%% consistent spelling of the heading.
%\begin{acks}
%Acks
%\end{acks}

%%
%% The next two lines define the bibliography style to be used, and
%% the bibliography file.
\bibliographystyle{ACM-Reference-Format}
\bibliography{./bibliography.bib}



\end{document}
\endinput
%%
%% End of file `sample-authordraft.tex'.
