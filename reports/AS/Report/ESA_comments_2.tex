% !TEX root = MAIN.tex

\section{Responses to ESA comments provided on 04.10.2021}
\label{sec:ESA:comments:1}

Comments IDs appear also in the main document next to the text modified to address the comment. To save space in the main text, the prefix \emph{TDR-D2-PABG-} has been abbreviated as \emph{C-P-}.

\setlength\LTleft{0pt}
\setlength\LTright{0pt}
\tiny 
%@{\extracolsep{\fill}}
\begin{longtable}{|p{1.5cm}|p{12cm}|@{}}
%\caption{\normalsize .}
%\label{table:comments:responses} 
\textbf{Comment ID}&\textbf{Response}\\
\\
\hline
TDR-D2-0&
\begin{minipage}{12cm}
Done.
\end{minipage}\\
\\
\hline

TDR-D2-1&
\begin{minipage}{12cm}
Done.
\end{minipage}\\
\\
\hline

TDR-D2-2&
\begin{minipage}{12cm}
Done.
\end{minipage}\\
\\
\hline


TDR-D2-3&
\begin{minipage}{12cm}
Done.
\end{minipage}\\
\\
\hline

TDR-D2-5&
\begin{minipage}{12cm}
Renamed MUTANT\_2 as MUTANT\_1.

\end{minipage}\\
\\
\hline

TDR-D2-6&
\begin{minipage}{12cm}
Mismatches between the text and test suite IDs had been corrected.
In general, we have two mutants and N test cases. Test suites from 1 to 4 kill all the mutants.

\end{minipage}\\
\\
\hline

TDR-D2-07&
\begin{minipage}{12cm}
We are keeping the Section "Data-driven mutation not based on buffers" just to keep track of the idea even if it is not implemented. Someone else, in the future, may take advantage of it.

\end{minipage}\\
\\
\hline


TDR-D2-PABG-08&
\begin{minipage}{12cm}
We restructured the document to avoid the misunderstanding. Data mutation probes for DAMAt are presented in Section~\ref{sec:generateAPI}. In Section~\ref{sec:FAQASDataMutationProbesASN}, we describe what might be the mutation probes for ASN1 grammar.

\end{minipage}\\
\\
\hline
                                       
\end{longtable}
\normalsize

\clearpage