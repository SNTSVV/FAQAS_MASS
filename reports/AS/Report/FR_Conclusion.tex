% !TEX root = MAIN.tex

\section{Conclusion}
\label{sec:conclusion}
%\addcontentsline{toc}{chapter}{Conclusion}

The FAQAS activity had been motivated by the need for high-quality software in space systems; indeed, the success of space missions depends on the quality of the system hardware as much on the dependability of its software. Before FAQAS there was no work on identifying and assessing feasible and effective mutation analysis and testing approaches for space software. 

The main output of the FAQAS activity had been a toolset that implements three main features: code-driven mutation analysis (MASS), data-driven mutation analysis (DAMAt), code-driven mutation testing (SEMuS).

The empirical evaluation conducted with the aid of industrial case study providers have highlighted the practical usefulness of the FAQAS toolset, which led to the identification of relevant test suite shortcomings (test input partitions not exercised and missing test oracles) and bugs in the case study subjects. 
Since all the case study subjects considered in our empirical evaluation are space software systems that either undergo an extensive testing procedure and are deployed on orbit, the identification of such shortcomings and bugs indicate that the current procedures in place are not sufficient to guarantee software quality.
Our results thus highlight the necessity for the adoption of an automated quality assessment toolset into development practices for space software --- the FAQAS toolset has demonstrated to be an effective solution.