% !TEX root =  ../MutationTestingSurvey.tex

%
%\setlength\LTleft{0pt}
%\setlength\LTright{0pt}
%\begin{longtable}{@{\extracolsep{\fill}}|p{2.5cm}|p{5cm}|p{5cm}|@{}}
%\toprule


\begin{table}[h]
\caption{Mutation Operators for Model-based Data-driven Mutation Testing. Based on~\cite{di2015generating}}
\label{table:dataModelMutationOperators}


\tiny
\begin{tabular}{|p{2.5cm}|p{5cm}|p{7cm}|}

\hline
\textbf{Operator}&\textbf{Description}&\textbf{Example}\\
\hline
\textbf{Class Instance Duplication (CID)}&
The operator \emph{Class Instance Duplication} duplicates an instance of a class belonging to a collection of elements. This operator copies a randomly chosen instance of a class in a collection and then inserts it at a random position in the collection. This operator simulates unexpected data in a collection.
&In Figure~\ref{fig:dataModel}, this operator can be applied to the associations between the classes \emph{Transmission} and \emph{Vcdu}, and between the classes \emph{VirtualChannel} and \emph{Packet}. In both cases the duplicated data generated by this operator simulates a transmission error.\\
\hline
\textbf{Class Instance Removal (CIR)}
&This mutation operator deletes a randomly selected instance of a class from a collection of elements. 
&In Figure~\ref{fig:dataModel}, this operator can be applied to the associations between the classes \emph{Transmission} and \emph{Vcdu}, and between the classes \emph{VirtualChannel} and \emph{Packet}. The removal of an instance of class \emph{Vcdu}, for example, simulates a transmission error that may lead to either missing or broken Packets. When processing erroneous data created with this mutation operator, SES-DAQ should report a \emph{COUNTER\_JUMP} error as indicated by the constraint in Figure~\ref{fig:costraint:firstHeader}. \\
\hline
\textbf{Class Instances Swapping (CIS)}
&Swaps the positions of two randomly chosen instances of a class in a collection of elements. 
&In Figure~\ref{fig:dataModel}, this operator can be applied to the associations between the classes \emph{Transmission} and \emph{Vcdu}, and between the classes \emph{VirtualChannel} and \emph{Packet}. The effect of swapping two packets belonging to the association between the classes \emph{VirtualChannel} and \emph{Packet} simulates the presence of transmission data sequence errors.\\
\hline

\textbf{Attribute Replacement with Random (ARR)}
&This mutation operator replaces the value of an identifier attribute in an instance of a class with a randomly chosen value. In principle all the attributes of a class can be replaced with randomly chosen values, but in the general case a randomly generated value is not necessarily erroneous.
It is applied to attributes tagged with the UML stereotype $Identifier$. The $Identifier$ stereotype enables software engineers to specify a numeric range for the random value to generate.
&In Figure~\ref{fig:dataModel}, this mutation operator can be applied to all the attributes tagged with the stereotype \emph{Identifier}. For example a random mutation of the attribute \emph{versionNumber} belonging to an instance of class \emph{Header} simulates an invalid frame version, which should be reported by the software. 
\\
\hline
\textbf{Attribute Replacement using Boundary Condition (ARBC)}
& This mutation operator changes the value of an attribute according to a boundary condition criterion. This operator is particularly useful for mutating attributes that should be bound within a range, these attributes are usually measures. It is applied to attributes tagged with the UML stereotype \emph{Measure}. This stereotype enables software engineers to indicate the minimum and maximum values allowed for the tagged attribute. The mutation operator generates four values out of range according to traditional boundary testing strategies: minimum value, minimum value minus one, maximum value, and maximum value plus one. The operator ensures that the generated value is in the range representable with the data type (e.g. unsigned bytes cannot represent negative values).
&In Figure~\ref{fig:dataModel}, this operator can be applied to all the attributes tagged with the UML stereotype \emph{Measure}. In the running example this operator can be applied to the attribute \emph{vcFrameCount} of class \emph{Header}. 
\\
\hline
\textbf{Attribute Bit Flipping (ABF)}
&This operator randomly selects an attribute that corresponds to transmitted data and alters the value of a randomly selected bit. This mutation operator is particularly effective for introducing errors in attributes that cannot be tagged as Identifiers or Measures.
The operator works by flipping a single bit of an attribute. 
&In Figure~\ref{fig:dataModel}, this mutation operator can be applied to the attribute \emph{packetData} of class \emph{Packet} of the running example.
The attribute \emph{packetData} is a byte array: the mutation of one of its bits
simulates the presence of a realistic transmission error that should be identified thanks to the presence of a redundancy check code.
\\
\hline



%\bottomrule                                                             

\end{tabular}
\end{table}
%\normalsize