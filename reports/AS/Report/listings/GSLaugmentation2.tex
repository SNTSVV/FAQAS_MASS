% !TEX root =  ../MAIN.tex

\begin{lstlisting}[style=CStyle, caption=Second example of code for the identification of inputs with CBMC., label=GSLaugmentationTwo]
#include <gs/util/base16.h>
#include <stdio.h>
#include <stdlib.h>
#include <limits.h>
//The following should not be included, otherwise CBMC will lead to wrong results
//#include <ctype.h>
#include <errno.h>
#include <assert.h>


//The following function had been defined because CBMC has a builtin definition of 
// some stdlib function including strtoul, but the definition is too generic
// and leads to non-deterministic behaviours
unsigned long
strtoul(nptr, endptr, base)
	const char *nptr;
	char **endptr;
	int base;
{
	const char *s = nptr;
	unsigned long acc;
	int c;
	unsigned long cutoff;
	int neg = 0, any, cutlim;

	/*
	 * See strtol for comments as to the logic used.
	 */
	do {
		c = *s++;
	} while (isspace(c));
	if (c == '-') {
		neg = 1;
		c = *s++;
	} else if (c == '+')
		c = *s++;
	if ((base == 0 || base == 16) &&
	    c == '0' && (*s == 'x' || *s == 'X')) {
		c = s[1];
		s += 2;
		base = 16;
	}
	if (base == 0)
		base = c == '0' ? 8 : 10;
	cutoff = (unsigned long)ULONG_MAX / (unsigned long)base;
	cutlim = (unsigned long)ULONG_MAX % (unsigned long)base;
	for (acc = 0, any = 0;; c = *s++) {
		if (isdigit(c))
			c -= '0';
		else if (isalpha(c))
			c -= isupper(c) ? 'A' - 10 : 'a' - 10;
		else
			break;
		if (c >= base)
			break;
		if (any < 0 || acc > cutoff || ( acc == cutoff && c > cutlim) )
			any = -1;
		else {
			any = 1;
			acc *= base;
			acc += c;
		}
	}
	if (any < 0) {
		acc = ULONG_MAX;
		errno = ERANGE;
	} else if (neg)
		acc = -acc;
	if (endptr != 0)
		*endptr = (char *)(any ? s - 1 : nptr);
	return (acc);
}


int base16_decode(const char *encoded, uint8_t *raw)
{
    uint8_t *raw_bytes = raw;
    if (encoded) {
        const char *encoded_bytes = encoded;
        char buf[3];
        char *endp;

	while (encoded_bytes[0]) {
            if (!encoded_bytes[1]) {
                return GS_ERROR_ARG;
            }
            memcpy(buf, encoded_bytes, 2);
            buf[2] = '\0';
            *(raw_bytes++) = (uint8_t) strtoul(buf, &endp, 16);
            if (*endp != '\0') {
                return GS_ERROR_ARG;
            }
            encoded_bytes += 2;
	}
    }
    return (int)(raw_bytes - raw);
}

int MUT_base16_decode(const char *encoded, uint8_t *raw)
{
    uint8_t *raw_bytes = raw;
    if (encoded) {
        const char *encoded_bytes = encoded;
        char buf[3];
        char *endp;

	while (encoded_bytes[0]) {
            if (!encoded_bytes[1]) {
                //return GS_ERROR_ARG;
            }
            memcpy(buf, encoded_bytes, 2);
            buf[2] = '\0';
            *(raw_bytes++) = (uint8_t) strtoul(buf, &endp, 16);
            if (*endp != '\0') {
                return GS_ERROR_ARG;
            }
            encoded_bytes += 2;
	}
    }
    return (int)(raw_bytes - raw);
}


#ifdef CBMC
int cbmc_main(){
    char encoded[4];
	char a = nondet_char();
	char b = nondet_char();
	char c = nondet_char();

	uint8_t raw;
	uint8_t mRaw;
	int ret;
	int mret;

    
	encoded[0]=a;
	encoded[1]=b;
	encoded[2]=c;
    encoded[3]=0;

	raw=0;
    ret = base16_decode(encoded,&raw);

	encoded[0]=a;
	encoded[1]=b;
	encoded[2]=c;
        encoded[3]=0;
	mRaw=0;
    mret = MUT_base16_decode(encoded,&mRaw);
    
    //to verify that the original function is deterministic, just replace with the following
    //mret = base16_decode(encoded,&mRaw);


	__CPROVER_output("input: encoded", encoded);
	__CPROVER_output("output original: ret", ret);
	__CPROVER_output("output original: mret", mret);
	assert( ret == mret );
        
}
#endif

int main(){
    char encoded[4];

	uint8_t raw;
	uint8_t mRaw;
	int ret;
	int mret;

    
	encoded[0]='\n';
	encoded[1]='6';
	encoded[2]='2';
	encoded[3]=0;

	raw=0;
    	ret = base16_decode(encoded,&raw);
    	//ret = MUT_base16_decode(encoded,&raw);

	assert( ret == GS_ERROR_ARG );

    	return 0;
}
\end{lstlisting}

