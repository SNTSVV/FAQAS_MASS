% !TEX root =  ../MAIN.tex

\begin{lstlisting}[style=CStyle, caption=Example of automatically generated ASN.1 data-driven mutation operations., label=ASN_mutations]

// TypeNested type
void _FAQAS_TypeNested_mutate(TypeNested *pVal) {

	// ALREADY_MUTATED is a global variable 
	// that traces if in the current execution we already performed data mutation
	if ( ALREADY_MUTATED ){
		return;
	}
	
	// intVal,VAT,1
	if ( ! has_been_mutated("TypeNested_1") ){
		// check that the value is not already 
		// what we want to generate

		if (pVal->intVal != 6 ){
			pVal->intVal = 6;
			save_mutation("TypeNested_1");

			return;
		}
	}
	
	// int2Val,VOR,1
	if ( ! has_been_mutated("TypeNested_2") ){
		// check that the value is not already 
		// what we want to generate

		if (pVal->intVal != 0 ){
			pVal->intVal = -1;
			save_mutation("TypeNested_2");

			return;
		}
	}

	// int2Val,VOR,2
	if ( ! has_been_mutated("TypeNested_3") ){

        printf("%lu\n", pVal -> intVal);

        if (pVal->intVal != 10){
            pVal->intVal = 51;
            save_mutation("TypeNested_3");

            return;
        }
    }

...

\end{lstlisting}

%// max E OR 1st operand constraint
%if ( strcmp(buf,"E_1") ) {                                                  
%	if ((*pVal) <= 255UL) {                                                                                                               
%	    printf("%lu\n", *pVal);
%
%	    if (*pVal != 255UL) {
%	        *pVal = 255UL;
%	        save_mutation();
%
%	        return;
%	    }           
%	}       
%}
%
%// n+1 E 2nd operand constraint
%if ( strcmp(buf, "E_2") ) {
%    if ((*pVal) == 1299UL) {                                                                                                              
%        printf("%lu\n", *pVal);
%
%        *pVal = 1299UL + 1UL;
%        save_mutation();
%        return;
%    }       
%} 