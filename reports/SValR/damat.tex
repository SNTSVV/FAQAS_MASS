% !TEX root = MAIN.tex

\chapter{\DAMA - Validation Execution Environment Definition and Results}

\begin{figure}[h]
  \centering
  \includegraphics[width=0.6\textwidth]{images/dataDrivenBufferProcess.pdf}
      \caption{DAMAt methodology.}
      \label{fig:damat}
\end{figure}

Figure~\ref{fig:damat} introduces \DAMA methodology.
Validation was performed by applying the \DAMA methodology to the case studies listed in the \emph{SPAP} document (chapter 5):
\begin{itemize}
  \item LXS System Test Suite for ESAIL.
  \item GSL Integration Test Suite for libgscsp.
  \item GSL System Test Suite for libparam.
\end{itemize}

Particularly, we verified that \DAMA was able to:
\begin{enumerate}
	\item Parse a fault model prepared by the user.
	\item Generate a mutation API with the functions that modify the data according to the provided fault model.
  \item Modify the buffer through calls to the mutation API.
	\item Generate and compile mutants.
	\item Execute the test suite against all the mutants and gather the results of the test cases.
	\item Generate the results of the mutation analysis.
\end{enumerate}

\begin{table}[h]
\caption{MASS Validation Report.}
\label{table:damat:results}
\scriptsize
\centering
\begin{tabular}{|@{\hspace{1pt}}p{33mm}|
@{\hspace{1pt}}>{\raggedleft\arraybackslash}p{12mm}@{\hspace{1pt}}|
>{\raggedleft\arraybackslash}p{12mm}@{\hspace{1pt}}|
>{\raggedleft\arraybackslash}p{12mm}@{\hspace{1pt}}|
>{\raggedleft\arraybackslash}p{12mm}@{\hspace{1pt}}|
}
\hline
\textbf{Methodology Step}&\textbf{ESAIL}&\textbf{libgscsp}&\textbf{libparam}\\
\hline
Parse the fault model&PASSED&PASSED&PASSED\\
Generate the mutation &PASSED&PASSED&PASSED\\
Modify the buffer&PASSED&PASSED&PASSED\\
Generate and compile the mutants&PASSED&PASSED&PASSED\\
Execute the test suite&PASSED&PASSED&PASSED\\
Generate the results&PASSED&PASSED&PASSED\\
\hline
\end{tabular}

\end{table}

The idea was to verify that every step of the methodology produced the correct input/outputs. For this reason, we provide Table~\ref{table:damat:results}, that shows the success/failure of every \DAMA step on the different case studies.

A detailed exposition of the validation results is presented in the \emph{D4} document (Chapter 2).

\clearpage


% Particularly, we verified that \DAMA was able to:
% \begin{enumerate}
% 	\item Parse a fault model prepared by the user.
% 	\item Generate a mutation API with the functions that modify the data according to the provided fault model.
%   \item Modify the buffer through calls to the mutation API.
% 	\item Generate and compile mutants.
% 	\item Execute the test suite against all the mutants and gather the results of the test cases.
% 	\item Generate the results of the mutation analysis.
% \end{enumerate}
%
% \begin{table}[h]
% \caption{MASS Validation Report.}
% \label{table:mass:results}
% \scriptsize
% \centering
% \begin{tabular}{|
% @{\hspace{1pt}}p{33mm}|
% @{\hspace{1pt}}>{\raggedleft\arraybackslash}p{12mm}@{\hspace{1pt}}|
% >{\raggedleft\arraybackslash}p{12mm}@{\hspace{1pt}}|
% >{\raggedleft\arraybackslash}p{12mm}@{\hspace{1pt}}|
% >{\raggedleft\arraybackslash}p{12mm}@{\hspace{1pt}}|
% >{\raggedleft\arraybackslash}p{12mm}@{\hspace{1pt}}|
%  >{\raggedleft\arraybackslash}p{12mm}@{\hspace{1pt}}|
% }
% \hline
% \textbf{Methodology Step}&\textbf{ESAIL}&\textbf{libgscsp}&\textbf{libparam}&\textbf{libutil}&\textbf{MLFS}&\textbf{ASN.1}\\
% \hline
% Parse the fault model&PASSED&PASSED&PASSED&PASSED&PASSED&PASSED\\
% Generate the mutation API&PASSED&PASSED&PASSED&PASSED&PASSED&PASSED\\
% Modify the buffer&PASSED&PASSED&PASSED&PASSED&PASSED&PASSED\\
% Generate and compile the mutants&PASSED&PASSED&PASSED&PASSED&PASSED&PASSED\\
% Execute the test suite&PASSED&PASSED&PASSED&PASSED&PASSED&PASSED\\
% Generate the results&PASSED&PASSED&PASSED&PASSED&PASSED&PASSED\\
% \hline
% \end{tabular}
%
% \end{table}
%
% The idea was to verify that every step of the methodology produced the correct input/outputs. For this reason, we provide Table~\ref{table:mass:results}, that shows the success/failure of every \MASS step on the different case studies.
%
% A detailed exposition of the validation results is presented in the \emph{D4} document (Chapter 2).
