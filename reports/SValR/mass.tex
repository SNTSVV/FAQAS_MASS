% !TEX root = MAIN.tex

\chapter{MASS - Validation Execution Environment Definition and Results}

\STARTCHANGEDFINAL

To validate \MASS we applied the methodology to the case studies listed in the \emph{SPAP} document (chapter 8):
\begin{itemize}
  \item LXS System Test Suite for ESAIL
  \item GSL Integration Test Suite for libgscsp
  \item GSL System Test Suite for libparam
  \item MLFS Unit Test Suite
  \item GSL Unit Test Suite for libutil
  \item ASN.1 Auto-generated Unit Test Suite
\end{itemize}

% \begin{table}[h]
% \caption{MASS Validation Report.}
% \label{table:mass:results} 
% \scriptsize
% \centering
% \begin{tabular}{|
% @{\hspace{1pt}}p{33mm}|
% @{\hspace{1pt}}>{\raggedleft\arraybackslash}p{12mm}@{\hspace{1pt}}|
% >{\raggedleft\arraybackslash}p{12mm}@{\hspace{1pt}}|
% >{\raggedleft\arraybackslash}p{12mm}@{\hspace{1pt}}|
% >{\raggedleft\arraybackslash}p{12mm}@{\hspace{1pt}}|
% >{\raggedleft\arraybackslash}p{12mm}@{\hspace{1pt}}|
%  >{\raggedleft\arraybackslash}p{12mm}@{\hspace{1pt}}|
% }
% \hline
% \textbf{Methodology Step}&\textbf{ESAIL}&\textbf{libgscsp}&\textbf{libparam}&\textbf{libutil}&\textbf{MLFS}&\textbf{ASN.1}\\ 
% \hline
% Collect SUT coverage&PASSED&PASSED&PASSED&PASSED&PASSED&PASSED\\
% Create Mutants&PASSED&PASSED&PASSED&PASSED&PASSED&PASSED\\
% Compile Mutants&PASSED&PASSED&PASSED&PASSED&PASSED&PASSED\\
% TCE&PASSED&PASSED&PASSED&PASSED&PASSED&PASSED\\
% Sample Mutants&PASSED&PASSED&PASSED&PASSED&PASSED&PASSED\\
% Prioritized Test Suite&PASSED&PASSED&PASSED&PASSED&PASSED&PASSED\\
% Identify Likely Equivalents&PASSED&PASSED&PASSED&PASSED&PASSED&PASSED\\
% Compute Mutation Score&PASSED&PASSED&PASSED&PASSED&PASSED&PASSED\\
% \hline
% \end{tabular}

% \end{table}

Particularly, we verified that \MASS complied with the requirements expressed in the \emph{Software System Specification} (\emph{SSS}), section 4.1.1.
To this purpose, two validation approaches were devised:

\begin{itemize}
  \item Unit Testing
  \item Application to Case Studies
\end{itemize}

These approaches are composed of validation tasks as shown in Table~\ref{table:mass:results} (Application to Case Studies), and Table~\ref{table:mass:unit_results}

% The idea was to verify that every step of the methodology produced the correct input/outputs. For this reason, we provide Table~\ref{table:mass:results}, that shows the success/failure of every \MASS step on the different case studies.

% A detailed exposition of the validation results is presented in the \emph{D4} document (Chapter 2).

The results provided in these tables correspond to Monday 20/09/2021.

The validation tasks are described in details in the \emph{Software Validation Specification} (SVS) document, chapter 5.

A detailed description of \MASS unit testing is available in the \emph{Software Unit Test Plan} (\emph{SUTP}) (Chapters 7-10) and a detailed report in the \emph{Software Unit Test Report} (Chapters 4 and 5).

\begin{table}[h]
\caption{\MASS Validation Report, Unit Testing.}
\label{table:mass:unit_results}
\scriptsize
\centering
\begin{tabular}{|l|l|l|l|}
\hline
\textbf{Validation Approach}&\textbf{Task}&\textbf{Result}\\
\hline
Unit Testing&Performing \emph{MASS-TD-SRCMutation-1}&PASSED\\
\hline
\end{tabular}

\end{table}


\begin{table}[h]
\caption{\MASS Validation Report, Application to Case Studies}
\label{table:mass:results}
\scriptsize
\centering
\begin{tabular}{|l|l|l|l|l|l|l|}
\hline
\textbf{Task}&\textbf{ESAIL}&\textbf{libparam} &\textbf{libgscsp}&\textbf{libutil}&\textbf{ASN.1}&\textbf{MLFS}\\
\hline
configuring MASS and running Launcher.sh&PASSED&PASSED&PASSED&PASSED&PASSED&PASSED\\
configuring MASS and running PrepareSUT.sh&PASSED&PASSED&PASSED&PASSED&PASSED&PASSED\\
configuring MASS and running GenerateMutants.sh&PASSED&PASSED&PASSED&PASSED&PASSED&PASSED\\
configuring MASS and running CompileOptimizedMutants.sh&PASSED&PASSED&PASSED&PASSED&PASSED&PASSED\\
configuring MASS and running OptimizedPostProcessing.sh&PASSED&PASSED&PASSED&PASSED&PASSED&PASSED\\
configuring MASS and running GeneratePTS.sh &PASSED&PASSED&PASSED&PASSED&PASSED&PASSED\\
\hline
\end{tabular}

\end{table}



\chapter{SEMUS - Validation Execution Environment Definition and Results}

To validate \SEMUS we applied the methodology to the following case studies:

\begin{itemize}
  \item MLFS Unit Test Suite
  \item ASN.1 Auto-generated Unit Test Suite
\end{itemize}

The idea was to verify that every step of the methodology produced the correct input/outputs. For this reason, we provide Table~\ref{table:semus:results}, that shows the success/failure of every \SEMUS step on the different case studies.

\begin{table}[h]
\caption{SEMuS Validation Report.}
\label{table:semus:results} 
\scriptsize
\centering
\begin{tabular}{|
@{\hspace{1pt}}p{33mm}|
@{\hspace{1pt}}>{\raggedleft\arraybackslash}p{12mm}@{\hspace{1pt}}|
>{\raggedleft\arraybackslash}p{12mm}@{\hspace{1pt}}|
}
\hline
\textbf{Methodology Step}&\textbf{ASN.1}&\textbf{MLFS}\\ 
\hline
Generate Test Templates&PASSED&PASSED\\
Generate Meta-Mutants&PASSED&PASSED\\
Invoke SEMu&PASSED&PASSED\\
Convert Ktest to Unit Test&PASSED&PASSED\\
\hline
\end{tabular}

\end{table}

A detailed exposition of the validation results is presented in the \emph{D4} document (Chapter 2).



\ENDCHANGEDFINAL
\clearpage
