% !TEX root = MAIN.tex
\chapter{Software Maintenance Plan}
\label{chapter:maintenance}
% This document is the Software Maintenance Plan (SMP) for the Basic mathematical Library (BL) and the Basic mathematical Library Test Suite (BLTS) of the Mathematical Library for Flight Software (MLFS) project. It is composed of this document itself, its accompanying resources for the maintenance tasks E1356-GTD-SMP-01a, and the template for NCR submission E1356-GTD-SMP-01b.
% Its purpose is to provide the definition of organizational aspects and management approach to the implementation of the maintenance tasks and to describe the approach to the implementation of the maintenance process for this software product.
% GTD GmbH developed the software but the maintenance of the MLFS library is not part of the contract, thus this plan will define how to act in case of a required maintenance but will not define resources and schedule aspects of the maintenance process.
% The software maintenance plan is a constituent of the maintenance file (MF).

% 6.3.8.1
% a. The organization responsible for maintenance shall be identified to allow a smooth transition into the operations and maintenance.
% NOTE An organization, with representatives from both supplier and customer, can be set up to support the maintenance activities. Attention is drawn to the importance of the flexibility of this organization to cope with the unexpected occurrence of problems and the identification of facilities and resources to be used for the maintenance activities.
% EXPECTED OUTPUT: Maintenance plan [MF, -; QR, AR, ORR].
% 6.3.8.2
% a. The maintenance organization shall specify the assurance, verification and validation activities applicable to maintenance interventions.
% EXPECTED OUTPUT: Maintenance plan [MF, -; QR, AR, ORR].
% 6.3.8.3
% a. The maintenance plans shall be verified against specified requirements for maintenance of the software product.
% NOTE The maintenance plans and procedures can address corrective, improving, adaptive and preventive maintenance, differentiating between “routine” and “emergency” maintenance activities.
% 6.3.8.4
% a. The maintenance plans and procedures shall include the following as a minimum:
% 1. scope of maintenance;
% 2. identification of the first version of the software product for which maintenance is to be done;
% 3. support organization;
% 4. maintenance life cycle;
% 5. maintenance activities;
% 6. quality measures to be applied during the maintenance;
% 7. maintenance records and reports.
% EXPECTED OUTPUT: Maintenance plan [MF, -; QR, AR, ORR].

% To support the adoption of the FAQAS framework, SnT will guarantee a maintenance period of 12 months after the end of the activity. The FAQAS framework will be made available through code hosting services (e.g., GitHub[36] and GitLab[37]), likely the Gitlab service of the University of Luxembourg [38]. Built-in issue tracking features will be used to track bug requests and fixes.
% SnT will guarantee corrective, adaptive and preventive maintenance [R5-2] except for exceptional case that imply a major rewriting of the toolset. Such cases will be likely included in the methodology definition as corner cases where the current toolsets cannot be applied.
% Activities related to training [R5-3], documentation [R5-4], and software releases [R5-5] will be performed according to SoW.

This Chapter contains the Software Maintenance Plan (SMP) for the FAQAS framework.

\section{Scope and Purpose}

Its purpose is to provide the definition of organizational aspects and management approach to the implementation of the maintenance tasks and to describe the approach to the implementation of the maintenance process for this software product.

\section{Application of the Plan}

 The FAQAS framework will be made available through code hosting services (i.e., the Gitlab service of ESA\footnote{https://gitrepos.estec.esa.int/external/FAQAS}). Built-in issue tracking features will be used to track bug requests and fixes.

\section{Maintenance Concept}

To support the adoption of the FAQAS framework, SnT will guarantee a maintenance period of 12 months after the end of the activity.
SnT will guarantee corrective, adaptive and preventive maintenance except for exceptional case that imply a major rewriting of the toolset.
Such cases will be included in the methodology definition as corner cases where the current toolsets cannot be applied.
