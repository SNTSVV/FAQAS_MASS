% !TEX root = MAIN.tex

\chapter{Software process assurance}

\section{	Software development cycle}
% a.	The SPAP shall refer to the software development cycle description in the software development plan.
% b.	If not covered in the software development plan, the life cycle shall be described.
% c.	The life cycle shall include a milestone immediately before the starting of the software validation.

\section{	Projects plans}
% a.	The SPAP shall describe all plans to be produced and used in the project.
% b.	The relationship between the project plans and a timely planning for their preparation and update shall be described.
\section{Software dependability and safety}
% a.	The SPAP shall contain a description and justification of the measures to be applied for the handling of critical software, including the analyses to be performed and the standards applicable for critical software.
\section{Software documentation and configuration management}
% a.	The SPAP shall describe the contribution of the software product assurance function to the proper implementation of documentation and configuration management.
% b.	The nonconformance control system shall be described or referenced. The point in the software life cycle from which the nonconformance procedures apply shall be specified.
% c.	The SPAP shall identify method and tool to protect the supplied software, a checksum-type key calculation for the delivered operational software, and a labelling method for the delivered media.
\section{	Process metrics}
% a.	The SPAP shall describe the process metrics derived from the defined quality models, the means to collect, store and analyze them, and the way they are used to manage the development processes.
\section{	Reuse of software}
% a.	The SPAP shall describe the approach for the reuse of existing software, including delta qualification.
\section{	Product assurance planning for individual processes and activities}
% a.	The following processes and activities shall be covered, taking into account the project scope and life cycle:
% 1.	software requirements analysis;
% 2.	software architectural design and design of software items;
% 3.	coding;
% 4.	testing and validation (including regression testing);
% 5.	verification;
% 6.	software delivery and acceptance;
% 7.	operations and maintenance.
\section{	Procedures and standards}
% a.	The SPAP shall describe or list by reference all procedures and standards applicable to the development of the software in the project.
% b.	The software product assurance measures to ensure adherence to the project procedures and standards shall be described.
% c.	The standards and procedures to be described or listed in accordance with B.2.1 a shall be as a minimum those covering the following aspects:
% 1.	project management;
% 2.	risk management;
% 3.	configuration and documentation management;
% 4.	verification and validation;
% 5.	requirements engineering;
% 6.	design;
% 7.	coding;
% 8.	metrication;
% 9.	nonconformance control;
% 10.	audits;
% 11.	alerts;
% 12.	procurement;
% 13.	reuse of existing software;
% 14.	use of methods and tools;
% 15.	numerical accuracy;
% 16.	delivery, installation and acceptance;
% 17.	operations;
% 18.	maintenance;
% 19.	device programming and marking.


\clearpage
