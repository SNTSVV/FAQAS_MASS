% !TEX root = MAIN.tex
\chapter{Organization}
\label{chapter:organization}

% The supplier shall ensure that an organizational structure is defined for software development, and that individuals have defined tasks and responsibilities.

The FAQAS framework is developed for ESA by institutions and companies with a strong background in the SatCom
scientific and industrial word, with synergic expertise in the key areas to be addressed during
the study:
\begin{itemize}
  \item Interdisciplinary Centre for Security, Reliability and Trust, University of Luxembourg (SnT), prime contractor of the study.
  \item GomSpace Luxembourg SARL (GSL)
  \item LuxSpace SARL (LXS).
\end{itemize}

The activity will be managed by Dr. Fabrizio Pastore, Research Scientist of the SVV group of SnT, which acts as the prime contractor.

\section{Responsibilities}
\label{sec:resp}

% The responsibility, the authority and the interrelation of personnel who manage, perform and verify work affecting software quality shall be defined and documented.
% The responsibilities and the interfaces of each organisation, either external or internal, involved in a project shall be defined and documented.
% The supplier shall identify the personnel responsible for software product assurance for the project (SW PA manager/engineer).

\section{Resources}
\label{sec:resources}

\subsection{Resources}
The human resources needed to ensure the project assurance are:
\begin{itemize}
  \item Software PA Manager, identified in section~\ref{sec:resp}%for example
\end{itemize}

The software needed to perform the related activities is listed in Table~\ref{table:software_resources}.

\begin{table}[H]
\centering
\begin{tabular}{||c|c|c|c||}
 \hline
 \textbf{Tool} & \textbf{Version} & \textbf{Purpose}\\
 \hline
 Git & version 2.30.1 & Version control and configuration management \\
 MS Office & version 16.54 & Report production and data analysis \\
 LaTeX & various & Report production \\
 \hline
\end{tabular}
\caption{Software resources for project assurance activities}
\label{table:software_resources}
\end{table}

\section{Methods and Tools}

% Methods and tools to be used for all the activities of the development cycle, (including requirements analysis, software specification, modelling, design, coding, validation, testing, configuration management, verification and product assurance) shall be identified by the supplier and agreed by the customer.
% EXPECTED OUTPUT: Software product assurance plan [PAF, SPAP; SRR, PDR].

\section{Training}

% The supplier shall ensure that the right composition and categories of appropriately trained personnel are available for the planned activities and tasks in a timely manner.
% The supplier shall determine the training subjects based on the specific tools, techniques, methodologies and computer resources to be used in the development and management of the software product.
% NOTE Personnel can undergo training to acquire skills and knowledge relevant to the specific field with which the software is to deal.
