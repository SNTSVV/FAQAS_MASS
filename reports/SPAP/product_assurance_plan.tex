% !TEX root = MAIN.tex
\chapter{Software Product Assurance Plan}

% The supplier shall develop a software product assurance plan in response to the software product assurance requirements in conformance with DRD in annex B.
% b. The software product assurance plan shall be either a standalone document or a section of the supplier overall product assurance plan.

% a. Testing shall be performed in accordance with a strategy for each testing level (i.e. unit, integration, validation against the technical specification, validation against the requirements baseline, acceptance), which includes:
% 1. the types of tests to be performed;
% NOTE For example: functional, boundary, performance, and usability tests.
% 2. the tests to be performed in accordance with the plans and procedures;
% 3. the means and organizations to perform assurance function for testing and validation.
% EXPECTED OUTPUT: Software product assurance plan [PAF, SPAP; PDR, CDR].

% 6.3.5.2
% a. Based on the criticality of the software, test coverage goals for each testing level shall be agreed between the customer and the supplier and their achievement monitored by metrics:
% 1. for unit level testing;
% 2. for integration level testing;
% 3. for validation against the technical specification and validation against the requirements baseline.
% EXPECTED OUTPUT: Software product assurance plan [PAF, SPAP; PDR, CDR].
% 6.3.5.3
% a. The supplier shall ensure through internal review that the test procedures and data are adequate, feasible and traceable and that they satisfy the requirements.
% EXPECTED OUTPUT: Software product assurance reports [PAF, -; -].
% 6.3.5.4
% a. Test readiness reviews shall be held before the commencement of test activities, as defined in the software development plan.
% EXPECTED OUTPUT: Test readiness review reports [DJF, -; TRR].

% 6.3.5.5
% a. Test coverage shall be checked with respect to the stated goals.
% EXPECTED OUTPUT: Software product assurance reports [PAF, -; -].
% b. Feedback from the results of test coverage evaluation shall be continuously provided to the software developers.
% 6.3.5.6
% a. The supplier shall ensure that nonconformances and software problem reports detected during testing are properly documented and reported to those concerned.
% EXPECTED OUTPUT: Nonconformance reports and software problem reports [DJF, -; CDR, QR, AR, ORR].
% 6.3.5.7
% a. The test coverage of configurable code shall be checked to ensure that the stated requirements are met in each tested configuration.
% EXPECTED OUTPUT: Statement of compliance with test plans and procedures [PAF, -; CDR, QR, AR, ORR].
% 6.3.5.8
% a. The completion of actions related to software problem reports generated during testing and validation shall be verified and recorded.
% EXPECTED OUTPUT: Software problem reports [DJF,

% 6.3.5.12
% a. The supplier shall ensure that tests are repeatable by verifying the storage and recording of tested software, support software, test environment, supporting documents and problems found.
% EXPECTED OUTPUT: Software product assurance reports [PAF, -; -].
% 6.3.5.13
% a. The supplier shall confirm in writing that the tests are successfully completed.
% EXPECTED OUTPUT: Testing and validation reports [DJF, -; CDR, QR, AR, ORR].
%
% 6.3.5.11
% a. The supplier shall ensure that:
% 1. tests are conducted in accordance with approved test procedures and data,
% 2. the configuration under test is correct,
% 3. the tests are properly documented, and
% 4. the test reports are up to date and valid.
% EXPECTED OUTPUT: Statement of compliance with test plans and procedures [PAF, -; CDR, QR, AR, ORR].

% 6.3.5.15
% a. Areas affected by any modification shall be identified and re‐tested (regression testing).
% 6.3.5.16
% a. In case of re‐testing, all test related documentation (test procedures, data and reports) shall be updated accordingly.
% EXPECTED OUTPUT: Updated test documentation [DJF, -; CDR, QR, AR, ORR].

% 6.3.5.18
% a. The need for regression testing and additional verification of the software shall be analysed after a change or update of any tool used to generate it.
% NOTE For example: source code or object code.
% EXPECTED OUTPUT: Updated test documentation

% 7.1.3 Assurance activities for product quality requirements
% a. The supplier shall define assurance activities to ensure that the product meets the quality requirements as specified in the technical specification.
% EXPECTED OUTPUT: Software product assurance plan [PAF, SPAP; SRR, PDR].

\section{Software product assurance programme implementation}

\subsection{Organization}
\label{subsec:organization}
% a. The SPAP shall describe the organization of software product assurance activities, including responsibility, authority and the interrelation of personnel who manage, perform and verify work affecting software quality.
% b. The following topics shall be included:
% 1. organizational structure;
% 2. interfaces of each organisation, either external or internal, involved in the project;
% 3. relationship to the system level product assurance and safety;
% 4. independence of the software product assurance function;
% 5. delegation of software product assurance tasks to a lower level supplier, if any

The product assurance activities will be performed following the team organization defined in Chapter~\ref{chapter:organization}.



\subsection{Reporting}
% The reporting related to product assurance processes includes:
SnT shall report on a regular basis on the status of the software product assurance programme implementation, as part of the overall reporting of the project.

\subsection{Quality models}
% a. The SPAP shall describe the quality models applicable to the project and how they are used to specify the quality requirements.
% Quality models shall be used to specify the software quality requirements.

\subsection{Methods and tools}
% a. The SPAP shall describe the methods and tools used for all the activities of the development cycle, and their level of maturity.

The methods and tools used for all the activities of the development cycle are described in the project Software Development Plan, in Chapter~\ref{chapter:software_development}
The PA Manager ensures that:
\begin{itemize}
  \item the development team has adequate experience and training to make use of  the methods and tools.
  \item the tools and methods are appropriate for the characteristics of the product in development.
  \item the tools and relative hardware are available throughout the lifetime of the product.
  \item the methods and tools are correctly used.
\end{itemize}


\subsection{Operation and Maintenance}
For the maintenance phase a software maintenance plan shall be prepared.
The outline shall be described in Chapter~\ref{chapter:maintenance}.

\subsection{Software problems}

% The supplier shall define and implement procedures for the logging, analysis and correction of all software problems encountered during software development.
% EXPECTED OUTPUT: Software problem
% NO, you need to put here as Yes and ndicate in the SPAP that software problms shall be indicated in the ISSUE section of https://gitlab.uni.lu/fpastore/FAQAS

% 5.2.5.2
% a. The software problem report shall contain the following information:
% 1. identification of the software item;
% 2. description of the problem;
% 3. recommended solution;
% 4. final disposition;
% 5. modifications implemented (e.g. documents, code, and tools);
% 6. tests re‐executed.
% EXPECTED OUTPUT: Software problem reporting procedures [PAF, -; PDR].
% 5.2.5.3
% a. The procedures for software problems shall define the interface with the nonconformance system (i.e. the circumstances under which a problem qualifies as a nonconformance).
% EXPECTED OUTPUT: Software problem reporting procedures [PAF, -; PDR].
% 5.2.5.4
% a. The supplier
