% !TEX root = MAIN.tex
\chapter{Software Product Assurance Plan}

\section{Software product assurance programme implementation}

\subsection{Organization}
\label{subsec:organization}
% a. The SPAP shall describe the organization of software product assurance activities, including responsibility, authority and the interrelation of personnel who manage, perform and verify work affecting software quality.
% b. The following topics shall be included:
% 1. organizational structure;
% 2. interfaces of each organisation, either external or internal, involved in the project;
% 3. relationship to the system level product assurance and safety;
% 4. independence of the software product assurance function;
% 5. delegation of software product assurance tasks to a lower level supplier, if any.

\subsection{Responsibilities}
\label{subsec:resp}

% The supplier shall identify the personnel responsible for software product assurance for the project (SW PA manager/engineer).

The SW PA manager is <Titolo Nome Cognome> from <Organizzazione>.

\subsection{Resources}
The human resources needed to ensure the project assurance are:
\begin{itemize}
  \item Software PA Manager as described in section~\ref{subsec:resp}%for example
\end{itemize}

The software needed to perform the related activities is listed in Table~\ref{table:software_resources}.

\begin{table}[H]
\centering
\begin{tabular}{||c|c|c|c||}
 \hline
 \textbf{Tool} & \textbf{Version} & \textbf{Purpose}\\
 \hline
 Git & version 2.30.1 & Version control and configuration management \\
 MS Office & version 16.54 & Report production and data analysis \\
 LaTeX & various & Report production \\
 \hline
\end{tabular}
\caption{Software resources for project assurance activities}
\label{table:software_resources}
\end{table}

\subsection{Reporting}
% The reporting related to product assurance processes includes:
SnT shall report on a regular basis on the status of the software product assurance programme implementation, as part of the overall reporting of the project.

\subsection{Quality models}
% a. The SPAP shall describe the quality models applicable to the project and how they are used to specify the quality requirements.

\subsection{Risk management}
% a. The SPAP shall describe the contribution of the software product assurance function to the project risk management.
<Nome Cognome>, the <Qualifica>, is responsible for maintaining the projects risks.


% \subsection{Supplier selection and control}
% a. The SPAP shall describe the contribution of the software product assurance function to the next level suppliers selection and control.

\subsection{Methods and tools}
% a. The SPAP shall describe the methods and tools used for all the activities of the development cycle, and their level of maturity.

The methods and tools used for all the activities of the development cycle are described in the project Software Development Plan, in Chapter~\ref{chapter:software_development}
The PA Manager ensures that:
\begin{itemize}
  \item the development team has adequate experience and training to make use of  the methods and tools.
  \item the tools and methods are appropriate for the characteristics of the product in development.
  \item the tools and relative hardware are available throughout the lifetime of the product.
  \item the methods and tools are correctly used.
\end{itemize}


\subsection{Operation and Maintenance}
For the maintenance phase a software maintenance plan shall be prepared.
The outline shall be described in Chapter~\ref{chapter:maintenance}.
