% !TEX root = MAIN.tex
\chapter{Software Verification Plan}
The purpose of this Software Verification Plan is to describe the approach and the organization of the software verification activitiess for the components of the FAQAS framework.

It addresses the follwing items:
\begin{itemize}
  \item the software products and life cycle activities
  subject to verification.
  \item the required verification tasks for each life cycle activity, software product, related resources, responsibilities, and schedule.
  \item the procedures for forwarding verification reports to the customer and other involved organizations.
\end{itemize}

\section{Software verification process overview}

\subsection{General}

The FAQAS Framework is comprised by category D classified SW so the verification activities shall be tailored to match this criticality level.

% a. The SVerP shall describe the approach to be utilized to implement the
% verification process throughout the software life cycle, the verification
% effort, and the level of independence for the verification tasks, as follows:
% NOTE 1 It is important to check the applicability of
% ECSS‐Q‐ST‐80 clause 5.3.1 (management of
% risks), 6.2.2 (software dependability and safety)
% and 6.2.6.13 (independent software verification
% and validation).
% NOTE 2 The verification effort is sized according to:
% • the potential for an undetected error in a
% system or software requirement to cause
% death or personal injury, mission failure, or
% financial or catastrophic equipment loss or
% damage;
% • the maturity of and risks associated with the
% software technology to be used;
% • availability of funds and resources.

\subsection{Organization}

% a. The SVerP shall describe the organization of the documentation review,
% proofs, and tracing activities.
% b. The following topics that shall be included:
% 1. roles;
% 2. reporting channels;
% 3. levels of authority for resolving problems;
% 4. organization relationships;
% 5. level of required and implemented independence.

\subsection{Master Schedule}

% a. A reference to the master schedule given in the software development
% plan shall be done.
% b. This SVerP shall describe the schedule for the planned verification
% activities.

\subsection{Resource summary}

% a. The SVerP shall summarize the resources to be used to perform the
% verification activities such as staff, hardware and software tools.

\subsection{Responsibilities}

% a. The SVerP shall describe the specific responsibilities.

\subsection{Identification of risks and level of independence.}

% a. The SVerP shall state (or refer to the SDP) the risks and level of
% independence.

\subsection{Tools, techniques and methods}

% a. The SVerP shall describe the software tools, techniques and methods
% used to execute the verification tasks throughout the software life cycle.

\subsection{Control procedures for verification process}

% a. The SVerP shall contain information (or reference to) about applicable
% management procedures concerning the following aspects:
% 1. problem reporting and resolution;
% 2. deviation and waiver policy;

\section{Verification activities}

\subsection{General}

% a. The SVerP shall address the verification activities of each software item.
% b. The SVerP shall address separately the activities to be performed for
% manually and automatically generated code.

\subsection{Software process verification}

% a. For each software process verification, the SVerP shall list:
% 1. the verification activities to be performed and how they are
% performed.
% 2. the required inputs to achieve the verification activities.
% 3. the intermediate and final outputs documenting the performed
% verification activities.
% 4. the methodologies, tools and facilities utilized to accomplish the
% verification activities.
% NOTE Examples of input and output are:
% • for software requirements (RB and TS) and
% architecture engineering:
% • input: draft SRS, draft software architectural
% design
% • output: software verification requirements
% report, software architectural design to
% requirements traceability
% • for software design and implementation
% engineering:
% • input: software components design, code,
% software user manual, software integration
% test plan
% • output: software code verification report,
% evaluation of software validation testing
% specification
% • for software delivery and acceptance:
% • input: software validation specification with
% respect to the requirements baseline,
% software acceptance testing documentation
% • output: software acceptance test report,
% software acceptance data package, problem
% reports, software release document,
% software configuration file
% • for software validation:
% • input: software validation specification with
% respect to the requirements baseline
% • output: software validation testing
% specifications

\subsection{Software quality requirements verification (as per ECSS‐QST‐80 clause 6.2.6.1)}

% <6.3.1> Activities
% a. The SVerP shall list the verification activities to be performed and how
% these are accomplished.
% NOTE Verification includes various techniques such as
% review, inspection, testing, walk‐through,
% cross‐reading, desk‐checking, model
% simulation, and many types of analysis such astraceability analysis, formal proof or fault tree
% analysis.
% <6.3.2> Inputs
% a. The SVerP shall list the required inputs to accomplish the verification
% activities.
% <6.3.3> Outputs
% a. The SVerP shall list the intermediate and final outputs documenting the
% performed verification activities.
% <6.3.4> Methodology, tools and facilities
% a. The SVerP shall describe the methodologies, tools and facilities utilized
% to accomplish the software quality requirements verification activities.
