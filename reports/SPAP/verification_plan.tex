% !TEX root = MAIN.tex
\chapter{Software Verification Plan}

% Activities for the verification of the quality requirements shall be specified in the definition of the verification plan.
% NOTE Verification includes various techniques such as review, inspection, testing, walk‐through, cross‐reading, desk‐checking, model simulation, and many types of analysis such as traceability analysis, formal proof or fault tree analysis.
% EXPECTED OUTPUT: Software verification plan [DJF, SVerP; PDR].

% a. Traceability matrices (as defined in ECSS‐E‐ST‐40 clause 5.8) shall be verified at each milestone.
% EXPECTED OUTPUT: Software product assurance milestone report [PAF, SPAMR; SRR, PDR, CDR, QR, AR, ORR].

% Questo capitolo é: Come facciamo la verfica della qualita del software. Loro lo fanno col code coverage.
% SS requirement engineering e sdd sono rivisti con pedro (project officer)
% SDD é stato rivisto col project officer
% coverage: since the faqas tool is not supposed to be run on sp[acecraft and done with different languages (pyhton etc) code coverage measurments are considered infeasible. verification of conformance with requirementes is based only on functional test cases which have bee derived systematically for the main components of the system. Namely the (i componenti che fanno la mutazione).


The purpose of this Software Verification Plan is to describe the approach and the organization of the software verification activitiess for the components of the FAQAS framework.

It addresses the follwing items:
\begin{itemize}
  \item the software products and life cycle activities
  subject to verification.
  \item the required verification tasks for each life cycle activity, software product, related resources, responsibilities, and schedule.
  \item the procedures for forwarding verification reports to the customer and other involved organizations.
\end{itemize}

\section{Software verification process overview}

\subsection{General}

The FAQAS Framework is comprised by category D classified SW so the verification activities shall be tailored to match this criticality level.

% a. The SVerP shall describe the approach to be utilized to implement the
% verification process throughout the software life cycle, the verification
% effort, and the level of independence for the verification tasks, as follows:
% NOTE 1 It is important to check the applicability of
% ECSS‐Q‐ST‐80 clause 5.3.1 (management of
% risks), 6.2.2 (software dependability and safety)
% and 6.2.6.13 (independent software verification
% and validation).
% NOTE 2 The verification effort is sized according to:
% • the potential for an undetected error in a
% system or software requirement to cause
% death or personal injury, mission failure, or
% financial or catastrophic equipment loss or
% damage;
% • the maturity of and risks associated with the
% software technology to be used;
% • availability of funds and resources.

\subsection{Organization}

The product assurance activities will be performed following the team organization defined in Chapter~\ref{chapter:organization}.

\subsection{Master Schedule}

The master schedule of the software develpment was outlined in Chapter~\ref{chapter:software_development}.

\subsection{Resource summary}

The resources used for the verifiication activities are reported in Section~\ref{sec:resources}.

\subsection{Responsibilities}

The roles and responsabilities linked to the activities described in this document are outlined in Section~\ref{sec:resp}.

% \subsection{Tools, techniques and methods}
%
% The software tools to be used for the verification activities are reported in Table~\ref{table:verification_tools}.
%
% \begin{table}[H]
% \centering
% \begin{tabular}{||c|c|c|c||}
%  \hline
%  \textbf{Tool} \\
%  \hline
%  Git version 2.30.1 \\
%  \hline
%  MS Office \\
%  \hline
%  LaTeX  \\
%  \hline
%  GNU GCC \\
%  \hline
%  Python 3.7 \\
%  \hline
%
% \end{tabular}
% \caption{Software resources for project assurance activities}
% \label{table:verification_tools}
% \end{table}

% \subsection{Control procedures for verification process}

\section{Verification activities}

\subsection{General}

This plan defines the verification activities to be performed on the FAQAS framework and its corresponding documentation.

\subsection{Software process verification}

% a. For each software process verification, the SVerP shall list:
% 1. the verification activities to be performed and how they are
% performed.
% 2. the required inputs to achieve the verification activities.
% 3. the intermediate and final outputs documenting the performed
% verification activities.
% 4. the methodologies, tools and facilities utilized to accomplish the
% verification activities.
% NOTE Examples of input and output are:
% • for software requirements (RB and TS) and
% architecture engineering:
% • input: draft SRS, draft software architectural
% design
% • output: software verification requirements
% report, software architectural design to
% requirements traceability
% • for software design and implementation
% engineering:
% • input: software components design, code,
% software user manual, software integration
% test plan
% • output: software code verification report,
% evaluation of software validation testing
% specification
% • for software delivery and acceptance:
% • input: software validation specification with
% respect to the requirements baseline,
% software acceptance testing documentation
% • output: software acceptance test report,
% software acceptance data package, problem
% reports, software release document,
% software configuration file
% • for software validation:
% • input: software validation specification with
% respect to the requirements baseline
% • output: software validation testing
% specifications

\subsubsection{Software Requirements Engineering}
The SSS was reviewed with the ESA Project Officer.

\subsubsection{Software Design and Implementation Engineering}
The SDD, SUM and SUTP were reviewed with the ESA Project Office.
The source code was verified throught unit and system testing.
Since the faqas tool is not supposed to be run on spacecrafts and it is realized with different programming languages (Python3, C, C++, Shell) code coverage measurments are considered impractical.
Verification of conformance with requirementes is based only on functional test cases which have bee derived systematically for the critical components of the system.

\subsubsection{Software Delivery and Acceptance}
The SRelD and SCF were reviewed with the ESA Project Officer.

\subsubsection{Software Validation}
The SVS and SValR were reviewed with the ESA Project Officer.

% \subsection{Software Quality Requirements Verification}


% <6.3.1> Activities
% a. The SVerP shall list the verification activities to be performed and how
% these are accomplished.
% NOTE Verification includes various techniques such as
% review, inspection, testing, walk‐through,
% cross‐reading, desk‐checking, model
% simulation, and many types of analysis such astraceability analysis, formal proof or fault tree
% analysis.
% <6.3.2> Inputs
% a. The SVerP shall list the required inputs to accomplish the verification
% activities.
% <6.3.3> Outputs
% a. The SVerP shall list the intermediate and final outputs documenting the
% performed verification activities.
% <6.3.4> Methodology, tools and facilities
% a. The SVerP shall describe the methodologies, tools and facilities utilized
% to accomplish the software quality requirements verification activities.
