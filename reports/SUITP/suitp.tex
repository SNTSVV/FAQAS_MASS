% !TEX root = MAIN.tex

\chapter{Software Unit Testing}

\section{Organization}
\label{sec:SUTSIT:org}

Unit tests are prepared by SnT in the form of executable Bash shell scripts. Test inputs are identified according to the procedures described in this document.

Since units are self-contained (i.e., they do not interact with other units but simply store results in files processed by other units), we avoid testing the integration of pairs of unit. Instead, we test the system as a whole, based on tutorial examples described in SUM. The system testing procedure is reported in the SVS - Software Validation Specification.

Testing is conducted by SnT personnel.

If any software problem occur during testing a development issue shall be raised in the GitLab Issue Tracker for the \FAQAS (see Section~\ref{chapter:control:procedure}), which shall be amended by SnT personnel.

%\section{Master schedule}


\section{Resource Summary}

Unit tests make use of the \FAQAS. Tests are performed in one hardware platform, a x86-64 desktop PC. Unit test execution time should not exceed one day for all target platforms combined.

\section{Responsibilities}

Unit tests tests are prepared by the SnT personnel.

The test cases are executed by an SnT specialist who has the responsibility of reporting any failure.

\section{Tool, techniques and methods}

Unit tests are specified in Bash files. Each unit test contains a launcher script that configures the environment for the correct execution of \FAQAS, and a source code example to mutate. The launcher script will invoke a dedicated operator Bash script, that will generate the corresponding mutants and will assess its results.


\section{Personnel and Personnel Training Requirements}

Unit testing is performed by a single person using a launcher script from the source folder of the SUT. Additional training is not needed.

\section{Risks and Contingencies}

The unit testing campaign is conducted in parallel with the development of the SUT and thus is not associated to any specific risk.
