% !TEX root = MAIN.tex

\chapter{MASS - Software Unit Test Procedures}

\section{General}

A single procedure has been identified to perform unit testing of MASS. It is described in the following.

\section{MASS Test Procedure}

\subsection{Test procedure Identifier}

\emph{MASS-TP-SRCMutation-1}


\subsection{Purpose}

This procedure aims to execute all the test cases reported in Table~\ref{table:matrix}.

\subsection{Procedure steps}

\subsubsection{Log}
%describe any special methods or format for logging the results of test execution, the incidents observed, and any other event pertinent to this test;
Since the test cases are executed within a console, the output of the test cases can be considered as is.
In case failures need to be reported, the console output should be copy-pasted.
\subsubsection{Set-up}
\emph{SRCMutation} comes with the test cases already set-up. The only precondition is to compile the software by running the command \emph{make}.
%: describe the sequence of actions to set up the procedure execution;
\subsubsection{Start}
%: describe the actions to begin the procedure execution; proceed: describe the actions during the procedure execution;
Test cases are executed by running the script \texttt{run\_unit\_tests.sh}.
\subsubsection{Test Result Acquisition}
Test results are printed out on the console.
%: describe how the test measurements is made;
\subsubsection{Shut Down}
Test cases are shut down through a SIGINT (i.e., pressing \texttt{CTRL-C})
%: describe the action to suspend testing when interruption is forced by unscheduled events;
\subsubsection{Restart}
Test cases can be restarted at any point by re-running the script \texttt{run\_unit\_tests.sh}.
%: identify any procedural restart points and describe the actions to restart the procedure at each of these points;
\subsubsection{Wrap-up}
The test execution terminates autonomously, no further action is needed.
%: describe the actions to terminate testing.
