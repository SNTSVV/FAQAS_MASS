% !TEX root = MAIN.tex

\chapter{DAMA - Software Unit Testing Approach}
\label{chap:approach_DAMA}


\section{Unit Testing Strategy}

% Unit testing aims to verify that the functional requirements of DAMA units are correctly implemented; test inputs are identified through the category-partition method.

Unit testing aims to verify that the functional requirements of DAMA units are correctly implemented; test inputs are identified through the pairwise combination testing method. Additional test cases were then added, focusing on inputs combinations of particular interest.


\section{Tasks and Items under Test}

Testing concerns the data-driven mutation API of DAMA, from now on defined as \emph{DDMutation}, which handles the \emph{Mutate Data} activity.

\section{Feature to be tested}

Testing concerns verifying the correct behavior of the data mutation operators implemented in \emph{DDMutation}, with regards to the instructions contained in the fault model and data model provided by the user, and to the C data type of the buffer targeted by the mutation.

\section{Feature not to be tested}

Testing does not concern the verification of the capability of \emph{DDMutation} to handle every possible numeric parameter that could be specified by the user in the fault model, since the number of possible combinations would render such a pursuit highly impractical if not impossible.
It is instead based on realistic use cases and inputs.
In the same way, testing do not cover every possible value of data that could be targeted by a mutation operator, but the focus is on testing the behaviour of the software with the supported C datatypes.


\section{Test Pass - Fail Criteria}

Unit testing pass if all the following are true:
\begin{itemize}
	\item Every test case is executed
	\item All test cases pass
	\item Exceptions and unexpected messages do not appear on screen and logs.
\end{itemize}



\section{Manually and Automatically Generated Code}

Part of the \emph{DDMutation} code is automatically generated for every specific \emph{FAQAS fault model} (see D2).
