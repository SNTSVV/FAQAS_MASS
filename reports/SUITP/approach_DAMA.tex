% !TEX root = MAIN.tex

\chapter{DAMAt - Software Unit Testing Approach}
\label{chap:approach_DAMAt}


\section{Unit Testing Strategy}

% Unit testing aims to verify that the functional requirements of DAMAt units are correctly implemented; test inputs are identified through the category-partition method.
Unit testing aims to verify that the functional requirements of DAMAt units are correctly implemented; test inputs are identified through the category partition method.
The pairwise combination testing method was used to test their integration.


\section{Tasks and Items under Test}

Testing concerns the data-driven mutation API of DAMAt, from now on defined as \emph{DDMutation}, which handles the \emph{Mutate Data} activity.
The API can be divided into two separate units: \emph{DDMutationData} which reads the target data from the buffer and write it back in once is mutated, and \emph{DDMutationFault} which applies the selected mutation operator to the data.

\section{Feature to be tested}

Testing concerns verifying the correct behavior of the \emph{DDMutationData} and \emph{DDMutationFault} units implemented in \emph{DDMutation}, with regards to the instructions contained in the fault model and data model provided by the user, and to the C data type of the buffer targeted by the mutation.

\section{Feature not to be tested}

Testing does not concern the verification of the capability of \emph{DDMutation} to handle every possible numeric parameter that could be specified by the user in the fault model and data model, since the number of possible combinations would render such a pursuit highly impractical if not impossible.
It is instead based on realistic use cases and inputs.
In the same way, testing do not cover every possible value that the buffer data that could be targeted by a mutation operator could assume.


\section{Test Pass - Fail Criteria}

\emph{DDMutation} passes its tests if all the following are true:
\begin{itemize}
	\item Every test case is executed
	\item All test cases pass
	\item Exceptions and unexpected messages do not appear on screen and logs.
\end{itemize}

\section{Manually and Automatically Generated Code}

Part of the \emph{DDMutation} code is automatically generated for every specific \emph{FAQAS fault model} (see D2). This code and the relative generation process are tested organically with the rest of the API.
