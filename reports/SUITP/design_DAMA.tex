% !TEX root = MAIN.tex

\chapter{DAMAt - Software Unit Test Case Design}
\label{chap:design_DAMAt}


\section{General}

The DAMAt test suite concerns the data-driven mutation API. It address its functional requirements. Two test designs have been identified. Two are based on the category partition method and target the units composing \emph{DDMutation}: \emph{DDMutationData} and \emph{DDMutationFault}.

\section{DAMAt - Test Design - Data-driven Mutation - DDMutationFault}

\subsection{Test design identifier}

The test design identifier is \emph{DAMAt-TD-DDMutation-1}

With this test design we aim to ensure that \emph{DDMutationFault}, which contains the implementation of the data-driven mutation operators, is implemented according to requirements of the \FAQAS.

\subsection{Features to be tested}

\emph{DDMutationFault} implements the data-driven mutation operators.

The set of data-driven mutation operators  is composed of the following fault classes: Value Above Threshold (VAT), Value Below Threshold (VBT), Value Out of Range (VOR), Fix Value Above Threshold (FVAT), Fix Value Below Threshold (FVBT), Fix Value Out of Range (FVOR),  Bit Flip (BF), Invalid Numeric Value (INV), Illegal Value (IV), Anomalous Signal Amplitude (ASA), Signal Shift (SS) and Hold Value (HV).
Table~\ref{table:damat:operators} shows the specifications for the mutation operators implemented by DAMAt.

% !TEX root =  ../MAIN.tex
% Please add the following required packages to your document preamble:
% \usepackage{booktabs}
\begin{table}[h!]

  \caption{Implemented set of data mutation operators.}
  \label{table:operators_DAMA}

\centering
\scriptsize
\resizebox{\textwidth}{!}{
\begin{tabular}{@{}lllll@{}}
\toprule
\textbf{Fault Class}                                     & \textbf{Types}                                                                                                                     & \textbf{Parameters}                                                                                                                                                                                                                    & \textbf{Description}                                                                                                                                                                                                                                                                                                                                                                                                           &  \\ \midrule
\multicolumn{1}{|l|}{Value   Above Threshold (VAT)}      & \multicolumn{1}{l|}{\begin{tabular}[c]{@{}l@{}}INT\\    \\ LONG\\    \\ FLOAT\\    \\ DOUBLE\\    \\ HEX\end{tabular}}             & \multicolumn{1}{l|}{\begin{tabular}[c]{@{}l@{}}T: threshold\\    \\ D: delta with respect to threshold\end{tabular}}                                                                                                                   & \multicolumn{1}{l|}{\begin{tabular}[c]{@{}l@{}}Data mutation operation: The mutation is   performed by replacing the\\    \\ current value (a number) with a value of   the same type that is equal to\\    \\ (T + D).\end{tabular}}                                                                                                                                                                                          &  \\ \cmidrule(r){1-4}
\multicolumn{1}{|l|}{Value   below threshold (VBT)}      & \multicolumn{1}{l|}{\begin{tabular}[c]{@{}l@{}}INT\\    \\ LONG\\    \\ FLOAT\\    \\ DOUBLE\\    \\ HEX\end{tabular}}             & \multicolumn{1}{l|}{\begin{tabular}[c]{@{}l@{}}T: threshold\\    \\ D: delta with respect to threshold\end{tabular}}                                                                                                                   & \multicolumn{1}{l|}{\begin{tabular}[c]{@{}l@{}}Data mutation operation: The mutation is   performed by replacing the\\    \\ current value (a number) with a value of   the same type that is equal to\\    \\ (T − D).\end{tabular}}                                                                                                                                                                                          &  \\ \cmidrule(r){1-4}
\multicolumn{1}{|l|}{Value out   of range (VOR)}         & \multicolumn{1}{l|}{\begin{tabular}[c]{@{}l@{}}INT\\    \\ LONG\\    \\ FLOAT\\    \\ DOUBLE\\    \\ HEX\end{tabular}}             & \multicolumn{1}{l|}{\begin{tabular}[c]{@{}l@{}}MIN: minimum valid value\\    \\ MAX: maximum valid value\\    \\ D: delta with respect to minimum/maximum   valid value\end{tabular}}                                                  & \multicolumn{1}{l|}{\begin{tabular}[c]{@{}l@{}}Data mutation operations (2): The   mutation is performed by replacing the\\    \\ current value (a number) with\\    \\ • a value of the same type that is equal   to (MIN − D)\\    \\ • a value of the same type that is equal   to (MAX + D)\end{tabular}}                                                                                                                  &  \\ \cmidrule(r){1-4}
\multicolumn{1}{|l|}{Bit flip (BF)}                      & \multicolumn{1}{l|}{BIN}                                                                                                           & \multicolumn{1}{l|}{\begin{tabular}[c]{@{}l@{}}MIN:   lower bit\\    \\ MAX:   higher bit\\    \\ STATE:   mutate only if the bit is in the given state\\    \\ VALUE:   integer specifying the number of bits to mutate\end{tabular}} & \multicolumn{1}{l|}{\begin{tabular}[c]{@{}l@{}}Data mutation operation: the operator   flips N randomly selected bit. If\\    \\ STATE is specified, the mutation is   applied only if the bit is in the specified\\    \\ state. Parameter VALUE specifies the   number of bits to mutate.\end{tabular}}                                                                                                                      &  \\ \cmidrule(r){1-4}
\multicolumn{1}{|l|}{Invalid numeric value (INV)}        & \multicolumn{1}{l|}{\begin{tabular}[c]{@{}l@{}}INT\\    \\ LONG\\    \\ FLOAT\\    \\ DOUBLE\\    \\ HEX\end{tabular}}             & \multicolumn{1}{l|}{\begin{tabular}[c]{@{}l@{}}MIN:   lower valid value\\    \\ MAX:   higher valid value\\    \\ D:   distribution to follow\\    \\ VALUE:   mean value for normal distribution\end{tabular}}                        & \multicolumn{1}{l|}{\begin{tabular}[c]{@{}l@{}}Data mutation operation: Mutation is   performed by replacing the current\\    \\ value with a different value randomly   sampled in the specified range. The\\    \\ parameter D specified the distribution   to follow when performing the mutation.\\    \\In our implementation 0 indicates   uniform, 1 indicates normal around the specified value (but in range).\end{tabular}} &  \\ \cmidrule(r){1-4}
\multicolumn{1}{|l|}{Illegal Value (IV)}                 & \multicolumn{1}{l|}{\begin{tabular}[c]{@{}l@{}}INT\\    \\ LONG\\    \\ FLOAT\\    \\ DOUBLE\\    \\ HEX\end{tabular}}             & \multicolumn{1}{l|}{VALUE: illegal value that is observed}                                                                                                                                                                             & \multicolumn{1}{l|}{\begin{tabular}[c]{@{}l@{}}Data mutation operation: Mutation is   performed by replacing the current\\    \\ value with the value VALUE, if different   than the current one.\end{tabular}}                                                                                                                                                                                                                &  \\ \cmidrule(r){1-4}
\multicolumn{1}{|l|}{Anomalous   Signal Amplitude (ASA)} & \multicolumn{1}{l|}{\begin{tabular}[c]{@{}l@{}}INT\\    \\ LONG\\    \\ FLOAT\\    \\ DOUBLE\\    \\ HEX\end{tabular}}             & \multicolumn{1}{l|}{\begin{tabular}[c]{@{}l@{}}T: change point\\    \\ D: delta to add/remove\\    \\ V: value to multiply\end{tabular}}                                                                                               & \multicolumn{1}{l|}{\begin{tabular}[c]{@{}l@{}}Data mutation operation: Mutation is   performed by replacing the current\\    \\ value (v) with the value (v′) computed   as follows: v=T + ((v − T) ∗ V) + D if v ≥ T v=T − ((T − v) ∗ V) − D if v \textless T\end{tabular}}                                                                                                                                                  &  \\ \cmidrule(r){1-4}
\multicolumn{1}{|l|}{Signal   Shift (SS)}                & \multicolumn{1}{l|}{\begin{tabular}[c]{@{}l@{}}INT\\    \\ LONG\\    \\ FLOAT\\    \\ DOUBLE\\    \\ HEX\end{tabular}}             & \multicolumn{1}{l|}{D: delta by which the signal should be   shifted}                                                                                                                                                                  & \multicolumn{1}{l|}{The value is modified by adding a value   D. It simulates an anomalous shift in the signal.}                                                                                                                                                                                                                                                                                                               &  \\ \cmidrule(r){1-4}
\multicolumn{1}{|l|}{Hold   Value (HV)}                  & \multicolumn{1}{l|}{\begin{tabular}[c]{@{}l@{}}INT\\    \\ LONG\\    \\ FLOAT\\    \\ DOUBLE\\    \\ HEX\\    \\ BIN\end{tabular}} & \multicolumn{1}{l|}{V: number of times to repeat the same   value}                                                                                                                                                                     & \multicolumn{1}{l|}{\begin{tabular}[c]{@{}l@{}}This operator keeps repeating an   observed value for V times. It emulates\\    \\ a constant signal replacing a signal   supposed to vary.\end{tabular}}                                                                                                                                                                                                                       &  \\ \bottomrule
\end{tabular}
}
\end{table}


Each data mutation operator performs data mutation by applying a data mutation operation (e.g., set a value above the upper range value).
Mutation operators might apply one or more data mutation operations.
Each data mutation operator can be configured with a set of parameters describing the type and charachteristics of the fault class. These parameters, provided by the user, specify the behaviour of the operators contained in the fault model with regards to the data described in the data model.

% The parameters are: \emph{Fault Model}, \emph{Data Item}, \emph{Span}, \emph{Type}, \emph{Fault Class}, \emph{Min}, \emph{Max}, \emph{Threshold}, \emph{Delta}, \emph{State} and \emph{Value}.
% \emph{Min, Max, Threshold, Delta, State} and \emph{Value} assume a different meaning depending on the Fault Class.
% \emph{DataItem, Span} and \emph{Type} describe the position, extension and type of the data targeted by the mutation.

\clearpage

\subsection{Approach refinements}

From the previous definitions, we derive the input classes to be used for the category partition method.
These are the inputs that affect the behaviour of \emph{DDMutationFault} in the most impactful way.
Iterating on their combinations provides the best trade-off between extensive coverage and efficiency.
They are \emph{Fault Class} and \emph{Data Type} and their possible values are represented below in tabular form (Table~\ref{table:ddmutation1_categories}).

\begin{table}[h!]
  \scriptsize
  \centering
  \caption{\emph{DDMutationFault} input categories and their values}
  \label{table:ddmutation1_categories}
\begin{tabular}{@{}ll@{}}
  \toprule
\textbf{Fault Class} & \textbf{Data Type} \\ \midrule
VAT           & INT       \\
VBT           & FLOAT     \\
VOR           & DOUBLE    \\
BF            & BIN       \\
INV           & HEX       \\
IV            & LONG      \\
ASA           &           \\
SS            &           \\
HV            &           \\ \bottomrule
\end{tabular}
\end{table}

\section{DAMAt - Test Design - Data-driven Mutation - DDMutationData}

\subsection{Test design identifier}
%
The test design identifier is \emph{DAMAt-TD-DDMutation-2}

With this test design we aim to ensure that \emph{DDMutationData}, which reads and writes the data targeted by the mutation into the buffer, is implemented according to requirements of the \FAQAS.
%
\subsection{Features to be tested}

\emph{DDMutationFault} implements the functions that read and write the data from the targeted buffer, which is implemented as an array of one of the supported C/C++ data types.
The data type of the elements of the array must be specified by the user when generating \emph{DDMutation}.
In the data model, the user specifies the number of elements of the array to consider using the \emph{Span} parameter and how the data will be casted prior of being mutated using the \emph{Type} parameter.


\subsection{Approach refinements}
%
The input categories to be used for the category partition method are \emph{Data Type}, \emph{Buffer Type} and \emph{Span}.
\emph{Data Type} is defined as the the value of the \emph{Type} parameter, as described in Table~\ref{table:damat:operators} while \emph{Buffer Type} is the C/C++ datatype used for implementing the buffer.

In the following, we provide the identified categories and their respective values in tabular form (Table~\ref{table:ddmutation2_categories}).

\begin{table}[h]
  \scriptsize
  \centering
  \caption{\emph{DDMutationData} input categories and their values}
  \label{table:ddmutation2_categories}
\begin{tabular}{@{}lll@{}}
\toprule
\textbf{Data Type} & \textbf{Buffer Type}   & \textbf{Span}   \\ \midrule
INT                & short int              & 1               \\
FLOAT              & unsigned short int     & \textgreater{}1 \\
DOUBLE             & unsigned int           &                 \\
BIN                & int                    &                 \\
HEX                & long int               &                 \\
LONG               & unsigned long int      &                 \\
                   & long long int          &                 \\
                   & unsigned long long int &                 \\
                   & signed char            &                 \\
                   & unsigned char          &                 \\
                   & float                  &                 \\
                   & double                 &                 \\ \bottomrule
\end{tabular}
\end{table}
