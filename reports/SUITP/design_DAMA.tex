% !TEX root = MAIN.tex

\chapter{DAMA - Software Unit Test Case Design}
\label{chap:design_DAMA}


\section{General}

% The DAMA unit test suite concerns the data-driven mutation component. It addresses its functional requirements. A single test design has been identified: it is based on the pairwise combination testing method and reported in the following sections.

The DAMA unit test suite concerns the data-driven mutation API. It address its functional requirements. A single test design has been identified: it is based on the category partition method and reported in the following sections.


\section{DAMA - Test Design - Data-driven Mutation - Operators}

\subsection{Test design identifier}

The test design identifier is \emph{DAMA-TD-DDMutation-1}

With this test design we aim to ensure that the data-driven mutation API template, which contains the data-driven mutation operators and performs the \emph{Mutate Data} activity, is implemented according to requirements of the \FAQAS.

\subsection{Features to be tested}

Table~\ref{table:operators_DAMA} shows the specifications for the mutation operators implemented by DAMA.

The set of data-driven mutation operators implemented in the DAMA API is composed of the following fault classes: Value Above Threshold (VAT), Value Below Threshold (VBT), Value Out of Range (VOR), Bit Flip (BF), Invalid Numeric Value (INV), Illegal Value (IV), Anomalous Signal Amplitude (ASA), Signal Shift (SS), Hold Value (HV), Array Swap (AS) and Array Random Swap (ARS).

% !TEX root =  ../MAIN.tex
% Please add the following required packages to your document preamble:
% \usepackage{booktabs}
\begin{table}[h!]

  \caption{Implemented set of data mutation operators.}
  \label{table:operators_DAMA}

\centering
\scriptsize
\resizebox{\textwidth}{!}{
\begin{tabular}{@{}lllll@{}}
\toprule
\textbf{Fault Class}                                     & \textbf{Types}                                                                                                                     & \textbf{Parameters}                                                                                                                                                                                                                    & \textbf{Description}                                                                                                                                                                                                                                                                                                                                                                                                           &  \\ \midrule
\multicolumn{1}{|l|}{Value   Above Threshold (VAT)}      & \multicolumn{1}{l|}{\begin{tabular}[c]{@{}l@{}}INT\\    \\ LONG\\    \\ FLOAT\\    \\ DOUBLE\\    \\ HEX\end{tabular}}             & \multicolumn{1}{l|}{\begin{tabular}[c]{@{}l@{}}T: threshold\\    \\ D: delta with respect to threshold\end{tabular}}                                                                                                                   & \multicolumn{1}{l|}{\begin{tabular}[c]{@{}l@{}}Data mutation operation: The mutation is   performed by replacing the\\    \\ current value (a number) with a value of   the same type that is equal to\\    \\ (T + D).\end{tabular}}                                                                                                                                                                                          &  \\ \cmidrule(r){1-4}
\multicolumn{1}{|l|}{Value   below threshold (VBT)}      & \multicolumn{1}{l|}{\begin{tabular}[c]{@{}l@{}}INT\\    \\ LONG\\    \\ FLOAT\\    \\ DOUBLE\\    \\ HEX\end{tabular}}             & \multicolumn{1}{l|}{\begin{tabular}[c]{@{}l@{}}T: threshold\\    \\ D: delta with respect to threshold\end{tabular}}                                                                                                                   & \multicolumn{1}{l|}{\begin{tabular}[c]{@{}l@{}}Data mutation operation: The mutation is   performed by replacing the\\    \\ current value (a number) with a value of   the same type that is equal to\\    \\ (T − D).\end{tabular}}                                                                                                                                                                                          &  \\ \cmidrule(r){1-4}
\multicolumn{1}{|l|}{Value out   of range (VOR)}         & \multicolumn{1}{l|}{\begin{tabular}[c]{@{}l@{}}INT\\    \\ LONG\\    \\ FLOAT\\    \\ DOUBLE\\    \\ HEX\end{tabular}}             & \multicolumn{1}{l|}{\begin{tabular}[c]{@{}l@{}}MIN: minimum valid value\\    \\ MAX: maximum valid value\\    \\ D: delta with respect to minimum/maximum   valid value\end{tabular}}                                                  & \multicolumn{1}{l|}{\begin{tabular}[c]{@{}l@{}}Data mutation operations (2): The   mutation is performed by replacing the\\    \\ current value (a number) with\\    \\ • a value of the same type that is equal   to (MIN − D)\\    \\ • a value of the same type that is equal   to (MAX + D)\end{tabular}}                                                                                                                  &  \\ \cmidrule(r){1-4}
\multicolumn{1}{|l|}{Bit flip (BF)}                      & \multicolumn{1}{l|}{BIN}                                                                                                           & \multicolumn{1}{l|}{\begin{tabular}[c]{@{}l@{}}MIN:   lower bit\\    \\ MAX:   higher bit\\    \\ STATE:   mutate only if the bit is in the given state\\    \\ VALUE:   integer specifying the number of bits to mutate\end{tabular}} & \multicolumn{1}{l|}{\begin{tabular}[c]{@{}l@{}}Data mutation operation: the operator   flips N randomly selected bit. If\\    \\ STATE is specified, the mutation is   applied only if the bit is in the specified\\    \\ state. Parameter VALUE specifies the   number of bits to mutate.\end{tabular}}                                                                                                                      &  \\ \cmidrule(r){1-4}
\multicolumn{1}{|l|}{Invalid numeric value (INV)}        & \multicolumn{1}{l|}{\begin{tabular}[c]{@{}l@{}}INT\\    \\ LONG\\    \\ FLOAT\\    \\ DOUBLE\\    \\ HEX\end{tabular}}             & \multicolumn{1}{l|}{\begin{tabular}[c]{@{}l@{}}MIN:   lower valid value\\    \\ MAX:   higher valid value\\    \\ D:   distribution to follow\\    \\ VALUE:   mean value for normal distribution\end{tabular}}                        & \multicolumn{1}{l|}{\begin{tabular}[c]{@{}l@{}}Data mutation operation: Mutation is   performed by replacing the current\\    \\ value with a different value randomly   sampled in the specified range. The\\    \\ parameter D specified the distribution   to follow when performing the mutation.\\    \\In our implementation 0 indicates   uniform, 1 indicates normal around the specified value (but in range).\end{tabular}} &  \\ \cmidrule(r){1-4}
\multicolumn{1}{|l|}{Illegal Value (IV)}                 & \multicolumn{1}{l|}{\begin{tabular}[c]{@{}l@{}}INT\\    \\ LONG\\    \\ FLOAT\\    \\ DOUBLE\\    \\ HEX\end{tabular}}             & \multicolumn{1}{l|}{VALUE: illegal value that is observed}                                                                                                                                                                             & \multicolumn{1}{l|}{\begin{tabular}[c]{@{}l@{}}Data mutation operation: Mutation is   performed by replacing the current\\    \\ value with the value VALUE, if different   than the current one.\end{tabular}}                                                                                                                                                                                                                &  \\ \cmidrule(r){1-4}
\multicolumn{1}{|l|}{Anomalous   Signal Amplitude (ASA)} & \multicolumn{1}{l|}{\begin{tabular}[c]{@{}l@{}}INT\\    \\ LONG\\    \\ FLOAT\\    \\ DOUBLE\\    \\ HEX\end{tabular}}             & \multicolumn{1}{l|}{\begin{tabular}[c]{@{}l@{}}T: change point\\    \\ D: delta to add/remove\\    \\ V: value to multiply\end{tabular}}                                                                                               & \multicolumn{1}{l|}{\begin{tabular}[c]{@{}l@{}}Data mutation operation: Mutation is   performed by replacing the current\\    \\ value (v) with the value (v′) computed   as follows: v=T + ((v − T) ∗ V) + D if v ≥ T v=T − ((T − v) ∗ V) − D if v \textless T\end{tabular}}                                                                                                                                                  &  \\ \cmidrule(r){1-4}
\multicolumn{1}{|l|}{Signal   Shift (SS)}                & \multicolumn{1}{l|}{\begin{tabular}[c]{@{}l@{}}INT\\    \\ LONG\\    \\ FLOAT\\    \\ DOUBLE\\    \\ HEX\end{tabular}}             & \multicolumn{1}{l|}{D: delta by which the signal should be   shifted}                                                                                                                                                                  & \multicolumn{1}{l|}{The value is modified by adding a value   D. It simulates an anomalous shift in the signal.}                                                                                                                                                                                                                                                                                                               &  \\ \cmidrule(r){1-4}
\multicolumn{1}{|l|}{Hold   Value (HV)}                  & \multicolumn{1}{l|}{\begin{tabular}[c]{@{}l@{}}INT\\    \\ LONG\\    \\ FLOAT\\    \\ DOUBLE\\    \\ HEX\\    \\ BIN\end{tabular}} & \multicolumn{1}{l|}{V: number of times to repeat the same   value}                                                                                                                                                                     & \multicolumn{1}{l|}{\begin{tabular}[c]{@{}l@{}}This operator keeps repeating an   observed value for V times. It emulates\\    \\ a constant signal replacing a signal   supposed to vary.\end{tabular}}                                                                                                                                                                                                                       &  \\ \bottomrule
\end{tabular}
}
\end{table}


Each data mutation operator performs data mutation by applying a data mutation operation (e.g., set a value above the upper range value).
Mutation operators might apply one or more data mutation operations.
Each data mutation operator can be configured with a set of parameters describing the type and charachteristics of the fault class. These parameters, provided by the user, specify the behaviour of the operators contained in the fault model with regards to the data described in the data model.
The parameters are: \emph{Fault Model}, \emph{Data Item}, \emph{Span}, \emph{Type}, \emph{Fault Class}, \emph{Min}, \emph{Max}, \emph{Threshold}, \emph{Delta}, \emph{State} and \emph{Value}.
\emph{Min, Max, Threshold, Delta, State} and \emph{Value} assume a different meaning depending on the Fault Class.
\emph{DataItem, Span} and \emph{Type} describe the position, extension and type of the data targeted by the mutation.

\clearpage

\subsection{Approach refinements}

% From the previous definitions we derive the input classes to be used for the pairwise method.
% They are \emph{Fault Class}, \emph{Data Type}, \emph{Buffer Type} and \emph{Span}.
%
% We provide the identified class values in tabular form (Table~\ref{table:classes_DAMA}) and also the applied constraints (Table~\ref{table:constraints_1_DAMA}, Table~\ref{table:constraints_2_DAMA} and Table~\ref{table:constraints_3_DAMA})
%
%
% % !TEX root =  ../MAIN.tex

% Please add the following required packages to your document preamble:
% \usepackage{booktabs}
\begin{table}[h]

  \scriptsize
  \centering
  \caption{DAMA chosen input classes.}
  \label{table:classes_DAMA}

\begin{tabular}{@{}llll@{}}
\toprule
\textbf{Fault   Class} & \textbf{Data Type} & \textbf{Buffer Type}   & \textbf{Span}   \\ \midrule
VAT                    & INT                & short int              & 1               \\
VBT                    & FLOAT              & unsigned short int     & \textgreater{}1 \\
VOR                    & DOUBLE             & unsigned int           &                 \\
BF                     & BIN                & int                    &                 \\
INV                    & HEX                & long int               &                 \\
IV                     & LONG               & unsigned long int      &                 \\
ASA                    &                    & long long int          &                 \\
SS                     &                    & unsigned long long int &                 \\
HV                     &                    & signed char            &                 \\
                       &                    & unsigned char          &                 \\
                       &                    & float                  &                 \\
                       &                    & double                 &                 \\ \bottomrule
\end{tabular}
\end{table}

%
% % !TEX root =  ../MAIN.tex


\begin{table}[ht]

  \scriptsize
  \centering
  \caption{Contraints for pairwise combination testing between the Fault Class and Span input classes.}
  \label{table:constraints_1_DAMA}

  \begin{tabular}{@{}lll@{}}
  \textbf{Buffer   Type} & \textbf{Constraint} & \textbf{Span}   \\ \midrule
  long long int          & cannot exist with       & \textgreater{}1 \\
  u long long int        & cannot exist with       & \textgreater{}1 \\
  double                 & cannot exist with       & \textgreater{}1 \\
  long int               & cannot exist with       & \textgreater{}1 \\
  u long int             & cannot exist with       & \textgreater{}1 \\
  float                  & cannot exist with       & \textgreater{}1 \\
  double                 & cannot exist with       & \textgreater{}1 \\ \bottomrule
  \end{tabular}

\end{table}


\begin{table}[ht]

\scriptsize
\centering
\caption{Contraints for pairwise combination testing between the Buffer Type and Data Type input classes.}
\label{table:constraints_2_DAMA}

\begin{tabular}{@{}lll@{}}
\textbf{Fault Class} & \textbf{Constraint} & \textbf{Data type} \\ \midrule
vat                  & cannot exist with with  & bin                \\
vbt                  & cannot exist with with  & bin                \\
vor                  & cannot exist with with  & bin                \\
inv                  & cannot exist with with  & bin                \\
iv                   & cannot exist with with  & bin                \\
asa                  & cannot exist with with  & bin                \\
ss                   & cannot exist with with  & bin                \\
bf                   & can exist only with & bin                \\ \bottomrule
\end{tabular}

\end{table}


\begin{table}[ht]

\scriptsize
\centering
\caption{Contraints for pairwise combination testing between the Buffer Type and Data Type input classes.}
\label{table:constraints_3_DAMA}

\begin{tabular}{@{}lll@{}}
\textbf{Buffer   Type} & \textbf{Constraint} & \textbf{Data Type} \\ \midrule
float                  & can exist only with     & float              \\
double                 & can exist only with     & double             \\ \bottomrule
\end{tabular}

\end{table}


From the previous definitions we derive the input classes to be used for the category partition method.
These are the input that affects the behaviour of the program in the most impactful way.
Iterating on their combinations provides the best trade-off between extensive coverage and efficiency.
They are \emph{Fault Class}, \emph{Data Type}, \emph{Buffer Type} and \emph{Span}.
We provide the identified class values in tabular form (Table~\ref{table:classes_DAMA})

% !TEX root =  ../MAIN.tex

% Please add the following required packages to your document preamble:
% \usepackage{booktabs}
\begin{table}[h]

  \scriptsize
  \centering
  \caption{DAMA chosen input classes.}
  \label{table:classes_DAMA}

\begin{tabular}{@{}llll@{}}
\toprule
\textbf{Fault   Class} & \textbf{Data Type} & \textbf{Buffer Type}   & \textbf{Span}   \\ \midrule
VAT                    & INT                & short int              & 1               \\
VBT                    & FLOAT              & unsigned short int     & \textgreater{}1 \\
VOR                    & DOUBLE             & unsigned int           &                 \\
BF                     & BIN                & int                    &                 \\
INV                    & HEX                & long int               &                 \\
IV                     & LONG               & unsigned long int      &                 \\
ASA                    &                    & long long int          &                 \\
SS                     &                    & unsigned long long int &                 \\
HV                     &                    & signed char            &                 \\
                       &                    & unsigned char          &                 \\
                       &                    & float                  &                 \\
                       &                    & double                 &                 \\ \bottomrule
\end{tabular}
\end{table}

