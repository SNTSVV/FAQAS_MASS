% !TEX root = MAIN.tex

\chapter{DAMAt - Software Unit Test Case Design}
\label{chap:design_DAMAt}


\section{General}

The DAMAt test suite concerns the data-driven mutation API. It address its functional requirements. Two test designs have been identified. Two are based on the category partition method and target the units composing \emph{DDMutation}: \emph{DDMutationData} and \emph{DDMutationFault}.

\section{DAMAt - Test Design - Data-driven Mutation - DDMutationFault}

\subsection{Test design identifier}

The test design identifier is \emph{DAMAt-TD-DDMutation-1}

With this test design we aim to ensure that \emph{DDMutationFault}, which contains the implementation of the data-driven mutation operators, is implemented according to requirements of the \FAQAS.

\subsection{Features to be tested}

\emph{DDMutationFault} implements the data-driven mutation operators.

The set of data-driven mutation operators  is composed of the following fault classes: Value Above Threshold (VAT), Value Below Threshold (VBT), Value Out of Range (VOR), Fix Value Above Threshold (FVAT), Fix Value Below Threshold (FVBT), Fix Value Out of Range (FVOR),  Bit Flip (BF), Invalid Numeric Value (INV), Illegal Value (IV), Anomalous Signal Amplitude (ASA), Signal Shift (SS) and Hold Value (HV).
Table~\ref{table:damat:operators} shows the specifications for the mutation operators implemented by DAMAt.

% !TEX root = ../MAIN.tex

\newcommand{\MINIPM}{3cm}
\newcommand{\MINIPW}{10cm}

\begin{table*}[h]
\caption{Data-driven mutation operators}
\label{table:operators}
\scriptsize
\begin{tabular}{|p{15mm}|p{10mm}|p{3cm}|p{10cm}|}
\hline
\textbf{Fault Class}&\textbf{Types}&\textbf{Parameters}&\textbf{Description}\\
\hline
Value above threshold (VAT)&
I,L,F,D,H
&
\begin{minipage}{\MINIPM}
T: threshold\\
\D: delta, difference with respect to threshold\\
\end{minipage}
&
\begin{minipage}{\MINIPW}
Replaces the current value with a value above the threshold T for a delta (\D). It simulates a value that is out of the nominal case and shall trigger a response from the system that shall be verified by the test case (e.g., the system may continue working but an alarm shall be triggered). Not applied if the value is already above the threshold.

\EMPH{Data mutation procedure:}
$v' =  (T+\Delta)   (\mathit{if} v \le T); v' =  v   (\mathit{otherwise})$;

%\EMPH{Data mutation procedure:}
%\[
%v' =
%    \begin{cases}
%      (T+D)    & \mathit{if} v \le T\\
%      v    & \mathit{otherwise}\\
%    \end{cases}
%\]

\end{minipage}
\\



\hline
Value below threshold (VBT)&
I,L,F,D,H
&
\begin{minipage}{\MINIPM}
T: threshold\\
\D: delta, difference with respect to threshold\\
\end{minipage}
&
\begin{minipage}{\MINIPW}
Replaces the current value with a value below the threshold T for a delta (\D). It simulates a value that is out of the nominal case and shall trigger a response from the system that shall be verified by the test case (e.g., the system may continue working but an alarm shall be triggered). Not applied if the value is already below the threshold.

\EMPH{Data mutation procedure:}
$v' =  (T-\Delta)  (\mathit{if} v \ge T); v' = v    (\mathit{otherwise})$

%\EMPH{Data mutation procedure:}
%\[
%v' =
%    \begin{cases}
%      (T-D)    & \mathit{if} v \ge T\\
%      v    & \mathit{otherwise}\\
%    \end{cases}
%\]
\end{minipage}
\\



\hline
Value out of range (VOR)&
I,L,F,D,H
&
\begin{minipage}{\MINIPM}
MIN: minimum valid value\\
MAX: maximum valid value\\
\D: delta, difference with respect to minimum/maximum valid value
\end{minipage}
&
\begin{minipage}{\MINIPW}
Replaces the current value with a value out of the range $[MIN;MAX]$. It simulates a value that is out of the nominal range and shall trigger a response from the system that shall be verified by the test case (e.g., the system may continue working but an alarm shall be triggered). Not applied if the value is already out of range.

\EMPH{Data mutation procedure 1:}
$v' =  (MIN-\Delta)    (\mathit{if} MIN \le v \le MAX); v' = v   (\mathit{otherwise})$\\

\EMPH{Data mutation procedure 2:}
$v' = (MAX+\Delta)   (\mathit{if} MIN \le v \le MAX); v' = v   (\mathit{otherwise})$\\

%\EMPH{Data mutation procedure 1:}
%\[
%v' =
%    \begin{cases}
%      (MIN-D)    & \mathit{if} MIN \le v \le MAX\\
%      v    & \mathit{otherwise}\\
%    \end{cases}
%\]
%
%\EMPH{Data mutation procedure 2:}
%\[
%v' =
%    \begin{cases}
%      (MAX+D)    & \mathit{if} MIN \le v \le MAX\\
%      v    & \mathit{otherwise}\\
%    \end{cases}
%\]

\end{minipage}
\\






\hline
Bit flip (BF)&
B
&
\begin{minipage}{\MINIPM}
MIN: lower bit\\
MAX: higher bit\\
STATE: mutate only if the bit is in the given state (i.e., 0 or 1). \\
VALUE: number of bits to mutate\\
\end{minipage}
&
\begin{minipage}{\MINIPW}
A number of bits randomly chosen in the positions between MIN and MAX (included) are flipped.
If STATE is specified, the mutation is applied only if  the bit is in the specified state; the value $-1$ indicates that any state shall be considered for mutation. Parameter VALUE specifies the number of bits to mutate.

\EMPH{Data mutation procedure:} the operator flips VALUE randomly selected bit if they are in the specified state.

\end{minipage}
\\

\hline
Invalid numeric value (INV)&
I,L,F,D,H
&
\begin{minipage}{\MINIPM}
MIN: lower valid value\\
MAX: higher valid value\\
\end{minipage}
&
\begin{minipage}{\MINIPW}
Replace the current value with a mutated value that is legal (i.e., in the specified range) but different than current value. It simulates the exchange of data that is not consistent with the state of the system.

\EMPH{Data mutation procedure:} Replace the current value with a different value randomly sampled in the specified range.
\end{minipage}
\\

\hline
Illegal Value (IV)
&
I,L,F,D,H
&
\begin{minipage}{\MINIPM}
VALUE: illegal value that is observed\\
\end{minipage}
&
\begin{minipage}{\MINIPW}
Replace the current value with a value that is equal to the parameter \emph{VALUE}.

\EMPH{Data mutation procedure:}
$v' = \mathit{VALUE}    (\mathit{if} v \ne \mathit{VALUE}); v' = v   (\mathit{otherwise})$\\

%\EMPH{Data mutation procedure:}
%\[
%v' =
%    \begin{cases}
%      \mathit{VALUE}    & \mathit{if} v \ne \mathit{VALUE}\\
%      v    & \mathit{otherwise}\\
%    \end{cases}
%\]
\end{minipage}
\\

\hline
Anomalous Signal Amplitude (ASA)
&
I,L,F,D,H
&
\begin{minipage}{\MINIPM}
T: change point\\
\D: delta, value to add/remove\\
VALUE: value to multiply\\
\end{minipage}
&
\begin{minipage}{\MINIPW}
The mutated value is derived by amplifying the observed value by a factor \emph{V} and by adding/removing a constant value \D from it. It is used to either amplify or reduce a signal in a constant manner to simulate unusual signals. The parameter \emph{T} indicates the observed value below which instead of adding  we subtract .

\EMPH{Data mutation procedure:}
$v' =  T+(  (v-T)*\mathit{VALUE}  ) + \Delta  (\mathit{if}\ v \ge T); v' = T - (  (T-v)*\mathit{VALUE}  ) - \Delta   (\mathit{if}\ v < T)$;
\end{minipage}

%\EMPH{Data mutation procedure:}
%\[
%v' =
%    \begin{cases}
%      T+(  (v-T)*\mathit{VALUE}  ) + \mathit{D}    & \mathit{if}\ v \ge T\\
%      T - (  (T-v)*\mathit{VALUE}  ) - \mathit{D}   & \mathit{if}\ v < T
%    \end{cases}
%\]
%\end{minipage}
\\


\hline
Signal Shift (SS)
&
I,L,F,D,H
&
\begin{minipage}{\MINIPM}
\D: delta, value by which the signal should be shifted\\
\end{minipage}
&
\begin{minipage}{\MINIPW}
The mutated value is derived by adding a value \D to the observed value. It simulates an anomalous shift in the signal.

\EMPH{Data mutation procedure:}
$v' = v + \Delta$
\end{minipage}
\\





\hline
Hold Value (HV)
&
\begin{minipage}{\MINIPW}
I,L,F,D,H
\end{minipage}
&
\begin{minipage}{\MINIPM}
V: number of times to repeat the same value\\
\end{minipage}
&
\begin{minipage}{\MINIPW}
This operator keeps repeating an observed value for $V$ times. It emulates a constant signal replacing a signal supposed to vary.

\EMPH{Data mutation procedure:}
$v' = \mathit{previous}\  v'   (\mathit{if}\ \mathit{counter} \le V) ; v' = v  \mathit{otherwise}$

%\EMPH{Data mutation procedure:}
%\[
%v' =
%    \begin{cases}
%      \mathit{previous}\  v'   & \mathit{if}\ \mathit{counter} \le V\\
%      v  & \mathit{otherwise}\
%    \end{cases}
%\]
\end{minipage}
\\

\hline
Fix value above threshold (FVAT)&
I,L,F,D,H
&
\begin{minipage}{\MINIPM}
T: threshold\\
\D: delta, difference with respect to threshold\\
\end{minipage}
&
\begin{minipage}{\MINIPW}
It is the complement of VAT and implements the same mutation procedure as VBT but we named it differently because it has a different purpose. Indeed, it is used to verify that test cases exercising exceptional cases are verified correctly. In the presence of a value above the threshold, it replaces the current value with a value below the threshold T for a delta \D.

\EMPH{Data mutation procedure:}
$v' =  v    (\mathit{if} v > T) ; V' = (T-\Delta)    (\mathit{otherwise})$

%\EMPH{Data mutation procedure:}
%\[
%v' =
%    \begin{cases}
%      v    & \mathit{if} v > T\\
%      (T-D)    & \mathit{otherwise}\\
%    \end{cases}
%\]

\end{minipage}
\\

\hline
Fix value below threshold (FVBT)&
I,L,F,D,H
&
\begin{minipage}{\MINIPM}
T: threshold\\
\D: delta, difference with respect to threshold\\
\end{minipage}
&
\begin{minipage}{\MINIPW}
It is the counterpart of FVAT for the operator VBT.
% and implements the same mutation operation as VAT but we named it differently because it has a different purpose. Indeed, it is used to verify that test cases exercising exceptional cases are verified correctly. In the presence of a value below the threshold it replaces the current value with a value above the threshold T for a delta D.

\EMPH{Data mutation procedure:}
$v' = v   (\mathit{if} v < T); v' = (T+\Delta) (\mathit{otherwise})$\\

%\EMPH{Data mutation procedure:}
%\[
%v' =
%    \begin{cases}
%      v    & \mathit{if} v < T\\
%      (T+D)    & \mathit{otherwise}\\
%    \end{cases}
%\]

\end{minipage}
\\



\hline
Fix value out of range (FVOR)&
I,L,F,D,H
&
\begin{minipage}{\MINIPM}
MIN: minimum valid value\\
MAX: maximum valid value\\
\end{minipage}
&
\begin{minipage}{\MINIPW}
It is the complement of VOR and implements the same mutation procedure as INV but we named it differently because it has a different purpose. Indeed, it is used to verify that test cases exercising exceptional cases are verified correctly.
%In the presence of a value out of the range  $[MIN;MAX]$ it replaces the current value with a random value within the range.

\EMPH{Data mutation procedure:}
$v' = v   (\mathit{if} MIN \le v \le MAX); v' = \mathit{random(MIN,MAX)}   (\mathit{otherwise})$

%\EMPH{Data mutation procedure:}
%\[
%v' =
%    \begin{cases}
%      v    & \mathit{if} MIN \le v \le MAX\\\\
%      \mathit{random(MIN,MAX)}    & \mathit{otherwise}\\
%    \end{cases}
%\]

\end{minipage}
\\

%\hline
%\TRFOUR{Array Swap (AS)}
%&
%\begin{minipage}{\MINIPW}
%ARRAY\_*\\
%\end{minipage}
%&
%\begin{minipage}{\MINIPW}
%MIN: position of element A\\
%MAX: position of element B\\
%VALUE: number of elements to move\\
%\end{minipage}
%&
%\begin{minipage}{\MINIPW}
%Replace a number of elements (number specified by VALUE) located starting from position MIN, with a number of elements located starting from position MAX, and viceversa.
%\EMPH{Data mutation procedure:} Mutation is performed by replacing the two set of elements with each other.
%\end{minipage}
%\\
%
%
%\hline
%\TRFOUR{Array Random Swap (ARS)}
%&
%\begin{minipage}{\MINIPW}
%ARRAY\_*\\
%\end{minipage}
%&
%\begin{minipage}{\MINIPW}
%MIN: min position of element A/B\\
%MAX: max position of element A/B\\
%VALUE: number of elements to move\\
%\end{minipage}
%&
%\begin{minipage}{\MINIPW}
%Replace a number of elements (number specified by VALUE) located in a position between MIN and MAX, with a number of elements located in a position between MIN and MAX. MIN and MAX specify a position with respect to the beginning and end of the array.  For example, MIN=0 indicates the first element of teh array, MIN=-2 indicates the second element of the array.
%\EMPH{Data mutation procedure:} Mutation is performed by replacing the two set of elements with each other.
%\end{minipage}
%\\



%Incorrect Identifier& Several transmission data fields have fixed values, for example fields identifying the transmitting satellite. Hardware/software errors may assign incorrect identifiers.\\
%%Incorrect Checksum& Hardware/software errors may result in an incorrect checksum for a Packet or VCDU.\\
%Incorrect Counter& Counters are used to track Packet or VCDU ordering. Hardware/software errors may assign incorrect counter values.\\
%Flipped Data Bits& Physical channel noise may flip one or more bits in the data transmission.\\
\hline
\end{tabular}
\textbf{Legend:} I: INT, L: LONG INT, F: FLOAT, D: DOUBLE, B: BIN, H: HEX


\end{table*}%


Each data mutation operator performs data mutation by applying a data mutation operation (e.g., set a value above the upper range value).
Mutation operators might apply one or more data mutation operations.
Each data mutation operator can be configured with a set of parameters describing the type and charachteristics of the fault class. These parameters, provided by the user, specify the behaviour of the operators contained in the fault model with regards to the data described in the data model.

% The parameters are: \emph{Fault Model}, \emph{Data Item}, \emph{Span}, \emph{Type}, \emph{Fault Class}, \emph{Min}, \emph{Max}, \emph{Threshold}, \emph{Delta}, \emph{State} and \emph{Value}.
% \emph{Min, Max, Threshold, Delta, State} and \emph{Value} assume a different meaning depending on the Fault Class.
% \emph{DataItem, Span} and \emph{Type} describe the position, extension and type of the data targeted by the mutation.

\clearpage

\subsection{Approach refinements}

From the previous definitions, we derive the input classes to be used for the category partition method.
These are the inputs that affect the behaviour of \emph{DDMutationFault} in the most impactful way.
Iterating on their combinations provides the best trade-off between extensive coverage and efficiency.
They are \emph{Fault Class} and \emph{Data Type} and their possible values are represented below in tabular form (Table~\ref{table:ddmutation1_categories}).

\begin{table}[h!]
  \scriptsize
  \centering
  \caption{\emph{DDMutationFault} input categories and their values}
  \label{table:ddmutation1_categories}
\begin{tabular}{@{}ll@{}}
  \toprule
\textbf{Fault Class} & \textbf{Data Type} \\ \midrule
VAT           & INT       \\
VBT           & FLOAT     \\
VOR           & DOUBLE    \\
BF            & BIN       \\
INV           & HEX       \\
IV            & LONG      \\
ASA           &           \\
SS            &           \\
HV            &           \\ \bottomrule
\end{tabular}
\end{table}

\section{DAMAt - Test Design - Data-driven Mutation - DDMutationData}

\subsection{Test design identifier}
%
The test design identifier is \emph{DAMAt-TD-DDMutation-2}

With this test design we aim to ensure that \emph{DDMutationData}, which reads and writes the data targeted by the mutation into the buffer, is implemented according to requirements of the \FAQAS.
%
\subsection{Features to be tested}

\emph{DDMutationFault} implements the functions that read and write the data from the targeted buffer, which is implemented as an array of one of the supported C/C++ data types.
The data type of the elements of the array must be specified by the user when generating \emph{DDMutation}.
In the data model, the user specifies the number of elements of the array to consider using the \emph{Span} parameter and how the data will be casted prior of being mutated using the \emph{Type} parameter.


\subsection{Approach refinements}
%
The input categories to be used for the category partition method are \emph{Data Type}, \emph{Buffer Type} and \emph{Span}.
\emph{Data Type} is defined as the the value of the \emph{Type} parameter, as described in Table~\ref{table:damat:operators} while \emph{Buffer Type} is the C/C++ datatype used for implementing the buffer.

In the following, we provide the identified categories and their respective values in tabular form (Table~\ref{table:ddmutation2_categories}).

\begin{table}[h]
  \scriptsize
  \centering
  \caption{\emph{DDMutationData} input categories and their values}
  \label{table:ddmutation2_categories}
\begin{tabular}{@{}lll@{}}
\toprule
\textbf{Data Type} & \textbf{Buffer Type}   & \textbf{Span}   \\ \midrule
INT                & short int              & 1               \\
FLOAT              & unsigned short int     & \textgreater{}1 \\
DOUBLE             & unsigned int           &                 \\
BIN                & int                    &                 \\
HEX                & long int               &                 \\
LONG               & unsigned long int      &                 \\
                   & long long int          &                 \\
                   & unsigned long long int &                 \\
                   & signed char            &                 \\
                   & unsigned char          &                 \\
                   & float                  &                 \\
                   & double                 &                 \\ \bottomrule
\end{tabular}
\end{table}
