% !TEX root = MAIN.tex

\chapter{MASS - Software Unit Test Case Specification}
\label{chap:spec}

\section{General}




Table~\ref{table:matrix} provides the list of unit test cases derived based on the procedure described in Chapter~\ref{chap:spec}. 
Column \emph{Term to replace} indicates the operator appearing in the line to be mutated. When a test case is supposed to mutate many terms, we report all of them.
Column \emph{Replacements} provides, for each operator, the expected replacements.
Note that \emph{SRCMutation} shall generate one distinct mutant for each element of the replacement column.



% !TEX root =  ../MAIN.tex

\setlength\LTleft{0pt}
\setlength\LTright{0pt}
\scriptsize
\begin{longtable}{|p{1cm}|p{5cm}|p{6cm}|p{2.5cm}|}

% \begin{table}[h]
% \scriptsize
% \centering
\caption{Unit test cases for SRCMutation.}
\label{table:matrix} \\

% \begin{tabular}{|llp{6cm}l|}
\hline 
\textbf{Operator}	&	\textbf{Term to replace}	&	\textbf{Replacements}	&	\textbf{Test Case} \\
\hline 
ABS	&	$a$	&	$-a$	&	abs\_val.sh \\
ABS	&	$a,b$	&	$-a,-b$	&	abs\_many.sh \\
AOR	&	$+$	&	$\{-,*,/,\%\}$	&	aor\_plus.sh \\
AOR	&	$-$	&	$\{+,*,/,\%\}$	&	aor\_minus.sh \\
AOR	&	$*$	&	$\{+,-,/,\%\}$	&	aor\_mult.sh \\
AOR	&	$/$	&	$\{+,-,*,\%\}$	&	aor\_div.sh \\
AOR	&	$\%$	&	$\{+,-,*,/\}$	&	aor\_mod.sh \\
AOR	&	$+=$	&	$\{-=,*=,/=,\%=\}$	&	aor\_plus\_assign.sh \\
AOR	&	$-=$	&	$\{+=,*=,/=,\%=\}$	&	aor\_minus\_assign.sh \\
AOR	&	$*=$	&	$\{+=,-=,/=,\%=\}$	&	aor\_mult\_assign.sh \\
AOR	&	$/=$	&	$\{+=,-=,*=,\%=\}$	&	aor\_div\_assign.sh \\
AOR	&	$\%=$	&	$\{+=,-=,*=,/=\}$	&	aor\_mod\_assign.sh \\
AOR	&	$+,-$	&	$\{+,-,*,\%\}$	&	aor\_many.sh \\
ICR	&	$10$	&	$\{1, -1, 0, 11, 9, -10\}$	&	icr\_val.sh \\
ICR	&	$10 + 1$	&	$\{1+1, -1+1, 10-1, 0+1, 10+0, 10+1, 10+2, 10-1+1, -10+1\}$	&	icr\_many.sh \\
LCR	&	$\&\&$	&	$||$	&	lcr\_logic\_or.sh \\
LCR	&	$||$	&	$\&\&$	&	lcr\_logic\_and.sh \\
LCR	&	$\&$	&	$\{|,\land\}$	&	lcr\_and.sh \\
LCR	&	$|$	&	$\{\&,\land\}$	&	lcr\_or.sh \\
LCR	&	$\land$	&	$\{\&,|\}$	&	lcr\_xor.sh \\
LCR	&	$\&=$	&	$\{|=, \land=\}$	&	lcr\_and\_assign.sh \\
LCR	&	$|=$	&	$\{\&=, \land=\}$	&	lcr\_or\_assign.sh \\
LCR	&	$\land=$	&	$\{\&=, |=\}$	&	lcr\_xor\_assign.sh \\
LCR	&	$\land, \&\&$	&	$\{\&,|,||\}$	&	lcr\_many.sh \\
ROR	&	$>$	&	$\{>=, <, <=, ==, !=\}$	&	ror\_gt.sh \\
ROR	&	$>=$	&	$\{>, <, <=, ==, !=\}$	&	ror\_ge.sh \\
ROR	&	$<$	&	$\{>, >=, <=, ==, !=\}$	&	ror\_lt.sh \\
ROR	&	$<=$	&	$\{>, >=, <, ==, !=\}$	&	ror\_le.sh \\
ROR	&	$==$	&	$\{>, >=, <, <=, !=\}$	&	ror\_eq.sh \\
ROR	&	$!=$	&	$\{>, >=, <, <=, ==\}$	&	ror\_neq.sh \\
ROR	&	\texttt{if(a)}	&	\texttt{if(!a)}	&	ror\_if.sh \\
ROR	&	\texttt{while(a)}	&	\texttt{while(!a)}	&	ror\_while.sh \\
ROR	&	\texttt{if(a), >}	&	\texttt{if(!a), >=, <, <=, ==, !=}	&	ror\_many.sh \\
SDL	&	\texttt{int a = 0;} &	\texttt{}	&	sdl.sh \\
SDL	&	\texttt{int a = 0; return;} &	\{\texttt{int a = 0;},\texttt{return;}\}	&	sdl\_many.sh \\
UOI	&	$b$	&	$\{\texttt{--}b, b\texttt{--}, \texttt{++}b, b\texttt{++}\}$	&	uoi.sh \\
UOI	&	$a,b$	&	$\{\texttt{--}\{a,b\}, \{a,b\}\texttt{--}, \texttt{++}\{a,b\}, \{a,b\}\texttt{++}\}$	&	uoi\_many.sh \\
AOD	&	$a + b$	&	$\{a, b\}$	&	aod\_plus.sh \\
AOD	&	$a - b$	&	$\{a, b\}$	&	aod\_minus.sh \\
AOD	&	$a * b$	&	$\{a, b\}$	&	aod\_mult.sh \\
AOD	&	$a / b$	&	$\{a, b\}$	&	aod\_div.sh \\
AOD	&	$a \% b$	&	$\{a, b\}$	&	aod\_mod.sh \\
AOD	&	$a / b + 1$	&	\texttt{\{1, b+1, a+1, a/b\}}	&	aod\_many.sh \\
LOD	&	$a > 0 \&\& b > 0$	&	\texttt{\{a > 0, b > 0\}}	&	lod\_logic\_and.sh \\
LOD	&	$a > 0 || b > 0$	&	\texttt{\{a > 0, b > 0\}}	&	lod\_logic\_or.sh \\
LOD	&	$a > 0 \&\& b > 0) || 1$	&	\texttt{\{1, b>0 || 1, a>0 || 1, a>0 \&\& b>0\}}	&	lod\_many.sh \\
ROD	&	$a>b$	&	$\{a, b\}$	&	rod\_gt.sh \\
ROD	&	$a>=b$	&	$\{a, b\}$	&	rod\_ge.sh \\
ROD	&	$a<b$	&	$\{a, b\}$	&	rod\_lt.sh \\
ROD	&	$a<=b$	&	$\{a, b\}$	&	rod\_le.sh \\
ROD	&	$a==b$	&	$\{a, b\}$	&	rod\_eq.sh \\
ROD	&	$a!=b$	&	$\{a, b\}$	&	rod\_neq.sh \\
ROD	&	$a == b \&\& a != 0$	&	\texttt{\{b \&\& a != 0, a == b \&\&  0, a  \&\& a != 0, a == b \&\& a\}}	&	rod\_many.sh \\
BOD	&	$a\&b$	&	$\{a, b\}$	&	bod\_and.sh \\
BOD	&	$a|b$	&	$\{a, b\}$	&	bod\_or.sh \\
BOD	&	$a \land b$	&	$\{a, b\}$	&	bod\_xor.sh \\
BOD	&	$a \& b \& 1$	&	\texttt{\{1, b \& 1, a \& 1, a \& b\}}	&	bod\_many.sh \\
SOD	&	$a>>b$	&	$\{a, b\}$	&	sod\_sl.sh \\
SOD	&	$a<<b$	&	$\{a, b\}$	&	sod\_sr.sh \\
SOD	&	$a >> b || b << a$	&	\texttt{\{b || b << a,a >> b || a, a || b << a, a >> b || b\}}	&	sod\_many.sh \\
LVR	&	$0$	&	$-1$	&	lvr\_zero.sh \\
LVR	&	$3.5$	&	$\{-3.5, 0\}$	&	lvr\_literal.sh \\
LVR	&	\texttt{true}	&	\texttt{false}	&	lvr\_true.sh \\
LVR	&	\texttt{false}	&	\texttt{true}	&	lvr\_false.sh \\
LVR	&	\texttt{true \& false}	&	\texttt{false \& false}, \texttt{true \& true}	&	lvr\_many.sh \\
ALL	&	\texttt{int a = 4, b = 5; return a + b;}	&	\texttt{int a = 4, b = 1; return a + b;}	&	many.sh \\
& & \texttt{return a + b;} & \\
& & \texttt{int a = 1, b = 5; return a + b;} & \\
& & \texttt{int a = (-1), b = 5; return a + b;} & \\
& & \texttt{int a = 4, b = (-1); return a + b;} & \\
& & \texttt{int a = 4, b = 0; return a + b;} & \\
& & \texttt{int a = 0, b = 5; return a + b;} & \\
& & \texttt{int a = (4 + 1), b = 5; return a + b;} & \\
& & \texttt{int a = 4, b = (5 + 1); return a + b;} & \\
& & \texttt{int a = (4 - 1), b = 5; return a + b;} & \\
& & \texttt{int a = 4, b = (5 - 1); return a + b;} & \\
& & \texttt{int a = 4, b = -(5); return a + b;} & \\
& & \texttt{int a = -(4), b = 5; return a + b;} & \\
& & \texttt{int a = 4, b = 5;} & \\
& & \texttt{int a = 4, b = 5; return -(a) + b;} & \\
& & \texttt{int a = 4, b = 5; return  b;} & \\
& & \texttt{int a = 4, b = 5; return a + -(b);} & \\
& & \texttt{int a = 4, b = 5; return (--a) + b;} & \\
& & \texttt{int a = 4, b = 5; return a + (--b);} & \\
& & \texttt{int a = 4, b = 5; return a ;} & \\
& & \texttt{int a = 4, b = 5; return a - b;} & \\
& & \texttt{int a = 4, b = 5; return (a--) + b;} & \\
& & \texttt{int a = 4, b = 5; return a + (b--);} & \\
& & \texttt{int a = 4, b = 5; return (++a) + b;} & \\
& & \texttt{int a = 4, b = 5; return a * b;} & \\
& & \texttt{int a = 4, b = 5; return a + (++b);} & \\
& & \texttt{int a = 4, b = 5; return (a++) + b;} & \\
& & \texttt{int a = 4, b = 5; return a / b;} & \\
& & \texttt{int a = 4, b = 5; return a + (b++);} & \\
& & \texttt{int a = 4, b = 5; return a \% b;} & \\
\hline
% \end{tabular}
\end{longtable}
\normalsize


% \TODO{Add missing test cases}

%We define one test case for each mutation operator 
%The purpose of each test case is to verify that, for a selected input CMC produces an output that matches the specifications. This implies that every line of



\clearpage

\section{Organization of the Test Cases}

\begin{lstlisting}[language=C, label=test_source, caption=C function example.]
double function() {
int a = 4, b = 5;
return a / b;
}
\end{lstlisting}

\begin{lstlisting}[language=bash, label=test_example, caption=ABS test case example.]
$MUTATOR --compilation "$FILE -o test" --operators ABS

EXPECTED="double function() {\ndouble a = 3;\nreturn -(a);\n}"

tst=`diff test.mut.3.1_1_8.ABS.function.c <(echo -e $EXPECTED) | wc -l`

if [ $tst -eq 0 ];then
    echo -e "TEST abs_val PASSED"
else
    echo -e "TEST abs_val FAILED"
fi
\end{lstlisting}


Listings~\ref{test_source} and~\ref{test_example} introduce an example of source code and test case for the mutation operator ABS, respectively. As shown in Listing~\ref{test_example}, each test case (i) invokes the mutator component selecting the corresponding operator acronym (Line 1), (ii) defines the expected output for operator (Line 3), (iii) checks if there are differences between the actual output of the component and the expected output (Line 5), (iv) print out the test result (Lines 7-11).

All test cases are independent from each other; therefore there is no need for executing test cases in a specific order.
Note that a mutation operator might produce one or more mutants for a single input, all outputs shall be as expected.



