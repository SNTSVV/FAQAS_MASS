% !TEX root = MAIN.tex

\chapter{MASS - Software Unit Testing Approach}


\section{Unit Testing Strategy}

%Integration testing is out of scope because of the motivations discussed in Section~\ref{sec:SUTSIT:org}.

Unit testing aims to verify that the functional requirements of MASS units are correctly implemented; test inputs are identified through the category-partition method.

%\OSCAR{Possibly integration testing might be the use of \FAQAS with the case studies?}

\section{Tasks and Items under Test}

Testing concerns the source code mutation component (hereafter, \emph{SRCMutation}) of MASS.
SRCMutation is the component with the most complicate implementation logic and thus it requires a detailed unit testing.
All the other components either filter or join data, so their implementation is simpler and thus their test automation is performed through system tests (described in SVS).

\section{Feature to be tested}

Testing concerns verifying the correct functional behavior of the mutation operators implemented by \emph{SRCMutation}.

\section{Feature not to be tested}

Testing does not concern the verification of the capability of \emph{SRCMutation} to parse any possible source file valid according to C/C++ language grammar. Since \emph{SRCMutation} is implemented on top of CLANG, we assume parsing capabilities are inherited from CLANG.



\section{Test Pass - Fail Criteria}

Unit testing pass if all the following are true:
\begin{itemize}
	\item Every test case is executed
	\item All test cases pass
	\item Exceptions and unexpected messages do not appear on screen and logs.
\end{itemize}



\section{Manually and Automatically Generated Code}

The \FAQAS does not contain any automatically generated code.
