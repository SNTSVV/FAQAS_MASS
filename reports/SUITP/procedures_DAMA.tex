% !TEX root = MAIN.tex

\chapter{DAMA - Software Unit Test Procedures}
\label{chap:proc_DAMA}

\section{General}

A single procedure has been identified to perform unit testing of DAMA. It is described in the following.

\section{DAMA Test Procedure}

\subsection{Test procedure Identifier}

\emph{DAMA-TP-DDMutation-1}


\subsection{Purpose}

This procedure aims to execute all the test cases reported in Table~\ref{table:matrix_DAMA}.

\subsection{Procedure steps}

\subsubsection{Log}
%describe any special methods or format for logging the results of test execution, the incidents observed, and any other event pertinent to this test;
Since the test cases are executed within a console, the output of the test cases can be considered as is.
In addition to that, every test case produces three files \texttt{<test name> compile.out"}, containing the output of the compiler, \texttt{<test name>.out}, which contains the output of the SUT, and \texttt{<test name>.valgrind.out}, containing the output of Valgrind, reporting information on eventual memory errors.
Moreover, the test procedure produces a file called \texttt{results.csv}, which contains a general summary of the results of all the test cases.

\subsubsection{Set-up}
\emph{DDMutation} comes with the test cases already set-up.  The Valgrind memory error detector can be installed to check the software for memory errors, althought the other components of the test suite will still function without it.
%: describe the sequence of actions to set up the procedure execution;
\subsubsection{Start}
%: describe the actions to begin the procedure execution; proceed: describe the actions during the procedure execution;
The complete test suite is executed by running the script \texttt{runTests.sh}.
Singular test cases can be executed by running the corresponding bash script in the "test" directory (i.e \texttt{run<test name>.sh}).
\subsubsection{Test Result Acquisition}
Results are printed out on the console and saved in the \texttt{results.csv} file.
Additional informations can be found in the \texttt{<test name>.compile.out} \texttt{<test name>.out} and \texttt{<test name>.valgrind.out} files, produced by all test cases.
%: describe how the test measurements is made;
\subsubsection{Shut Down}
Test cases are shut down through a SIGINT (i.e., pressing \texttt{CTRL-C})
%: describe the action to suspend testing when interruption is forced by unscheduled events;
\subsubsection{Restart}
Test cases can be restarted from \texttt{test01} by re-running the script \texttt{runTests.sh}.
%: identify any procedural restart points and describe the actions to restart the procedure at each of these points;
\subsubsection{Wrap-up}
The test execution terminates autonomously, no further action is needed.
%: describe the actions to terminate testing.
