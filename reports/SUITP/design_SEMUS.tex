% !TEX root = MAIN.tex

\chapter{SEMuS - Software Unit Test Case Design}
\label{chap:spec:semus}


\section{General}

The \SEMUS unit test suite concerns all the components of the tool. It address its functional requirements. A single test design has been identified: it runs the test generation process on a example source file, and verifies the generated outputs, the procedure is detailed in the following sections.



\section{SEMuS - Test Design}

\subsection{Test design identifier}

The test design identifier is \emph{SEMuS-TD-TGMutation-1}

With this test design we aim to ensure that each component of \SEMUS is implemented according to its requirements.

\subsection{Features to be tested}

% !TEX root =  ../MAIN.tex

\begin{table}[h]
\scriptsize
\centering
\caption{\SEMUS components with its inputs and outputs.}
\label{table:semus:components}

\begin{tabular}{lll}
\hline 
\textbf{Component}	&	\textbf{Inputs}	&	\textbf{Outputs}\\
\hline 
\emph{Test Template Generator}&SUT source code; Template configuration&Test templates\\ 
\emph{GenerateMutants}&SUT source code&Set of mutants\\
\emph{Pre-SEMu}&Set of mutants; Test templates&Meta-mutant\\
\emph{KLEE-SEMu}&Meta-mutant&KLEE tests\\
\emph{KTest to Unit Test}&KLEE tests& C Unit tests\\
\hline
\end{tabular}
\end{table}


Table~\ref{table:semus:components} shows the \SEMUS components and details its inputs and outputs. The \emph{Test Template Generator} receives as input the SUT source code and the template configuration (i.e., the JSON configuration files), and it outputs the different test templates for driving the test generation. The \emph{GenerateMutants} component receives as input the SUT source code and produces as output the set of mutants. The \emph{Pre-SEMu} receives the set of mutants, and the test templates, and generates a meta-mutant that contains all the mutants into one source code, plus the test template for the function under test. Then, the \emph{KLEE-SEMu} receives this meta-mutant, and proceeds to generate test inputs for the mutants under analysis and produces the KLEE tests, which are finally processed by the \emph{KTest to Unit Test} that generates the definitive C unit tests.


\clearpage

% \subsection{Approach refinements}