% !TEX root =  ../MAIN.tex
% Please add the following required packages to your document preamble:
% \usepackage{booktabs}
\begin{table}[h!]

  \caption{Implemented set of data mutation operators.}
  \label{table:operators_DAMA}

\centering
\scriptsize
\resizebox{\textwidth}{!}{
\begin{tabular}{@{}lllll@{}}
\toprule
\textbf{Fault Class}                                     & \textbf{Types}                                                                                                                     & \textbf{Parameters}                                                                                                                                                                                                                    & \textbf{Description}                                                                                                                                                                                                                                                                                                                                                                                                           &  \\ \midrule
\multicolumn{1}{|l|}{Value   Above Threshold (VAT)}      & \multicolumn{1}{l|}{\begin{tabular}[c]{@{}l@{}}INT\\    \\ LONG\\    \\ FLOAT\\    \\ DOUBLE\\    \\ HEX\end{tabular}}             & \multicolumn{1}{l|}{\begin{tabular}[c]{@{}l@{}}T: threshold\\    \\ D: delta with respect to threshold\end{tabular}}                                                                                                                   & \multicolumn{1}{l|}{\begin{tabular}[c]{@{}l@{}}Data mutation operation: The mutation is   performed by replacing the\\    \\ current value (a number) with a value of   the same type that is equal to\\    \\ (T + D).\end{tabular}}                                                                                                                                                                                          &  \\ \cmidrule(r){1-4}
\multicolumn{1}{|l|}{Value   below threshold (VBT)}      & \multicolumn{1}{l|}{\begin{tabular}[c]{@{}l@{}}INT\\    \\ LONG\\    \\ FLOAT\\    \\ DOUBLE\\    \\ HEX\end{tabular}}             & \multicolumn{1}{l|}{\begin{tabular}[c]{@{}l@{}}T: threshold\\    \\ D: delta with respect to threshold\end{tabular}}                                                                                                                   & \multicolumn{1}{l|}{\begin{tabular}[c]{@{}l@{}}Data mutation operation: The mutation is   performed by replacing the\\    \\ current value (a number) with a value of   the same type that is equal to\\    \\ (T − D).\end{tabular}}                                                                                                                                                                                          &  \\ \cmidrule(r){1-4}
\multicolumn{1}{|l|}{Value out   of range (VOR)}         & \multicolumn{1}{l|}{\begin{tabular}[c]{@{}l@{}}INT\\    \\ LONG\\    \\ FLOAT\\    \\ DOUBLE\\    \\ HEX\end{tabular}}             & \multicolumn{1}{l|}{\begin{tabular}[c]{@{}l@{}}MIN: minimum valid value\\    \\ MAX: maximum valid value\\    \\ D: delta with respect to minimum/maximum   valid value\end{tabular}}                                                  & \multicolumn{1}{l|}{\begin{tabular}[c]{@{}l@{}}Data mutation operations (2): The   mutation is performed by replacing the\\    \\ current value (a number) with\\    \\ • a value of the same type that is equal   to (MIN − D)\\    \\ • a value of the same type that is equal   to (MAX + D)\end{tabular}}                                                                                                                  &  \\ \cmidrule(r){1-4}
\multicolumn{1}{|l|}{Bit flip (BF)}                      & \multicolumn{1}{l|}{BIN}                                                                                                           & \multicolumn{1}{l|}{\begin{tabular}[c]{@{}l@{}}MIN:   lower bit\\    \\ MAX:   higher bit\\    \\ STATE:   mutate only if the bit is in the given state\\    \\ VALUE:   integer specifying the number of bits to mutate\end{tabular}} & \multicolumn{1}{l|}{\begin{tabular}[c]{@{}l@{}}Data mutation operation: the operator   flips N randomly selected bit. If\\    \\ STATE is specified, the mutation is   applied only if the bit is in the specified\\    \\ state. Parameter VALUE specifies the   number of bits to mutate.\end{tabular}}                                                                                                                      &  \\ \cmidrule(r){1-4}
\multicolumn{1}{|l|}{Invalid numeric value (INV)}        & \multicolumn{1}{l|}{\begin{tabular}[c]{@{}l@{}}INT\\    \\ LONG\\    \\ FLOAT\\    \\ DOUBLE\\    \\ HEX\end{tabular}}             & \multicolumn{1}{l|}{\begin{tabular}[c]{@{}l@{}}MIN:   lower valid value\\    \\ MAX:   higher valid value\\    \\ D:   distribution to follow\\    \\ VALUE:   mean value for normal distribution\end{tabular}}                        & \multicolumn{1}{l|}{\begin{tabular}[c]{@{}l@{}}Data mutation operation: Mutation is   performed by replacing the current\\    \\ value with a different value randomly   sampled in the specified range. The\\    \\ parameter D specified the distribution   to follow when performing the mutation.\\    \\In our implementation 0 indicates   uniform, 1 indicates normal around the specified value (but in range).\end{tabular}} &  \\ \cmidrule(r){1-4}
\multicolumn{1}{|l|}{Illegal Value (IV)}                 & \multicolumn{1}{l|}{\begin{tabular}[c]{@{}l@{}}INT\\    \\ LONG\\    \\ FLOAT\\    \\ DOUBLE\\    \\ HEX\end{tabular}}             & \multicolumn{1}{l|}{VALUE: illegal value that is observed}                                                                                                                                                                             & \multicolumn{1}{l|}{\begin{tabular}[c]{@{}l@{}}Data mutation operation: Mutation is   performed by replacing the current\\    \\ value with the value VALUE, if different   than the current one.\end{tabular}}                                                                                                                                                                                                                &  \\ \cmidrule(r){1-4}
\multicolumn{1}{|l|}{Anomalous   Signal Amplitude (ASA)} & \multicolumn{1}{l|}{\begin{tabular}[c]{@{}l@{}}INT\\    \\ LONG\\    \\ FLOAT\\    \\ DOUBLE\\    \\ HEX\end{tabular}}             & \multicolumn{1}{l|}{\begin{tabular}[c]{@{}l@{}}T: change point\\    \\ D: delta to add/remove\\    \\ V: value to multiply\end{tabular}}                                                                                               & \multicolumn{1}{l|}{\begin{tabular}[c]{@{}l@{}}Data mutation operation: Mutation is   performed by replacing the current\\    \\ value (v) with the value (v′) computed   as follows: v=T + ((v − T) ∗ V) + D if v ≥ T v=T − ((T − v) ∗ V) − D if v \textless T\end{tabular}}                                                                                                                                                  &  \\ \cmidrule(r){1-4}
\multicolumn{1}{|l|}{Signal   Shift (SS)}                & \multicolumn{1}{l|}{\begin{tabular}[c]{@{}l@{}}INT\\    \\ LONG\\    \\ FLOAT\\    \\ DOUBLE\\    \\ HEX\end{tabular}}             & \multicolumn{1}{l|}{D: delta by which the signal should be   shifted}                                                                                                                                                                  & \multicolumn{1}{l|}{The value is modified by adding a value   D. It simulates an anomalous shift in the signal.}                                                                                                                                                                                                                                                                                                               &  \\ \cmidrule(r){1-4}
\multicolumn{1}{|l|}{Hold   Value (HV)}                  & \multicolumn{1}{l|}{\begin{tabular}[c]{@{}l@{}}INT\\    \\ LONG\\    \\ FLOAT\\    \\ DOUBLE\\    \\ HEX\\    \\ BIN\end{tabular}} & \multicolumn{1}{l|}{V: number of times to repeat the same   value}                                                                                                                                                                     & \multicolumn{1}{l|}{\begin{tabular}[c]{@{}l@{}}This operator keeps repeating an   observed value for V times. It emulates\\    \\ a constant signal replacing a signal   supposed to vary.\end{tabular}}                                                                                                                                                                                                                       &  \\ \bottomrule
\end{tabular}
}
\end{table}
