% !TEX root = MAIN.tex

\chapter{\SEMUS - Software Unit Test Case Specification}
\label{chap:specs:semus}

\section{General}




Table~\ref{table:matrix:semus} provides the list tasks performed during unit testing based on the procedure described in Chapter~\ref{chap:spec:semus}.
Column \emph{Step} indicates the task being performed in the unit test. Column \emph{Description} provides a brief description of the step. Columns \emph{Input} and \emph{Output} indicates the expected files for each step of the test. 

% !TEX root =  ../MAIN.tex

\setlength\LTleft{0pt}
\setlength\LTright{0pt}

\scriptsize

\begin{longtable}{|p{4cm}|p{4cm}|}

\caption{Unit test case steps for SEMuS.}
\label{table:matrix:semus} \\

\hline
\textbf{Step}	&	\textbf{Result}	\\
\hline
\texttt{setup}&PASSED\\

\texttt{gen\_templates}&PASSED\\

\texttt{test\_generation\_pipeline}&PASSED\\

\texttt{check\_results}&PASSED\\
\hline
\end{longtable}
\normalsize


\clearpage

\section{Organization of the Test Cases}

\begin{lstlisting}[language=C, label=test_source_semus, caption=C function example.]
typedef struct {
    int x;
    int y;
    char z[5];
} MyType;

int func1(int v, MyType *mytype_var)
{
    int ret = 0;
    int x = 6*v; float y = 2.5, /* */ z=9 * x*1;

    x+=y
        || z;
    if (x+y > z
            && 2*x+y==7) {
        x++;
        if (x || v){
            ret = 1;
        }else
            ret = 2;
    }

    for (int i=0; i<y+z && i<5; ++i) 
    {
        ret =3;
        y+=z+i;
    }
    switch(ret) {
        case 3:
            ret = y; break;
        case 2:
            ret = mytype_var->x; break;
        case 1: 
            ret = mytype_var->y + x; break;
        default:
            mytype_var->z[0] = 'a';
    }
    return ret;
}
\end{lstlisting}

\begin{lstlisting}[language=C, label=semus_output, caption=SEMuS output.]
/* Wrapping main template for the function func1 defined in the file /opt/faqas_semu/case_studies/tests/WORKSPACE/DOWNLOADED/src.c */
/* The following mutants (IDs) were targeted to generated this test: [9] */

#include <stdio.h>
#include <string.h>



int main(int argc, char** argv)
{
    (void)argc;
    (void)argv;

    // Declare variable to hold function returned value
    int result_faqas_semu;

    // Declare arguments and make input ones symbolic
    int v;
    MyType mytype_var;
    memset(&v, 0, sizeof(v));
    memset(&mytype_var, 0, sizeof(mytype_var));
    const unsigned char v_faqas_semu_test_data[] = {0x00, 0x00, 0x00, 0x00};
    const unsigned char mytype_var_faqas_semu_test_data[] = {0x00, 0x00, 0x00, 0x00, 0x00, 0x00, 0x00, 0x00, 0x00, 0x00, 0x00, 0x00, 0x00, 0x00, 0x00, 0x00};
    memcpy(&v, v_faqas_semu_test_data, sizeof(v)); // Integer val is 0
    memcpy(&mytype_var, mytype_var_faqas_semu_test_data, sizeof(mytype_var));

    // Call function under test
    result_faqas_semu = func1(v, &mytype_var);

    // Make some output
    printf("FAQAS-SEMU-TEST_OUTPUT: x=%d, y=%d, z=%s\n", mytype_var.x, mytype_var.y, mytype_var.z);
    printf("FAQAS-SEMU-TEST_OUTPUT: result_faqas_semu = %d\n", result_faqas_semu);
    return (int)result_faqas_semu;
}
\end{lstlisting}

Listings~\ref{test_source_semus} shows the example of source code that is used for testing the test generation process. We considered this example because it exercises different type of mutation operators and constructs of the C language.

As explained in the previous section, after launching the \texttt{test\_generation\_pipeline} step of the unit test (see Table~\ref{table:matrix:semus}). The \texttt{check\_results} step check every intermediate result of \SEMUS, that is, checking the content of the test templates, checking that every mutant is produced, checking the expected output of the meta-mutant, and checking the content of the generated C unit test cases. Listing~\ref{semus_output} shows an example of expected output for the source code tested in Listing~\ref{test_source_semus}.
