% !TEX root = MAIN.tex

\chapter{DAMAt - Software Unit Test Case Specification}
\label{chap:spec_DAMAt}

\section{General}

Table~\ref{table:matrix_DATA} and Table~\ref{table:matrix_FAULT} provide the list of unit test cases derived based on the procedures described in Chapter~\ref{chap:design_DAMAt}, respectively for \emph{DDMutationData} and \emph{DDMutationFault}.
Table~\ref{table:matrix_Pairwise}, provides the list of test cases for their integration.

% Column \emph{Test ID} refers to the test folder and executable numbers: \emph{test01\_1}, for example, refers to the executable \texttt{main_1} in the \texttt{test01} directory.
\emph{Buffer Type} contains information on the C datatype used for the element of the buffer targeted by the mutation.
\emph{Fault Model} indicates the name chosen for the fault model.\
The remaining columns (i.e. \emph{Data Item}, \emph{Type}, \emph{Fault Class}, \emph{Min}, \emph{Max}, \emph{Threshold}, \emph{Delta}, \emph{State}  and \emph{Value}) contain the parameters describing the fault class, and were chosen to be representative of the real life use-cases.

% !TEX root =  ../MAIN.tex

\setlength\LTleft{0pt}
\setlength\LTright{0pt}
\scriptsize

% \begin{longtable}{|p{1cm}|l|p{1cm}|p{0.6cm}|p{0.6cm}|p{1cm}|p{0.6cm}|p{0.5cm}|p{0.5cm}|l|p{0.5cm}|p{0.5cm}|p{0.5cm}|}
\begin{longtable}{|l|l|l|p{0.5cm}|p{0.5cm}|l|p{0.5cm}|p{0.5cm}|p{0.5cm}|l|l|p{0.5cm}|l|}

% \begin{table}[h]
% \scriptsize
% \centering
\caption{Unit test cases for \emph{DDMutationData}.}
\label{table:matrix_DATA} \\

\hline
\textbf{Test ID} & \textbf{Buffer Type} & \textbf{Fault Model} & \textbf{Data Item} & \textbf{Span} & \textbf{Type} & \textbf{Fault Class} & \textbf{Min} & \textbf{Max} & \textbf{Threshold} & \textbf{Delta} & \textbf{State} & \textbf{Value}\\
\hline
test48\_0 & SHORT INT & IfHK & 0 & 1 & INT & SS & 0 & 0 & 0 & 0 & 0 & 0 \\
test48\_1 & SHORT INT & IfHK & 1 & 2 & FLOAT & SS & 0 & 0 & 0 & 0 & 0 & 0 \\
test48\_2 & SHORT INT & IfHK & 3 & 4 & DOUBLE & SS & 0 & 0 & 0 & 0 & 0 & 0 \\
test48\_3 & SHORT INT & IfHK & 7 & 2 & HEX & SS & 0 & 0 & 0 & 0 & 0 & 0 \\
test48\_4 & SHORT INT & IfHK & 9 & 2 & INT & SS & 0 & 0 & 0 & 0 & 0 & 0 \\
test48\_5 & SHORT INT & IfHK & 11 & 1 & BIN & BF & 0 & 0 & 0 & 0 & 0 & 0 \\
test48\_6 & SHORT INT & IfHK & 12 & 4 & LONG & SS & 0 & 0 & 0 & 0 & 0 & 0 \\
test49\_0 & UNS. SH. INT & IfHK & 0 & 1 & INT & SS & 0 & 0 & 0 & 0 & 0 & 0 \\
test49\_1 & UNS. SH. INT & IfHK & 1 & 2 & FLOAT & SS & 0 & 0 & 0 & 0 & 0 & 0 \\
test49\_2 & UNS. SH. INT & IfHK & 3 & 4 & DOUBLE & SS & 0 & 0 & 0 & 0 & 0 & 0 \\
test49\_3 & UNS. SH. INT & IfHK & 7 & 2 & HEX & SS & 0 & 0 & 0 & 0 & 0 & 0 \\
test49\_4 & UNS. SH. INT & IfHK & 9 & 2 & INT & SS & 0 & 0 & 0 & 0 & 0 & 0 \\
test49\_5 & UNS. SH. INT & IfHK & 11 & 1 & BIN & BF & 0 & 0 & 0 & 0 & 0 & 0 \\
test49\_6 & UNS. SH. INT & IfHK & 12 & 4 & LONG & SS & 0 & 0 & 0 & 0 & 0 & 0 \\
test50\_0 & UNS. INT & IfHK & 0 & 1 & INT & SS & 0 & 0 & 0 & 0 & 0 & 0 \\
test50\_1 & UNS. INT & IfHK & 1 & 1 & FLOAT & SS & 0 & 0 & 0 & 0 & 0 & 0 \\
test50\_2 & UNS. INT & IfHK & 2 & 2 & DOUBLE & SS & 0 & 0 & 0 & 0 & 0 & 0 \\
test50\_3 & UNS. INT & IfHK & 4 & 1 & HEX & SS & 0 & 0 & 0 & 0 & 0 & 0 \\
test50\_4 & UNS. INT & IfHK & 5 & 1 & BIN & BF & 0 & 0 & 0 & 0 & 0 & 0 \\
test50\_5 & UNS. INT & IfHK & 6 & 2 & LONG & SS & 0 & 0 & 0 & 0 & 0 & 0 \\
test51\_0 & INT & IfHK & 0 & 1 & INT & SS & 0 & 0 & 0 & 0 & 0 & 0 \\
test51\_1 & INT & IfHK & 1 & 1 & FLOAT & SS & 0 & 0 & 0 & 0 & 0 & 0 \\
test51\_2 & INT & IfHK & 2 & 2 & DOUBLE & SS & 0 & 0 & 0 & 0 & 0 & 0 \\
test51\_3 & INT & IfHK & 4 & 1 & HEX & SS & 0 & 0 & 0 & 0 & 0 & 0 \\
test51\_4 & INT & IfHK & 5 & 1 & BIN & BF & 0 & 0 & 0 & 0 & 0 & 0 \\
test51\_5 & INT & IfHK & 6 & 2 & LONG & SS & 0 & 0 & 0 & 0 & 0 & 0 \\
test52\_0 & L. INT & IfHK & 0 & 1 & INT & SS & 0 & 0 & 0 & 0 & 0 & 0 \\
test52\_1 & L. INT & IfHK & 1 & 1 & FLOAT & SS & 0 & 0 & 0 & 0 & 0 & 0 \\
test52\_2 & L. INT & IfHK & 2 & 1 & DOUBLE & SS & 0 & 0 & 0 & 0 & 0 & 0 \\
test52\_3 & L. INT & IfHK & 4 & 1 & HEX & SS & 0 & 0 & 0 & 0 & 0 & 0 \\
test52\_4 & L. INT & IfHK & 5 & 1 & BIN & BF & 0 & 0 & 0 & 0 & 0 & 0 \\
test52\_5 & L. INT & IfHK & 6 & 1 & LONG & SS & 0 & 0 & 0 & 0 & 0 & 0 \\
test53\_0 & UNS. L. INT & IfHK & 0 & 1 & INT & SS & 0 & 0 & 0 & 0 & 0 & 0 \\
test53\_1 & UNS. L. INT & IfHK & 1 & 1 & FLOAT & SS & 0 & 0 & 0 & 0 & 0 & 0 \\
test53\_2 & UNS. L. INT & IfHK & 2 & 1 & DOUBLE & SS & 0 & 0 & 0 & 0 & 0 & 0 \\
test53\_3 & UNS. L. INT & IfHK & 4 & 1 & HEX & SS & 0 & 0 & 0 & 0 & 0 & 0 \\
test53\_4 & UNS. L. INT & IfHK & 5 & 1 & BIN & BF & 0 & 0 & 0 & 0 & 0 & 0 \\
test53\_5 & UNS. L. INT & IfHK & 6 & 1 & LONG & SS & 0 & 0 & 0 & 0 & 0 & 0 \\
test54\_0 & L. L. INT & IfHK & 0 & 1 & INT & SS & 0 & 0 & 0 & 0 & 0 & 0 \\
test54\_1 & L. L. INT & IfHK & 1 & 1 & FLOAT & SS & 0 & 0 & 0 & 0 & 0 & 0 \\
test54\_2 & L. L. INT & IfHK & 2 & 1 & DOUBLE & SS & 0 & 0 & 0 & 0 & 0 & 0 \\
test54\_3 & L. L. INT & IfHK & 4 & 1 & HEX & SS & 0 & 0 & 0 & 0 & 0 & 0 \\
test54\_4 & L. L. INT & IfHK & 5 & 1 & BIN & BF & 0 & 0 & 0 & 0 & 0 & 0 \\
test54\_5 & L. L. INT & IfHK & 6 & 1 & LONG & SS & 0 & 0 & 0 & 0 & 0 & 0 \\
test55\_0 & UNS. L. L. INT & IfHK & 0 & 1 & INT & SS & 0 & 0 & 0 & 0 & 0 & 0 \\
test55\_1 & UNS. L. L. INT & IfHK & 1 & 1 & FLOAT & SS & 0 & 0 & 0 & 0 & 0 & 0 \\
test55\_2 & UNS. L. L. INT & IfHK & 2 & 1 & DOUBLE & SS & 0 & 0 & 0 & 0 & 0 & 0 \\
test55\_3 & UNS. L. L. INT & IfHK & 4 & 1 & HEX & SS & 0 & 0 & 0 & 0 & 0 & 0 \\
test55\_4 & UNS. L. L. INT & IfHK & 5 & 1 & BIN & BF & 0 & 0 & 0 & 0 & 0 & 0 \\
test55\_5 & UNS. L. L. INT & IfHK & 6 & 1 & LONG & SS & 0 & 0 & 0 & 0 & 0 & 0 \\
test56\_0 & S. CHAR & IfHK & 0 & 1 & INT & SS & 0 & 0 & 0 & 0 & 0 & 0 \\
test56\_1 & S. CHAR & IfHK & 1 & 4 & FLOAT & SS & 0 & 0 & 0 & 0 & 0 & 0 \\
test56\_2 & S. CHAR & IfHK & 5 & 8 & DOUBLE & SS & 0 & 0 & 0 & 0 & 0 & 0 \\
test56\_3 & S. CHAR & IfHK & 14 & 4 & HEX & SS & 0 & 0 & 0 & 0 & 0 & 0 \\
test56\_4 & S. CHAR & IfHK & 19 & 4 & INT & SS & 0 & 0 & 0 & 0 & 0 & 0 \\
test56\_5 & S. CHAR & IfHK & 24 & 1 & BIN & BF & 0 & 0 & 0 & 0 & 0 & 0 \\
test56\_6 & S. CHAR & IfHK & 29 & 4 & LONG & SS & 0 & 0 & 0 & 0 & 0 & 0 \\
test57\_0 & UNS. CHAR & IfHK & 0 & 1 & INT & SS & 0 & 0 & 0 & 0 & 0 & 0 \\
test57\_1 & UNS. CHAR & IfHK & 1 & 4 & FLOAT & SS & 0 & 0 & 0 & 0 & 0 & 0 \\
test57\_2 & UNS. CHAR & IfHK & 5 & 8 & DOUBLE & SS & 0 & 0 & 0 & 0 & 0 & 0 \\
test57\_3 & UNS. CHAR & IfHK & 14 & 4 & HEX & SS & 0 & 0 & 0 & 0 & 0 & 0 \\
test57\_4 & UNS. CHAR & IfHK & 19 & 4 & INT & SS & 0 & 0 & 0 & 0 & 0 & 0 \\
test57\_5 & UNS. CHAR & IfHK & 24 & 1 & BIN & BF & 0 & 0 & 0 & 0 & 0 & 0 \\
test57\_6 & UNS. CHAR & IfHK & 29 & 4 & LONG & SS & 0 & 0 & 0 & 0 & 0 & 0 \\
test58\_0 & FLOAT & IfHK & 0 & 1 & INT & SS & 0 & 0 & 0 & 0 & 0 & 0 \\
test58\_1 & FLOAT & IfHK & 1 & 1 & FLOAT & SS & 0 & 0 & 0 & 0 & 0 & 0 \\
test58\_2 & FLOAT & IfHK & 2 & 2 & DOUBLE & SS & 0 & 0 & 0 & 0 & 0 & 0 \\
test58\_3 & FLOAT & IfHK & 4 & 1 & HEX & SS & 0 & 0 & 0 & 0 & 0 & 0 \\
test58\_4 & FLOAT & IfHK & 5 & 1 & BIN & BF & 0 & 0 & 0 & 0 & 0 & 0 \\
test58\_5 & FLOAT & IfHK & 6 & 2 & LONG & SS & 0 & 0 & 0 & 0 & 0 & 0 \\
\hline
\end{longtable}
\normalsize


% !TEX root =  ../MAIN.tex

\setlength\LTleft{0pt}
\setlength\LTright{0pt}
\scriptsize

% \begin{longtable}{|p{1cm}|l|p{1cm}|p{0.6cm}|p{0.6cm}|p{1cm}|p{0.6cm}|p{0.5cm}|p{0.5cm}|l|p{0.5cm}|p{0.5cm}|p{0.5cm}|}
\begin{longtable}{|l|l|l|p{0.5cm}|p{0.5cm}|l|p{0.5cm}|p{0.5cm}|p{0.5cm}|l|l|p{0.5cm}|l|}

% \begin{table}[h]
% \scriptsize
% \centering
\caption{Unit test cases for \emph{DDMutationFault}.}
\label{table:matrix_FAULT} \\

\hline
\textbf{Test ID} & \textbf{Buffer Type} & \textbf{Fault Model} & \textbf{Data Item} & \textbf{Span} & \textbf{Type} & \textbf{Fault Class} & \textbf{Min} & \textbf{Max} & \textbf{Threshold} & \textbf{Delta} & \textbf{State} & \textbf{Value}\\
\hline
test01\_0 & INT & IfHK & 0 & 1 & BIN & BF & 0 & 0 & -1 & -1 & -1 & 1 \\
test01\_1 & INT & IfHK & 1 & 1 & INT & VOR & 0 & 5 & -1 & 1 & -1 & -1 \\
test01\_2 & INT & IfHK & 1 & 1 & INT & VOR & 0 & 5 & -1 & 1 & -1 & -1 \\
test01\_3 & INT & IfHK & 2 & 2 & BIN & BF & 0 & 0 & -1 & -1 & -1 & 1 \\
test01\_4 & INT & IfHK & 4 & 1 & BIN & BF & 0 & 0 & -1 & -1 & -1 & 1 \\
test02\_0 & INT & IfHK & 1 & 1 & INT & VAT & 0 & 0 & 10 & 15 & 0 & 0 \\
test02\_1 & INT & IfHK & 4 & 1 & INT & VBT & 0 & 0 & 0 & 15 & 0 & 0 \\
test02\_2 & INT & IfHK & 3 & 1 & INT & IV & 0 & 0 & 0 & 0 & 0 & 69 \\
test02\_3 & INT & IfHK & 2 & 1 & INT & VOR & -5 & 5 & 0 & 2 & 0 & 0 \\
test02\_4 & INT & IfHK & 2 & 1 & INT & VOR & -5 & 5 & 0 & 2 & 0 & 0 \\
test03\_0 & INT & IfHK & 1 & 1 & INT & INV & 5 & 5 & 0 & 0 & 0 & 0 \\
test03\_1 & INT & IfHK & 2 & 1 & INT & INV & -5 & -5 & 0 & 0 & 0 & 0 \\
test03\_2 & INT & IfHK & 3 & 1 & INT & INV & -5 & 5 & 0 & 0 & 0 & 0 \\
test03\_3 & INT & IfHK & 4 & 1 & INT & INV & -5 & 5 & 0 & 0 & 0 & 0 \\
test04\_0 & DOUBLE & IfHK & 1 & 1 & DOUBLE & VAT & 0 & 0 & 10.3 & 15.2 & 0 & 0 \\
test04\_1 & DOUBLE & IfHK & 4 & 1 & DOUBLE & VBT & 0 & 0 & 0 & 15.5 & 0 & 0 \\
test04\_2 & DOUBLE & IfHK & 3 & 1 & DOUBLE & IV & 0 & 0 & 0 & 0 & 0 & 69.69 \\
test04\_3 & DOUBLE & IfHK & 2 & 1 & DOUBLE & VOR & -5.5 & 5.5 & 0 & 2 & 0 & 0 \\
test05\_0 & DOUBLE & IfHK & 1 & 1 & DOUBLE & INV & 5.5 & 5.5 & 0 & 0 & 0 & 0 \\
test05\_1 & DOUBLE & IfHK & 2 & 1 & DOUBLE & INV & -5.5 & 5.5 & 0 & 0 & 0 & 0 \\
test05\_2 & DOUBLE & IfHK & 3 & 1 & DOUBLE & INV & -5.5 & 5.5 & 0 & 0 & 0 & 0 \\
test05\_3 & DOUBLE & IfHK & 4 & 1 & DOUBLE & INV & -5.5 & 5.5 & 0 & 0 & 0 & 0 \\
test06\_0 & INT & IfHK & 1 & 1 & FLOAT & VAT & 0 & 0 & 10.3 & 15.2 & 0 & 0 \\
test06\_1 & INT & IfHK & 4 & 1 & FLOAT & VBT & 0 & 0 & 0 & 15.5 & 0 & 0 \\
test06\_2 & INT & IfHK & 3 & 1 & FLOAT & IV & 0 & 0 & 0 & 0 & 0 & 69.69 \\
test06\_3 & INT & IfHK & 2 & 1 & FLOAT & VOR & -5.5 & 5.5 & 0 & 2 & 0 & 0 \\
test06\_4 & INT & IfHK & 2 & 1 & FLOAT & VOR & -5.5 & 5.5 & 0 & 2 & 0 & 0 \\
test07\_0 & FLOAT & IfHK & 1 & 1 & FLOAT & VAT & 0 & 0 & 10.3 & 15.2 & 0 & 0 \\
test07\_1 & FLOAT & IfHK & 4 & 1 & FLOAT & VBT & 0 & 0 & 0 & 15.5 & 0 & 0 \\
test07\_2 & FLOAT & IfHK & 3 & 1 & FLOAT & IV & 0 & 0 & 0 & 0 & 0 & 69.69 \\
test07\_3 & FLOAT & IfHK & 2 & 1 & FLOAT & VOR & -5.5 & 5.5 & 0 & 2 & 0 & 0 \\
test07\_4 & FLOAT & IfHK & 2 & 1 & FLOAT & VOR & -5.5 & 5.5 & 0 & 2 & 0 & 0 \\
test08\_0 & INT & IfHK & 0 & 1 & INT & SS & 0 & 0 & 0 & 10 & 0 & 0 \\
test08\_1 & INT & IfHK & 1 & 1 & INT & SS & 0 & 0 & 0 & 10 & 0 & 0 \\
test08\_2 & INT & IfHK & 2 & 1 & INT & SS & 0 & 0 & 0 & 10 & 0 & 0 \\
test08\_3 & INT & IfHK & 3 & 1 & INT & SS & 0 & 0 & 0 & 10 & 0 & 0 \\
test08\_4 & INT & IfHK & 4 & 1 & INT & SS & 0 & 0 & 0 & 10 & 0 & 0 \\
test09\_0 & FLOAT & IfHK & 1 & 1 & DOUBLE & INV & 5.5 & 5.5 & 0 & 0 & 0 & 0 \\
test09\_1 & FLOAT & IfHK & 2 & 1 & FLOAT & INV & -5.5 & 5.5 & 0 & 0 & 0 & 0 \\
test09\_2 & FLOAT & IfHK & 3 & 1 & FLOAT & INV & -5.5 & 5.5 & 0 & 0 & 0 & 0 \\
test09\_3 & FLOAT & IfHK & 4 & 1 & FLOAT & INV & -5.5 & 5.5 & 0 & 0 & 0 & 0 \\
test10\_0 & DOUBLE & IfHK & 0 & 1 & DOUBLE & SS & 0 & 0 & 0 & 10.1 & 0 & 0 \\
test10\_1 & DOUBLE & IfHK & 1 & 1 & DOUBLE & SS & 0 & 0 & 0 & 10.1 & 0 & 0 \\
test10\_2 & DOUBLE & IfHK & 2 & 1 & DOUBLE & SS & 0 & 0 & 0 & 10.1 & 0 & 0 \\
test10\_3 & DOUBLE & IfHK & 3 & 1 & DOUBLE & SS & 0 & 0 & 0 & 10.1 & 0 & 0 \\
test10\_4 & DOUBLE & IfHK & 4 & 1 & DOUBLE & SS & 0 & 0 & 0 & 10.1 & 0 & 0 \\
test11\_0 & FLOAT & IfHK & 0 & 1 & FLOAT & SS & 0 & 0 & 0 & 10.1 & 0 & 0 \\
test11\_1 & FLOAT & IfHK & 1 & 1 & FLOAT & SS & 0 & 0 & 0 & 10.1 & 0 & 0 \\
test11\_2 & FLOAT & IfHK & 2 & 1 & FLOAT & SS & 0 & 0 & 0 & 10.1 & 0 & 0 \\
test11\_3 & FLOAT & IfHK & 3 & 1 & FLOAT & SS & 0 & 0 & 0 & 10.1 & 0 & 0 \\
test11\_4 & FLOAT & IfHK & 4 & 1 & FLOAT & SS & 0 & 0 & 0 & 10.1 & 0 & 0 \\
test12\_0 & INT & IfHK & 1 & 1 & INT & SS & 0 & 0 & 0 & 10 & 0 & 0 \\
test13\_0 & DOUBLE & IfHK & 1 & 1 & DOUBLE & SS & 0 & 0 & 0 & 10.5 & 0 & 0 \\
test14\_0 & FLOAT & IfHK & 1 & 1 & FLOAT & SS & 0 & 0 & 0 & 10.5 & 0 & 0 \\
test15\_0 & L. L. INT & IfHK & 1 & 1 & DOUBLE & SS & 0 & 0 & 0 & 20.5 & 0 & 0 \\
test16\_0 & L. L. INT & IfHK & 1 & 1 & DOUBLE & SS & 0 & 0 & 0 & 0 & 0 & 0 \\
test17\_0 & INT & IfHK & 1 & 1 & BIN & BF & 1 & 1 & 0 & 0 & 1 & 1 \\
test17\_1 & INT & IfHK & 1 & 1 & BIN & BF & 2 & 2 & 0 & 0 & 0 & 1 \\
test17\_2 & INT & IfHK & 2 & 1 & BIN & BF & 3 & 3 & 0 & 0 & 0 & 1 \\
test17\_3 & INT & IfHK & 0 & 1 & BIN & BF & 3 & 3 & 0 & 0 & 0 & 1 \\
test17\_4 & INT & IfHK & 4 & 1 & BIN & BF & 3 & 3 & 0 & 0 & 0 & 1 \\
test18\_0 & DOUBLE & IfHK & 1 & 1 & BIN & BF & 1 & 1 & 0 & 0 & 1 & 1 \\
test18\_2 & DOUBLE & IfHK & 1 & 1 & BIN & BF & 2 & 2 & 0 & 0 & 0 & 1 \\
test18\_2 & DOUBLE & IfHK & 2 & 1 & BIN & BF & 3 & 3 & 0 & 0 & 0 & 1 \\
test18\_3 & DOUBLE & IfHK & 0 & 1 & BIN & BF & 3 & 3 & 0 & 0 & 0 & 1 \\
test18\_4 & DOUBLE & IfHK & 4 & 1 & BIN & BF & 3 & 3 & 0 & 0 & 0 & 1 \\
test19\_0 & FLOAT & IfHK & 1 & 1 & BIN & BF & 1 & 1 & 0 & 0 & 1 & 1 \\
test19\_1 & FLOAT & IfHK & 1 & 1 & BIN & BF & 2 & 2 & 0 & 0 & 0 & 1 \\
test19\_2 & FLOAT & IfHK & 2 & 1 & BIN & BF & 3 & 3 & 0 & 0 & 0 & 1 \\
test19\_3 & FLOAT & IfHK & 0 & 1 & BIN & BF & 3 & 3 & 0 & 0 & 0 & 1 \\
test19\_4 & FLOAT & IfHK & 4 & 1 & BIN & BF & 3 & 3 & 0 & 0 & 0 & 1 \\
test20\_0 & INT & IfHK & 0 & 1 & INT & ASA & 0 & 0 & 3 & 5 & 0 & 2 \\
test20\_1 & INT & IfHK & 1 & 1 & INT & ASA & 0 & 0 & 3 & 5 & 0 & 2 \\
test20\_2 & INT & IfHK & 2 & 1 & INT & ASA & 0 & 0 & 3 & 5 & 0 & 2 \\
test20\_3 & INT & IfHK & 3 & 1 & INT & ASA & 0 & 0 & 3 & 5 & 0 & 2 \\
test20\_4 & INT & IfHK & 4 & 1 & INT & ASA & 0 & 0 & 3 & 5 & 0 & 2 \\
test21\_0 & SHORT INT & IfHK & 1 & 2 & INT & SS & 0 & 0 & 0 & 0 & 0 & 0 \\
test22\_0 & SHORT INT & IfHK & 1 & 2 & INT & SS & 0 & 0 & 0 & 10 & 0 & 0 \\
test23\_0 & INT & IfHK & 1 & 2 & DOUBLE & SS & 0 & 0 & 0 & 0 & 0 & 0 \\
test24\_0 & SHORT INT & IfHK & 1 & 2 & FLOAT & SS & 0 & 0 & 0 & 0 & 0 & 0 \\
test25\_0 & FLOAT & IfHK & 0 & 1 & FLOAT & ASA & 0 & 0 & 3 & 5 & 0 & 2 \\
test25\_1 & FLOAT & IfHK & 1 & 1 & FLOAT & ASA & 0 & 0 & 3 & 5 & 0 & 2 \\
test25\_2 & FLOAT & IfHK & 2 & 1 & FLOAT & ASA & 0 & 0 & 3 & 5 & 0 & 2 \\
test25\_3 & FLOAT & IfHK & 3 & 1 & FLOAT & ASA & 0 & 0 & 3 & 5 & 0 & 2 \\
test25\_4 & FLOAT & IfHK & 4 & 1 & FLOAT & ASA & 0 & 0 & 3 & 5 & 0 & 2 \\
test26\_0 & DOUBLE & IfHK & 0 & 1 & DOUBLE & ASA & 0 & 0 & 3 & 5 & 0 & 2 \\
test26\_1 & DOUBLE & IfHK & 1 & 1 & DOUBLE & ASA & 0 & 0 & 3 & 5 & 0 & 2 \\
test26\_2 & DOUBLE & IfHK & 2 & 1 & DOUBLE & ASA & 0 & 0 & 3 & 5 & 0 & 2 \\
test26\_3 & DOUBLE & IfHK & 3 & 1 & DOUBLE & ASA & 0 & 0 & 3 & 5 & 0 & 2 \\
test26\_4 & DOUBLE & IfHK & 4 & 1 & DOUBLE & ASA & 0 & 0 & 3 & 5 & 0 & 2 \\
test27\_0 & INT & IfHK & 1 & 1 & INT & HV & 0 & 0 & 0 & 0 & 0 & 5 \\
test28\_0 & DOUBLE & IfHK & 1 & 1 & DOUBLE & HV & 0 & 0 & 0 & 0 & 0 & 5 \\
test29\_0 & FLOAT & IfHK & 1 & 1 & FLOAT & HV & 0 & 0 & 0 & 0 & 0 & 5 \\
test30\_0 & INT & IfHK & 0 & 1 & FLOAT & VAT & 0 & 0 & 10 & 5 & 0 & 0 \\
test32\_0 & L. L. INT & SensorA & 1 & 1 & INT & SS & 0 & 0 & 0 & 1 & 0 & 0 \\
test32\_1 & L. L. INT & SensorB & 1 & 1 & INT & SS & 0 & 0 & 0 & 1 & 0 & 0 \\
test32\_2 & L. L. INT & SensorC & 1 & 1 & INT & SS & 0 & 0 & 0 & 1 & 0 & 0 \\
test32\_3 & L. L. INT & ActuatorA & 1 & 1 & INT & SS & 0 & 0 & 0 & 1 & 0 & 0 \\
test32\_4 & L. L. INT & ActuatorB & 1 & 1 & INT & SS & 0 & 0 & 0 & 1 & 0 & 0 \\
test32\_5 & L. L. INT & ActuatorC & 1 & 1 & INT & SS & 0 & 0 & 0 & 1 & 0 & 0 \\
test35\_0 & L. L. INT & SensorA & 1 & 1 & INT & SS & 0 & 0 & 0 & 1 & 0 & 0 \\
test35\_1 & L. L. INT & SensorB & 1 & 1 & INT & SS & 0 & 0 & 0 & 1 & 0 & 0 \\
test35\_2 & L. L. INT & SensorC & 1 & 1 & INT & SS & 0 & 0 & 0 & 1 & 0 & 0 \\
test35\_3 & L. L. INT & ActuatorA & 1 & 1 & INT & SS & 0 & 0 & 0 & 1 & 0 & 0 \\
test35\_4 & L. L. INT & ActuatorB & 1 & 1 & INT & SS & 0 & 0 & 0 & 1 & 0 & 0 \\
test35\_5 & L. L. INT & ActuatorC & 1 & 1 & INT & SS & 0 & 0 & 0 & 1 & 0 & 0 \\
test36\_0 & L. L. INT & SensorA & 1 & 1 & INT & SS & 0 & 0 & 0 & 1000 & 0 & 0 \\
test36\_1 & L. L. INT & SensorB & 1 & 1 & FLOAT & SS & 0 & 0 & 0 & 1000 & 0 & 0 \\
test36\_2 & L. L. INT & SensorC & 1 & 1 & DOUBLE & SS & 0 & 0 & 0 & 1000 & 0 & 0 \\
test36\_3 & L. L. INT & ActuatorA & 1 & 1 & INT & SS & 0 & 0 & 0 & 1000 & 0 & 0 \\
test36\_4 & L. L. INT & ActuatorB & 1 & 1 & FLOAT & SS & 0 & 0 & 0 & 1000 & 0 & 0 \\
test37\_5 & L. L. INT & ActuatorC & 1 & 1 & DOUBLE & SS & 0 & 0 & 0 & 1000 & 0 & 0 \\
test37\_0 & L. L. INT & SensorA & 1 & 1 & INT & SS & 0 & 0 & 0 & 1000 & 0 & 0 \\
test37\_1 & L. L. INT & SensorB & 1 & 1 & FLOAT & SS & 0 & 0 & 0 & 1000 & 0 & 0 \\
test37\_2 & L. L. INT & SensorC & 1 & 1 & DOUBLE & SS & 0 & 0 & 0 & 1000 & 0 & 0 \\
test37\_3 & L. L. INT & ActuatorA & 1 & 1 & INT & SS & 0 & 0 & 0 & 1000 & 0 & 0 \\
test37\_4 & L. L. INT & ActuatorB & 1 & 1 & FLOAT & SS & 0 & 0 & 0 & 1000 & 0 & 0 \\
test37\_5 & L. L. INT & ActuatorC & 1 & 1 & DOUBLE & SS & 0 & 0 & 0 & 1000 & 0 & 0 \\
test38\_0 & INT & IfHK & 1 & 1 & HEX & VAT & 0x0 & 0x0 & 0x10 & 0xf & 0x0 & 0x0 \\
test38\_1 & INT & IfHK & 4 & 1 & HEX & VBT & 0x0 & 0x0 & 0x45 & 0xf & 0x0 & 0x0 \\
test38\_2 & INT & IfHK & 3 & 1 & HEX & IV & 0x0 & 0x0 & 0x0 & 0x0 & 0x0 & 0x45 \\
test38\_3 & INT & IfHK & 2 & 1 & HEX & VOR & 0x5 & 0x5 & 0x0 & 0x2 & 0x0 & 0x0 \\
test38\_4 & INT & IfHK & 2 & 1 & HEX & VOR & 0x5 & 0x5 & 0x0 & 0x2 & 0x0 & 0x0 \\
test39\_0 & INT & IfHK & 0 & 1 & INT & SS & 0x0 & 0x0 & 0x0 & 0xa & 0x0 & 0x0 \\
test39\_1 & INT & IfHK & 1 & 1 & INT & SS & 0x0 & 0x0 & 0x0 & 0xa & 0x0 & 0x0 \\
test39\_2 & INT & IfHK & 2 & 1 & INT & SS & 0x0 & 0x0 & 0x0 & 0xa & 0x0 & 0x0 \\
test39\_3 & INT & IfHK & 3 & 1 & INT & SS & 0x0 & 0x0 & 0x0 & 0xa & 0x0 & 0x0 \\
test39\_4 & INT & IfHK & 4 & 1 & INT & SS & 0x0 & 0x0 & 0x0 & 0xa & 0x0 & 0x0 \\
test40\_0 & INT & IfHK & 0 & 1 & INT & ASA & 0x0 & 0x0 & 0x3 & 0x5 & 0x0 & 0x2 \\
test40\_1 & INT & IfHK & 1 & 1 & INT & ASA & 0x0 & 0x0 & 0x3 & 0x5 & 0x0 & 0x2 \\
test40\_2 & INT & IfHK & 2 & 1 & INT & ASA & 0x0 & 0x0 & 0x3 & 0x5 & 0x0 & 0x2 \\
test40\_3 & INT & IfHK & 3 & 1 & INT & ASA & 0x0 & 0x0 & 0x3 & 0x5 & 0x0 & 0x2 \\
test40\_4 & INT & IfHK & 4 & 1 & INT & ASA & 0x0 & 0x0 & 0x3 & 0x5 & 0x0 & 0x2 \\
test42\_0 & INT & IfHK & 0 & 1 & BIN & BF & 0 & 0 & -1 & -1 & -1 & 1 \\
test42\_1 & INT & IfHK & 1 & 1 & INT & VOR & 0 & 5 & -1 & 1 & -1 & -1 \\
test42\_2 & INT & IfHK & 2 & 1 & BIN & BF & 0 & 0 & -1 & -1 & -1 & 1 \\
test42\_3 & INT & IfHK & 4 & 1 & BIN & BF & 0 & 0 & -1 & -1 & -1 & 1 \\
test42\_4 & INT & UNUSED & 0 & 1 & INT & IV & 0 & 0 & 0 & 0 & 0 & 69 \\
test42\_5 & INT & UNUSED & 1 & 1 & INT & IV & 0 & 0 & 0 & 0 & 0 & 69 \\
test42\_6 & INT & UNUSED & 2 & 1 & INT & IV & 0 & 0 & 0 & 0 & 0 & 69 \\
test42\_7 & INT & UNUSED & 3 & 1 & INT & IV & 0 & 0 & 0 & 0 & 0 & 69 \\
test44\_0 & INT & IfHK & 0 & 1 & BIN & BF & 0 & 0 & -1 & -1 & -1 & 1 \\
test44\_1 & INT & IfHK & 1 & 1 & INT & VOR & 0 & 5 & -1 & 1 & -1 & -1 \\
test44\_2 & INT & IfHK & 2 & 1 & BIN & BF & 0 & 0 & -1 & -1 & -1 & 1 \\
test44\_3 & INT & IfHK & 4 & 1 & BIN & BF & 0 & 0 & -1 & -1 & -1 & 1 \\
test44\_4 & INT & IfTT & 0 & 1 & INT & IV & 0 & 0 & 0 & 0 & 0 & 69 \\
test44\_5 & INT & IfTT & 3 & 1 & INT & IV & 0 & 0 & 0 & 0 & 0 & 69 \\
test45\_0 & L. INT & IfHK & 1 & 1 & LONG & VAT & 0 & 0 & 10 & 15 & 0 & 0 \\
test45\_1 & L. INT & IfHK & 3 & 1 & LONG & IV & 0 & 0 & 0 & 0 & 0 & 45 \\
test45\_2 & L. INT & IfHK & 2 & 1 & LONG & VOR & 5 & 5 & 0 & 2 & 0 & 0 \\
test45\_3 & L. INT & IfHK & 4 & 1 & LONG & VBT & 0 & 0 & 45 & 15 & 0 & 0 \\
test46\_0 & L. INT & IfHK & 0 & 1 & INT & SS & 0 & 0 & 0 & 64748 & 0 & 0 \\
test46\_1 & L. INT & IfHK & 1 & 1 & INT & SS & 0 & 0 & 0 & 64748 & 0 & 0 \\
test46\_2 & L. INT & IfHK & 2 & 1 & INT & SS & 0 & 0 & 0 & 64748 & 0 & 0 \\
test46\_3 & L. INT & IfHK & 3 & 1 & INT & SS & 0 & 0 & 0 & 64748 & 0 & 0 \\
test46\_4 & L. INT & IfHK & 4 & 1 & INT & SS & 0 & 0 & 0 & 64748 & 0 & 0 \\
test47\_0 & L. INT & IfHK & 0 & 1 & INT & ASA & 0 & 0 & 3 & 5 & 0 & 2 \\
test47\_1 & L. INT & IfHK & 1 & 1 & INT & ASA & 0 & 0 & 3 & 5 & 0 & 2 \\
test47\_2 & L. INT & IfHK & 2 & 1 & INT & ASA & 0 & 0 & 3 & 5 & 0 & 2 \\
test47\_3 & L. INT & IfHK & 3 & 1 & INT & ASA & 0 & 0 & 3 & 5 & 0 & 2 \\
test47\_4 & L. INT & IfHK & 4 & 1 & INT & ASA & 0 & 0 & 3 & 5 & 0 & 2 \\
test62\_1 & IfHK & 0 & 1 & INT & VAT & NA & NA & -1 & 5 & NA & NA \\
test62\_2 & IfHK & 1 & 1 & INT & VBT & NA & NA & 1 & 5 & NA & NA \\
test62\_3 & IfHK & 2 & 1 & INT & VOR & 5 & 10 & NA & 1 & NA & NA \\
test62\_4 & IfHK & 3 & 1 & INT & VOR & -5 & -1 & NA & 1 & NA & NA \\
test63\_1 & IfHK & 0 & 1 & DOUBLE & VAT & NA & NA & -1 & 5 & NA & NA \\
test63\_2 & IfHK & 1 & 1 & DOUBLE & VBT & NA & NA & 1 & 5 & NA & NA \\
test63\_3 & IfHK & 2 & 1 & DOUBLE & VOR & 5 & 10 & NA & 1 & NA & NA \\
test63\_4 & IfHK & 3 & 1 & DOUBLE & VOR & -5 & -1 & NA & 1 & NA & NA \\
test64\_1 & IfHK & 0 & 1 & LONG & VAT & NA & NA & -1 & 5 & NA & NA \\
test64\_2 & IfHK & 1 & 1 & LONG & VBT & NA & NA & 1 & 5 & NA & NA \\
test64\_3 & IfHK & 2 & 1 & LONG & VOR & 5 & 10 & NA & 1 & NA & NA \\
test64\_4 & IfHK & 3 & 1 & LONG & VOR & -5 & -1 & NA & 1 & NA & NA \\
test65\_1 & IfHK & 0 & 1 & FLOAT & VAT & NA & NA & -1 & 5 & NA & NA \\
test65\_2 & IfHK & 1 & 1 & FLOAT & VBT & NA & NA & 1 & 5 & NA & NA \\
test65\_3 & IfHK & 2 & 1 & FLOAT & VOR & 5 & 10 & NA & 1 & NA & NA \\
test65\_4 & IfHK & 3 & 1 & FLOAT & VOR & -5 & -1 & NA & 1 & NA & NA \\
test66\_1 & IfHK & 0 & 1 & INT & FVAT & NA & NA & -1 & 5 & NA & NA \\
test66\_2 & IfHK & 1 & 1 & INT & FVBT & NA & NA & 1 & 5 & NA & NA \\
test66\_3 & IfHK & 2 & 1 & INT & FVOR & 5 & 10 & NA & 1 & NA & NA \\
test66\_4 & IfHK & 3 & 1 & INT & FVOR & -5 & -1 & NA & 1 & NA & NA \\
test67\_1 & IfHK & 0 & 1 & DOUBLE & FVAT & NA & NA & -1 & 5 & NA & NA \\
test67\_2 & IfHK & 1 & 1 & DOUBLE & FVBT & NA & NA & 1 & 5 & NA & NA \\
test67\_3 & IfHK & 2 & 1 & DOUBLE & FVOR & 5 & 10 & NA & 1 & NA & NA \\
test67\_4 & IfHK & 3 & 1 & DOUBLE & FVOR & -5 & -1 & NA & 1 & NA & NA \\
test68\_1 & IfHK & 0 & 1 & LONG & FVAT & NA & NA & -1 & 5 & NA & NA \\
test68\_2 & IfHK & 1 & 1 & LONG & FVBT & NA & NA & 1 & 5 & NA & NA \\
test68\_3 & IfHK & 2 & 1 & LONG & FVOR & 5 & 10 & NA & 1 & NA & NA \\
test68\_4 & IfHK & 3 & 1 & LONG & FVOR & -5 & -1 & NA & 1 & NA & NA \\
test69\_1 & IfHK & 0 & 1 & FLOAT & FVAT & NA & NA & -1 & 5 & NA & NA \\
test69\_2 & IfHK & 1 & 1 & FLOAT & FVBT & NA & NA & 1 & 5 & NA & NA \\
test69\_3 & IfHK & 2 & 1 & FLOAT & FVOR & 5 & 10 & NA & 1 & NA & NA \\
test69\_4 & IfHK & 3 & 1 & FLOAT & FVOR & -5 & -1 & NA & 1 & NA & NA \\
\hline
\end{longtable}
\normalsize


% !TEX root = ../MAIN.tex

\setlength\LTleft{0pt}
\setlength\LTright{0pt}
\scriptsize

% \begin{longtable}{|p{1cm}|l|p{1cm}|p{0.6cm}|p{0.6cm}|p{1cm}|p{0.6cm}|p{0.5cm}|p{0.5cm}|l|p{0.5cm}|p{0.5cm}|p{0.5cm}|}
\begin{longtable}{|l|l|l|l|l|l|}

% \begin{table}[h]
% \scriptsize
% \centering
\caption{Test cases for the integration of \emph{DDMutationData} and \emph{DDMutationFault} .}
\label{table:matrix_Pairwise} \\

\hline
\textbf{Test ID} & \textbf{Buffer Type} & \textbf{Fault Model} & \textbf{Fault Class} & \textbf{Data Type} & \textbf{Span}\\
\hline
test60\_0 & double & Pairwise & SS & DOUBLE & 1 \\
test61\_0 & float & Pairwise & VAT & FLOAT & 1 \\
test62\_0 & int & Pairwise & VAT & FLOAT & 1 \\
test62\_1 & int & Pairwise & VBT & INT & 1 \\
test62\_2 & int & Pairwise & IV & LONG & \textgreater{}1 \\
test62\_3 & int & Pairwise & ASA & HEX & 1 \\
test62\_4 & int & Pairwise & SS & DOUBLE & \textgreater{}1 \\
test62\_5 & int & Pairwise & HV & FLOAT & \textgreater{}1 \\
test62\_6 & int & Pairwise & VBT & LONG & 1 \\
test62\_7 & int & Pairwise & VOR & HEX & 1 \\
test62\_8 & int & Pairwise & INV & DOUBLE & \textgreater{}1 \\
test63\_0 & long int & Pairwise & VAT & HEX & 1 \\
test63\_1 & long int & Pairwise & VOR & FLOAT & 1 \\
test63\_2 & long int & Pairwise & SS & LONG & 1 \\
test64\_0 & long long int & Pairwise & VAT & HEX & 1 \\
test64\_1 & long long int & Pairwise & VOR & FLOAT & 1 \\
test64\_2 & long long int & Pairwise & INV & INT & 1 \\
test64\_3 & long long int & Pairwise & ASA & LONG & 1 \\
test64\_4 & long long int & Pairwise & HV & DOUBLE & 1 \\
test65\_0 & short int & Pairwise & INV & HEX & \textgreater{}1 \\
test65\_1 & short int & Pairwise & IV & DOUBLE & 1 \\
test65\_2 & short int & Pairwise & ASA & FLOAT & \textgreater{}1 \\
test65\_3 & short int & Pairwise & SS & INT & 1 \\
test65\_4 & short int & Pairwise & HV & LONG & 1 \\
test65\_5 & short int & Pairwise & HV & BIN & \textgreater{}1 \\
test65\_6 & short int & Pairwise & VAT & DOUBLE & 1 \\
test65\_7 & short int & Pairwise & VBT & DOUBLE & 1 \\
test65\_8 & short int & Pairwise & VOR & INT & 1 \\
test65\_9 & short int & Pairwise & VOR & FLOAT & \textgreater{}1 \\
test65\_10 & short int & Pairwise & INV & INT & 1 \\
test66\_0 & signed char & Pairwise & VBT & LONG & \textgreater{}1 \\
test66\_1 & signed char & Pairwise & INV & DOUBLE & \textgreater{}1 \\
test66\_2 & signed char & Pairwise & IV & FLOAT & 1 \\
test66\_3 & signed char & Pairwise & ASA & INT & \textgreater{}1 \\
test66\_4 & signed char & Pairwise & HV & BIN & 1 \\
test66\_5 & signed char & Pairwise & VAT & HEX & 1 \\
test66\_6 & signed char & Pairwise & VBT & DOUBLE & \textgreater{}1 \\
test66\_7 & signed char & Pairwise & VOR & FLOAT & 1 \\
test66\_8 & signed char & Pairwise & INV & INT & \textgreater{}1 \\
test67\_0 & unsigned char & Pairwise & INV & HEX & 1 \\
test67\_1 & unsigned char & Pairwise & IV & DOUBLE & \textgreater{}1 \\
test67\_2 & unsigned char & Pairwise & ASA & FLOAT & 1 \\
test67\_3 & unsigned char & Pairwise & SS & INT & 1 \\
test67\_4 & unsigned char & Pairwise & HV & BIN & \textgreater{}1 \\
test67\_5 & unsigned char & Pairwise & BF & BIN & 1 \\
test67\_6 & unsigned char & Pairwise & VAT & LONG & \textgreater{}1 \\
test67\_7 & unsigned char & Pairwise & VBT & HEX & 1 \\
test67\_8 & unsigned char & Pairwise & VOR & DOUBLE & \textgreater{}1 \\
test67\_9 & unsigned char & Pairwise & INV & FLOAT & 1 \\
test68\_0 & unsigned int & Pairwise & VBT & FLOAT & 1 \\
test68\_1 & unsigned int & Pairwise & ASA & LONG & 1 \\
test68\_2 & unsigned int & Pairwise & SS & HEX & \textgreater{}1 \\
test68\_3 & unsigned int & Pairwise & HV & DOUBLE & 1 \\
test68\_4 & unsigned int & Pairwise & VAT & INT & 1 \\
test68\_5 & unsigned int & Pairwise & VOR & LONG & \textgreater{}1 \\
test68\_6 & unsigned int & Pairwise & INV & HEX & 1 \\
test69\_0 & unsigned long int & Pairwise & VBT & HEX & 1 \\
test69\_1 & unsigned long int & Pairwise & INV & FLOAT & 1 \\
test69\_2 & unsigned long int & Pairwise & HV & LONG & 1 \\
test69\_3 & unsigned long int & Pairwise & VBT & FLOAT & 1 \\
test70\_0 & unsigned long long int & Pairwise & VBT & HEX & 1 \\
test70\_1 & unsigned long long int & Pairwise & VOR & DOUBLE & 1 \\
test70\_2 & unsigned long long int & Pairwise & IV & INT & 1 \\
test70\_3 & unsigned long long int & Pairwise & SS & LONG & 1 \\
test71\_0 & unsigned short int & Pairwise & INV & LONG & 1 \\
test71\_1 & unsigned short int & Pairwise & IV & HEX & \textgreater{}1 \\
test71\_2 & unsigned short int & Pairwise & ASA & DOUBLE & 1 \\
test71\_3 & unsigned short int & Pairwise & SS & FLOAT & \textgreater{}1 \\
test71\_4 & unsigned short int & Pairwise & HV & INT & 1 \\
test71\_5 & unsigned short int & Pairwise & BF & BIN & \textgreater{}1 \\
test71\_6 & unsigned short int & Pairwise & VAT & LONG & 1 \\
test71\_7 & unsigned short int & Pairwise & VBT & HEX & \textgreater{}1 \\
test71\_8 & unsigned short int & Pairwise & VOR & DOUBLE & 1 \\
test71\_9 & unsigned short int & Pairwise & INV & FLOAT & 1 \\
\hline
\end{longtable}
\normalsize

