% !TEX root = MAIN.tex

\chapter{Software Release Overview}

The released software CI (ITT-1-9873-ESA-FAQAS-MASS.zip, ITT-1-9873-ESA-FAQAS-SEMuS.zip, and ITT-1-9873-ESA-FAQAS-DAMAt.zip) are identified in the associated SCF. The version of the delivered software is v1.0.

The delivered software is functionally complete regarding the technical specifications, and has been delivered an validated with space software case studies. No functional issues have been identified. For a concrete description of problems and limitations of the library see Section~\ref{sec:limitations}.

The delivered software contains source code files, build scripts, and readme files. For precise instructions on how to install and use the software, please refer to the Software User Manual (SUM).


\chapter{Status of the Software Configuration Item}

\section{Evolution Since Previous Version}

This software CI is the first fully functional release of the FAQAS-Framework. We do not have previous versions for the FAQAS-Framework.

\section{Known Problems or Limitations}
\label{sec:limitations}

The delivered software CI has been designed and implemented in compliance with ESA ECSS-E-ST-40 and ECSS-Q-ST-80, and has been verified and validated. Limitations still present in the software are described in the following sections.

\subsection{MASS}

Even though, \MASS mutates C and C++ source files, it does not manage correctly all C++ constructs (e.g., scope resolution operator). However, this does not affect the mutation analysis process, since incorrect mutations are discarded by the approach. 
This limitation will be assessed in a follow up project.

\subsection{SEMuS}

\SEMUS has a limitation when generating test inputs for functions that process parameters of floating-point types. This is an inherited limitation from \SEMU, and more specifically, from KLEE. The direct consequence of this limitation is that some test inputs generated by \SEMUS might not kill the mutant. This limitation will be assessed in a follow up project.


