% !TEX root = MutationTestingSurvey.tex

\chapter{Mutation Testing Benchmarks}
\label{chapter:industry}


\section{Mutation Testing Benchmarks}

In Section \ref{sec:limitations} we presented the limitations, and concretely the scalability issues of the mutation testing process. We also explained that the scalability issues of the mutation testing process are directly related to the size of the system under test and the respective test suite used to assess its adequacy. 

To cope with the limitations of the mutation testing process, in Section \ref{sec:limitations} we also introduced several approaches and strategies present in the scientific literature. 
Each of these studies, usually introduce an empirical evaluation aimed to prove or disprove a specific theory proposed by the authors. Since the size of these software under test and test suite are important to prove the validity of the proposed approach on real life software (e.g., space context software), in this chapter we present a benchmark of the most relevant empirical evaluations performed in the last eight years, collected from most important software engineering venues and journals (e.g., ICSE, ICST, ISSTA, ISSRE, AST, TSE and TOSEM), in the context of mutation testing process for C software. Section \ref{sec:benchmark} introduces a summary of the benchmark with the respective analysis.

\section{Benchmarks Present in the Literature}
\label{sec:benchmark}

In the following, we introduce Table \ref{table:benchmarks}, the table shows a summary of the most relevant benchmarks targeting C software systems, used in the software engineering literature in the last eight years. 

For each benchmark identified in the literature, we report (1) the case study (i.e., the software under system), (2) the size of the system being analyzed in terms of lines of code (LOC), (3) the size of the used test suite in terms of number of test cases, (4) the original goal of the evaluation for which the benchmark was performed, and (5) the actual reference to the paper in which the benchmark was originally presented. 

From Table \ref{table:benchmarks} can be seen that the most used benchmark for assessing C software systems has been the Siemens suite, the programs belonging to this suite are commonly used in the software testing community, because they also provide a set of real fault useful for the validation of software testing techniques.
From Table \ref{table:benchmarks}, it also can be appreciated the wide variety of case studies selected in the literature, while some authors decide to assess their techniques using very simple programs such as \textit{abs} \cite{tokumoto2016muvm}. Other authors preferred using Unix applications such as the Coreutils package \cite{hariri2019comparing,papadakis2018mutation,chekam2017empirical}. 
On the other hand, some authors selected very complex programs such as OpenSSL and LLVM \cite{denisov2018mull}, but usually the experiments are exercised only on small components of these large applications.

Concerning the size of the selected softwares under test, the most relevant studies has been presented by Papadakis and Chekam \cite{papadakis2018mutation,chekam2017empirical,papadakis2018mutant}. Specifically, these authors employed the case studies Coreutils, Findutils, Grep, Make and Codeflaws (83 KLOC, 18 KLOC, 9 KLOC, 35 KLOC and 266 KLOC respectively). 
Concerning the size of the test suite employed to assess their adequacy, again Papadakis and Chekam \cite{papadakis2018mutation,chekam2017empirical,papadakis2018mutant} present the most comprehensive studies with test suite sizes ranging from 58 131, to 122 261 test cases.

% !TEX root = MutationTestingSurvey.tex

\chapter{Mutation Testing Benchmarks}
\label{chapter:industry}

In Section~\ref{sec:limitations} we have provided an overview of state-of the-art solutions to perform mutation testing and address limitations of the mutation testing process.
Each of these solutions had been evaluated against a set of case study systems deemed representative for the usage context. Such case study systems consist of a software under test (SUT) and a test suite for the software.
Hereafter, we'll use the term \INDEX{benchmark} to indicate the set of case study systems used in the empirical evaluation presented in a research paper.
%introduce an empirical evaluation aimed to prove or disprove a specific theory proposed by the authors. 
These benchmarks can be used as reference for future mutation testing developments. 
Also, a deep understanding of the characteristics of the benchmarks considered in the literature, may provide insights on the generalizability of the results to space software (e.g, results evaluated against a real time application are more likely generalizable to the case of space software).


% !TEX root =  ../MutationTestingSurvey.tex

\setlength\LTleft{0pt}
\setlength\LTright{0pt}
\scriptsize 
\begin{longtable}{@{\extracolsep{\fill}}|p{1.2cm}|p{6cm}|p{4.3cm}|p{1.2cm}|@{}}
\caption{\normalsize List of conferences and journals considered in our survey.}
\label{table:papers} \\
\hline

	\textbf{Acronym} & \textbf{Name}	&	\textbf{\begin{tabular}[c]{@{}l@{}}Type of Venue\\(Conference Proceedings, Journal)\end{tabular}}	&	\textbf{Publisher}\\

\hline
	ICST & International Conference on Software Testing, Validation and Verification &	Conference Proceedings	&	IEEE\\
	ICSTW & International Conference on Software Testing, Verification and Validation Workshops &	Conference Proceedings	&	IEEE\\
	ICSE & International Conference on Software Engineering &	Conference Proceedings	&	IEEE/ACM\\
	ASE & International Conference on Automated Software Engineering & Conference Proceedings	& IEEE/ACM\\
	ISSTA & International Symposium on Software Testing and Analysis & Conference Proceedings	&	ACM\\
	TSE & Transactions on Software Engineering & Journal	&	IEEE\\
	%TOSEM & Transactions on Software Engineering and Methodology & Journal	& ACM\\
	TR & Transactions on Reliability & Journal & IEEE\\
	IST & Information and Software Technology & Journal	&	Elsevier\\
	SCP & Science of Computer Programming & Journal	&	Elsevier\\
	STVR & Software Testing, Verification and Reliability & Journal	&	Wiley\\
\hline                                                           
\end{longtable}
\normalsize


In this chapter we provide a survey of the benchmarks considered in empirical evaluations described in research work presented in top software engineering conferences and journals, between 2013 and 2019. Table~\ref{table:papers} provides the list of conferences and journals considered in our survey. For every venue, we have considered all the empirical evaluations considering C and C++ case study systems. Table~\ref{table:papers} provides the acronym and name of the venue, the type of venue, i.e., whether if it is a journal of a conference proceedings, and finally the publisher of the venue. 

%\DONE{Add the table of conferences, describe the columns in the text. Columns should be: name, type of venue (conference proceedings,journal),publisher.}

%\DONE{I rewrote, I do not expect you to introduce a table. You may have writte someting like "In this section we describe of the most relevant benchmarks targeting C software systems, they are summarized in Table..". I rewrote differently.}

\REVTWO{C40}{In the following, we introduce Sections~\ref{section:industry:code} and~\ref{section:industry:data}, which present detailed information about benchmarks on code-driven and data-driven mutation testing, respectively.}

%\DONE{check the date}

\section{Code-Driven Mutation Testing Benchmarks}
\label{section:industry:code}

% !TEX root = MAIN.tex

\chapter{Software Validation Process Planning}

\subsection{General}

The approach followed to validate the \FAQAS is to apply it to different case studies provided by LuxSpace and GOMSpace.
Table~\ref{tab:caseStudies} provides the list of case studies along with an indication of the type of mutation testing (i.e., code-driven or data-driven) they will be targeted for. More information on the case studies are present in the \emph{D2}.

\begin{table}[htp]
\caption{Case studies for the FAQAS activity.}
\label{tab:caseStudies}
\begin{center}
\begin{tabular}{|p{1.2cm}|p{6cm}|p{2.5cm}|p{2.5cm}|}
\hline
\textbf{Partner}&\textbf{Case study}&\textbf{Code-driven}&\textbf{Data-driven}\\
\hline
LXS&System Test Suite for ESAIL&Y&Y\\
LXS&Unit Test Suite for ESAIL&Y&N\\
GSL&Unit Test Suite for libUtil&Y&N\\
GSL&Integration Test Suite for libgscsp&Y&Y\\
GSL&System Test Suite for libparam&Y&Y\\
ESA&MLFS mathematical library&Y&N\\
ESA&ASN1 Compiler&Y&N\\
\hline
\end{tabular}
\end{center}
\end{table}

\clearpage


Table~\ref{table:case_studies} provides the list of the case study systems considered in the literature. For each case study we report (1) the case study (i.e., the software under test - SUT), (2) the size of the SUT (the size may vary according to the specific study), (3) the size of the test suite for the SUT\footnote{The size may vary from an empirical evaluation to another; also, in some cases the test suite details were not available (N/A).} (4) the number of papers that report using the case study, and (5) the references to the papers that report the case study.

%\DONE{You have to describe what types of programs it contains, when it was developed.}

From Table~\ref{table:case_studies} can be seen that the Siemens, Make and Space are the most common case studies with 6, 5 and 4 uses, respectively. 
The most used case study is the \INDEX{Siemens suite}. The programs belonging to this suite are commonly used to evaluate state-of-the-art solutions because they it includes faulty versions affected by faults introduced by engineers during development. The suite is available through the Subject Infrastructure Repository (SIR) from the University of Nebraska-Lincoln\footnote{https://sir.csc.ncsu.edu/portal/index.php}, in particular the suite contains a diverse collection of C programs that include code involving integer and floating-point operations, pointers, memory allocation, loops and complex conditional expressions.The Siemens suite was introduced in 1994 by Hutchins et al.~\cite{hutchins1994experiments}, the programs from the suite come with a large pool of test cases written initially by Hutchins et al. and then augmented by Rothermel et al.~\cite{rothermel1998empirical}.

Table~\ref{table:benchmarks} provides the list of benchmarks identified in the literature. We report (1) the case studies of the benchmark (i.e., the software under test - SUT), (2) the size of SUT in terms of lines of code (LOC), (3) the size of the SUT test suite in terms of number of test cases, (4) the original goal of the evaluation, and (5) the actual reference to the paper in which the benchmark was originally presented. 



%\DONE{"Appreciate" is a little too much :)}
%From Table~\ref{table:benchmarks}, it also can be appreciated the wide variety of case studies selected in the literature, while some authors decide to assess their techniques using very simple programs such as \textit{abs}~\cite{tokumoto2016muvm}. Other authors preferred using Unix applications such as the Coreutils package~\cite{hariri2019comparing,papadakis2018mutation,chekam2017empirical}. 
%On the other hand, some authors selected very complex programs such as OpenSSL and LLVM~\cite{denisov2018mull}, but usually the experiments are exercised only on small components of these large applications.

The size of the benchmark case studies varies a lot. They include simple algorithms such as \textit{abs} (6 LOC)~\cite{tokumoto2016muvm}, large Unix utilities such as the Coreutils package~\cite{hariri2019comparing,papadakis2018mutation,chekam2017empirical}, and programs implementing complex functions such as OpenSSL and LLVM~\cite{denisov2018mull}.
However, when large and complex software systems are considered in the empirical evaluation, the evaluation concerns only a subset of the components of these large applications.
For example, in the study performed by Kintis et al.~\cite{kintis2017detecting} they considered the assessment of Vim, a Unix text editor of 362 KLOC, however, because of the size of the program, the authors decided to restrict the analysis only to a couple of components such as \texttt{spell} and \texttt{eval}, 16 and 22 KLOC, respectively. 

%\DONE{Provide an example. Describe what they did in a paper where they selected a component, otherwise the sentence above might not be understood}

Concerning the size of the selected software under test, the most relevant studies have been presented by Papadakis and Chekam~\cite{papadakis2018mutation,chekam2017empirical,papadakis2018mutant}. Specifically, these authors considered the case studies Coreutils, Findutils, Grep, Make and Codeflaws (83 KLOC, 18 KLOC, 9 KLOC, 35 KLOC and 266 KLOC respectively). 
Concerning the size of the test suite employed to assess their adequacy, again Papadakis and Chekam~\cite{papadakis2018mutation,chekam2017empirical,papadakis2018mutant} present the most comprehensive studies with test suite sizes ranging from 58\,131, to 122\,261 test cases.
Despite scalability remains an open problem for mutation testing, the work of Papadakis and Chekam~\cite{papadakis2018mutation,chekam2017empirical,papadakis2018mutant} shows that optimization techniques are scaling up to large software systems.

Concerning the adoption of \INDEX{industrial case studies}, the most recent work is that of \cite{delgado2018evaluation} where mutation testing has been applied to 15 functions of a Commercial Off The Shelf Component used in nuclear systems. 
Another paper evaluating the applicability of mutation testing to safety critical systems is that of Daran and Thavenod-Fosse~\cite{daran1996software}, who conducted a study to identify if mutations are correlated with real faults; the experimentation was carried out on a critical software from the civil nuclear field. Andrews et al.~\cite{andrews2005mutation}, who explored the relation between hand-seeded and real faults in the Space software. Space is a software developed at the European Space Agency that it has been used as case study in software engineering papers since 1998 \cite{frankl1998further}.
Baker and Habli~\cite{baker2012empirical} conducted experiments on two safety-critical airborne systems, C and Ada, that had satisfied the coverage requirements for certification. In their experiments, they found an effective subset of mutation operators able to detect multiple deficiencies in test suites already assessed by experts. 

\REVTWO{C41}{Even though, many efforts has been done to make code-driven mutation testing a scalable solution, we conclude from this benchmark section, that unfortunately no study has yet applied mutation testing to complex and large real industrial software. Furthermore, in the context of space software, the applicability of mutation testing on large-scale satellite systems has not been empirically evaluated yet.}

\REVTWO{C42}{From this section, we conclude that mutation testing as a technique is applied mainly within research environments, and that in particular, mutation testing has not been adopted formally within industrial software development environments.}

% carried out an empirical evaluation based on two safety-critical airborne systems that had satisfied the coverage requirements for certification. Those systems were developed using high-integrity subsets for C (MISRA C [33]) and Ada. In their experiments, they found an effective subset of mutation operators that was able to detect different deficiencies in tests suites which had already met statement and MC/DC coverage and had been manually peer-reviewed.

% !TEX root = MutationTestingSurvey.tex

\chapter{Mutation Testing Benchmarks}
\label{chapter:industry}

In Section~\ref{sec:limitations} we have provided an overview of state-of the-art solutions to perform mutation testing and address limitations of the mutation testing process.
Each of these solutions had been evaluated against a set of case study systems deemed representative for the usage context. Such case study systems consist of a software under test (SUT) and a test suite for the software.
Hereafter, we'll use the term \INDEX{benchmark} to indicate the set of case study systems used in the empirical evaluation presented in a research paper.
%introduce an empirical evaluation aimed to prove or disprove a specific theory proposed by the authors. 
These benchmarks can be used as reference for future mutation testing developments. 
Also, a deep understanding of the characteristics of the benchmarks considered in the literature, may provide insights on the generalizability of the results to space software (e.g, results evaluated against a real time application are more likely generalizable to the case of space software).


% !TEX root =  ../MutationTestingSurvey.tex

\setlength\LTleft{0pt}
\setlength\LTright{0pt}
\scriptsize 
\begin{longtable}{@{\extracolsep{\fill}}|p{1.2cm}|p{6cm}|p{4.3cm}|p{1.2cm}|@{}}
\caption{\normalsize List of conferences and journals considered in our survey.}
\label{table:papers} \\
\hline

	\textbf{Acronym} & \textbf{Name}	&	\textbf{\begin{tabular}[c]{@{}l@{}}Type of Venue\\(Conference Proceedings, Journal)\end{tabular}}	&	\textbf{Publisher}\\

\hline
	ICST & International Conference on Software Testing, Validation and Verification &	Conference Proceedings	&	IEEE\\
	ICSTW & International Conference on Software Testing, Verification and Validation Workshops &	Conference Proceedings	&	IEEE\\
	ICSE & International Conference on Software Engineering &	Conference Proceedings	&	IEEE/ACM\\
	ASE & International Conference on Automated Software Engineering & Conference Proceedings	& IEEE/ACM\\
	ISSTA & International Symposium on Software Testing and Analysis & Conference Proceedings	&	ACM\\
	TSE & Transactions on Software Engineering & Journal	&	IEEE\\
	%TOSEM & Transactions on Software Engineering and Methodology & Journal	& ACM\\
	TR & Transactions on Reliability & Journal & IEEE\\
	IST & Information and Software Technology & Journal	&	Elsevier\\
	SCP & Science of Computer Programming & Journal	&	Elsevier\\
	STVR & Software Testing, Verification and Reliability & Journal	&	Wiley\\
\hline                                                           
\end{longtable}
\normalsize


In this chapter we provide a survey of the benchmarks considered in empirical evaluations described in research work presented in top software engineering conferences and journals, between 2013 and 2019. Table~\ref{table:papers} provides the list of conferences and journals considered in our survey. For every venue, we have considered all the empirical evaluations considering C and C++ case study systems. Table~\ref{table:papers} provides the acronym and name of the venue, the type of venue, i.e., whether if it is a journal of a conference proceedings, and finally the publisher of the venue. 

%\DONE{Add the table of conferences, describe the columns in the text. Columns should be: name, type of venue (conference proceedings,journal),publisher.}

%\DONE{I rewrote, I do not expect you to introduce a table. You may have writte someting like "In this section we describe of the most relevant benchmarks targeting C software systems, they are summarized in Table..". I rewrote differently.}

\REVTWO{C40}{In the following, we introduce Sections~\ref{section:industry:code} and~\ref{section:industry:data}, which present detailed information about benchmarks on code-driven and data-driven mutation testing, respectively.}

%\DONE{check the date}

\section{Code-Driven Mutation Testing Benchmarks}
\label{section:industry:code}

% !TEX root = MAIN.tex

\chapter{Software Validation Process Planning}

\subsection{General}

The approach followed to validate the \FAQAS is to apply it to different case studies provided by LuxSpace and GOMSpace.
Table~\ref{tab:caseStudies} provides the list of case studies along with an indication of the type of mutation testing (i.e., code-driven or data-driven) they will be targeted for. More information on the case studies are present in the \emph{D2}.

\begin{table}[htp]
\caption{Case studies for the FAQAS activity.}
\label{tab:caseStudies}
\begin{center}
\begin{tabular}{|p{1.2cm}|p{6cm}|p{2.5cm}|p{2.5cm}|}
\hline
\textbf{Partner}&\textbf{Case study}&\textbf{Code-driven}&\textbf{Data-driven}\\
\hline
LXS&System Test Suite for ESAIL&Y&Y\\
LXS&Unit Test Suite for ESAIL&Y&N\\
GSL&Unit Test Suite for libUtil&Y&N\\
GSL&Integration Test Suite for libgscsp&Y&Y\\
GSL&System Test Suite for libparam&Y&Y\\
ESA&MLFS mathematical library&Y&N\\
ESA&ASN1 Compiler&Y&N\\
\hline
\end{tabular}
\end{center}
\end{table}

\clearpage


Table~\ref{table:case_studies} provides the list of the case study systems considered in the literature. For each case study we report (1) the case study (i.e., the software under test - SUT), (2) the size of the SUT (the size may vary according to the specific study), (3) the size of the test suite for the SUT\footnote{The size may vary from an empirical evaluation to another; also, in some cases the test suite details were not available (N/A).} (4) the number of papers that report using the case study, and (5) the references to the papers that report the case study.

%\DONE{You have to describe what types of programs it contains, when it was developed.}

From Table~\ref{table:case_studies} can be seen that the Siemens, Make and Space are the most common case studies with 6, 5 and 4 uses, respectively. 
The most used case study is the \INDEX{Siemens suite}. The programs belonging to this suite are commonly used to evaluate state-of-the-art solutions because they it includes faulty versions affected by faults introduced by engineers during development. The suite is available through the Subject Infrastructure Repository (SIR) from the University of Nebraska-Lincoln\footnote{https://sir.csc.ncsu.edu/portal/index.php}, in particular the suite contains a diverse collection of C programs that include code involving integer and floating-point operations, pointers, memory allocation, loops and complex conditional expressions.The Siemens suite was introduced in 1994 by Hutchins et al.~\cite{hutchins1994experiments}, the programs from the suite come with a large pool of test cases written initially by Hutchins et al. and then augmented by Rothermel et al.~\cite{rothermel1998empirical}.

Table~\ref{table:benchmarks} provides the list of benchmarks identified in the literature. We report (1) the case studies of the benchmark (i.e., the software under test - SUT), (2) the size of SUT in terms of lines of code (LOC), (3) the size of the SUT test suite in terms of number of test cases, (4) the original goal of the evaluation, and (5) the actual reference to the paper in which the benchmark was originally presented. 



%\DONE{"Appreciate" is a little too much :)}
%From Table~\ref{table:benchmarks}, it also can be appreciated the wide variety of case studies selected in the literature, while some authors decide to assess their techniques using very simple programs such as \textit{abs}~\cite{tokumoto2016muvm}. Other authors preferred using Unix applications such as the Coreutils package~\cite{hariri2019comparing,papadakis2018mutation,chekam2017empirical}. 
%On the other hand, some authors selected very complex programs such as OpenSSL and LLVM~\cite{denisov2018mull}, but usually the experiments are exercised only on small components of these large applications.

The size of the benchmark case studies varies a lot. They include simple algorithms such as \textit{abs} (6 LOC)~\cite{tokumoto2016muvm}, large Unix utilities such as the Coreutils package~\cite{hariri2019comparing,papadakis2018mutation,chekam2017empirical}, and programs implementing complex functions such as OpenSSL and LLVM~\cite{denisov2018mull}.
However, when large and complex software systems are considered in the empirical evaluation, the evaluation concerns only a subset of the components of these large applications.
For example, in the study performed by Kintis et al.~\cite{kintis2017detecting} they considered the assessment of Vim, a Unix text editor of 362 KLOC, however, because of the size of the program, the authors decided to restrict the analysis only to a couple of components such as \texttt{spell} and \texttt{eval}, 16 and 22 KLOC, respectively. 

%\DONE{Provide an example. Describe what they did in a paper where they selected a component, otherwise the sentence above might not be understood}

Concerning the size of the selected software under test, the most relevant studies have been presented by Papadakis and Chekam~\cite{papadakis2018mutation,chekam2017empirical,papadakis2018mutant}. Specifically, these authors considered the case studies Coreutils, Findutils, Grep, Make and Codeflaws (83 KLOC, 18 KLOC, 9 KLOC, 35 KLOC and 266 KLOC respectively). 
Concerning the size of the test suite employed to assess their adequacy, again Papadakis and Chekam~\cite{papadakis2018mutation,chekam2017empirical,papadakis2018mutant} present the most comprehensive studies with test suite sizes ranging from 58\,131, to 122\,261 test cases.
Despite scalability remains an open problem for mutation testing, the work of Papadakis and Chekam~\cite{papadakis2018mutation,chekam2017empirical,papadakis2018mutant} shows that optimization techniques are scaling up to large software systems.

Concerning the adoption of \INDEX{industrial case studies}, the most recent work is that of \cite{delgado2018evaluation} where mutation testing has been applied to 15 functions of a Commercial Off The Shelf Component used in nuclear systems. 
Another paper evaluating the applicability of mutation testing to safety critical systems is that of Daran and Thavenod-Fosse~\cite{daran1996software}, who conducted a study to identify if mutations are correlated with real faults; the experimentation was carried out on a critical software from the civil nuclear field. Andrews et al.~\cite{andrews2005mutation}, who explored the relation between hand-seeded and real faults in the Space software. Space is a software developed at the European Space Agency that it has been used as case study in software engineering papers since 1998 \cite{frankl1998further}.
Baker and Habli~\cite{baker2012empirical} conducted experiments on two safety-critical airborne systems, C and Ada, that had satisfied the coverage requirements for certification. In their experiments, they found an effective subset of mutation operators able to detect multiple deficiencies in test suites already assessed by experts. 

\REVTWO{C41}{Even though, many efforts has been done to make code-driven mutation testing a scalable solution, we conclude from this benchmark section, that unfortunately no study has yet applied mutation testing to complex and large real industrial software. Furthermore, in the context of space software, the applicability of mutation testing on large-scale satellite systems has not been empirically evaluated yet.}

\REVTWO{C42}{From this section, we conclude that mutation testing as a technique is applied mainly within research environments, and that in particular, mutation testing has not been adopted formally within industrial software development environments.}

% carried out an empirical evaluation based on two safety-critical airborne systems that had satisfied the coverage requirements for certification. Those systems were developed using high-integrity subsets for C (MISRA C [33]) and Ada. In their experiments, they found an effective subset of mutation operators that was able to detect different deficiencies in tests suites which had already met statement and MC/DC coverage and had been manually peer-reviewed.

% !TEX root = MutationTestingSurvey.tex

\chapter{Mutation Testing Benchmarks}
\label{chapter:industry}

In Section~\ref{sec:limitations} we have provided an overview of state-of the-art solutions to perform mutation testing and address limitations of the mutation testing process.
Each of these solutions had been evaluated against a set of case study systems deemed representative for the usage context. Such case study systems consist of a software under test (SUT) and a test suite for the software.
Hereafter, we'll use the term \INDEX{benchmark} to indicate the set of case study systems used in the empirical evaluation presented in a research paper.
%introduce an empirical evaluation aimed to prove or disprove a specific theory proposed by the authors. 
These benchmarks can be used as reference for future mutation testing developments. 
Also, a deep understanding of the characteristics of the benchmarks considered in the literature, may provide insights on the generalizability of the results to space software (e.g, results evaluated against a real time application are more likely generalizable to the case of space software).


\input{tables/papers}

In this chapter we provide a survey of the benchmarks considered in empirical evaluations described in research work presented in top software engineering conferences and journals, between 2013 and 2019. Table~\ref{table:papers} provides the list of conferences and journals considered in our survey. For every venue, we have considered all the empirical evaluations considering C and C++ case study systems. Table~\ref{table:papers} provides the acronym and name of the venue, the type of venue, i.e., whether if it is a journal of a conference proceedings, and finally the publisher of the venue. 

%\DONE{Add the table of conferences, describe the columns in the text. Columns should be: name, type of venue (conference proceedings,journal),publisher.}

%\DONE{I rewrote, I do not expect you to introduce a table. You may have writte someting like "In this section we describe of the most relevant benchmarks targeting C software systems, they are summarized in Table..". I rewrote differently.}

\REVTWO{C40}{In the following, we introduce Sections~\ref{section:industry:code} and~\ref{section:industry:data}, which present detailed information about benchmarks on code-driven and data-driven mutation testing, respectively.}

%\DONE{check the date}

\section{Code-Driven Mutation Testing Benchmarks}
\label{section:industry:code}

\input{tables/case_studies}

Table~\ref{table:case_studies} provides the list of the case study systems considered in the literature. For each case study we report (1) the case study (i.e., the software under test - SUT), (2) the size of the SUT (the size may vary according to the specific study), (3) the size of the test suite for the SUT\footnote{The size may vary from an empirical evaluation to another; also, in some cases the test suite details were not available (N/A).} (4) the number of papers that report using the case study, and (5) the references to the papers that report the case study.

%\DONE{You have to describe what types of programs it contains, when it was developed.}

From Table~\ref{table:case_studies} can be seen that the Siemens, Make and Space are the most common case studies with 6, 5 and 4 uses, respectively. 
The most used case study is the \INDEX{Siemens suite}. The programs belonging to this suite are commonly used to evaluate state-of-the-art solutions because they it includes faulty versions affected by faults introduced by engineers during development. The suite is available through the Subject Infrastructure Repository (SIR) from the University of Nebraska-Lincoln\footnote{https://sir.csc.ncsu.edu/portal/index.php}, in particular the suite contains a diverse collection of C programs that include code involving integer and floating-point operations, pointers, memory allocation, loops and complex conditional expressions.The Siemens suite was introduced in 1994 by Hutchins et al.~\cite{hutchins1994experiments}, the programs from the suite come with a large pool of test cases written initially by Hutchins et al. and then augmented by Rothermel et al.~\cite{rothermel1998empirical}.

Table~\ref{table:benchmarks} provides the list of benchmarks identified in the literature. We report (1) the case studies of the benchmark (i.e., the software under test - SUT), (2) the size of SUT in terms of lines of code (LOC), (3) the size of the SUT test suite in terms of number of test cases, (4) the original goal of the evaluation, and (5) the actual reference to the paper in which the benchmark was originally presented. 



%\DONE{"Appreciate" is a little too much :)}
%From Table~\ref{table:benchmarks}, it also can be appreciated the wide variety of case studies selected in the literature, while some authors decide to assess their techniques using very simple programs such as \textit{abs}~\cite{tokumoto2016muvm}. Other authors preferred using Unix applications such as the Coreutils package~\cite{hariri2019comparing,papadakis2018mutation,chekam2017empirical}. 
%On the other hand, some authors selected very complex programs such as OpenSSL and LLVM~\cite{denisov2018mull}, but usually the experiments are exercised only on small components of these large applications.

The size of the benchmark case studies varies a lot. They include simple algorithms such as \textit{abs} (6 LOC)~\cite{tokumoto2016muvm}, large Unix utilities such as the Coreutils package~\cite{hariri2019comparing,papadakis2018mutation,chekam2017empirical}, and programs implementing complex functions such as OpenSSL and LLVM~\cite{denisov2018mull}.
However, when large and complex software systems are considered in the empirical evaluation, the evaluation concerns only a subset of the components of these large applications.
For example, in the study performed by Kintis et al.~\cite{kintis2017detecting} they considered the assessment of Vim, a Unix text editor of 362 KLOC, however, because of the size of the program, the authors decided to restrict the analysis only to a couple of components such as \texttt{spell} and \texttt{eval}, 16 and 22 KLOC, respectively. 

%\DONE{Provide an example. Describe what they did in a paper where they selected a component, otherwise the sentence above might not be understood}

Concerning the size of the selected software under test, the most relevant studies have been presented by Papadakis and Chekam~\cite{papadakis2018mutation,chekam2017empirical,papadakis2018mutant}. Specifically, these authors considered the case studies Coreutils, Findutils, Grep, Make and Codeflaws (83 KLOC, 18 KLOC, 9 KLOC, 35 KLOC and 266 KLOC respectively). 
Concerning the size of the test suite employed to assess their adequacy, again Papadakis and Chekam~\cite{papadakis2018mutation,chekam2017empirical,papadakis2018mutant} present the most comprehensive studies with test suite sizes ranging from 58\,131, to 122\,261 test cases.
Despite scalability remains an open problem for mutation testing, the work of Papadakis and Chekam~\cite{papadakis2018mutation,chekam2017empirical,papadakis2018mutant} shows that optimization techniques are scaling up to large software systems.

Concerning the adoption of \INDEX{industrial case studies}, the most recent work is that of \cite{delgado2018evaluation} where mutation testing has been applied to 15 functions of a Commercial Off The Shelf Component used in nuclear systems. 
Another paper evaluating the applicability of mutation testing to safety critical systems is that of Daran and Thavenod-Fosse~\cite{daran1996software}, who conducted a study to identify if mutations are correlated with real faults; the experimentation was carried out on a critical software from the civil nuclear field. Andrews et al.~\cite{andrews2005mutation}, who explored the relation between hand-seeded and real faults in the Space software. Space is a software developed at the European Space Agency that it has been used as case study in software engineering papers since 1998 \cite{frankl1998further}.
Baker and Habli~\cite{baker2012empirical} conducted experiments on two safety-critical airborne systems, C and Ada, that had satisfied the coverage requirements for certification. In their experiments, they found an effective subset of mutation operators able to detect multiple deficiencies in test suites already assessed by experts. 

\REVTWO{C41}{Even though, many efforts has been done to make code-driven mutation testing a scalable solution, we conclude from this benchmark section, that unfortunately no study has yet applied mutation testing to complex and large real industrial software. Furthermore, in the context of space software, the applicability of mutation testing on large-scale satellite systems has not been empirically evaluated yet.}

\REVTWO{C42}{From this section, we conclude that mutation testing as a technique is applied mainly within research environments, and that in particular, mutation testing has not been adopted formally within industrial software development environments.}

% carried out an empirical evaluation based on two safety-critical airborne systems that had satisfied the coverage requirements for certification. Those systems were developed using high-integrity subsets for C (MISRA C [33]) and Ada. In their experiments, they found an effective subset of mutation operators that was able to detect different deficiencies in tests suites which had already met statement and MC/DC coverage and had been manually peer-reviewed.

\input{tables/industry}

\clearpage

\section{Data-Driven Mutation Testing Benchmarks}
\label{section:industry:data}

In this section we provide a survey of the benchmarks considered both in empirical evaluations described in research work and in industrial cases for data-driven mutation testing.

Table~\ref{table:benchmarks_datadriven} provides the list of the case study systems considered in the literature and industry. Similar to the previous section, for each case study we report (1) the case study (i.e., the software under test - SUT), (2) the size of the SUT (the size can be expressed in terms of bytecode instructions, lines of code (LOC), or executable size (KB or MB)), (3) the size of the test suite for the SUT (unfortunately, in most cases the test suite details were not available (N/A)), and (4) the references to the papers that report the case study.

Table~\ref{table:benchmarks_datadriven} presents case studies that were applied to UML models~\cite{di2017augmenting}, block models~\cite{pham2016model} and grammars~\cite{AFL:industrialcases}.

Concerning the size and typology of SUT, the case studies reported in Table~\ref{table:benchmarks_datadriven} varies a lot.
For instance, experimentation on block models has been applied to different types of desktop applications such as VLC, Adobe Reader, Real Player, and Windows Media Player~\cite{pham2016model} which sizes goes from 60 KB to 2.32 MB. 

Experimentation on grammars has been widely studied, but for sake of simplicity we only reported the most important case studies for AFL tool~\cite{AFL:industrialcases}. Particularly, AFL has been applied to a very wide range of software typology, for example it has been applied on programming languages such as PHP, web browsers such as Firefox, libraries such as OpenSSL, and even application server such as MySQL Server. 

Regarding case studies applied to space context software, we highlight SES-DAQ, a DAQ system developed by SES that processes bytestreams of transmitted satellite data. The version of SES-DAQ used for the experimentation performed by Di Nardo et al.~\cite{di2017augmenting} had a size of 32\,469 bytecode instructions.

\input{tables/industry_datadriven}




\clearpage

\section{Data-Driven Mutation Testing Benchmarks}
\label{section:industry:data}

In this section we provide a survey of the benchmarks considered both in empirical evaluations described in research work and in industrial cases for data-driven mutation testing.

Table~\ref{table:benchmarks_datadriven} provides the list of the case study systems considered in the literature and industry. Similar to the previous section, for each case study we report (1) the case study (i.e., the software under test - SUT), (2) the size of the SUT (the size can be expressed in terms of bytecode instructions, lines of code (LOC), or executable size (KB or MB)), (3) the size of the test suite for the SUT (unfortunately, in most cases the test suite details were not available (N/A)), and (4) the references to the papers that report the case study.

Table~\ref{table:benchmarks_datadriven} presents case studies that were applied to UML models~\cite{di2017augmenting}, block models~\cite{pham2016model} and grammars~\cite{AFL:industrialcases}.

Concerning the size and typology of SUT, the case studies reported in Table~\ref{table:benchmarks_datadriven} varies a lot.
For instance, experimentation on block models has been applied to different types of desktop applications such as VLC, Adobe Reader, Real Player, and Windows Media Player~\cite{pham2016model} which sizes goes from 60 KB to 2.32 MB. 

Experimentation on grammars has been widely studied, but for sake of simplicity we only reported the most important case studies for AFL tool~\cite{AFL:industrialcases}. Particularly, AFL has been applied to a very wide range of software typology, for example it has been applied on programming languages such as PHP, web browsers such as Firefox, libraries such as OpenSSL, and even application server such as MySQL Server. 

Regarding case studies applied to space context software, we highlight SES-DAQ, a DAQ system developed by SES that processes bytestreams of transmitted satellite data. The version of SES-DAQ used for the experimentation performed by Di Nardo et al.~\cite{di2017augmenting} had a size of 32\,469 bytecode instructions.

% !TEX root =  ../MutationTestingSurvey.tex


\setlength\LTleft{0pt}
\setlength\LTright{0pt}
\small 
\begin{longtable}{@{\extracolsep{\fill}}|p{4cm}|p{3.5cm}|p{3cm}|p{1.8cm}|@{}}
\caption{\normalsize Summary of Data-Driven Mutation Testing Benchmarks.}
\label{table:benchmarks_datadriven} \\

\hline

\textbf{Case Study (SUT)}	&	\textbf{Size of SUT}	&	\textbf{\begin{tabular}[c]{@{}l@{}}Size of Test Suite\\(test cases)\end{tabular}}	 & \textbf{Reference}	\\
\hline

SES-DAQ & 32\,469 bytecode instructions & 32 & \cite{di2017augmenting,di2015evolutionary,di2015generating} \\

VLC 2.0.7 & 184 KB & N/A & \cite{pham2016model} \\
Libpng 1.5.4 & 176 KB & N/A & \cite{pham2016model} \\
XnView 1.98 & 4.46 MB & N/A & \cite{pham2016model} \\
Adobe Reader 9.2& 2.32 MB & N/A & \cite{pham2016model} \\
Windows Media Player 9.0 & 1.22 MB & N/A & \cite{pham2016model} \\
Real Player SP 1.0 & 60 KB & N/A & \cite{pham2016model} \\
MIDI Player 0.35& 336 KB & N/A & \cite{pham2016model} \\
Orbital Viewer 1.04 & 538 KB & N/A & \cite{pham2016model} \\

PHP 5.4.36 & 735 KLOC & N/A & \cite{AFL:industrialcases}\\
Firefox 32.0.1 & 62 MB & N/A & \cite{AFL:industrialcases}\\
OpenSSL 1.0.2 & 264 KLOC & N/A & \cite{AFL:industrialcases}\\
PuTTY 0.54 & 68 KLOC & N/A & \cite{AFL:industrialcases}\\
MySQL Server 5.7 & 373 MB & N/A & \cite{AFL:industrialcases}\\
indent 2.2.11 & 24 KLOC & N/A & \cite{AFL:industrialcases}\\

HotelRS & 1.5 KLOC & 33  & \cite{Appelt:SQLI:ISSTA:2014}\\
Sugar-CRM & 352 KLOC & 33 & \cite{Appelt:SQLI:ISSTA:2014}\\

JavaScript Interpreter (IE7) & 113\,562 machine instructions& 2\,800 & \cite{Godefroid:GrammarBasedFuzzying:2008}\\

\textbf{Windows NT utilities}
\begin{minipage}[t]{2.5cm}
attrib\\
chkdsk\\
comp\\
expand\\
fc\\
find\\
help\\
label\\
replace\\
\end{minipage}
 & 
 \begin{minipage}[t]{2.5cm}
 \hfill\\
21 KB\\
25 KB\\
25 KB\\
65 KB\\
24 KB\\
17 KB\\
12 KB\\
17 KB\\
21 KB\\
\end{minipage}
 &  \begin{minipage}[t]{2.5cm}
 \hfill\\
 N/A
\end{minipage} &  \begin{minipage}[t]{2.5cm}
 \hfill\\
 \cite{ghosh1998testing}
\end{minipage}\\
 

\hline                                                           
\end{longtable}


\normalsize




\clearpage

\section{Data-Driven Mutation Testing Benchmarks}
\label{section:industry:data}

In this section we provide a survey of the benchmarks considered both in empirical evaluations described in research work and in industrial cases for data-driven mutation testing.

Table~\ref{table:benchmarks_datadriven} provides the list of the case study systems considered in the literature and industry. Similar to the previous section, for each case study we report (1) the case study (i.e., the software under test - SUT), (2) the size of the SUT (the size can be expressed in terms of bytecode instructions, lines of code (LOC), or executable size (KB or MB)), (3) the size of the test suite for the SUT (unfortunately, in most cases the test suite details were not available (N/A)), and (4) the references to the papers that report the case study.

Table~\ref{table:benchmarks_datadriven} presents case studies that were applied to UML models~\cite{di2017augmenting}, block models~\cite{pham2016model} and grammars~\cite{AFL:industrialcases}.

Concerning the size and typology of SUT, the case studies reported in Table~\ref{table:benchmarks_datadriven} varies a lot.
For instance, experimentation on block models has been applied to different types of desktop applications such as VLC, Adobe Reader, Real Player, and Windows Media Player~\cite{pham2016model} which sizes goes from 60 KB to 2.32 MB. 

Experimentation on grammars has been widely studied, but for sake of simplicity we only reported the most important case studies for AFL tool~\cite{AFL:industrialcases}. Particularly, AFL has been applied to a very wide range of software typology, for example it has been applied on programming languages such as PHP, web browsers such as Firefox, libraries such as OpenSSL, and even application server such as MySQL Server. 

Regarding case studies applied to space context software, we highlight SES-DAQ, a DAQ system developed by SES that processes bytestreams of transmitted satellite data. The version of SES-DAQ used for the experimentation performed by Di Nardo et al.~\cite{di2017augmenting} had a size of 32\,469 bytecode instructions.

% !TEX root =  ../MutationTestingSurvey.tex


\setlength\LTleft{0pt}
\setlength\LTright{0pt}
\small 
\begin{longtable}{@{\extracolsep{\fill}}|p{4cm}|p{3.5cm}|p{3cm}|p{1.8cm}|@{}}
\caption{\normalsize Summary of Data-Driven Mutation Testing Benchmarks.}
\label{table:benchmarks_datadriven} \\

\hline

\textbf{Case Study (SUT)}	&	\textbf{Size of SUT}	&	\textbf{\begin{tabular}[c]{@{}l@{}}Size of Test Suite\\(test cases)\end{tabular}}	 & \textbf{Reference}	\\
\hline

SES-DAQ & 32\,469 bytecode instructions & 32 & \cite{di2017augmenting,di2015evolutionary,di2015generating} \\

VLC 2.0.7 & 184 KB & N/A & \cite{pham2016model} \\
Libpng 1.5.4 & 176 KB & N/A & \cite{pham2016model} \\
XnView 1.98 & 4.46 MB & N/A & \cite{pham2016model} \\
Adobe Reader 9.2& 2.32 MB & N/A & \cite{pham2016model} \\
Windows Media Player 9.0 & 1.22 MB & N/A & \cite{pham2016model} \\
Real Player SP 1.0 & 60 KB & N/A & \cite{pham2016model} \\
MIDI Player 0.35& 336 KB & N/A & \cite{pham2016model} \\
Orbital Viewer 1.04 & 538 KB & N/A & \cite{pham2016model} \\

PHP 5.4.36 & 735 KLOC & N/A & \cite{AFL:industrialcases}\\
Firefox 32.0.1 & 62 MB & N/A & \cite{AFL:industrialcases}\\
OpenSSL 1.0.2 & 264 KLOC & N/A & \cite{AFL:industrialcases}\\
PuTTY 0.54 & 68 KLOC & N/A & \cite{AFL:industrialcases}\\
MySQL Server 5.7 & 373 MB & N/A & \cite{AFL:industrialcases}\\
indent 2.2.11 & 24 KLOC & N/A & \cite{AFL:industrialcases}\\

HotelRS & 1.5 KLOC & 33  & \cite{Appelt:SQLI:ISSTA:2014}\\
Sugar-CRM & 352 KLOC & 33 & \cite{Appelt:SQLI:ISSTA:2014}\\

JavaScript Interpreter (IE7) & 113\,562 machine instructions& 2\,800 & \cite{Godefroid:GrammarBasedFuzzying:2008}\\

\textbf{Windows NT utilities}
\begin{minipage}[t]{2.5cm}
attrib\\
chkdsk\\
comp\\
expand\\
fc\\
find\\
help\\
label\\
replace\\
\end{minipage}
 & 
 \begin{minipage}[t]{2.5cm}
 \hfill\\
21 KB\\
25 KB\\
25 KB\\
65 KB\\
24 KB\\
17 KB\\
12 KB\\
17 KB\\
21 KB\\
\end{minipage}
 &  \begin{minipage}[t]{2.5cm}
 \hfill\\
 N/A
\end{minipage} &  \begin{minipage}[t]{2.5cm}
 \hfill\\
 \cite{ghosh1998testing}
\end{minipage}\\
 

\hline                                                           
\end{longtable}


\normalsize






