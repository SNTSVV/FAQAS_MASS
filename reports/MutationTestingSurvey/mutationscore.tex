% !TEX root = MutationTestingSurvey.tex

\subsection{Mutation Score Calculation}
\label{sub:mutationscore}

%\DONE{Is there anything we can add here?}

\MREVISION{C6}{The \INDEX{mutation score} captures, in percentage points, the quality of a test suite. It measures the percentage of mutants that had been killed by the test suites.} 
The mutation testing process is driven by the mutation score; the process iterates multiple times until the mutation score reaches a certain threshold. 

According to recent studies, there is a relation between the mutation score and the fault revelation ability of mutation testing.
\MREVISION{C7}{A recent empirical evaluation, for example, has shown that \textit{achieving high mutation scores improves significantly the fault detection capability of a test suite}
~\cite{papadakis2018mutation}. 
%\DONE{The following sentence is incomplete. You cannot say "similar fault detection ability that statement and branch coverage". Do you mean 100\% branch coverage (i.e., branch adequate)? Also putting together branch and statement is tricky, because branch is a stronger criterion. }
This evaluation shows that from a large set of test suites, the subsets of test suites with highest 10\% of branch coverage had a fault revelation rate of 0.542, while the subsets of test suites with highest 10\% of mutation coverage had a fault revelation of 0.639. Instead, when considering the subsets of test suites with highest 20\% of branch and mutation coverage, the fault revelation rates dropped in both cases to 0.524 and 0.565, respectively.
The drawback of this result, is, however, that the fault detection ability of mutation testing is high only with a high mutation score.}


%This evaluation shows that a test suite that reaches a mutation score of 80\% has a similar  similar fault detection capability of one that achieves 80\% branch coverage. 
%\TODO{What does it mean outperforms? Can we be more precise?}
%Instead, a test suite that reaches a mutation score of 90\% outperforms this code coverage criterion. 


%To reliably compute the mutation score it is necessary to identify equivalent and redundant mutants. 
As seen in Section~\ref{sec:opt:equivalent} and~\ref{sec:opt:redundant}, equivalent and redundant mutants can affect (i.e., cause an overestimation or underestimation) the mutation score~\cite{papadakis2016threats,kintis2017detecting}.
Hence, it is desirable to identify the presence of such mutants before estimating the fault revelation ability of a test suite.
