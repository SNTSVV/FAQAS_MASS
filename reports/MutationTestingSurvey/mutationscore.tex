% !TEX root = MutationTestingSurvey.tex

\subsection{Mutation Score Calculation}
\label{sub:mutationscore}

\DONE{Is there anything we can add here?}

\MREVISION{C6}{The mutation score indicates quantitatively the quality of a test suite, the score is estimated by taking into account the proportion of killed mutants with respect to total amount of mutants.} 
As introduced in Section \ref{sec:process}, the mutation testing process is leaded by the mutation score, that is, the process will iterate multiple times until the test suite adequacy is such that the mutation score reaches a certain threshold. 

According to recent studies, there is a relation between the mutation score and fault revelation ability of mutation testing.
\MREVISION{C7}{Particularly, Papadakis et al. \cite{papadakis2018mutation} states that \textit{achieving higher mutation scores improves significantly the fault detection}. In an experimental evaluation they demonstrate that a test suite that reaches a mutation score of 80\% has similar fault detection ability that statement and branch coverage, instead a test suite that reaches a mutation score of 90\% or 95\% outperforms code coverage test criterion. This means that the fault detection ability of mutation testing can be trusted only at higher mutation score levels.}

Computing the mutation score includes the task of identifying equivalent and redundant mutants. As seen in Section \ref{sec:opt:equivalent} and \ref{sec:opt:redundant}, these type of mutants can obscure and overestimate or underestimate the true score \cite{papadakis2016threats,kintis2017detecting}.
Hence, these mutants must be assessed before estimating the fault revelation ability of a test suite through the mutation testing process.
