% !TEX root = MutationTestingSurvey.tex

\subsection{Mutation Score Calculation}
\label{sub:mutationscore}

%\DONE{Is there anything we can add here?}

\MREVISION{C6}{The \INDEX{mutation score} captures, in percentage points, the quality of a test suite. It measures the percentage of mutants that had been killed by the test suite.} 
The mutation testing process is driven by the mutation score; the process iterates multiple times until the mutation score reaches a certain threshold. 

According to recent studies, there is a relation between the mutation score and the fault revelation ability of mutation testing.
\MREVISION{C7}{A recent empirical evaluation, for example, has shown that \textit{achieving a high mutation score improves significantly the fault detection capability of a test suite}
~\cite{papadakis2018mutation}. 
%\DONE{The following sentence is incomplete. You cannot say "similar fault detection ability that statement and branch coverage". Do you mean 100\% branch coverage (i.e., branch adequate)? Also putting together branch and statement is tricky, because branch is a stronger criterion. }
%This evaluation shows that from a large set of test suites, the subsets of test suites with highest 10\% of branch coverage had a fault revelation rate of 0.542, while the subsets of test suites with highest 10\% of mutation coverage had a fault detection rate of 0.639. Instead, when considering the subsets of test suites with highest 20\% of branch and mutation coverage, the fault revelation rates dropped in both cases to 0.524 and 0.565, respectively.
The evaluation shows that, from a large set of test suites, the top 10\% ranked according to branch coverage (i.e., the ones with the highest branch coverage) have a \INDEX{fault detection
 rate} (i.e., the portion of real faults being detected by the test suite)  of 0.542, while the top 10\% ranked according to mutation coverage (i.e., the ones with the highest mutation score) have a fault detection rate of 0.639. 
If we consider, instead, test suites with lower branch and mutation coverage (i.e., top 20\% for both branch and mutation coverage), the fault detection rate drops to 0.524 and 0.565, for branch and mutation coverage, respectively.
 The drawback of this result, is, however, that only a high mutation score ensures to have test suites with a fault detection rate above 0.6.}
\REVTWO{C6}{Similar results are achieved also by Checkam et al., who show that,
among randomly selected test suites, only the ones achieving top 5\% mutation score achieve better fault detection rate than the others~\cite{Chekam:17}. 
These studies also show that, even when automated test case generation is in place, 100\% mutation score is unlikely achievable. In the work of Checkam et al., for example, the median mutation score is 0.57.}
\REVTWO{C7}{Despite the literature does not include studies concerning the identification of a threshold that guarantees that the test suite has a high fault detection rate (e.g., higher than branch coverage adequacy), the data reported in Checkam's work show that the mutation score of the best test suites is above 75\%. The value 75\% might thus be considered as a \INDEX{threshold} for terminating the mutation testing process.}

%This evaluation shows that a test suite that reaches a mutation score of 80\% has a similar  similar fault detection capability of one that achieves 80\% branch coverage. 
%\TODO{What does it mean outperforms? Can we be more precise?}
%Instead, a test suite that reaches a mutation score of 90\% outperforms this code coverage criterion. 


%To reliably compute the mutation score it is necessary to identify equivalent and redundant mutants. 
As seen in Section~\ref{sec:opt:equivalent} and~\ref{sec:opt:redundant}, equivalent and redundant mutants can affect (i.e., cause an overestimation or underestimation) the mutation score~\cite{papadakis2016threats,kintis2017detecting}.
Hence, it is desirable to identify the presence of such mutants before estimating the fault revelation ability of a test suite.
