% !TEX root = MutationTestingSurvey.tex

\subsection{Mutation Score Calculation}
\label{sub:mutationscore}

%\DONE{Is there anything we can add here?}

\MREVISION{C6}{The \INDEX{mutation score} captures, in percentage points, the quality of a test suite. It measures the percentage of mutants that had been killed by the test suites.} 
The mutation testing process is driven by the mutation score; the process iterates multiple times until the mutation score reaches a certain threshold. 

According to recent studies, there is a relation between the mutation score and the fault revelation ability of mutation testing.
\MREVISION{C7}{A recent empirical evaluation, for example, has shown that \textit{achieving high mutation scores improves significantly the fault detection capability of a test suite}
~\cite{papadakis2018mutation}. 
\TODO{The following sentence is incomplete. You cannot say "similar fault detection ability that statement and branch coverage". Do you mean 100\% branch coverage (i.e., branch adequate)? Also putting together branch and statement is tricky, because branch is a stronger criterion. }
This evaluation shows that a test suite that reaches a mutation score of 80\% has a similar  similar fault detection capability of one that achieves XX\% branch coverage. 
\TODO{Why do we have to put both 90 and 95?}
Instead, a test suite that reaches a mutation score of 90\% or 95\% outperforms this code coverage criterion. The drawback of this result, is, however, that the fault detection ability of mutation testing is high only with a high mutation score.}

%To reliably compute the mutation score it is necessary to identify equivalent and redundant mutants. 
As seen in Section \ref{sec:opt:equivalent} and \ref{sec:opt:redundant}, equivalent and redundant mutants can affect (i.e., cause an overestimation or underestimation) the mutation score \cite{papadakis2016threats,kintis2017detecting}.
Hence, it is desirable to identify the presence of such mutants before estimating the fault revelation ability of a test suite.
