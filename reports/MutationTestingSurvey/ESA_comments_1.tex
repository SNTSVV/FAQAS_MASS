% !TEX root = MutationTestingSurvey.tex

\section*{Response to ESA comments provided on 03.03.2020}
\label{sec:ESA:comments:1}


\setlength\LTleft{0pt}
\setlength\LTright{0pt}
\tiny 
%@{\extracolsep{\fill}}
\begin{longtable}{|p{1.2cm}|p{12cm}|@{}}
\caption{\normalsize .}
\label{table:codeoperators} \\

C1&
\begin{minipage}{8cm}
Please, consider the following suggestions:\\
- Add a table to keep the change record or the document (the table could have e.g. Issue number, Date, Change description, and Author)\\
- Add a Table of Contents\\
- Add an introduction chapter, with scope of the document, applicable and reference documents, and acronyms and abbreviations\\
\end{minipage}
\\
&
We have added table of contents and revisions.
We have added acronyms to the introduction.

\\

C2
&When generating the pdf file, make sure to have the index of the document on the left, so that the document is easier to navigate.
\TODO{Cannot understand}
\\

C3&
Try to make the document a bit more friendly to read (e.g. by adding some sentences in bold (see the highlighted ones for this paragraph), making bigger
separation between paragraphs, ...).
\\
We have implemented the two suggestions proposed.
\\


C4&
It is perhaps interesting to add a chapter with some basic definitions?
e.g. equivalent mutant, redundant mutant, mutation score, killed mutant, live mutant, weak mutation, ...
\\
We have added an index with the references to the page where terms are introduced. We decided to not add a glossary because it would lead to a duplication of definitions.

\TODO{OScar, please add 'index' keywords where appropriate in the source.}

\\
C5&
Please, consider to add more examples. Similar to what is presented in the slides in the different progress meetings, where often the different concepts
are introduced with an example.
\\
We have added more examples as suggested.
\\

C6&
Elaborate a bit more on the mutation score concept (e.g. what is achievable in reality e.g. can we target 100\%?, how this number can be correlated with
the quality of a test suite e.g. if a test suite presents 80\% score, can it be considered very good?
\\
&
\TODO{Please add something based on references. You should anyway, add a chapter on mutation score.}
\\


C7&
Please, elaborate on the concept of having a threshold to stop the process.
\\
&
\TODO{Please try to say something, based on references.}
\\

C8:&
Please, comment a bit on the possibility to mutate the IR, executable, assembly code; and pros and cons.\\
&
\TODO{Please try to say something, based on references.}
\\

C9&
Is this really the only criteria to select the mutation operators?
\\
\TODO{Can we say anything?}
\\

C10&
\begin{minipage}{8cm}
"the context of space software and embedded systems"
This is to be refined:\\
- embedded systems is clear\\
- space software term can be ambiguous... (ground desktop SW, ground embedded SW, on-board embedded SW, ...)\\
\end{minipage}
\\
In the introduction, we have clarified how the term space software is used in this book (i.e., to refer to on-board embedded SW).
\\


C11&
\begin{minipage}{8cm}
a. There is a minimum set of operators. Is there anything for the Maximum?
b. There are many operators. Any preliminary idea of the ones we should try (or the strategy to follow) within this activity?
c. High-order operators: Is there any strategy to combine operators to define the high order ones?
d. Is there any logic followed for naming the different mutation operators? (e.g. OBBN). Are those terms consistently used by practitioners?
\end{minipage}
\\
We have addressed the comments above. Concerning \emph{C11.a} we can only provide references to work that rely on sampling of mutants, work on the identification of an upper bound for the number of operators is not available, to the best of our knowledge.
\TODO{Concerning C11.b, At the end of section 1.3, can you add the summary table that we used in the slides where we provide a strategy to prioritize the operators? You should also add a paragraph that describes the table.}
\TODO{Can you add something, somewhere for C11.c?}
\\


C12&
\begin{minipage}{8cm}
a. There are many techniques presented for the different limitations. Do you have any idea already on the techniques that you would like to try within
FAQAS?
b. Is there any particular technique that you would like to perform some research on?
\end{minipage}
\\
\begin{minipage}{8cm}
(C12.a) In the context of the project, we will evaluate the following approaches:
\begin{itemize}
\item Selection of a subset of operators (e.g., deletion and sufficient)
\item Random sampling of mutants
\item Among the live mutants, identification of non-equivalent mutants based on code coverage
\item Among the live mutants, identification of non-equivalent mutants by means of symbolic execution or bounded model checking
\item Among the live mutants, identification of redundant mutants based on code coverage
\end{itemize}

(C12.b) The effectiveness of the selected solutions for complex, space software, will be reported in a research paper.
\end{minipage}

C13&
\begin{minipage}{8cm}
a. It is not trivial to understand why Equivalent Mutants may lead mutation testing infeasible.\\
b. The fact of having a high number of equivalent / redundant mutants suggests that the quality of the source code is not that great; or that the used mutation
operators are not good. \\
c. Also, the observability of the system may not be good enough to see the effects of the mutation.\\
\end{minipage}

Concerning .b and .c, it is not possible to provide a general rule relating the presence of equivalent mutants and the quality of code. For example, in while loops, we may observe the statement \texttt{var++} in a single line (see example isPalindrome). Such statement may lead to equivalent mutants generated by the operator that changes \texttt{var++} into \texttt{++var}. However, the code might be well written. Also, the operator, which may lead to equivalent mutants in this case, may be useful in others. A refinement of the scope in which the operator is applied might be possible (e.g., not when the increment is alone in a statement, nor when it within a for loop), however it might be hard to generalize.

\TODO{Concerning .b and .c, can you check the paper that states that the identification of eq mut is infeasible? It should help us to clarify, that, in general it is natural to have variables that cannot propagate the effect of mutations to the outputs.}
\\

C14&
Do you have general numbers? e.g. what was the percentage of killed mutants?
\\
\TODO{can you check the numbers in the paper and add something?}
\\

C15&
As for equivalent mutants, it is not trivial to understand the consequences of redundant mutants. It also suggest poor source code / mutation selection /
system visibility.
\\
\TODO{Is there any research in this?}
\\

C16&
Within the report, there are several numbers like this 9\%, or the 45\% in equivalent mutants, etc. Do you plan to do some research within FAQAS to check
on those numbers?
\\
\TODO{Check TODO in chapter}
\\


C17&
My first impression after reading this chapter is that Data Mutation would be quite applicable to our software systems (perhaps a bit more from the point of
view of verifying robustness of the system, rather than assessing an existing test suite).
Do you already have an idea on how to apply this within this activity and how deep to go in the approach?
\\
\TODO{TODO Fabrizio}
\\


C18&
Only Software faults? This can be true for data exchanged among SW components; but if the data comes from somewhere else, it can be any kind of
issue.
\\
We have clarified in the paragraph.
\\

C19&
Could we say that Objective 1 is linked to Test Suite Assessment and Objective 2 is linked so system robustness testing? If so, it is better to state it.
\\
This is not exactly the case, we have clarified in the text.
\\

%Author: Pedro Barrios Subject: Sticky Note Date: 28/02/2020, 09:18:11
%Comment #4:
%It is perhaps interesting to add a chapter with some basic definitions?
%e.g. equivalent mutant, redundant mutant, mutation score, killed mutant, live mutant, weak mutation, ...
%Author: Pedro Barrios Subject: Sticky Note Date: 28/02/2020, 09:18:18
%Comment #5:
%Please, consider to add more examples.

\bottomrule                                                             
\end{longtable}
\normalsize

\clearpage