% !TEX root = MutationTestingSurvey.tex
\chapter{ESA Revisions}
\section{Responses to ESA comments provided on 03.03.2020}
\label{sec:ESA:comments:1}


\setlength\LTleft{0pt}
\setlength\LTright{0pt}
\tiny 
%@{\extracolsep{\fill}}
\begin{longtable}{|p{1.2cm}|p{12cm}|@{}}
%\caption{\normalsize .}
\label{table:comments:responses} \\
C1&
\begin{minipage}{8cm}
Please, consider the following suggestions:\\
- Add a table to keep the change record or the document (the table could have e.g. Issue number, Date, Change description, and Author)\\
- Add a Table of Contents\\
- Add an introduction chapter, with scope of the document, applicable and reference documents, and acronyms and abbreviations\\
\end{minipage}
\\
\cmidrule{2-2}
&
\begin{minipage}{8cm}
We have added table of contents and revisions.\\
We have added acronyms to the introduction.
\end{minipage}
\\
\hline
C2
&When generating the pdf file, make sure to have the index of the document on the left, so that the document is easier to navigate.
\\
\cmidrule{2-2}
&
\DONE{Cannot understand what is the point of this comment. Oscar: I can see the bookmarks of the document on the left, I think that is what he wants.}
\\
\hline
C3&
Try to make the document a bit more friendly to read (e.g. by adding some sentences in bold (see the highlighted ones for this paragraph), making bigger
separation between paragraphs, ...).
\\
\cmidrule{2-2}
&We have implemented the two suggestions proposed.
\\
\hline
C4&
It is perhaps interesting to add a chapter with some basic definitions?
e.g. equivalent mutant, redundant mutant, mutation score, killed mutant, live mutant, weak mutation, ...
\\
\cmidrule{2-2}
&
We have added an index with the references to the page where terms are introduced. We decided to not add a glossary because it would lead to a duplication of definitions.

\TODO{Oscar, please add the 'backslash INDEX' command where appropriate in the source. Oscar: I added already several terms, I'll keep doing it during the next days.}

\\
\hline
C5&
Please, consider to add more examples. Similar to what is presented in the slides in the different progress meetings, where often the different concepts
are introduced with an example.
\\
\cmidrule{2-2}
&We have added more examples as suggested.
\\
\hline
C6&
Elaborate a bit more on the mutation score concept (e.g. what is achievable in reality e.g. can we target 100\%?, how this number can be correlated with
the quality of a test suite e.g. if a test suite presents 80\% score, can it be considered very good?
\\
\cmidrule{2-2}
&
\TODO{Please add something based on references. You should anyway, add a chapter on mutation score.}
\\

\hline
C7&
Please, elaborate on the concept of having a threshold to stop the process.
\\
\cmidrule{2-2}
&
\TODO{Please try to say something, based on references.}
\\

\hline
C8&
Please, comment a bit on the possibility to mutate the IR, executable, assembly code; and pros and cons.\\
\cmidrule{2-2}
&
\DONE{Please try to say something, based on references.}
We added a discussion on the pros and cons of mutating compiled code on Section \ref{sub:compileTime}.
\\
\hline
C9&
Is this really the only criteria to select the mutation operators?\\
\cmidrule{2-2}
&\DONE{Can we say anything?}
We added an explanation on how the syntactical changes introduced by mutation operators are related to a specific fault model.
\\
\hline
C10&
\begin{minipage}{8cm}
"the context of space software and embedded systems"
This is to be refined:\\
- embedded systems is clear\\
- space software term can be ambiguous... (ground desktop SW, ground embedded SW, on-board embedded SW, ...)\\
\end{minipage}
\\
\cmidrule{2-2}
&In the introduction, we have clarified how the term space software is used in this book (i.e., to refer to on-board embedded SW).
\\

\hline
C11&
\begin{minipage}{8cm}
a. There is a minimum set of operators. Is there anything for the Maximum?
b. There are many operators. Any preliminary idea of the ones we should try (or the strategy to follow) within this activity?
c. High-order operators: Is there any strategy to combine operators to define the high order ones?
d. Is there any logic followed for naming the different mutation operators? (e.g. OBBN). Are those terms consistently used by practitioners?
\end{minipage}
\\
\cmidrule{2-2}
&
\begin{minipage}{10cm}
We have addressed the comments above. Concerning \emph{C11.a} we can only provide references to work that rely on sampling of mutants, work on the identification of an upper bound for the number of operators is not available, to the best of our knowledge. Concerning \emph{C11.b} we introduced a table summarizing the mutation operators and their applicability to the FAQAS activity. Concerning \emph{C11.c} we made clear that each higher-order mutation operator introduce a strategy for create them, we also include an example. For \emph{C11.d} we included an explanation in \ref{sec:operators}.
\DONE{Concerning C11.b, At the end of section 1.3, can you add the summary table that we used in the slides where we provide a strategy to prioritize the operators? You should also add a paragraph that describes the table.}
\DONE{Can you add something, somewhere for C11.c?}
\end{minipage}
\\

\hline
C12&
\begin{minipage}{8cm}
a. There are many techniques presented for the different limitations. Do you have any idea already on the techniques that you would like to try within
FAQAS?\\
b. Is there any particular technique that you would like to perform some research on?
\end{minipage}
\\
\cmidrule{2-2}
&
\begin{minipage}{8cm}
(C12.a) In the context of the project, we will evaluate the following approaches:
\begin{itemize}
\item Selection of a subset of operators (e.g., deletion and sufficient)
\item Random sampling of mutants
\item Among the live mutants, identification of non-equivalent mutants based on code coverage
\item Among the live mutants, identification of non-equivalent mutants by means of symbolic execution or bounded model checking
\item Among the live mutants, identification of redundant mutants based on code coverage
\end{itemize}

(C12.b) The effectiveness of the selected solutions for complex, space software, will be reported in a research paper.
\end{minipage}
\\
\hline
C13&
\begin{minipage}{8cm}
a. It is not trivial to understand why Equivalent Mutants may lead mutation testing infeasible.\\
b. The fact of having a high number of equivalent / redundant mutants suggests that the quality of the source code is not that great; or that the used mutation
operators are not good. \\
c. Also, the observability of the system may not be good enough to see the effects of the mutation.\\
\end{minipage}
\\
\cmidrule{2-2}
&
\begin{minipage}{10cm}
Concerning .b and .c, it is not possible to provide a general rule relating the presence of equivalent mutants and the quality of code. For example, in while loops, we may observe the statement \texttt{var++} in a single line (see example isPalindrome). Such statement may lead to equivalent mutants generated by the operator that changes \texttt{var++} into \texttt{++var}. However, the code might be well written. Also, the operator, which may lead to equivalent mutants in this case, may be useful in others. A refinement of the scope in which the operator is applied might be possible (e.g., not when the increment is alone in a statement, nor when it within a for loop), however it might be hard to generalize.

\DONE{Concerning .b and .c, can you check the paper that states that the identification of eq mut is infeasible? It should help us to clarify, that, in general it is natural to have variables that cannot propagate the effect of mutations to the outputs.}
\end{minipage}
\\
\hline
C14&
Do you have general numbers? e.g. what was the percentage of killed mutants?
\\
\cmidrule{2-2}
&\DONE{can you check the numbers in the paper and add something?}
We added more details about the experimental results of \cite{schuler2013covering}.
\\
\hline
C15&
As for equivalent mutants, it is not trivial to understand the consequences of redundant mutants. It also suggest poor source code / mutation selection /
system visibility.
\\
\cmidrule{2-2}
&\TODO{Is there any research in this?}
\\
\hline
C16&
Within the report, there are several numbers like this 9\%, or the 45\% in equivalent mutants, etc. Do you plan to do some research within FAQAS to check
on those numbers?
\\
\cmidrule{2-2}
&\DONE{Check TODO in chapter}
\\

\hline
C17&
My first impression after reading this chapter is that Data Mutation would be quite applicable to our software systems (perhaps a bit more from the point of
view of verifying robustness of the system, rather than assessing an existing test suite).
Do you already have an idea on how to apply this within this activity and how deep to go in the approach?
\\
\cmidrule{2-2}
&\TODO{TODO for Fabrizio}
\\

\hline
C18&
Only Software faults? This can be true for data exchanged among SW components; but if the data comes from somewhere else, it can be any kind of
issue.
\\
\cmidrule{2-2}
&We have clarified in the paragraph.
\\
\hline
C19&
Could we say that Objective 1 is linked to Test Suite Assessment and Objective 2 is linked so system robustness testing? If so, it is better to state it.
\\
\cmidrule{2-2}
&This is not exactly the case, we have clarified in the text.
\\

\hline
C20&
As for code mutation, please elaborate a bit more on this \% operators applied (why it is important), and why we need a threshold to stop the process.
\\
\cmidrule{2-2}
&We have clarified in Section~\ref{sec:data:mutationscore}.
\\
\hline
C21&
The fact of needing modifications at runtime, is it not a hard constraint for Data Mutation?
Depending on where the mutation is done, we may need to modify the code itself, or the test harnesses. This, in my opinion, adds an extra complexity
compared to Code Mutation.
\\
\cmidrule{2-2}
&We agree, the changes for C28 apply also for this case.
\\

\hline
C22&
Could you please add the definition of a test oracle? We are not used to this naming e.g. is it the set of inputs and their corresponding outputs for a test
case?
\\
\cmidrule{2-2}
&Thank you. We have added a definition for test oracles in section 1.1. (the first time it appears).
\\
\hline
C23&
\begin{minipage}{10cm}
a. Is this lack impacting somehow the objectives of this activity, or would it be enough to try Data Mutation with the fault injection techniques already
existing?\\
b. Is the lack stated in this paragraph a potential research area for SnT?\\
c. Is it an area that you would like to explore a bit more?\\
\end{minipage}
\\
\cmidrule{2-2}
&
\begin{minipage}{10cm}
(C23.a) We believe that current approaches should be reusable. We refer especially to references~\cite{di2015evolutionary} and \cite{di2015generating}, which have been developed at SnT. To address scalability issues, however, such work may need to be adapted or reimplemented in a simpler fashion.\\
(C23.b and C23.c) Data-driven mutation is a research direction for SnT. FAQAS might be a starting point for a broader future project. We believe that data-driven mutation testing might be useful for a large set of cyber-physical systems, including space systems; indeed, they work by analysing an modelling a large quantity of data, their system test suites require the execution of simulators to generate streams of sensor data that follow specific environmental rules.  Data-driven mutation testing might be a solution to evaluate the quality of these test suites.
\end{minipage}
\\
\hline
C24&
As for code mutation, this is to be refined.
\\
\cmidrule{2-2}
&We believe that the change in the introduction to address C10 should clarify the misunderstanding.
\\

\hline
C25&
This column is missing in the table
\\
\cmidrule{2-2}
&Thank you. We have fixed the table.
\\

\hline
C26&
Is there any research needed to improve this table i.e. to come up with more mutation operators; or it is enough for this activity?
\\
&We believe this set of mutation operators to be sufficient for the project.
\\
\hline
C27&
Is there any convention on the operator names?
\\
\cmidrule{2-2}
&In this case, since operators are defined in different contexts, there is no convention in the literature.
\\

\hline
C28&
\begin{minipage}{10cm}
a. Is Run-Time Scalability a limitation in data mutation?\\
b. My impression is that in order to do data mutation, we must have a thorough knowledge on the SUT and its operating environment (e.g. hardware
around, the communication protocols, etc). Is this a limitation on the approach?\\
c. The integration of Data Mutation in a specific validation environment looks more challenging than for Code Mutation. Is that a limitation of the
approach?
\end{minipage}
\\
\cmidrule{2-2}
&Yes, we have clarified it at the end of Section~\ref{sec:data:test_suite_evaluation}
\\
\hline
C29&
Why is Data Modeling a limitation? Is it because of the need to define a data model and that is time consuming?
\\
\cmidrule{2-2}
&Yes, we have clarified it at the end of Section~\ref{sec:data:test_suite_evaluation}.
\\
\hline
C30&
This concept is difficult to understand. Maybe it would help by adding an example?
\\
\cmidrule{2-2}
&We have added an example of BNF grammar in Section~\cite{sec:dataModeling}.
\\

%Author: Pedro Barrios Subject: Sticky Note Date: 28/02/2020, 09:18:11
%Comment #4:
%It is perhaps interesting to add a chapter with some basic definitions?
%e.g. equivalent mutant, redundant mutant, mutation score, killed mutant, live mutant, weak mutation, ...
%Author: Pedro Barrios Subject: Sticky Note Date: 28/02/2020, 09:18:18
%Comment #5:
%Please, consider to add more examples.

\bottomrule                                                             
\end{longtable}
\normalsize

\clearpage