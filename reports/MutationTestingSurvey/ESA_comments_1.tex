% !TEX root = MutationTestingSurvey.tex

\section*{Response to ESA comments provided on 03.03.2020}
\label{sec:ESA:comments:1}


\setlength\LTleft{0pt}
\setlength\LTright{0pt}
\tiny 
%@{\extracolsep{\fill}}
\begin{longtable}{|p{1.2cm}|p{12cm}|@{}}
\caption{\normalsize .}
\label{table:codeoperators} \\

C1&
\begin{minipage}{8cm}
Please, consider the following suggestions:\\
- Add a table to keep the change record or the document (the table could have e.g. Issue number, Date, Change description, and Author)\\
- Add a Table of Contents\\
- Add an introduction chapter, with scope of the document, applicable and reference documents, and acronyms and abbreviations\\
\end{minipage}
\\
&
We have added table of contents and revisions.
We have added acronyms to the introduction.

\\

C2
&When generating the pdf file, make sure to have the index of the document on the left, so that the document is easier to navigate.
\TODO{Cannot understand}
\\

C3&
Try to make the document a bit more friendly to read (e.g. by adding some sentences in bold (see the highlighted ones for this paragraph), making bigger
separation between paragraphs, ...).
\\
We have implemented the two suggestions proposed.
\\


C4&
It is perhaps interesting to add a chapter with some basic definitions?
e.g. equivalent mutant, redundant mutant, mutation score, killed mutant, live mutant, weak mutation, ...
\\
We have added an index with the references to the page where terms are introduced. We decided to not add a glossary because it would lead to a duplication of definitions.

\TODO{OScar, please add 'index' keywords where appropriate in the source.}

\\
C5&
Please, consider to add more examples. Similar to what is presented in the slides in the different progress meetings, where often the different concepts
are introduced with an example.
\\
We have added more examples as suggested.
\\

C6&
Elaborate a bit more on the mutation score concept (e.g. what is achievable in reality e.g. can we target 100\%?, how this number can be correlated with
the quality of a test suite e.g. if a test suite presents 80\% score, can it be considered very good?
\\
&
\TODO{Please add something based on references. You should anyway, add a chapter on mutation score.}
\\


C7&
Please, elaborate on the concept of having a threshold to stop the process.
\\
&
\TODO{Please try to say something, based on references.}
\\

C8:&
Please, comment a bit on the possibility to mutate the IR, executable, assembly code; and pros and cons.\\
&
\TODO{Please try to say something, based on references.}
\\

C9&
Is this really the only criteria to select the mutation operators?
\\
\TODO{Can we say anything?}
\\

C10:
\begin{minipage}{8cm}
"the context of space software and embedded systems"
This is to be refined:\\
- embedded systems is clear\\
- space software term can be ambiguous... (ground desktop SW, ground embedded SW, on-board embedded SW, ...)\\
\end{minipage}
\\
In the introduction, we have clarified how the term space software is used in this book (i.e., to refer to on-board embedded SW).
\\

%Author: Pedro Barrios Subject: Sticky Note Date: 28/02/2020, 09:18:11
%Comment #4:
%It is perhaps interesting to add a chapter with some basic definitions?
%e.g. equivalent mutant, redundant mutant, mutation score, killed mutant, live mutant, weak mutation, ...
%Author: Pedro Barrios Subject: Sticky Note Date: 28/02/2020, 09:18:18
%Comment #5:
%Please, consider to add more examples.

\bottomrule                                                             
\end{longtable}
\normalsize

\clearpage