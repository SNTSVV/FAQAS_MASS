% XeLaTeX can use any Mac OS X font. See the setromanfont command below.
% Input to XeLaTeX is full Unicode, so Unicode characters can be typed directly into the source.

% The next lines tell TeXShop to typeset with xelatex, and to open and save the source with Unicode encoding.

%!TEX TS-program = xelatex
%!TEX encoding = UTF-8 Unicode

\documentclass[11pt,twoside]{book}

\usepackage{geometry}                % See geometry.pdf to learn the layout options. There are lots.
\geometry{a4paper}                   % ... or a4paper or a5paper or ... 
%\geometry{landscape}                % Activate for for rotated page geometry
%\usepackage[parfill]{parskip}    % Activate to begin paragraphs with an empty line rather than an indent
\usepackage{amssymb}
\usepackage{todonotes}
\usepackage{booktabs}
\usepackage{longtable}
\usepackage{url}


\usepackage{imakeidx}
\usepackage[utf8]{inputenc}
\usepackage[T1]{fontenc}
\makeindex

\usepackage[font=normalsize]{caption}  


\usepackage{xcolor}
\usepackage{listings}

\definecolor{mGreen}{rgb}{0,0.6,0}
\definecolor{mGray}{rgb}{0.5,0.5,0.5}
\definecolor{mPurple}{rgb}{0.58,0,0.82}
\definecolor{backgroundColour}{rgb}{0.95,0.95,0.92}

\lstdefinestyle{CStyle}{
    backgroundcolor=\color{backgroundColour},   
    commentstyle=\color{mGreen},
    keywordstyle=\color{magenta},
    numberstyle=\tiny\color{mGray},
    stringstyle=\color{mPurple},
    basicstyle=\scriptsize\ttfamily,
    breakatwhitespace=false,         
    breaklines=true,                 
    captionpos=b,                    
    keepspaces=true,                 
    numbers=left,                    
    numbersep=5pt,                  
    showspaces=false,                
    showstringspaces=false,
    showtabs=false,                  
    tabsize=2,
    language=C
}



% Will Robertson's fontspec.sty can be used to simplify font choices.
% To experiment, open /Applications/Font Book to examine the fonts provided on Mac OS X,
% and change "Hoefler Text" to any of these choices.

\usepackage{fontspec,xltxtra,xunicode}
\defaultfontfeatures{Mapping=tex-text}
\setromanfont[Mapping=tex-text]{Optima}
\setsansfont[Scale=MatchLowercase,Mapping=tex-text]{Optima}
\setmonofont[Scale=MatchLowercase]{Andale Mono}

%%%% Graphics
\usepackage{graphicx}
\usepackage{tikz}
\definecolor{unilublue}{RGB}{55,149,218}
\definecolor{sntred}{RGB}{219,46,27}
\definecolor{sntpurple}{RGB}{86,30,130}
%\definecolor{sntblue}{RGB}{50,130,207}
\definecolor{sntblue}{RGB}{55,149,218}

\pgfdeclareimage[width=30mm]{logo-snt}{logos/logo-snt}
\pgfdeclareimage[width=30mm]{logo-uni-lu}{logos/logo-uni-lu}

%%%% Fancy
\usepackage{fancyhdr}

%%%% Increase page length
\addtolength{\textheight}{1in}

%%%% Sections
\usepackage{sectsty}
\allsectionsfont{\sffamily}

%%%% Title
\newcommand{\titleone}{\textsf{Fault-based Automated Quality Assurance of Test Suites:}}
\newcommand{\titletwo}{\textsf{A Survey of Mutation Testing and Test Suite Augmentation Approaches}}
\newcommand{\titlethree}{\textsf{for Satellite and Embedded Systems}}

\newcommand{\todoinline}[1]{\todo[color=orange,inline]{ \textbf{Note}: #1 }}


\begin{document}

\pagestyle{fancy}
\renewcommand{\sectionmark}[1]{\markright{\textit{#1}}}

\renewcommand{\headrulewidth}{2pt}% 2pt header rule
\renewcommand{\headrule}{\hbox to\headwidth{%
  \color{sntblue}\leaders\hrule height \headrulewidth\hfill}}

\fancyhf{}

%\lhead{\fancyplain{}{\setlength{\unitlength}{1mm}
%\begin{picture}(0,0)
%\put(0,-4){\includegraphics[width=50pt]{logos/logo-snt}}
%\end{picture}}} 

\lhead{\fancyplain{}{\textit{}}}

\rhead{\fancyplain{}{\rightmark }}

\fancyfoot[C]{%
\begin{tikzpicture}[remember picture,overlay]
\path [fill=sntred]    ([xshift=88pt,yshift=20pt]current page.south west) rectangle
                       ([xshift=229pt,yshift=50pt] current page.south west);
\path [fill=sntpurple] ([xshift=229pt,yshift=20pt] current page.south west) rectangle
                       ([xshift=370pt,yshift=50pt] current page.south west);
\path [fill=sntblue]   ([xshift=370pt,yshift=20pt] current page.south west) rectangle
                       ([xshift=510pt,yshift=50pt] current page.south west);
\end{tikzpicture}
}

\fancyfoot[RO]{
\begin{tikzpicture}[remember picture,overlay]
\node [circle, ultra thick, fill=white, draw=sntblue] at ([xshift=530pt,yshift=35pt] current page.south west) {\thepage};
\end{tikzpicture}}

\fancyfoot[LE]{
\begin{tikzpicture}[remember picture,overlay]
\node [circle, ultra thick, fill=white, draw=sntblue] at ([xshift=65pt,yshift=35pt] current page.south west) {\thepage};
\end{tikzpicture}}


\newcommand{\CHANGED}[1]{\textcolor{blue}{#1}}

\thispagestyle{empty}

\begin{tikzpicture}[remember picture,overlay]
\path [fill=sntred]    ([xshift=30pt,yshift=20pt]current page.south west) rectangle
                       ([xshift=170pt,yshift=50pt] current page.south west);
\path [fill=sntpurple] ([xshift=170pt,yshift=20pt] current page.south west) rectangle
                       ([xshift=310pt,yshift=50pt] current page.south west);
\path [fill=sntblue]   ([xshift=310pt,yshift=20pt] current page.south west) rectangle
                       ([xshift=460pt,yshift=51pt] current page.south west);
\path [fill=unilublue] ([xshift=30pt,yshift= 50pt] current page.south west) --
                       ([xshift=460pt,yshift= 50pt] current page.south west)
                       [rounded corners=20pt] --
                       ([xshift=460pt,yshift=630pt] current page.south west)
                       [sharp corners] --
                       ([xshift=30pt,yshift=630pt] current page.south west);
\node [fill=white,rounded corners=0pt,inner xsep=6pt,inner ysep=3pt]
      at ([xshift=415pt,yshift=120pt] current page.south west)
      {\pgfuseimage{logo-uni-lu}};

\node [fill=white,rounded corners=2pt,inner xsep=6pt,inner ysep=3pt]
      at ([xshift=410pt,yshift=670pt] current page.south west)
      {\pgfuseimage{logo-snt}};

%\node [circle, fill=white, draw=sntblue] at ([xshift=550pt,yshift=35pt] current page.south west) {a};

\node[draw=none,fill=none,right] at (-0.5, -3){\color{white}\LARGE\bf\titleone};
\node[draw=none,fill=none,right] at (-0.5, -4){\color{white}\LARGE\bf\titletwo};
\node[draw=none,fill=none,right] at (-0.5, -5){\color{white}\LARGE\bf\titlethree};

\node[draw=none,fill=none,right] at (-0.5, -6){\color{white}\Large\textsf{F. Pastore, O. Cornejo, E. Viganò}};
\node[draw=none,fill=none,right] at (-0.5, -7){\color{white}\Large\textsf{Interdisciplinary Centre for Security, Reliability and Trust}};
\node[draw=none,fill=none,right] at (-0.5, -8){\color{white}\Large\textsf{University of Luxembourg}};
\node[draw=none,fill=none,right] at (9.8, -10){\color{white}\textsf{\DOCUMENTID}};
\node[draw=none,fill=none,right] at (9.8, -11){\color{white}\Large\textsf{Issue 1, Rev. 2}};
\node[draw=none,fill=none,right] at (9.8, -12){\color{white}\Large\textsf{\today}};

\node[draw=none,fill=none,right] at (-0.5, -16.8){\color{white}\tiny\textsf{EUROPEAN SPACE AGENCY. CONTRACT REPORT.}}; 
\node[draw=none,fill=none,right] at (-0.5, -17.1){\color{white}\tiny\textsf{The work described in this report was done under ESA contract.}}; 
\node[draw=none,fill=none,right] at (-0.5, -17.4){\color{white}\tiny\textsf{Responsibility for the contents resides in the author or organisation that prepared it.}};
\node[draw=none,fill=none,right] at (-0.5, -17.7){\color{white}\tiny\textsf{The copyright in this document is vested in the University of Luxembourg.}};
\node[draw=none,fill=none,right] at (-0.5, -18.0){\color{white}\tiny\textsf{This document may only be reproduced in whole or in part, or stored in a retrieval system, or transmitted}};
\node[draw=none,fill=none,right] at (-0.5, -18.3){\color{white}\tiny\textsf{in any form,~\  or by any means electronic,~\  mechanical, photocopying or otherwise,~\ either with the prior }}; 
\node[draw=none,fill=none,right] at (-0.5, -18.6){\color{white}\tiny\textsf{permission of ~\ the University of Luxembourg ~\ or in accordance with the terms of ESTEC ~\ Contract No.}};
\node[draw=none,fill=none,right] at (-0.5, -18.9){\color{white}\tiny\textsf{4000128969/19/NL/AS.}};


\end{tikzpicture}

\newpage

\printindex

% !TEX root = MAIN.tex

\chapter{Introduction}

The purpose of this  Software Validation Specification is to describe the testing, analysis, inspection and
review of design specifications, and is used to document the software validation specification concerning the requirements baseline.

\chapter{Applicable and reference documents}

\begin{itemize}
\item{D1 - Mutation testing survey}
\item{D2 - Study of mutation testing applicability to space software}
\item{D4 - Validation of the toolset}
\item{SSS - Software System Specification}
\item{SUM - Software User Manual}
\item{SUTP - Software Unit Test Plan}
\end{itemize}

\chapter{Terms, definitions, and abbreviated terms}

\begin{itemize}
\item{FAQAS}: activity ITT-1-9873-ESA
\item{FAQAS-framework}: software system to be released at the end of WP4 of FAQAS
\item{D2}: Deliverable D2 of FAQAS, \emph{Study of mutation testing applicability to space software}
\item{D4}: Deliverable D4 of FAQAS, \emph{Validation of the toolset}
\item{SSS}: Software System Specification.
\item{KLEE}: Third-party test generation tool, details are provided in D2.
\item{MLFS}: Mathematical Library for Flight Software.
\item{SUT}: Software under test, i.e, the software that should be mutated employing mutation testing.
\item{SUM}: Software User Manual.
\item{SUTP}: Software Unit Test Plan
\item{WP}: Work package.
\end{itemize}

\chapter{Software Overview}

This documents concern the components of the FAQAS framework:
\begin{itemize}
  \item MASS
  \item \DAMA
  \item SEMuS
%  \item DAMTE
\end{itemize}

For detailed information on their structure and usage see D2 and SUM.

\chapter{Software validation specification task identification}

\STARTCHANGEDFINAL

\section{Task and criteria}
\label{sec:taskCrit}

The validation tasks are meant to ensure that the requirement expressed in the SSS are met.
They are accomplished by following validation approaches:
\begin{itemize}
  \item veryfing the requirements for the most critical and complex software components througt \EMPH{unit testing} as detailed in the SUTP.
  \emph{unit testing} is composed of three tasks:
  \begin{itemize}
    \item performing \emph{\DAMA-TD-DDMutation-1}
    \item performing \emph{\DAMA-TD-DDMutation-2}
    \item performing \emph{\MASS-TD-SRCMutation-1}
    \item performing \emph{SEMuS-TD-TGMutation-1}
  \end{itemize}
  \item verifying the requirements for the rest of the components through \EMPH{application to the case studies} as detailed in the D4 and the SUM.
  Application to the case studies is composed of the following tasks, detailed in \ref{sec:case_studies}
  \begin{itemize}
    \item \emph{configuring \MASS and running Launcher.sh - \MASS}
    \item \emph{configuring \MASS and running PrepareSUT.sh - \MASS}
    \item \emph{configuring \MASS and running GenerateMutants.sh - \MASS}
    \item \emph{configuring \MASS and running CompileOptimizedMutants.sh - \MASS}
    \item \emph{configuring \MASS and running OptimizedPostProcessing.sh - \MASS}
    \item \emph{configuring \MASS and running GeneratePTS.sh - \MASS}
    \item \emph{configuring generate\_template\_config.json and running generate\_direct.py - SEMuS}
    \item \emph{manually editing of the generated test templates - SEMuS} 
    \item \emph{instrumenting the source code - \DAMA}
    \item \emph{configuring and running} a given component, for example \texttt{\DAMA\_compile.sh}.
  \end{itemize}
  \item verifying general requirements through \EMPH{manual inspection}. For this approach no further specific task can be identified.
\end{itemize}



\section{Application to the case studies}
\label{sec:case_studies}

In the following sections the validation task that fall under the \EMPH{application to the case studies} category are described, along with their pass-fail criteria.


\subsection{Configuring \MASS and running Launcher.sh - \MASS}
\label{sec:configuring:mass}

The objective of this task is to validate that \MASS can properly prepare the SUT, collect code coverage information, and correctly apply the eight steps of the methodology for assessing the quality of the SUT test suite.

% !TEX root =  ../MAIN.tex

\begin{table}[tb]
\footnotesize
\centering
\caption{\MASS minimal set of parameters to be configured.}
\label{table:to_configure}
\begin{tabular}{llp{7.5cm}}
\hline
\textbf{Script Name}  & \textbf{Parameter} &  \textbf{Description} \\
\hline
& BUILD\_SYSTEM &  Specifies the building system type.\\
& PROJ &  Path of the SUT root  directory.\\
& PROJ\_SRC &  Path of the SUT source directory.\\
& PROJ\_TST &  Path of the SUT test  directory.\\
& PROJ\_COV &  Path of the directory with SUT coverage information.\\
& PROJ\_BUILD &  Path of the directory where the compiled binary is stored.\\
& ORIGINAL\_MAKEFILE &  Path to the original build script.\\
& COMPILATION\_CMD &  Compilation command of the SUT.\\
& TCE\_COMPILE\_CMD &  Compilation command for TCE analysis.\\
& PRIORITIZED &  Specifies whether \MASS should be executed with a prioritized test suite.\\
& SAMPLING &  Specifies the mutant sampling technique.\\
\hline
PrepareSUT.sh & None &  Commands shall be provided manually.\\
\hline
mutation\_additional\_functions.sh & run\_tst\_case  &  Implementation of the Bash function run\_tst\_case that executes the test case passed as a parameter.\\
\hline
\end{tabular}
\end{table}

The files and variables to be configured for \MASS are specified in Table~\ref{table:to_configure}. As detailed in the SUM document, these files (i.e., mass\_conf.sh, PrepareSUT.sh, mutation\_additional\_functions.sh) support \MASS to correctly process the paths of the SUT, SUT compilation commands, SUT test suite execution commands, and enable the mutation analysis process.

After configuring the three files, the following command shall be executed for launching the mutation analysis process. 

\begin{lstlisting}[language=bash]
  $ ./Launcher.sh
\end{lstlisting}

Note that this command (1) shall be executed from \MASS workspace, (2) shall automatically perform the following eight steps:

\begin{enumerate}
  \item PrepareSUT
  \item GenerateMutants
  \item CompileOptimizedMutants
  \item OptimizedPostProcessing
  \item GeneratePTS
  \item ExecuteMutants
  \item IdentifyEquivalents
  \item MutationScore
\end{enumerate}

A typical \MASS output is presented in Listing~\ref{mass:output}.

\begin{lstlisting}[language=bash, label=mass:output, caption=\MASS output.]
##### MASS Output #####
## Total mutants generated: 28071
## Total mutants filtered by TCE: 6918
## Sampling type: fsci
## Total mutants analyzed: 461
## Total killed mutants: 369
## Total live mutants: 92
## Total likely equivalent mutants: 53
## MASS mutation score (%): 90.44
## List A of useful undetected mutants: /opt/MLFS/RESULTS/useful_list_a
## List B of useful undetected mutants: /opt/MLFS/RESULTS/useful_list_b
## Number of statements covered: 1973
## Statement coverage (%): 100
## Minimum lines covered per source file: 2
## Maximum lines covered per source file: 138
\end{lstlisting}

\subsubsection{Pass-Fail criteria}

The following criteria must be respected in order to declare this task a pass:

\begin{itemize}
  \item The eight steps of the methodology shall be performed. In turn, the following shall be produced for each steps:
    \begin{itemize}
      \item PrepareSUT: this step shall produce code coverage files, and list of test cases to be executed at \texttt{COV\_FILES} folder.
      \item GenerateMutants: this step shall produce mutant files at \texttt{src-mutants} folder.
      \item CompileOptimizedMutants: inside \texttt{COMPILED} there should be one folder for each optimization level containing (1) list of SHA512 hashes for every compiled mutant, (2) list of non-compiled mutants.
      \item OptimizedPostProcessing: inside \texttt{COMPILED} there should be (1) list of equivalent mutants, (2) list of redundant mutants, (3) list of unique mutants.
      \item GeneratePTS: this step shall produce reduced and prioritized test suites for the SUT at folder \texttt{PRIORITIZED}.
      \item ExecuteMutants: inside \texttt{MUTATION} folder there should be: (1) code coverage for each mutant, (2) traces of killed and live mutants, (3) list of executed mutants.
      \item IdentifyEquivalents: inside folder \texttt{DETECTION} there should be: (1) list of equivalent mutants, (2) mutation traces without equivalent mutants.
      \item MutationScore: inside folder \texttt{RESULTS} there should be: (1) final MASS report, (2) list of killed and live mutants, (3) code coverage report, (4) list of useful mutants. 
    \end{itemize}
  \item A final report shall be reported at the end of the mutation analysis process.
  \item The MASS log shall report no error.
\end{itemize}

\subsection{Configuring \MASS and running PrepareSUT.sh - \MASS}

This validation task consists of configuring \MASS, as specified in the Section~\ref{sec:configuring:mass}, and then executing the PrepareSUT.sh command of \MASS, this command prepare the SUT and collect information about the SUT test suite. Note that this script shall be prepared by the SUT engineer. 
The objective of this task is to validate that \MASS can properly prepare the SUT and collect code coverage information about the SUT test suite, which is necessary for enabling \MASS optimizations.

\begin{lstlisting}[language=bash]
  $ ./PrepareSUT.sh
\end{lstlisting}

The command shall create a \texttt{COV\_FILES} folder inside the \MASS workspace; such folder will contain GCOV files for each test case, and a list of test cases to be executed.

\subsubsection{Pass-Fail criteria}

\begin{itemize}
  \item The execution of PrepareSUT shall be successful, that is, the script shall terminate with a 0 error code.
  \item The required output shall be observed inside \texttt{COV\_FILES} folder.
\end{itemize}

\subsection{Configuring \MASS and running GenerateMutants.sh - \MASS}

This validation task consists of configuring \MASS, as specified in the Section~\ref{sec:configuring:mass}, and then executing the GenerateMutants.sh command of \MASS, this command generates the code coverage matrices and launches the generation of mutants. A prerequisite to this task is to have executed the PrepareSUT command.
The objective of this task is to validate that \MASS can properly process code coverage information and generate code-driven mutants of the SUT.

\begin{lstlisting}[language=bash]
  $ ./GenerateMutants.sh
\end{lstlisting}

The command shall create a \texttt{src-mutants} folder inside the \MASS workspace; such folder will contain one folder for each source under test, and each of this folder shall contain a set of mutants for the source code.

A typical mutant source file will be named using (1) the name of the source file, (2) the mutated line, (3) the instance of the mutation (e.g., the AOR operator can replace a $+$ operator into four instances (a) $-$, (b) $*$, (c) $/$, and (d) $\%$) and the position in the source file, (4) the operator being used, (5) the affected function. An example of mutant would be \url{source.mut.126.1_6_19.ICR.function_1.c}

\subsubsection{Pass-Fail criteria}

\begin{itemize}
  \item The execution of GenerateMutants shall be successful, that is, the script shall terminate with a 0 error code.
  \item The required output shall be observed inside \texttt{src-mutants} folder.
\end{itemize}

\subsection{Configuring \MASS and running CompileOptimizedMutants.sh - \MASS}

This validation task consists of configuring \MASS, as specified in the Section~\ref{sec:configuring:mass}, and then executing the CompileOptimizedMutants.sh command of \MASS, this command compiles the mutants with multiple optimisation levels and generate intermediate files to be processed then by OptimizedPostProcessing. A prerequisite to this task is to have executed the GenerateMutants command.
The objective of this task is to validate that \MASS can properly compile the mutants and generate a list of SHA512 hashes for each mutant, in order to properly identify equivalent and redundant mutants during the OptimizedPostProcessing step.

\begin{lstlisting}[language=bash]
  $ ./CompileOptimizedMutants.sh
\end{lstlisting}

The command shall create a \texttt{COMPILED} folder inside the \MASS workspace; such folder will contain one folder for each optimization level containing (1) list of SHA512 hashes for every compiled mutant, (2) list of non-compiled mutants.

\subsubsection{Pass-Fail criteria}

\begin{itemize}
  \item The execution of CompileOptimizedMutants shall be successful, that is, the script shall terminate with a 0 error code.
  \item The required output shall be observed inside \texttt{COMPILED} folder.
\end{itemize}

\subsection{Configuring \MASS and running OptimizedPostProcessing.sh - \MASS}

This validation task consists of configuring \MASS, as specified in the Section~\ref{sec:configuring:mass}, and then executing the CompileOptimizedMutants.sh command of \MASS, this command iterates over the SHA512 hashes generated by the CompileOptimizedMutants command, and produce a list of unique mutants. A prerequisite to this task is to have executed the CompileOptimizedMutants command.
The objective of this task is to validate that \MASS can properly identify equivalent and redundant mutants to be then analyzed during ExecuteMutants.

\begin{lstlisting}[language=bash]
  $ ./OptimizedPostProcessing.sh
\end{lstlisting}

The command shall create (1) a list of equivalent mutants, (2) a list of redundant mutants, and (3) a list of unique mutants inside the \texttt{COMPILED} folder within the \MASS workspace.

\subsubsection{Pass-Fail criteria}

\begin{itemize}
  \item The execution of OptimizedPostProcessing shall be successful, that is, the script shall terminate with a 0 error code.
  \item The required output shall be observed inside \texttt{COMPILED} folder.
\end{itemize}

\subsection{Configuring \MASS and running GeneratePTS.sh - \MASS}

This validation task consists of configuring \MASS, as specified in the Section~\ref{sec:configuring:mass}, and then executing the CompileOptimizedMutants.sh command of \MASS, this command generates a reduced and prioritized test suite for the SUT. A prerequisite to this task is to have executed the OptimizedPostProcessing command.
The objective of this task is to validate that \MASS can properly generate a reduced and prioritized set of test cases to be used during the ExecuteMutants step.

\begin{lstlisting}[language=bash]
  $ ./GeneratePTS.sh
\end{lstlisting}

The command shall create two files, (1) a reduced, and (2) prioritized test suite for the SUT, at folder \texttt{PRIORITIZED}.

\subsubsection{Pass-Fail criteria}

\begin{itemize}
  \item The execution of GeneratePTS shall be successful, that is, the script shall terminate with a 0 error code.
  \item The required output shall be observed inside \texttt{PRIORITIZED} folder.
\end{itemize}



\subsection{Configuring generate\_template\_config.json and running generate\_direct.py - SEMuS}

% !TEX root =  ../MAIN.tex

\begin{table}[t]
\tiny
\centering
\caption{SEMuS parameters to be configured.}
\label{table:to_conf_semus}
\begin{tabular}{lp{8.5cm}}
\hline
\textbf{Parameter}  &  \textbf{Description} \\
\hline
FAQAS\_SEMU\_CASE\_STUDY\_TOPDIR &  Root folder of the case study \\
FAQAS\_SEMU\_CASE\_STUDY\_WORKSPACE &  SEMuS workspace for the case study \\
FAQAS\_SEMU\_OUTPUT\_TOPDIR & SEMuS output folder, to be placed inside the workspace \\
FAQAS\_SEMU\_GENERATED\_MUTANTS\_TOPDIR & Root folder for storing the generated mutants \\
FAQAS\_SEMU\_REPO\_ROOTDIR &  Root folder of the case study source code\\
FAQAS\_SEMU\_ORIGINAL\_SOURCE\_FILE & Path of the source file under analysis \\
FAQAS\_SEMU\_COMPILE\_COMMAND\_SPECIFIED\_SOURCE\_FILE & Name of the source file under analysis \\
FAQAS\_SEMU\_GENERATED\_MUTANTS\_DIR & Folder for storing the generated mutants for the specified source file\\
FAQAS\_SEMU\_BUILD\_CODE\_FUNC\_STR & Bash function for building the source file under analysis, to be specified in string format\\
FAQAS\_SEMU\_BUILD\_LLVM\_BC & Bash function for building the source file to LLVM bitcode \\
FAQAS\_SEMU\_META\_MU\_TOPDIR &  Root folder for the meta mutant \\
FAQAS\_SEMU\_GENERATED\_META\_MU\_SRC\_FILE &  Path of the source file (i.e., C file) of the meta mutant \\
FAQAS\_SEMU\_GENERATED\_META\_MU\_BC\_FILE &  Path of the source file (i.e., LLVM bitcode file) of the meta mutant \\
FAQAS\_SEMU\_GENERATED\_META\_MU\_MAKE\_SYM\_TOP\_DIR &  Folder for storing intermediate files for the generated inputs \\
FAQAS\_SEMU\_GENERATED\_TESTS\_TOPDIR &  Folder for storing the generated inputs \\
FAQAS\_SEMU\_TEST\_GEN\_TIMEOUT & Timeout in seconds for the test generation process \\
FAQAS\_SEMU\_HEURISTICS\_CONFIG & Configuration array for SEMu heuristics \\
FAQAS\_SEMU\_TEST\_GEN\_MAX\_MEMORY & Maximum test generation memory in MB \\
FAQAS\_SEMU\_STOP\_TG\_ON\_MEMORY\_LIMIT & Parameter to stop test generation when the memory limit is reached \\
FAQAS\_SEMU\_TG\_MAX\_MEMORY\_INHIBIT & Parameter to stop forking states when the memory limit is reached \\
\hline
\end{tabular}
\end{table}



% !TEX root =  ../MAIN.tex

\begin{table}[t]
\tiny
\centering
\caption{Test template generator parameters to be configured.}
\label{table:ttg_semus}
\begin{tabular}{lp{8.5cm}}
\hline
\textbf{Parameter}  &  \textbf{Description} \\
\hline
TYPES\_TO\_INTCONVERT &  Specify how to convert a type to int. \\
TYPES\_TO\_PRINTCODE &  Specify how to print a type.  \\
OUT\_ARGS\_NAMES & Specify the names of function arguments that are used as function output. \\
IN\_OUT\_ARGS\_NAMES & Specify the names of function arguments that are used both as function input and output \\
TYPE\_TO\_INITIALIZATIONCODE &  Specify the initialization statement of a type.\\
VOID\_ARG\_SUBSTITUTE\_TYPE & Specify the underlying type for a void pointer. \\
TYPE\_TO\_SYMBOLIC\_FIELDS\_ACCESS & Specify, for pointer parameters, the number of elements it points to. \\
\hline
\end{tabular}
\end{table}




The objective of this task is to validate that \SEMUS can properly generate test templates that can be processed by the underlying test generation tool (i.e., KLEE).

The files and variables to be configured for \SEMUS are specified in Tables~\ref{table:to_conf_semus} and~\ref{table:ttg_semus}. As detailed in the SUM document, these files (i.e., generate\_template\_config.json and running generate\_direct.py) support \SEMUS to correctly process the paths of the SUT, SUT compilation commands, the configuration of \SEMUS itself, and how to print the values of the functions under test, so \SEMUS can determine if a mutant has been killed.

After configuring the two files, the following command shall be executed for launching the test template generation. 

\begin{lstlisting}[language=bash]
 $ case_studies/$SUT/util_codes/generate_direct.py ../WORKSPACE/DOWNLOADED/casestudy/test.c direct \
                    " -I../WORKSPACE/DOWNLOADED/casestudy/" -c generate_template_config.json
\end{lstlisting}

Note that this command shall generate inside the directory \texttt{case\_studies/\$SUT/util\_codes} one folder for each source under analysis, and inside of these folders, one template for each function under test.

\subsubsection{Pass-Fail criteria}

\begin{itemize}
  \item The test templates shall be generated.
  \item The generate\_direct.py shall terminate with a 0 error code.
\end{itemize}

\subsection{Instrumenting the source code - \DAMA}
\label{subsec:instrumenting}

The variables for running the \DAMA pipeline on ESAIL must be set in the \emph{\DAMA\_configure.sh} file, as reported in Listing~\ref{lst:configure_esail}. The significance of these variables is described in the SUM.

Then the commands represented in Listing~\ref{lst:instrument_esail_cmds} must be run to generate the mutation probes.

\begin{lstlisting}[language=bash, label={lst:instrument_esail_cmds}]
bash DAMA_probe_generation.sh
\end{lstlisting}

The generated probes must be inserted in the target function, as reported in the SUM (Chapter 13).
The Software Under Test (LibParam or ESAIL) shall be compiled with the macro \texttt{-DMUTATIONOPT=-1} enabled.

\subsubsection{Pass-Fail criteria}

The following criteria must be respected in order to declare this task a pass:
\begin{itemize}
  \item The compilation shall be successful.
  \item The compilation log shall report no error.
\end{itemize}

\subsection{Configuring and running \emph{\DAMA\_compile.sh}}

A prerequisite for this task is having successfully performed \EMPH{Instrumenting the source code - \DAMA}, as described in Section~\ref{subsec:instrumenting}.

\subsubsection{ESAIL}

The commands for compiling the SVF, reported in Listing~\ref{lst:compile_esail} must be included in the \emph{\DAMA\_compile.sh} script in the appropriate section as described in the SUM (Chapter 13).


  \begin{lstlisting}[language=bash, label={lst:compile_esail}]

  compilation_folder="/home/svf/Svf"

  pushd $compilation_folder

  make install-debug

      if [ $? -eq 0 ]; then
          echo $x " compilation OK"
      else
          echo $x " compilation FAILED"
      fi

  popd

  \end{lstlisting}

Then the commands represented in Listing~\ref{lst:compile_esail_cmds} must be run.

  \begin{lstlisting}[language=bash, label={lst:compile_esail_cmds}]

  bash \DAMA_compile.sh "0" "TRUE"

  \end{lstlisting}

\subsubsection{LibParam}

The commands for configuring the LibParam test suite, reported in Listing~\ref{lst:compile_param} must be included in the \emph{\DAMA\_compile.sh} script in the appropriate section as described in the SUM (Chapter 13


\begin{lstlisting}[language=bash, label={lst:compile_param}]

TEST_FOLDER="/home/csp/libparam/tst"

pushd $TEST_FOLDER

for f in *; do
    if [ -d "$f" ] && [ "$f" != "include" ]; then

        pushd $f
        echo "cleaning..."
        ./waf clean

        echo "configuring..."
        if [ $singleton == "TRUE" ]; then
        ./waf configure --mutation-opt $mutant_id --singleton $singleton
        else
        ./waf configure --mutation-opt $mutant_id
        fi

        if [ $? -eq 0 ]; then
            echo $x " configuration OK"
        else
            echo $x " configuration FAILED"
        fi
        popd
    fi
done

popd

\end{lstlisting}

Then the commands represented in Listing~\ref{lst:compile_param_cmds} must be run.

  \begin{lstlisting}[language=bash, label={lst:compile_param_cmds}]

  bash \DAMA_compile.sh "0" "TRUE"

  \end{lstlisting}

\subsubsection{Pass-Fail criteria}

The following criteria must be respected in order to declare this task a pass:
\begin{itemize}
  \item The compilation log shall report no error.
  \item The compilation log shall report the phrase \texttt{compilation OK}.
\end{itemize}


\subsection{Configuring and running \emph{\DAMA\_run\_test.sh}}

A prerequisite for this task is having successfully performed \EMPH{Instrumenting the source code - \DAMA}, as described in Section~\ref{subsec:instrumenting}.

\subsubsection{ESAIL}

The commands for running ESAIL's test suite, reported in Listing~\ref{lst:run_esail} must be included in the \emph{\DAMA\_run\_test.sh} script in the appropriate section as described in the SUM (Chapter 13).


  \begin{lstlisting}[language=bash, label={lst:run_esail}]

  ESAIL=/home/svf/Obsw/Test/lib/esail.sh
  PARSE_RESULTS=/home/svf/Obsw/Test/lib/parse_results.sh
  echo -n "${mutant_id};COMPILED;${tst};" >> $results_file

  timeout $TIMEOUT $ESAIL --obsw /home/svf/Obsw/Source/_binaries/OBSW.exe --fast -n -c --source /home/svf/Obsw/Source --version 04010000 -t $tst &
  ESAIL_PID=$!

  wait $ESAIL_PID
  EXEC_RET_CODE=$?

  mutant_end_time=$(($(date +%s%N)/1000000))
  mutant_elapsed="$(($mutant_end_time-$mutant_start_time))"

  \end{lstlisting}

Then the commands represented in Listing~\ref{lst:run_esail_cmds} must be run.

  \begin{lstlisting}[language=bash, label={lst:run_esail_cmds}]

  bash \DAMA_compile.sh "-1" "TRUE" && bash \DAMA_run_tests.sh "-1" "./test_list.csv" "./"

  \end{lstlisting}

The file \texttt{test\_list.csv} must contain the list of all test cases, as exposed in Listings~\ref{lst:test_esail}.

  \begin{lstlisting}[label={lst:test_esail}]

  731;46287
  6343;6161251
  6000;535195
  612;50098
  791;27125
  767;67364
  743;71586
  683;66213
  1841;102287
  3045;60261
  2561;164595
  719;74033
  5702;280961
  2819;203036
  695;62058
  707;44825
  1222;94890
  2435;134397
  755;53084
  779;156284
  543;71224
  6616;35423
  524;32621
  561;27344
  497;31588
  943;25435
  3394;63879
  5176;30804
  5656;34471
  5218;38539
  4361;63238
  2733;127361
  6161;562902
  6262;100839
  3077;83077
  6117;61088
  2848;421200
  4497;156531
  3590;190508
  2985;265498
  5917;127059
  3004;101740
  3917;200336
  2890;522174
  3034;507824
  5374;276736
  2867;441252
  2753;187527
  2928;311267
  3023;79797
  3610;1283052
  2947;995408
  2909;290593
  2966;133739
  3651;293401
  3853;132605
  6138;760249
  3621;46376
  6600;51881
  5190;196980
  4294;46522
  4270;35237
  6583;144017
  6275;143658
  5393;47192
  2604;45503
  2411;83456
  3950;39675
  146;26984
  140;23420
  3969;39893
  113;28140
  74;26170
  154;22556
  126;24476
  1465;68582
  1377;633878
  1364;47647
  2327;254152
  2339;83202
  2395;284150
  6630;228215
  2254;36753
  6362;102271
  2277;53371
  5570;32536
  160;27916
  659;98164
  6315;25372
  1242;645102
  647;27085
  225;27339
  6401;58932
  1807;56712
  4931;46262
  6038;63676
  1505;2975291
  671;84703
  990;34764
  2007;38421
  1702;86591
  5791;90984
  3996;28060
  3358;113929
  5606;31639
  1545;28645
  1572;135209
  1742;26743
  3340;136863
  197;98851
  2375;50526
  4815;74190
  1293;27614
  3313;66150
  5645;26901
  3376;1068552
  3291;584423
  2093;189396
  953;63629
  6330;25552
  6736;2446525

  \end{lstlisting}

\subsubsection{LibParam}

The commands for running the LibParam test suite, reported in Listing~\ref{lst:run_param} must be included in the \emph{\DAMA\_run\_test.sh} script in the appropriate section as described in the SUM (Chapter 13).


  \begin{lstlisting}[language=bash, label={lst:run_param}]

  tmp_log="$results_dir"/tmp_log

    TEST_FOLDER="/home/csp/libparam/tst"

    pushd $TEST_FOLDER

    pushd $tst

    touch $tmp_log

    timeout $TIMEOUT ./waf --mutation-opt=$mutant_id  --singleton=TRUE 2>&1 | tee $tmp_log

    EXEC_RET_CODE=$?

    mutant_end_time=$(($(date +%s%N)/1000000))
    mutant_elapsed="$(($mutant_end_time-$mutant_start_time))"

    if [ $EXEC_RET_CODE -ne 124 ]; then
        if grep "successfully" $tmp_log
        then
            EXEC_RET_CODE=0
            echo "PASSED"
        else
            EXEC_RET_CODE=1
            echo "FAILED"
        fi
    fi
    popd
    rm $tmp_log
    popd

  \end{lstlisting}

Then the commands represented in Listing~\ref{lst:run_param_cmds} must be run.

% mutant_id=$1
% tests_list=$2
% \DAMA_FOLDER=$3

  \begin{lstlisting}[language=bash, label={lst:run_param_cmds}]

  bash \DAMA_compile.sh "-1" "TRUE" && bash \DAMA_run_tests.sh "-1" "./test_list.csv" "./"

  \end{lstlisting}

The file \texttt{test\_list.csv} must contain the list of all test cases, as exposed in Listings~\ref{lst:test_param}.

  \begin{lstlisting}[label={lst:test_param}]

  bindings,4427.200000
  example,4145.400000
  file_store,2954.000000
  i2c,2793.400000
  log,2881.200000
  param3,3007.800000
  param4,3273.600000
  rparam3,5099.400000
  rparam4,8003.200000
  serialize,2656.400000
  spi,2825.800000
  store,2774.400000
  store_load,3182.000000
  vmem_store_checksum_first,3005.800000
  vmem_store_checksum_last,2963.400000

  \end{lstlisting}

  \subsubsection{Pass-Fail criteria}

  The following criteria must be respected in order to declare this task a pass:

  \begin{itemize}
    \item All test cases in Listing~\ref{lst:test_param} (for LibParam) and Listing~\ref{lst:test_esail} (for ESAIL) shall be executed.
    \item All test cases listed in Listing~\ref{lst:test_param} (for LibParam) and Listing~\ref{lst:test_esail} (for ESAIL) shall pass.
  \end{itemize}

\subsection{Configuring and running \emph{\DAMA\_obtain\_coverage.sh}}

A prerequisite for this task is having successfully performed \EMPH{Instrumenting the source code - \DAMA}, as described in Section~\ref{subsec:instrumenting}.

\subsubsection{ESAIL}

The variables for running the \DAMA-pipeline on ESAIL must be set in the \emph{\DAMA\_configure.sh} file, as reported in Listing~\ref{lst:configure_esail}. The significance of these variables is described in the SUM (Chapter 13).

  \begin{lstlisting}[language=bash, label={lst:configure_esail}]

  # the location of the csv with all the test identifiers and the execution time
  tests_list=$\DAMA_FOLDER/tests.csv

  # the location of the csv containing the definitions of the mutation operators
  fault_model=$\DAMA_FOLDER/fault_model.csv

  # the datatype of the elements of the target buffer
  buffer_type="unsigned long long int"

  # padding: can be used to skip the first n bit of a buffer, normally set to 0
  padding=0

  # singleton: can set to true to load the fault model into a singleton   variable, normally set to "TRUE", can also  be set to "FALSE"
  singleton="TRUE"

  \end{lstlisting}

Then the commands represented in Listing~\ref{lst:coverage_esail_cmds} must be run.

  \begin{lstlisting}[language=bash, label={lst:coverage_esail_cmds}]

  bash \DAMA_obtain_coverage.sh

  \end{lstlisting}

\subsubsection{LibParam}

The variables for running the \DAMA-pipeline on ESAIL must be set in the \emph{\DAMA\_configure.sh} file, as reported in Listing~\ref{lst:configure_param}. The significance of these variables is described in the SUM (Chapter 13).

  \begin{lstlisting}[language=bash, label={lst:configure_param}]

  # the location of the csv with all the test identifiers and the execution time
  tests_list=$\DAMA_FOLDER/tests_param.csv

  # the location of the csv containing the definitions of the mutation operators
  fault_model=$\DAMA_FOLDER/LIBP-FM.csv

  # the datatype of the elements of the target buffer
  buffer_type="unsigned long long int"

  # padding: can be used to skip the first n bit of a buffer, normally set to 0
  padding=0

  # singleton: can set to true to load the fault model into a singleton variable, normally set to "TRUE", can also  be set to "FALSE"
  singleton="TRUE"

  \end{lstlisting}

Then the commands represented in Listing~\ref{lst:coverage_param_cmds} must be run.

  \begin{lstlisting}[language=bash, label={lst:coverage_param_cmds}]

  bash pipeline_scripts/\DAMA_obtain_coverage.sh ./

  \end{lstlisting}

\subsubsection{Pass-Fail criteria}

These criteria must be respected to consider this task a pass:
\begin{itemize}
  \item The compilation lof shall show no error.
  \item The compilation log shall report the phrase \texttt{compilation OK}.
  \item No test cases shall fail.
  \item A \texttt{testlist} folder shall be generated. Inside this folder there shall be a file called \texttt{test\_<mutant>} for every mutant.
  \item Every file shall contain a subset of the tests listed in Listing~\ref{lst:test_param} (for LibParam) or in Listing~\ref{lst:test_esail} (for ESAIL).
\end{itemize}




\subsection{Configuring and running \emph{\DAMA\_mutants\_launcher.sh}}

A prerequisite for this task is having successfully performed \EMPH{Instrumenting the source code - \DAMA}, as described in Section~\ref{subsec:instrumenting}.

\subsubsection{ESAIL}

The variables for running the \DAMA-pipeline on ESAIL must be set in the \emph{\DAMA\_configure.sh} file, as reported in Listing~\ref{lst:configure_esail}. The significance of these variables is described in the SUM (Chapter 13).

Then the commands represented in Listing~\ref{lst:launcher_esail_cmds} must be run.

  \begin{lstlisting}[language=bash, label={lst:launcher_esail_cmds}]

  bash \DAMA_mutants_launcher.sh ./

  \end{lstlisting}

  The compilation logs shall report the phrase \texttt{compilation OK}.
  A \texttt{testlist} folder shall be generated. Inside this folder there shall be a file called \texttt{test\_<mutationID>} for every mutant containing a subset of the tests listed in Listing~\ref{lst:test_esail}.
  A \texttt{results} folder shall be generated and it shall contain the files described in the SUM (Section 9.2).

\subsubsection{LibParam}

The variables for running the \DAMA-pipeline on ESAIL must be set in the \emph{\DAMA\_configure.sh} file, as reported in Listing~\ref{lst:configure_param}. The significance of these variables is described in the SUM (Chapter 13).

Then the commands represented in Listing~\ref{lst:launcher_param_cmds} must be run.

  \begin{lstlisting}[language=bash, label={lst:launcher_param_cmds}]

  bash \DAMA_mutants_launcher.sh ./

  \end{lstlisting}

\subsubsection{Pass-Fail criteria}

The following criteria shall be respected in order to consider this task a pass:

\begin{itemize}
  \item The compilation logs shall report the phrase \texttt{configuration OK}.
  \item A \texttt{testlist} folder shall be generated.
  \item Inside this folder there shall be a file called \texttt{test\_<mutationID>} for every mutant containing a subset of the tests listed in Listing~\ref{lst:test_param} for LibParam or in in Listing~\ref{lst:test_esail} for ESAIL.
  \item A \texttt{results} folder shall be generated and it shall contain the files described in the SUM (Section 9.2).
\end{itemize}

\ENDCHANGEDFINAL

\section{Features to be tested}
The scope of the SVS includes all baseline requirements expressed in the SSS.

% \section{Features not to be tested}
% The SVS w.r.t. TS or RB shall describe all the features and significant
% combinations not to be tested.

\section{Test pass and fail criteria}
The pass-fail criteria for the tasks belonging to the \emph{unit testing} validation approach are detailed in the SUTP.

Regarding the tasks belonging to the \emph{application to the case studies} validation approach, the pass criteria are described in Section~\ref{sec:case_studies}.


% \section{Items that cannot be validated by test}
% a. The SVS w.r.t. TS or RB shall list the tasks and items under tests that
% cannot be validated by a test.
% b. Each of them shall be properly justified
% c. For each of them, an analysis, inspection, or review of design shall be
% proposed.
%
% % \section{Manually and automatically generated code}
% % a. The SVS shall address separately the activities to be performed for
% % manually and automatically generated code, although they have the
% % same objective (ECSS‐Q‐ST‐80 clause 6.2.8.2 and 6.2.8.7).
%
% \chapter{Software validation testing specification design}
%
% \section{General}
% a. The SVS w.r.t. TS or RB shall define software
% validation testing specification design, giving the design grouping
% criteria such as function, component, or equipment management.
% b. For each identified test design, the SVS w.r.t. TS or RB shall provide the
% information given in <6.2>.
%
% \section{Organization of each identified test design}
%
% NOTE The SVS w.r.t. TS or RB defines each validation

\clearpage


% !TEX root = MutationTestingSurvey.tex

\chapter{Code Mutation Testing}
\label{chapter:codemutation}

% !TEX root = MutationTestingSurvey.tex

\section{Mutation Testing Process}
\label{sec:process}

	\begin{figure}
	\centering
		\includegraphics[width=\textwidth]{images/process}
		\caption{Mutation Testing Process.}
		\label{fig:code:process}
	\end{figure}

Figure \ref{fig:code:process} shows the reference code-driven mutation testing process that will be considered in this book. The process depicted in Figure \ref{fig:code:process} has been inspired by the mutation testing process described in related work \cite{offutt2001mutation,papadakis2019mutation}. The process is based on two main sub-processes, \emph{Test Suite Evaluation} and \emph{Test Suite Augmentation}. Sections~\ref{sub:test_suite_evaluation} and  provide an overview of each of them, Sections~\ref{sub:test_suite_evaluation}~to~\ref{sub:test_suite_augmentation} describe in details the building blocks and the issues to overcome in order to efficiently apply code oriented data mutation.

\subsection{Test Suite Evaluation} % (fold)
\label{sub:test_suite_evaluation}

The Test Suite Evaluation process concerns the automatic generation of modified versions (i.e., the mutants) of the software under test (SUT) and the evaluation of the quality of the SUT test suite. It consists of three activities: \emph{create mutants}, \emph{execute mutants}, and \emph{analyze results}. 
These activities are typically automated by toolsets that often include strategies to address scalability issues. Mutation testing activities and related optimizations are depicted in Figure~\ref{fig:code:process} and described in the following paragraphs.

The Test Suite Evaluation process starts with engineers providing the SUT and a selection of the mutation operators to consider. The set of mutation of available mutation operators typically depends on the toolset implementing the mutation testing process. 
A typical set of mutation operators implemented by most of the existing toolsets consists of the relational (ROR), logical (LCR), arithmetic (AOR), absolute (ABS) and unary insertion (UOI) operators \cite{rothermel1996experimental}. 
%Several research paper target the definition of mutation operators; relevant for this ESA activity are works on the definition of mutants for floating point operators \cite{dan2012semantic}, operators synthesized by processing the revision history of C projects \cite{brown2017care}, mutation operators targeting memory operations \cite{wu2017memory}, deletion operators \cite{delamaro2014designing}, operators targeting components integration \cite{grechanik2016mutation}. Finally, recent attention has been put towards the development of higher-order mutation operators \cite{harman2010manifesto,ghiduk2017higher}. First order mutation seeds faults generated by a single syntactic change to the original program. Higher order mutation combines first order mutants to simulate more complex faults, motivated by a desire to capture subtle faults \cite{jia2009higher}. 
Section~\ref{sec:operators} provides an overview of the mutation operators defined in the literature that can be applied in the context of space and embedded systems.

In the mutation testing process, the activity \emph{create mutants} concerns the application of the mutation operators to the source code of the SUT; it leads to the generation of modified versions of the SUT (i.e., the mutants) that should be compiled and then executed against the test suite to evaluate the test suite quality. 

Unfortunately, the mutation process leads to a high number of mutants to be generated, which leads to scalability issues due to the time required to compile and execute the different versions generated. Recent surveys provide an overview of existing optimization techniques \cite{ferrari2018systematic}; the most relevant optimization approaches target the compilation processes, the execution of mutants, and mutants selection. 

Two main mutant selection approaches have been defined: the selection of mutation operators and the random selection of mutants \cite{zhang2010operator}. The first approach consists of the empirical identification of a subset of mutation operators that is sufficient to predict the mutation score \cite{siami2008sufficient,barbosa2001toward}. The second approach consists of randomly selecting a certain percentage of mutants from the generated ones \cite{wong1995reducing}, possibly with a uniform distribution of the different mutation operators \cite{zhang2010operator}. Empirical results with academic case studies \cite{zhang2010operator} show that the first approach is not superior to random selection when selecting the same number of mutants. Other work \cite{zhang2013operator} show that the combination of operator-based selection and random sampling leads to better results since it leads to high mutation score (above 98\%) while reducing the average mutation testing time to 6.54\%. The use of higher-order mutants is another solution to reduce the overall number of mutants. Other optimizations are framework specific, for example Mull limits mutations to reachable code \cite{hariri2018srciror}. Section~\ref{sec:opt:selection} provides an overview of mutant selection approaches.

To reduce the time spent in the compilation of the generated mutants, mutant schemata \cite{untch1993mutation} are adopted to encode all the mutants in a single file and parametrize the mutant execution so that mutants are compiled in a single pass and selected at runtime. Section~\ref{sub:compileTime} provides an overview of compile-time optimization approaches.


Another optimization concerns the identification of equivalent and redundant mutants. Equivalent mutants are mutants that behave as the original program, while redundant mutants are mutants that lead to the same test failures. 
To detect if a program and one of its mutants are
equivalent is undecidable~\cite{Budd:1982}; however, heuristics to partially address the problems have been defined in the literature.
Trivial compiler optimization might be adopted to detect both equivalent and redundant mutants; it relies on the idea that source code that leads to the same program behaviour often belongs to the same optimized compiled code \cite{papadakis2015trivial}. Other approaches concern the adoption of symbolic execution \cite{papadakis2012mutation,kurtz2015static} and the runtime monitoring of the SUT (e.g., mutants that lead to the same execution paths are likely equivalent \cite{schuler2013covering}). Finally, Shin et al. proposed the \emph{distinguishing mutation adequacy criterion}, which aims to ensure that the test suite includes enough different test inputs so that every mutant is distinguished by each other, if feasible~\cite{shin2017theoretical}. 
Solutions to address the problem of identifying equivalent and redundant mutants are detailed in Sections~\ref{sec:opt:equivalent} and~\ref{sec:opt:redundant}, respectively.



The execution of mutants implies the execution of the test suite of SUT against all the generated mutants. 
Optimizations for the execution of mutants concern the scalability of the mutants execution process and 
the identification at run-time of equivalent and redundant mutants.
A well-known optimization concerning the scalability of the mutants execution process is the split-stream optimization, which consists of generating a modified version of the SUT that creates multiple processes (one for each mutant) only when the mutated code is reached \cite{tokumoto2016muvm}. With split-stream, the code shared among multiple mutants is executed only once thus saving time and resources. Other execution optimizations consist of minimizing the number of processes being created by sharing one single process among mutants that bring the system into the same state \cite{wang2017faster}.
Section~\ref{sec:opt:execution} provides an overview of techniques to address run-time scalability issues.



The analysis of test results concerns the identification of mutants that lead to the failure of at least one test case of the SUT test suite; these mutants are said to be \emph{killed}. Mutants that do not lead to the failure of any test case are said to be \emph{live mutants}. The identification of killed and live mutants enable the definition of a mutation adequacy criterion as follow, \emph{a test suite is mutation-adequate if all mutants are killed by at least one test of the test suite}. 
Also, the percentage of killed mutants is used to quantitatively measure the quality of a test suite. This measure is referred to as \emph{mutation score}.
Because of equivalent and redundant mutants, mutation-adequacy is difficult to achieve while the mutation score might not be representative of test suites quality~\cite{papadakis2016threats}. Section~\ref{sub:mutationscore} provides an overview of solutions addressing the problems related to the computation of the mutation score.



The capability of a test case to kill a mutant often depends on the observability of the system state. 
To overcome the limitations due to observability, different strategies for distinguishing program executions (i.e, to determine if the execution of two test cases led to different results) have been defined. These strategies are known as strong, weak, firm, and flexible mutation testing.
Strong mutation coverage indicate that the computation of the mutation score is based on the percentage of mutants identified by test failures, i.e., based on difference between the expected and the observed output of the system.  Weak mutation coverage consists of verifying if the state of the system has been altered, with respect to the original code, after the execution of the mutated statement. Firm mutation coverage consists of verifying if the change in the state of the system propagates after the mutated code, e.g., at function boundaries. Flexible weak mutation consists of checking if the mutated code leads to object corruption \cite{mateo2012validating}. The main difference among these four coverage strategies is that only strong mutation coverage enables engineers to assess the quality of test cases in their entirety, i.e., by evaluating both the capability of triggering an erroneous behavior and the capability of reporting the erroneous behaviour thanks to complete test oracles. The other strategies only evaluate the capability of the test suites of triggering the erroneous behavior. 

% subsection test_suite_evaluation (end)

\subsection{Test Suite Augmentation} % (fold)
\label{sub:test_suite_augmentation}

The Test Suite Augmentation process concerns the definition of test cases that kill live mutants. Although the test suite can be augmented manually by engineers after inspecting the source code of the live mutants, we focus on the possibility of automating such process. 
The Test Suite Augmentation process consists of four activities Identify Test Inputs, Generate Test Oracles, Execute the SUT, Fix the SUT. The first two activities lead to the definition of new complete test cases, the execution of the SUT enable engineers to determine if the newly defined test cases spot faults not identified  by the original test suite, which is one of the benefit of mutation testing. Finally, fixing the SUT is performed manually by the engineer in case of test failures.


%To this end, mutants could be ranked according to their importance in order to ensure that, for a given test budget, the most relevant mutants are considered first. MuRanker \cite{namin2015muranker}, for example, ranks mutants according to their predicted difficulty and complexity in being detected. 

Concerning the automated generation of test cases for mutated C programs, existing work investigated the adoption of the KLEE symbolic execution engine \cite{holling2016nequivack} and the use of bounded model checking \cite{riener2011test}. Other work combines dynamic symbolic execution (DSE) with search-based software testing (SBST) to generate test inputs that lead to strong mutations \cite{harman2011strong}. 

Concerning the generation of test oracles, a state-of-the-art approach consists of the generation of assertions that verify the value of variables that enable the killing of mutants \cite{fraser2011mutation}. Such approach has been adopted in the context of Java programs but not for C or embedded systems.


If test failures are not observed, engineers evaluate the quality of the newly generated test suite by observing the mutation score and inspecting live mutants. 

Section~\ref{sec:testGeneration} provides an overview of approaches for the automated generation of test cases.

% subsection test_suite_augmentation (end)


% !TEX root =  ../Main.tex

\newcommand{\op}{\mathit{op}}
\newcommand{\ArithmeticSet}{ \texttt{+}, \texttt{-}, \texttt{*}, \texttt{/}, \texttt{\%} }
\newcommand{\LogicalSet}{ \texttt{&&}, \texttt{||} }
\newcommand{\RelationalSet}{ \texttt{>}, \texttt{>=}, \texttt{<}, \texttt{<=}, \texttt{==}, \texttt{!=} }
\newcommand{\BitWiseSet}{ \texttt{\&}, \texttt{|}, \land }
\newcommand{\ShiftSet}{ \texttt{>>}, \texttt{<<} }


\begin{table}[h]
\caption{Implemented set of mutation operators.}
\label{table:operators} 
\centering
\scriptsize
\begin{tabular}{|@{}p{4mm}@{}|@{}p{2cm}@{\hspace{1pt}}|@{}p{11.1cm}@{}|}
\hline
&\textbf{Operator} & \textbf{Description$^{*}$} \\
\hline
\multirow{7}{*}{\rotatebox{90}{\emph{Sufficient Set}}}&ABS               & $\{(v, -v)\}$	\\
\cline{2-3}
&AOR               & $\{(\op_1, op_2) \,|\, \op_1, \op_2 \in \{ \ArithmeticSet \} \land \op_1 \neq \op_2 \} $       \\
&    			  & $\{(\op_1, \op_2) \,|\, \op_1, \op_2 \in \{\texttt{+=}, \texttt{-=}, \texttt{*=}, \texttt{/=}, \texttt{\%} \texttt{=}\} \land \op_1 \neq \op_2 \} $       \\
\cline{2-3}
&ICR               & $\{i, x) \,|\, x \in \{1, -1, 0, i + 1, i - 1, -i\}\}$           \\
\cline{2-3}
&LCR               & $\{(\op_1, \op_2) \,|\, \op_1, \op_2 \in \{ \texttt{\&\&}, || \} \land \op_1 \neq \op_2 \}$            \\
&				  & $\{(\op_1, \op_2) \,|\, \op_1, \op_2 \in \{ \texttt{\&=}, \texttt{|=}, \texttt{\&=}\} \land \op_1 \neq \op_2 \}$            \\
&				  & $\{(\op_1, \op_2) \,|\, \op_1, \op_2 \in \{ \texttt{\&}, \texttt{|}, \texttt{\&\&}\} \land \op_1 \neq \op_2 \}$            \\
\cline{2-3}
&ROR               & $\{(\op_1, \op_2) \,|\, \op_1, \op_2 \in \{ \RelationalSet \}\}$            \\
&				  & $\{ (e, !(e)) \,|\, e \in \{\texttt{if(e)}, \texttt{while(e)}\} \}$ \\
\cline{2-3}
&SDL               & $\{(s, \texttt{remove}(s))\}$            \\
\cline{2-3}
&UOI               & $\{ (v, \texttt{--}v), (v, v\texttt{--}), (v, \texttt{++}v), (v, v\texttt{++}) \}$            \\   
\hline
\hline
\multirow{5}{*}{\rotatebox{90}{\emph{OODL}}}&AOD               & $\{((t_1\,op\,t_2), t_1), ((t_1\,op\,t_2), t_2) \,|\, op \in \{ \ArithmeticSet \} $       \\ 
\cline{2-3}
&LOD               & $\{((t_1\,op\,t_2), t_1), ((t_1\,op\,t_2), t_2) \,|\, op \in \{  \} \}$       \\ 
\cline{2-3}
&ROD               & $\{((t_1\,op\,t_2), t_1), ((t_1\,op\,t_2), t_2) \,|\, op \in \{ \RelationalSet \} \}$       \\ 
\cline{2-3}
&BOD               & $\{((t_1\,op\,t_2), t_1), ((t_1\,op\,t_2), t_2) \,|\, op \in \{ \BitWiseSet \} \}$       \\ 
\cline{2-3}
&SOD               & $\{((t_1\,op\,t_2), t_1), ((t_1\,op\,t_2), t_2) \,|\, op \in \{ \ShiftSet \} \}$       \\ 
%\hline
%COR               & $\{(\op_1, \op_2) \,|\, \op_1, \op_2 \in \{ \texttt{\&\&}, \texttt{||}, \land \} \land \op_1 \neq \op_2 \}$            \\
\hline
\hline
\multirow{3}{*}{\rotatebox{90}{\emph{Other}}}&LVR			& $\{(l_1, l_2) \,|\, (l_1, l_2) \in \{(0,-1), (l_1,-l_1), (l_1, 0), (\mathit{true}, \mathit{false}), (\mathit{false}, \mathit{true})\}\}$\\
&&\\
&&\\
\hline
\end{tabular}

$^{*}$Each pair in parenthesis shows how a program element is modified by the mutation operator. Th eleft element of the pair is replaced with the right element. We follow standard syntax~\cite{kintis2018effective}. Program elements are literals ($l$), integer literals ($i$), boolean expressions ($e$), operators ($\op$), statements ($s$), variables ($v$), and terms ( $t_i$, which might be either variables or literals).
\end{table}

% !TEX root = MutationTestingSurvey.tex

\section{Limitations of Code-driven Mutation Testing}
\label{sec:limitations}

% !TEX root = MutationTestingSurvey.tex

\subsection{Run-time Scalability}
\label{sec:opt:execution}

In the mutation testing process, running the mutants is considered to be the most expensive phase, consider having for a program under test $n$ mutants, and a test suite consisting of $m$ tests, this means that we have to perform at maximum $n \times m$ executions.
For example, consider the ESAIL SVF test suite that requires approximately 10 hours to run for the original program, if we produce 100,000 mutants, it means that we would require 1,000,000 hours or 114 years to cover all mutant executions.
To cope with the high cost of mutation testing research has make some advances in this area. For example, mutants that are not exercised by any test case should not be considered, since there is no chance of being killed, this approach has been implement in the Proteum mutation tool \cite{delamaro1996proteum}. In the same way, mutants that have been already killed by a test case should not be executed again with other test cases, since this would only increase the overhead of the technique. Similarly, Just et al. \cite{just2012using} and Zhang et al. \cite{zhang2013faster} proposed prioritizing test cases, that is, ordering test cases in a way that mutants are killed as early as possible, e.g., Just et al. \cite{just2012using} proposed to execute first the test cases with a shorter runtime. Papadakis and Malevris \cite{papadakis2011automatically} gave a step further in this research line, and proposed that a way to prioritise the different mutants is by assessing in advance which of the mutants are actually modifying the variables that check the killing condition of the mutant, i.e., the final assertion of a test case, and prioritise them over mutants that do not modify variables related to the killing condition.

To decrease the run-time cost of mutation testing, two additional types of killing conditions can be considered: weak and firm mutation. Unlike strong mutation, i.e., the mutant and the original program must show an observable difference (e.g., the result of an assertion), in weak mutation \cite{ammann2016introduction} instead, the mutant is considered killed if the program state is modified immediately after the mutation, the advantage of weak mutation is that a complete execution to determine the killing condition is not required, such as strong mutation, the main drawback is that sacrifices test effectiveness with test effort. Similarly to weak mutation, in firm mutation \cite{ammann2016introduction}, the assessment of the killing condition is placed in between weak and strong mutation. In a nutshell, weak mutation is considered less effective than firm mutation, and firm mutation is considered less effective than strong mutation.

To further avoid unnecessary test runtime, several mutation tools include time threshold for running executions, in general if a test case employs three times during mutation execution with respect the original program, the execution is terminated.

King and Offutt \cite{king1991fortran} presented split-stream execution, a technique that takes advantage of the fact that a program and its mutants share most of the execution parts, so instead of having one executable per mutant it could be possible to generate a modified version of the SUT that creates multiple processes (one for each mutant) only when the mutated code is reached \cite{tokumoto2016muvm}, in this way saving time and resources. 

% \todoinline{How easy it is to determine if we have weak mutation in C programs? I think that one of the solutions is linked to split stream. If weak mutation is not achieved, i.e., if the program state is not infected, there is no point to execute the test case till the end. We can simply report it as not killed and save time.}

An orthogonal solution is to reduce the test cases conforming a test suite to decrease the run-time cost of the mutation process. For instance, it is possible to remove redundant test cases, e.g., test cases that do not alter mutation score in case of test case deletion. Shi et al. \cite{shi2014balancing} assessed the effects of reducing the size of test suites, and confirmed empirically on an experiment guided on 18 projects with a total of 261,235 tests, that is actually possible to maintain the mutation score and reduce test suite size without loss of fault detection.


% !TEX root = MutationTestingSurvey.tex

\subsection{Compile-time Scalability}
\label{sub:compileTime}

\endinput





% !TEX root = MutationTestingSurvey.tex

\subsection{Equivalent Mutants}
\label{sec:opt:equivalent}

\endinput

Mutants that are semantically equivalent to the original program despite being syntactically different. The process of identifying equivalent mutant is a undecidable problem \cite{madeyski2013overcoming}.

\subsubsection{Overcoming Equivalent Mutants} % (fold)
\label{sub:overcoming_equivalent_mutants}

The only way to overcome equivalent mutants is to define heuristics for removing them.

\textbf{Detecting equivalent mutants technique (harder to implement)}

\begin{itemize}
	\item \textbf{Trivial Compiler Optimisation} \cite{papadakis2015trivial, kintis2017detecting,papadakis2019mutation}: code optimisations transforms mutants to the optimised version. Hence, equivalent mutants are transformed to the same optimised version. TCE identifies at least 30\% of equivalent mutants.
	Baldwin and Sayward \cite{baldwin1979heuristics}, proposed six types of compiler optimization rules, the rules were studied by Offutt and Craft \cite{offutt1994using} and they discover that 10\% of all mutants were equivalent for 15 subject programs.
	
	\item \textbf{Constraint satisfaction problem}: the problem is formulated as a constraint satisfaction problem by analyzing the path condition of a mutant. The mutant is defined equivalent iff the input constraint is unsatisfiable. The technique detected 47\% of equivalent mutants among 11 subject programs \cite{offutt1996detecting,offutt1997automatically}.

	Holling et al. \cite{holling2016nequivack} presented an approach for identifying non-equivalent mutants and improving the confidence of the mutation score. Using static analysis and symbolic execution (KLEE \cite{cadar2008klee}) they define a six-steps procedure to determine which mutants are \textit{non-equivalent} or \textit{unknown}. \textit{Non-equivalent} mutants are identified if they find a counter-example input for which the outpus of a pair of functions (the original function and the mutant one) is different. If no counter-example is find, then is classified as \textit{unknown}.

	\item Program slicing \cite{voas1997software, hierons1999using, harman2001relationship}: generation of a sliced program that denotes the answer to an equivalent mutant. (READ)
	\item Using semantic differences in terms of a running profile to detect non-equivalent mutants \cite{ellims2007csaw} (READ)
	\item Margrave's change-impact analysis \cite{martin2007fault} (READ)
	\item Using Lesar model-checker for eliminating equivalent mutants \cite{du2008towards} (READ)
\end{itemize}

\textbf{Avoiding equivalent mutant generation techniques}

\begin{itemize}
	\item \textbf{Selective mutation} \cite{mresa1999efficiency}:
	Randomly selecting 10\%, 20\%, 30\%, 40\%, 50\% and 60\% of the mutants results in a fault loss of approximately 26\%, 16\%, 13\%, 10\%, 7\% and 6\% respectively \cite{papadakis2010empirical}.

	Rothermel et al. \cite{rothermel1996experimental} and then Andrews et al. \cite{andrews2005mutation} proposed a small set of operators that is a sufficiently accurate approximation of the results obtained by using all possible operators (e.g., Replace numerical constant, negate jump condition, replace arithmetic operator, omit method calls).

	Namin et al. \cite{siami2008sufficient} proposed a statistical analysis procedure for identifying a subset of operators that could predict mutation score (reduce mutants in a 93\%), the developed tool aims C programs.

	Delamaro et al. \cite{delamaro2014designing} designed new deletion operators, and found that they form a cost-effective alternative to other operators, i.e., they produce less equivalent mutants.

	\item \textbf{Avoiding equivalent mutation generation using program dependence analyisis} \cite{harman2001relationship}: use of program dependence analysis for careful selection of the variables and statements to be mutated and where these variables need to be inspected, in order to avoid the equivalent mutant problem. The idea is to avoid mutants that fail to propagate \textit{corrupted data} to the inspection set at the probe point which will be equivalent.

	They use a method called JR-dependence, which allow them to relate variable and nodes pairs, rather than simple to relate nodes. This is good for mutation testing, because we want to know the set of variables which can be used to kill a mutant and which set of variables store values which cannot be used to kill a mutant.

	\item Co-evolutionary approach \cite{adamopoulos2004overcome}: define a fitness function that sets a poor fitness value to an equivalent mutans. Through this function equivalent mutants are removed during the co-evolution process and only mutants that are hard to kill and tests cases that are good at detecting mutants are selected.

	\item Using equivalency conditions to eliminate equivalent mutants for object oriented mutation operators \cite{offutt2006class} (READ)

	\item Using a fault hierarchy to improve the efficiency of the DNF logic mutation testing \cite{kaminski2009using} (READ)

	\item \textbf{Distinguishing the equivalent mutants by semantic exception hierarchy} (Java - Javalanche) \cite{grun2009impact}. They discovered that mutation that alter the dynamic control flow are less likely to be equivalent; the higher the imapct, less chance to be equivalent. They focuses on developing a mutation tool based on sufficient mutation operators, mutant schemata, coverage based data.

	They measure changes in program behavior between the mutant and the original version (i.e., control flow). They compute the code coverage of a program, the program records the statement coverage for each test case and every mutation, that is, the number of times a statement is executed.
	By comparing the coverage of the original execution with the coverage of the mutated execution, they determine the coverage difference (number of classes that have different coverage). 

	\item \textbf{High order mutation testing} \cite{jia2009higher,kintis2010evaluating,offutt1992investigations,papadakis2010empirical}:
	
	Mutants that are composed by combining two or more mutants at the same time. 

	Offutt demonstrated that \textit{the set of test data developed for FOMs actually killed a higher percentage of mutants when applied to SOMs} \cite{offutt1992investigations}.

	Jia and Harman identified six different types of HOMs \cite{jia2009higher} and presented a categorization of HOMs. They introduced the concept of subsuming and strongly subsuming HOMs.

	Polo et al. \cite{polo2009decreasing} studied three strategies to combine FOMs and generate mutants and found that they can achieve significant cost reductions without losing any effectivenes (they reduced the number of mutants in a approximately 50\%, without much decrease in the quality of the test suite).

	Papadakis and Malevris \cite{papadakis2010empirical} worked on a approach for the C programming language that leads to the reduction of approximately 80-90\% of the generated equivalent mutants and a fault detecetion ability loss from 11-15\%. 

	Kintis et al. \cite{kintis2010evaluating} developed a solution for the Java language, they state that SOMs achieve higher collateral coverage for strong mutation as compared with third or higher order mutants. With their approach they obtained a mutant reduction of between 65-87\% and a loss of test effectiveness from 1.75-4.2\%.

	Mateo et al \cite{mateo2012validating,madeyski2013overcoming} found that second order mutants (SOM) are significantly more efficient that first order mutants (FOM).

\end{itemize}	

\textbf{Suggesting equivalent mutants techniques}

\begin{itemize}
	\item Using bayesian-learning based guidelines to help to determine equivalent mutants \cite{maldonado2005bayesian}
	\item Examining the impact of equivalent mutants on coverage \cite{grun2009impact}
	\item Using the impact of dynamic invariants \cite{schuler2009efficient}
	\item Examining changes in coverage to distinguish equivalent mutants \cite{schuler2010covering,schuler2013covering} (most effective approach, if the mutation changes coverage, there is 75\% chances of being non-equivalent)
	\item Identifying Killable Mutants (symbolic execution based) \cite{papadakis2012mutation, holling2016nequivack}: Execute mutants symbolically and assess whether these can be killable with symbolic input data. 
	Other approaches consider the use of software clones to detect equivalent mutants \cite{kintis2013identifying} (READ).
	\item Data Flow Patterns \cite{kintis2014using,kintis2015medic}: equivalent mutants have specific data-flow patterns which form data-flow anomalies, through static data-flow analysis we can eliminate equivalent mutants.
\end{itemize}

\begin{table*}[ht]
\centering
\scriptsize
\begin{tabular}{lllllllp{4cm}}
\toprule
Author(s)          & Year   & Language & \begin{tabular}[c]{@{}l@{}}Largest\\Subject\end{tabular} & \begin{tabular}[c]{@{}l@{}}\#Eq. \\ Mutants\end{tabular} & \begin{tabular}[c]{@{}l@{}}Available \\ Tool\end{tabular} & Category                                                 & Findings                                                                                      \\
\midrule
Baldwin \& Sayward \cite{baldwin1979heuristics} & 1979   &          &                                                           &                                                          &                                                           & Detect                                                   & Compiler optimization can be used to detect equivalent mutants                                \\
Acree  \cite{acree1980mutation}       & 1980   & Fortran  &                                                           & 25                                                       &                                                           & Detect                                                   & Human make mistakes when they identify equivalent mutants                                     \\
Offutt \& Craft \cite{offutt1994using}   & 1994   & Fortran  & 52                                                        & 255                                                      &                                                           & Detect                                                   & Compiler optimisation can detect on average 45\% of equivalent mutants                        \\
Offutt \& Pan \cite{offutt1996detecting,offutt1997automatically}     & 1996-7 & Fortran  & 29                                                        & 695                                                      & Yes                                                       & Detect                                                   & Constraint-based testing can detect on average 47\% of equivalent mutants                     \\
Voas \& McGraw \cite{voas1997software}    & 1997   &          &                                                           &                                                          &                                                           & Detect                                                   & Slicing may be helpful in detecting equivalent mutants                                        \\
Hierons et al. \cite{hierons1999using}      & 1999   &          &                                                           &                                                          &                                                           & \begin{tabular}[c]{@{}l@{}}Detect/\\ Reduce\end{tabular} & Program slicing can be used to detect and assist the identification of equivalent mutants     \\
Harman et al.  \cite{harman2001relationship}    & 2001   &          &                                                           &                                                          &                                                           & \begin{tabular}[c]{@{}l@{}}Detect/\\ Reduce\end{tabular} & Dependence analysis can be used to detect and assist the identification of equivalent mutants \\
Adamopoulos et al  & 2004 \cite{adamopoulos2004overcome}  &          &                                                           &                                                          &                                                           & Reduce                                                   & Co-evolution can help in reducing the effects of equivalent mutants                           \\
Grun et al. \cite{grun2009impact}       & 2009   & Java     & 12,449                                                     & 8                                                        & Yes                                                       & Reduce                                                   & Coverage Impact can be used to classify killable mutants                                      \\
Schuler et al. \cite{schuler2009efficient}    & 2009   & Java     & 94,902                                                     & 10                                                       & Yes                                                       & Reduce                                                   & Invariants violations can be used to classify killable mutants                                \\
Schuler \& Zeller \cite{schuler2010covering,schuler2013covering} & 2010-2 & Java     & 94,902                                                     & 63                                                       & Yes                                                       & Reduce                                                   & Coverage impact can be used to classify killable mutants                                      \\
Nica \& Wotawa \cite{nica2012using}    & 2012   & Java     & 380                                                       & 1,424                                                     &                                                           & Detect                                                   & Constraint-based testing can detect equivalent mutants                                        \\
Kintis et al. \cite{kintis2012isolating,kintis2015employing}     & 2012-4 & Java     & 94,902                                                     & 89                                                       &                                                           & Reduce                                                   & Higher order mutants can be used to classify killable mutants                                 \\
Kintis \& Malevris \cite{kintis2014using} & 2014   & Java     & 25,909                                                     & 84                                                       &                                                           & Detect                                                   & Data-flow patterns can detect 69\% of the equivalent mutants introduced by the AOIS operator  \\
Papadakis et al. \cite{papadakis2014mitigating}    & 2014   & C        & 513                                                       & 5,589                                                     &                                                           & Reduce                                                   & Coverage impact can be used to classify killable mutants                                      \\
Papadakis et al. \cite{papadakis2015trivial}    & 2014   & C        & 362,769                                                    & 9,551                                                     & Yes                                                       & Detect                                                   & Compilers can be used to effectively automate the mutant equivalence detection               \\
\bottomrule
\end{tabular}
\end{table*}



% !TEX root = MutationTestingSurvey.tex

\subsection{Solutions to Minimize Redundant Mutants}
\label{sec:opt:redundant}

The term \emph{redundant mutant} is used to refer to mutants that show the same behaviour of other mutants, i.e., they
are killed every time other mutants are also being killed. 
%Redundant mutants are another particular category of mutants, these mutants do not contribute to the testing process, since they are killed every time other mutants are also being killed. 
\emph{The main drawback of redundant mutants is that they can artificially inflate the apparent ability of a test technique to detect faults, in other words they tend to skew the mutation score measurement leading to serious threats to the validity of empirical research}~\cite{papadakis2016threats}.

\DONE{Is it possible to add an example?}

\input{listings/redundants}

In Section \ref{sec:process} we introduced a mutation testing example for the function \texttt{isPalindrome}. 
Listing \ref{redudantexample1} and \ref{redudantexample2} show excerpts from mutants \textit{M4} and \textit{M5} obtained with the \textit{SSDL} operator. In this case, both mutants are considered not equivalent with respect to the original program, but they are redundant between each other, because they are being killed by the same test cases, that is, the test cases exercising the inputs \texttt{abba} and \texttt{aba}.

\DONE{The following sentence is not good. Can we say something more, for example how many SUT they considered, which mutation operators they considered?}

To highlight that redundant mutants are a recurrent problem, Kintis et al. \cite{kintis2010evaluating} showed in an experiment that 9\% of mutants were redundant. For the experimental evaluation the authors considered 15 SUT (372 LOC) and 6\,127 test cases, and applied the sufficient set of operators for generating the mutants. 

\DONE{The definition below is not clear. I cannot understand teh difference between them}
In the literature, redundant mutants are divided into two categories. The first is the category of duplicated mutants, that is, mutants equivalent between them but not equivalent with respect to the original program. The second category concerns subsumed mutants, that is, mutants that are not equivalent between them, but that are killed together with the same test cases. 

Previous studies showed redundancies between mutation operators and proved that a certain subset of operators is sufficient to measure test effectiveness. For instance, Rothermel et al. \cite{rothermel1996experimental} and then Andrews et al. \cite{andrews2005mutation} proposed a small set of operators that is lead to a sufficiently accurate approximation of the results obtained by using all possible operators (e.g., replace numerical constant, negate jump condition, replace arithmetic operator, omit method calls). In the same direction, Namin et al. \cite{siami2008sufficient} proposed a statistical analysis procedure for identifying a subset of operators that could predict mutation score, their approach reduced mutants in a 93\% on C programs. 

More recently, Delamaro et al. \cite{delamaro2014designing} designed deletion operators, and found that they form a cost-effective alternative to other operators, i.e., they produce less redundant mutants. The deletion operator by itself has been proven to be the most effective for fault detection \cite{delamaro2014designing}.

%With a different perspective, 
Papadakis and Malevris \cite{papadakis2012mutation} and then Kurtz et al. \cite{kurtz2015static} proposed a path selection strategy (i.e., they generate new test inputs) for selecting the test cases able to effectively kill mutants using symbolic execution, and to consequently decrease the number of redundant mutants. 
The authors suggest constrained versions of the logical, relational and unary operators for generating less redundant mutants. 
In a similar manner, Just et al. \cite{just2012redundant,just2015higher} proved that these three operators are better at detecting faults that the rest of mutation operators.

Delgado et al. \cite{delgado2017assessment} show that some operators naturally produce more redundant mutants than others and
developed a selective approach for reducing the number of mutants without loss of effectiveness for C++ programs. 

The approach introduces a degree of redundancy for every mutation operator that helps developers to choosing the mutation operators with a lower degree of redundancy, based on the test cases defined in the project.

Another solution to reduce the number of redundant mutants is the application of trivial compiler optimisations \cite{papadakis2015trivial, kintis2017detecting,papadakis2019mutation}. 
It can identify duplicate mutants by comparing the optimised object code of each mutant. The empirical study guided by Kintis et al. \cite{kintis2017detecting} showed that by using compiler optimisations it is possible to reveal 21\% and 5.4\% of C and Java mutants, respectively.


Finally, Shin et al. suggest to avoid discarding redundant mutants but, instead, augment the test suite with additional test cases so that 
each mutant can trigger a test failure that cannot be observed with other mutants~\cite{Shin:TSE:DCriterion:2018}. 
They introduce the distinguishing mutation adequacy criterion to characterize test suites in which every mutant triggers a test failure that is not observed with other mutants.
Empirical results show that test suites that satisfy the distinguishing mutation adequacy criterion have a higher
 fault detection effectiveness than test suites that simply satisfy mutation coverage.






% !TEX root = MutationTestingSurvey.tex

\subsection{Mutation Score Calculation}
\label{sub:mutationscore}

%\DONE{Is there anything we can add here?}

\MREVISION{C6}{The \INDEX{mutation score} captures, in percentage points, the quality of a test suite. It measures the percentage of mutants that had been killed by the test suite.} 
The mutation testing process is driven by the mutation score; the process iterates multiple times until the mutation score reaches a certain threshold. 

According to recent studies, there is a relation between the mutation score and the fault revelation ability of mutation testing.
\MREVISION{C7}{A recent empirical evaluation, for example, has shown that \textit{achieving a high mutation score improves significantly the fault detection capability of a test suite}
~\cite{papadakis2018mutation}. 
%\DONE{The following sentence is incomplete. You cannot say "similar fault detection ability that statement and branch coverage". Do you mean 100\% branch coverage (i.e., branch adequate)? Also putting together branch and statement is tricky, because branch is a stronger criterion. }
%This evaluation shows that from a large set of test suites, the subsets of test suites with highest 10\% of branch coverage had a fault revelation rate of 0.542, while the subsets of test suites with highest 10\% of mutation coverage had a fault detection rate of 0.639. Instead, when considering the subsets of test suites with highest 20\% of branch and mutation coverage, the fault revelation rates dropped in both cases to 0.524 and 0.565, respectively.
The evaluation shows that, from a large set of test suites, the top 10\% ranked according to branch coverage (i.e., the ones with the highest branch coverage) have a \INDEX{fault detection
 rate} (i.e., the portion of real faults being detected by the test suite)  of 0.542, while the top 10\% ranked according to mutation coverage (i.e., the ones with the highest mutation score) have a fault detection rate of 0.639. 
If we consider, instead, test suites with lower branch and mutation coverage (i.e., top 20\% for both branch and mutation coverage), the fault detection rate drops to 0.524 and 0.565, for branch and mutation coverage, respectively.
 The drawback of this result, is, however, that only a high mutation score ensures to have test suites with a fault detection rate above 0.6.}
\REVTWO{C6}{Similar results are achieved also by Checkam et al., who show that,
among randomly selected test suites, only the ones achieving top 5\% mutation score achieve better fault detection rate than the others~\cite{Chekam:17}. 
These studies also show that, even when automated test case generation is in place, 100\% mutation score is unlikely achievable. In the work of Checkam et al., for example, the median mutation score is 0.57.}
\REVTWO{C7}{Despite the literature does not include studies concerning the identification of a threshold that guarantees that the test suite has a high fault detection rate (e.g., higher than branch coverage adequacy), the data reported in Checkam's work show that the mutation score of the best test suites is above 75\%. The value 75\% might thus be considered as a \INDEX{threshold} for terminating the mutation testing process.}

%This evaluation shows that a test suite that reaches a mutation score of 80\% has a similar  similar fault detection capability of one that achieves 80\% branch coverage. 
%\TODO{What does it mean outperforms? Can we be more precise?}
%Instead, a test suite that reaches a mutation score of 90\% outperforms this code coverage criterion. 


%To reliably compute the mutation score it is necessary to identify equivalent and redundant mutants. 
As seen in Section~\ref{sec:opt:equivalent} and~\ref{sec:opt:redundant}, equivalent and redundant mutants can affect (i.e., cause an overestimation or underestimation) the mutation score~\cite{papadakis2016threats,kintis2017detecting}.
Hence, it is desirable to identify the presence of such mutants before estimating the fault revelation ability of a test suite.




% !TEX root = MAIN.tex

\chapter{Data-driven Mutation Testing}
\label{chapter:datamutation}

% !TEX root = MutationTestingSurvey.tex

\section{Overview of the Data-driven Mutation Testing Process}
\label{sec:dataProcess}

	\begin{figure}
	\centering
		\includegraphics[width=\textwidth]{images/dataProcess}
		\caption{Data-driven Mutation Testing Process.}
		\label{fig:data:process}
	\end{figure}



This Chapter defines a test suite assessment process based on the injection of faults in the data processed by software components; we refer to this process as \INDEX{data-driven mutation testing}. 
%The definition of data-driven mutation testing is a unique contribution of this book; it has not been presented in previous literature work.

Data-driven mutation testing aims to assess test suites by simulating faults that affect the data produced, received, or exchanged by the software and its components.
It is based on a \INDEX{fault model} capturing the type of data faults that might affect the system. The fault model is produced by software engineers based on their domain knowledge and experience~\cite{di2015generating}. The considered faults might be due to programming errors, hardware problems, or critical situations in the environment (e.g., noise in the channel). The data is then automatically  mutated (i.e., modified) by a set of operators that aim to replicate the faults in the fault model. For example, the \INDEX{bit flip operator} flips a randomly selected bit of a field of the transmitted data (see Section~\ref{sec:faultModel}). 
%Mutation operators can be applied multiple times, on different data chunks or over repeated executions of a test case, ti


Figure \ref{fig:data:process} shows the reference \INDEX{data-driven mutation testing process} that will be considered in FAQAS. The process is based on two main sub-processes, \INDEX{test suite evaluation} and \INDEX{test suite augmentation}, which are described in Sections~\ref{sec:data:test_suite_evaluation}~and~\ref{sec:data:test_suite_augmentation}, respectively. Differently from the code-driven mutation testing process introduced in Section~\ref{sec:process}, the data-driven mutation testing process has not been formalized by existing software testing literature. An extensive discussion of related work has been presented in deliverable D1.

\MREVISION{C-P-29}{The type of faults that might be simulated by data-driven mutation testing are \INDEX{CPU faults}, \INDEX{memory faults}, \INDEX{data processing faults}, and \INDEX{communication faults}. 
However, it is worth noting that these faults are the means to perform mutation testing (i.e., evaluate test suites), they are not the purpose of data-driven mutation testing. Data-driven mutation testing does not aim at simulating such faults to determine if the system is robust but it aims to determine if, in the presence of such faults, the test suite fails as it would be expected. The choice of the faults to simulate depends on the type of test suite to evaluate.}

\CHANGED{In FAQAS we focus on evaluating the quality of \INDEX{functional test suites}. In the following, we briefly discuss the reasons why we do not evaluate \INDEX{robustness testing} test suites, which, among the different types of test suites, is the closest to mutation testing. Indeed,  it is often performed through mutation.
Space systems are expected to be robust with respect to CPU faults and memory faults; for this reasons such faults are often the means for performing robustness testing. Robustness testing is often performed by relying on ad-hoc fault injection systems (e.g., by corrupting memory through debugger features), which may require the test cases to be manual performed. For these reasons evaluating the quality of robustness test cases with an automated and generic strategy appear infeasible.}

%\CHANGED{To perform data-driven mutation testing of functional test suites all the different types of faults might be considered (e.g., CPU faults, memory faults, data processing faults, and communication faults). However, CPU faults and memory faults simply result into erroneous results, which are simulated by replacing valid values or performing bit flips.}

\CHANGED{With data-driven mutation testing we aim to simulate higher-level problems that typically affect complex data structures. The main reason is that other types of problems (e.g., computation of erroneous results) are already targeted by code-driven mutation. For this reason, we aim to perform data-driven mutation testing at the boundary of software components.
An ideal target for data-driven mutation testing are loosely coupled software components; typically the ones that run on separated pieces of hardware (e.g., the on board controller and the ADCS).
Despite other cases can be envisaged (e.g., pure software components that run on the same hardware), we target components running on separated hardware since we believe they are more likely affected by problems due to an incorrect understanding of requirements specifications or error-handling (e.g., because developed by separate teams).}
%Since data-driven mutation testing alters the data produced, received, or exchanged by the software or its components, 

\CHANGED{For this reason, in FAQAS, we apply data-driven mutation testing to evaluate test suites that trigger the execution and communication between multiple components (e.g., system or integration test cases). In FAQAS, data-driven mutation testing is not meant to be applied to assess unit test suites.}

\clearpage

\renewcommand\APPR{\emph{DAMAt}\xspace}

\section{Test Suite Evaluation} % (fold)
\label{sec:data:test_suite_evaluation}

\STARTCHANGEDWPT

% !TEX root = MAIN.tex

\chapter{Introduction}

The purpose of this  Software Validation Specification is to describe the testing, analysis, inspection and
review of design specifications, and is used to document the software validation specification concerning the requirements baseline.

\chapter{Applicable and reference documents}

\begin{itemize}
\item{D1 - Mutation testing survey}
\item{D2 - Study of mutation testing applicability to space software}
\item{D4 - Validation of the toolset}
\item{SSS - Software System Specification}
\item{SUM - Software User Manual}
\item{SUTP - Software Unit Test Plan}
\end{itemize}

\chapter{Terms, definitions, and abbreviated terms}

\begin{itemize}
\item{FAQAS}: activity ITT-1-9873-ESA
\item{FAQAS-framework}: software system to be released at the end of WP4 of FAQAS
\item{D2}: Deliverable D2 of FAQAS, \emph{Study of mutation testing applicability to space software}
\item{D4}: Deliverable D4 of FAQAS, \emph{Validation of the toolset}
\item{SSS}: Software System Specification.
\item{KLEE}: Third-party test generation tool, details are provided in D2.
\item{MLFS}: Mathematical Library for Flight Software.
\item{SUT}: Software under test, i.e, the software that should be mutated employing mutation testing.
\item{SUM}: Software User Manual.
\item{SUTP}: Software Unit Test Plan
\item{WP}: Work package.
\end{itemize}

\chapter{Software Overview}

This documents concern the components of the FAQAS framework:
\begin{itemize}
  \item MASS
  \item \DAMA
  \item SEMuS
%  \item DAMTE
\end{itemize}

For detailed information on their structure and usage see D2 and SUM.

\chapter{Software validation specification task identification}

\STARTCHANGEDFINAL

\section{Task and criteria}
\label{sec:taskCrit}

The validation tasks are meant to ensure that the requirement expressed in the SSS are met.
They are accomplished by following validation approaches:
\begin{itemize}
  \item veryfing the requirements for the most critical and complex software components througt \EMPH{unit testing} as detailed in the SUTP.
  \emph{unit testing} is composed of three tasks:
  \begin{itemize}
    \item performing \emph{\DAMA-TD-DDMutation-1}
    \item performing \emph{\DAMA-TD-DDMutation-2}
    \item performing \emph{\MASS-TD-SRCMutation-1}
    \item performing \emph{SEMuS-TD-TGMutation-1}
  \end{itemize}
  \item verifying the requirements for the rest of the components through \EMPH{application to the case studies} as detailed in the D4 and the SUM.
  Application to the case studies is composed of the following tasks, detailed in \ref{sec:case_studies}
  \begin{itemize}
    \item \emph{configuring \MASS and running Launcher.sh - \MASS}
    \item \emph{configuring \MASS and running PrepareSUT.sh - \MASS}
    \item \emph{configuring \MASS and running GenerateMutants.sh - \MASS}
    \item \emph{configuring \MASS and running CompileOptimizedMutants.sh - \MASS}
    \item \emph{configuring \MASS and running OptimizedPostProcessing.sh - \MASS}
    \item \emph{configuring \MASS and running GeneratePTS.sh - \MASS}
    \item \emph{configuring generate\_template\_config.json and running generate\_direct.py - SEMuS}
    \item \emph{manually editing of the generated test templates - SEMuS} 
    \item \emph{instrumenting the source code - \DAMA}
    \item \emph{configuring and running} a given component, for example \texttt{\DAMA\_compile.sh}.
  \end{itemize}
  \item verifying general requirements through \EMPH{manual inspection}. For this approach no further specific task can be identified.
\end{itemize}



\section{Application to the case studies}
\label{sec:case_studies}

In the following sections the validation task that fall under the \EMPH{application to the case studies} category are described, along with their pass-fail criteria.


\subsection{Configuring \MASS and running Launcher.sh - \MASS}
\label{sec:configuring:mass}

The objective of this task is to validate that \MASS can properly prepare the SUT, collect code coverage information, and correctly apply the eight steps of the methodology for assessing the quality of the SUT test suite.

% !TEX root =  ../MAIN.tex

\begin{table}[tb]
\footnotesize
\centering
\caption{\MASS minimal set of parameters to be configured.}
\label{table:to_configure}
\begin{tabular}{llp{7.5cm}}
\hline
\textbf{Script Name}  & \textbf{Parameter} &  \textbf{Description} \\
\hline
& BUILD\_SYSTEM &  Specifies the building system type.\\
& PROJ &  Path of the SUT root  directory.\\
& PROJ\_SRC &  Path of the SUT source directory.\\
& PROJ\_TST &  Path of the SUT test  directory.\\
& PROJ\_COV &  Path of the directory with SUT coverage information.\\
& PROJ\_BUILD &  Path of the directory where the compiled binary is stored.\\
& ORIGINAL\_MAKEFILE &  Path to the original build script.\\
& COMPILATION\_CMD &  Compilation command of the SUT.\\
& TCE\_COMPILE\_CMD &  Compilation command for TCE analysis.\\
& PRIORITIZED &  Specifies whether \MASS should be executed with a prioritized test suite.\\
& SAMPLING &  Specifies the mutant sampling technique.\\
\hline
PrepareSUT.sh & None &  Commands shall be provided manually.\\
\hline
mutation\_additional\_functions.sh & run\_tst\_case  &  Implementation of the Bash function run\_tst\_case that executes the test case passed as a parameter.\\
\hline
\end{tabular}
\end{table}

The files and variables to be configured for \MASS are specified in Table~\ref{table:to_configure}. As detailed in the SUM document, these files (i.e., mass\_conf.sh, PrepareSUT.sh, mutation\_additional\_functions.sh) support \MASS to correctly process the paths of the SUT, SUT compilation commands, SUT test suite execution commands, and enable the mutation analysis process.

After configuring the three files, the following command shall be executed for launching the mutation analysis process. 

\begin{lstlisting}[language=bash]
  $ ./Launcher.sh
\end{lstlisting}

Note that this command (1) shall be executed from \MASS workspace, (2) shall automatically perform the following eight steps:

\begin{enumerate}
  \item PrepareSUT
  \item GenerateMutants
  \item CompileOptimizedMutants
  \item OptimizedPostProcessing
  \item GeneratePTS
  \item ExecuteMutants
  \item IdentifyEquivalents
  \item MutationScore
\end{enumerate}

A typical \MASS output is presented in Listing~\ref{mass:output}.

\begin{lstlisting}[language=bash, label=mass:output, caption=\MASS output.]
##### MASS Output #####
## Total mutants generated: 28071
## Total mutants filtered by TCE: 6918
## Sampling type: fsci
## Total mutants analyzed: 461
## Total killed mutants: 369
## Total live mutants: 92
## Total likely equivalent mutants: 53
## MASS mutation score (%): 90.44
## List A of useful undetected mutants: /opt/MLFS/RESULTS/useful_list_a
## List B of useful undetected mutants: /opt/MLFS/RESULTS/useful_list_b
## Number of statements covered: 1973
## Statement coverage (%): 100
## Minimum lines covered per source file: 2
## Maximum lines covered per source file: 138
\end{lstlisting}

\subsubsection{Pass-Fail criteria}

The following criteria must be respected in order to declare this task a pass:

\begin{itemize}
  \item The eight steps of the methodology shall be performed. In turn, the following shall be produced for each steps:
    \begin{itemize}
      \item PrepareSUT: this step shall produce code coverage files, and list of test cases to be executed at \texttt{COV\_FILES} folder.
      \item GenerateMutants: this step shall produce mutant files at \texttt{src-mutants} folder.
      \item CompileOptimizedMutants: inside \texttt{COMPILED} there should be one folder for each optimization level containing (1) list of SHA512 hashes for every compiled mutant, (2) list of non-compiled mutants.
      \item OptimizedPostProcessing: inside \texttt{COMPILED} there should be (1) list of equivalent mutants, (2) list of redundant mutants, (3) list of unique mutants.
      \item GeneratePTS: this step shall produce reduced and prioritized test suites for the SUT at folder \texttt{PRIORITIZED}.
      \item ExecuteMutants: inside \texttt{MUTATION} folder there should be: (1) code coverage for each mutant, (2) traces of killed and live mutants, (3) list of executed mutants.
      \item IdentifyEquivalents: inside folder \texttt{DETECTION} there should be: (1) list of equivalent mutants, (2) mutation traces without equivalent mutants.
      \item MutationScore: inside folder \texttt{RESULTS} there should be: (1) final MASS report, (2) list of killed and live mutants, (3) code coverage report, (4) list of useful mutants. 
    \end{itemize}
  \item A final report shall be reported at the end of the mutation analysis process.
  \item The MASS log shall report no error.
\end{itemize}

\subsection{Configuring \MASS and running PrepareSUT.sh - \MASS}

This validation task consists of configuring \MASS, as specified in the Section~\ref{sec:configuring:mass}, and then executing the PrepareSUT.sh command of \MASS, this command prepare the SUT and collect information about the SUT test suite. Note that this script shall be prepared by the SUT engineer. 
The objective of this task is to validate that \MASS can properly prepare the SUT and collect code coverage information about the SUT test suite, which is necessary for enabling \MASS optimizations.

\begin{lstlisting}[language=bash]
  $ ./PrepareSUT.sh
\end{lstlisting}

The command shall create a \texttt{COV\_FILES} folder inside the \MASS workspace; such folder will contain GCOV files for each test case, and a list of test cases to be executed.

\subsubsection{Pass-Fail criteria}

\begin{itemize}
  \item The execution of PrepareSUT shall be successful, that is, the script shall terminate with a 0 error code.
  \item The required output shall be observed inside \texttt{COV\_FILES} folder.
\end{itemize}

\subsection{Configuring \MASS and running GenerateMutants.sh - \MASS}

This validation task consists of configuring \MASS, as specified in the Section~\ref{sec:configuring:mass}, and then executing the GenerateMutants.sh command of \MASS, this command generates the code coverage matrices and launches the generation of mutants. A prerequisite to this task is to have executed the PrepareSUT command.
The objective of this task is to validate that \MASS can properly process code coverage information and generate code-driven mutants of the SUT.

\begin{lstlisting}[language=bash]
  $ ./GenerateMutants.sh
\end{lstlisting}

The command shall create a \texttt{src-mutants} folder inside the \MASS workspace; such folder will contain one folder for each source under test, and each of this folder shall contain a set of mutants for the source code.

A typical mutant source file will be named using (1) the name of the source file, (2) the mutated line, (3) the instance of the mutation (e.g., the AOR operator can replace a $+$ operator into four instances (a) $-$, (b) $*$, (c) $/$, and (d) $\%$) and the position in the source file, (4) the operator being used, (5) the affected function. An example of mutant would be \url{source.mut.126.1_6_19.ICR.function_1.c}

\subsubsection{Pass-Fail criteria}

\begin{itemize}
  \item The execution of GenerateMutants shall be successful, that is, the script shall terminate with a 0 error code.
  \item The required output shall be observed inside \texttt{src-mutants} folder.
\end{itemize}

\subsection{Configuring \MASS and running CompileOptimizedMutants.sh - \MASS}

This validation task consists of configuring \MASS, as specified in the Section~\ref{sec:configuring:mass}, and then executing the CompileOptimizedMutants.sh command of \MASS, this command compiles the mutants with multiple optimisation levels and generate intermediate files to be processed then by OptimizedPostProcessing. A prerequisite to this task is to have executed the GenerateMutants command.
The objective of this task is to validate that \MASS can properly compile the mutants and generate a list of SHA512 hashes for each mutant, in order to properly identify equivalent and redundant mutants during the OptimizedPostProcessing step.

\begin{lstlisting}[language=bash]
  $ ./CompileOptimizedMutants.sh
\end{lstlisting}

The command shall create a \texttt{COMPILED} folder inside the \MASS workspace; such folder will contain one folder for each optimization level containing (1) list of SHA512 hashes for every compiled mutant, (2) list of non-compiled mutants.

\subsubsection{Pass-Fail criteria}

\begin{itemize}
  \item The execution of CompileOptimizedMutants shall be successful, that is, the script shall terminate with a 0 error code.
  \item The required output shall be observed inside \texttt{COMPILED} folder.
\end{itemize}

\subsection{Configuring \MASS and running OptimizedPostProcessing.sh - \MASS}

This validation task consists of configuring \MASS, as specified in the Section~\ref{sec:configuring:mass}, and then executing the CompileOptimizedMutants.sh command of \MASS, this command iterates over the SHA512 hashes generated by the CompileOptimizedMutants command, and produce a list of unique mutants. A prerequisite to this task is to have executed the CompileOptimizedMutants command.
The objective of this task is to validate that \MASS can properly identify equivalent and redundant mutants to be then analyzed during ExecuteMutants.

\begin{lstlisting}[language=bash]
  $ ./OptimizedPostProcessing.sh
\end{lstlisting}

The command shall create (1) a list of equivalent mutants, (2) a list of redundant mutants, and (3) a list of unique mutants inside the \texttt{COMPILED} folder within the \MASS workspace.

\subsubsection{Pass-Fail criteria}

\begin{itemize}
  \item The execution of OptimizedPostProcessing shall be successful, that is, the script shall terminate with a 0 error code.
  \item The required output shall be observed inside \texttt{COMPILED} folder.
\end{itemize}

\subsection{Configuring \MASS and running GeneratePTS.sh - \MASS}

This validation task consists of configuring \MASS, as specified in the Section~\ref{sec:configuring:mass}, and then executing the CompileOptimizedMutants.sh command of \MASS, this command generates a reduced and prioritized test suite for the SUT. A prerequisite to this task is to have executed the OptimizedPostProcessing command.
The objective of this task is to validate that \MASS can properly generate a reduced and prioritized set of test cases to be used during the ExecuteMutants step.

\begin{lstlisting}[language=bash]
  $ ./GeneratePTS.sh
\end{lstlisting}

The command shall create two files, (1) a reduced, and (2) prioritized test suite for the SUT, at folder \texttt{PRIORITIZED}.

\subsubsection{Pass-Fail criteria}

\begin{itemize}
  \item The execution of GeneratePTS shall be successful, that is, the script shall terminate with a 0 error code.
  \item The required output shall be observed inside \texttt{PRIORITIZED} folder.
\end{itemize}



\subsection{Configuring generate\_template\_config.json and running generate\_direct.py - SEMuS}

% !TEX root =  ../MAIN.tex

\begin{table}[t]
\tiny
\centering
\caption{SEMuS parameters to be configured.}
\label{table:to_conf_semus}
\begin{tabular}{lp{8.5cm}}
\hline
\textbf{Parameter}  &  \textbf{Description} \\
\hline
FAQAS\_SEMU\_CASE\_STUDY\_TOPDIR &  Root folder of the case study \\
FAQAS\_SEMU\_CASE\_STUDY\_WORKSPACE &  SEMuS workspace for the case study \\
FAQAS\_SEMU\_OUTPUT\_TOPDIR & SEMuS output folder, to be placed inside the workspace \\
FAQAS\_SEMU\_GENERATED\_MUTANTS\_TOPDIR & Root folder for storing the generated mutants \\
FAQAS\_SEMU\_REPO\_ROOTDIR &  Root folder of the case study source code\\
FAQAS\_SEMU\_ORIGINAL\_SOURCE\_FILE & Path of the source file under analysis \\
FAQAS\_SEMU\_COMPILE\_COMMAND\_SPECIFIED\_SOURCE\_FILE & Name of the source file under analysis \\
FAQAS\_SEMU\_GENERATED\_MUTANTS\_DIR & Folder for storing the generated mutants for the specified source file\\
FAQAS\_SEMU\_BUILD\_CODE\_FUNC\_STR & Bash function for building the source file under analysis, to be specified in string format\\
FAQAS\_SEMU\_BUILD\_LLVM\_BC & Bash function for building the source file to LLVM bitcode \\
FAQAS\_SEMU\_META\_MU\_TOPDIR &  Root folder for the meta mutant \\
FAQAS\_SEMU\_GENERATED\_META\_MU\_SRC\_FILE &  Path of the source file (i.e., C file) of the meta mutant \\
FAQAS\_SEMU\_GENERATED\_META\_MU\_BC\_FILE &  Path of the source file (i.e., LLVM bitcode file) of the meta mutant \\
FAQAS\_SEMU\_GENERATED\_META\_MU\_MAKE\_SYM\_TOP\_DIR &  Folder for storing intermediate files for the generated inputs \\
FAQAS\_SEMU\_GENERATED\_TESTS\_TOPDIR &  Folder for storing the generated inputs \\
FAQAS\_SEMU\_TEST\_GEN\_TIMEOUT & Timeout in seconds for the test generation process \\
FAQAS\_SEMU\_HEURISTICS\_CONFIG & Configuration array for SEMu heuristics \\
FAQAS\_SEMU\_TEST\_GEN\_MAX\_MEMORY & Maximum test generation memory in MB \\
FAQAS\_SEMU\_STOP\_TG\_ON\_MEMORY\_LIMIT & Parameter to stop test generation when the memory limit is reached \\
FAQAS\_SEMU\_TG\_MAX\_MEMORY\_INHIBIT & Parameter to stop forking states when the memory limit is reached \\
\hline
\end{tabular}
\end{table}



% !TEX root =  ../MAIN.tex

\begin{table}[t]
\tiny
\centering
\caption{Test template generator parameters to be configured.}
\label{table:ttg_semus}
\begin{tabular}{lp{8.5cm}}
\hline
\textbf{Parameter}  &  \textbf{Description} \\
\hline
TYPES\_TO\_INTCONVERT &  Specify how to convert a type to int. \\
TYPES\_TO\_PRINTCODE &  Specify how to print a type.  \\
OUT\_ARGS\_NAMES & Specify the names of function arguments that are used as function output. \\
IN\_OUT\_ARGS\_NAMES & Specify the names of function arguments that are used both as function input and output \\
TYPE\_TO\_INITIALIZATIONCODE &  Specify the initialization statement of a type.\\
VOID\_ARG\_SUBSTITUTE\_TYPE & Specify the underlying type for a void pointer. \\
TYPE\_TO\_SYMBOLIC\_FIELDS\_ACCESS & Specify, for pointer parameters, the number of elements it points to. \\
\hline
\end{tabular}
\end{table}




The objective of this task is to validate that \SEMUS can properly generate test templates that can be processed by the underlying test generation tool (i.e., KLEE).

The files and variables to be configured for \SEMUS are specified in Tables~\ref{table:to_conf_semus} and~\ref{table:ttg_semus}. As detailed in the SUM document, these files (i.e., generate\_template\_config.json and running generate\_direct.py) support \SEMUS to correctly process the paths of the SUT, SUT compilation commands, the configuration of \SEMUS itself, and how to print the values of the functions under test, so \SEMUS can determine if a mutant has been killed.

After configuring the two files, the following command shall be executed for launching the test template generation. 

\begin{lstlisting}[language=bash]
 $ case_studies/$SUT/util_codes/generate_direct.py ../WORKSPACE/DOWNLOADED/casestudy/test.c direct \
                    " -I../WORKSPACE/DOWNLOADED/casestudy/" -c generate_template_config.json
\end{lstlisting}

Note that this command shall generate inside the directory \texttt{case\_studies/\$SUT/util\_codes} one folder for each source under analysis, and inside of these folders, one template for each function under test.

\subsubsection{Pass-Fail criteria}

\begin{itemize}
  \item The test templates shall be generated.
  \item The generate\_direct.py shall terminate with a 0 error code.
\end{itemize}

\subsection{Instrumenting the source code - \DAMA}
\label{subsec:instrumenting}

The variables for running the \DAMA pipeline on ESAIL must be set in the \emph{\DAMA\_configure.sh} file, as reported in Listing~\ref{lst:configure_esail}. The significance of these variables is described in the SUM.

Then the commands represented in Listing~\ref{lst:instrument_esail_cmds} must be run to generate the mutation probes.

\begin{lstlisting}[language=bash, label={lst:instrument_esail_cmds}]
bash DAMA_probe_generation.sh
\end{lstlisting}

The generated probes must be inserted in the target function, as reported in the SUM (Chapter 13).
The Software Under Test (LibParam or ESAIL) shall be compiled with the macro \texttt{-DMUTATIONOPT=-1} enabled.

\subsubsection{Pass-Fail criteria}

The following criteria must be respected in order to declare this task a pass:
\begin{itemize}
  \item The compilation shall be successful.
  \item The compilation log shall report no error.
\end{itemize}

\subsection{Configuring and running \emph{\DAMA\_compile.sh}}

A prerequisite for this task is having successfully performed \EMPH{Instrumenting the source code - \DAMA}, as described in Section~\ref{subsec:instrumenting}.

\subsubsection{ESAIL}

The commands for compiling the SVF, reported in Listing~\ref{lst:compile_esail} must be included in the \emph{\DAMA\_compile.sh} script in the appropriate section as described in the SUM (Chapter 13).


  \begin{lstlisting}[language=bash, label={lst:compile_esail}]

  compilation_folder="/home/svf/Svf"

  pushd $compilation_folder

  make install-debug

      if [ $? -eq 0 ]; then
          echo $x " compilation OK"
      else
          echo $x " compilation FAILED"
      fi

  popd

  \end{lstlisting}

Then the commands represented in Listing~\ref{lst:compile_esail_cmds} must be run.

  \begin{lstlisting}[language=bash, label={lst:compile_esail_cmds}]

  bash \DAMA_compile.sh "0" "TRUE"

  \end{lstlisting}

\subsubsection{LibParam}

The commands for configuring the LibParam test suite, reported in Listing~\ref{lst:compile_param} must be included in the \emph{\DAMA\_compile.sh} script in the appropriate section as described in the SUM (Chapter 13


\begin{lstlisting}[language=bash, label={lst:compile_param}]

TEST_FOLDER="/home/csp/libparam/tst"

pushd $TEST_FOLDER

for f in *; do
    if [ -d "$f" ] && [ "$f" != "include" ]; then

        pushd $f
        echo "cleaning..."
        ./waf clean

        echo "configuring..."
        if [ $singleton == "TRUE" ]; then
        ./waf configure --mutation-opt $mutant_id --singleton $singleton
        else
        ./waf configure --mutation-opt $mutant_id
        fi

        if [ $? -eq 0 ]; then
            echo $x " configuration OK"
        else
            echo $x " configuration FAILED"
        fi
        popd
    fi
done

popd

\end{lstlisting}

Then the commands represented in Listing~\ref{lst:compile_param_cmds} must be run.

  \begin{lstlisting}[language=bash, label={lst:compile_param_cmds}]

  bash \DAMA_compile.sh "0" "TRUE"

  \end{lstlisting}

\subsubsection{Pass-Fail criteria}

The following criteria must be respected in order to declare this task a pass:
\begin{itemize}
  \item The compilation log shall report no error.
  \item The compilation log shall report the phrase \texttt{compilation OK}.
\end{itemize}


\subsection{Configuring and running \emph{\DAMA\_run\_test.sh}}

A prerequisite for this task is having successfully performed \EMPH{Instrumenting the source code - \DAMA}, as described in Section~\ref{subsec:instrumenting}.

\subsubsection{ESAIL}

The commands for running ESAIL's test suite, reported in Listing~\ref{lst:run_esail} must be included in the \emph{\DAMA\_run\_test.sh} script in the appropriate section as described in the SUM (Chapter 13).


  \begin{lstlisting}[language=bash, label={lst:run_esail}]

  ESAIL=/home/svf/Obsw/Test/lib/esail.sh
  PARSE_RESULTS=/home/svf/Obsw/Test/lib/parse_results.sh
  echo -n "${mutant_id};COMPILED;${tst};" >> $results_file

  timeout $TIMEOUT $ESAIL --obsw /home/svf/Obsw/Source/_binaries/OBSW.exe --fast -n -c --source /home/svf/Obsw/Source --version 04010000 -t $tst &
  ESAIL_PID=$!

  wait $ESAIL_PID
  EXEC_RET_CODE=$?

  mutant_end_time=$(($(date +%s%N)/1000000))
  mutant_elapsed="$(($mutant_end_time-$mutant_start_time))"

  \end{lstlisting}

Then the commands represented in Listing~\ref{lst:run_esail_cmds} must be run.

  \begin{lstlisting}[language=bash, label={lst:run_esail_cmds}]

  bash \DAMA_compile.sh "-1" "TRUE" && bash \DAMA_run_tests.sh "-1" "./test_list.csv" "./"

  \end{lstlisting}

The file \texttt{test\_list.csv} must contain the list of all test cases, as exposed in Listings~\ref{lst:test_esail}.

  \begin{lstlisting}[label={lst:test_esail}]

  731;46287
  6343;6161251
  6000;535195
  612;50098
  791;27125
  767;67364
  743;71586
  683;66213
  1841;102287
  3045;60261
  2561;164595
  719;74033
  5702;280961
  2819;203036
  695;62058
  707;44825
  1222;94890
  2435;134397
  755;53084
  779;156284
  543;71224
  6616;35423
  524;32621
  561;27344
  497;31588
  943;25435
  3394;63879
  5176;30804
  5656;34471
  5218;38539
  4361;63238
  2733;127361
  6161;562902
  6262;100839
  3077;83077
  6117;61088
  2848;421200
  4497;156531
  3590;190508
  2985;265498
  5917;127059
  3004;101740
  3917;200336
  2890;522174
  3034;507824
  5374;276736
  2867;441252
  2753;187527
  2928;311267
  3023;79797
  3610;1283052
  2947;995408
  2909;290593
  2966;133739
  3651;293401
  3853;132605
  6138;760249
  3621;46376
  6600;51881
  5190;196980
  4294;46522
  4270;35237
  6583;144017
  6275;143658
  5393;47192
  2604;45503
  2411;83456
  3950;39675
  146;26984
  140;23420
  3969;39893
  113;28140
  74;26170
  154;22556
  126;24476
  1465;68582
  1377;633878
  1364;47647
  2327;254152
  2339;83202
  2395;284150
  6630;228215
  2254;36753
  6362;102271
  2277;53371
  5570;32536
  160;27916
  659;98164
  6315;25372
  1242;645102
  647;27085
  225;27339
  6401;58932
  1807;56712
  4931;46262
  6038;63676
  1505;2975291
  671;84703
  990;34764
  2007;38421
  1702;86591
  5791;90984
  3996;28060
  3358;113929
  5606;31639
  1545;28645
  1572;135209
  1742;26743
  3340;136863
  197;98851
  2375;50526
  4815;74190
  1293;27614
  3313;66150
  5645;26901
  3376;1068552
  3291;584423
  2093;189396
  953;63629
  6330;25552
  6736;2446525

  \end{lstlisting}

\subsubsection{LibParam}

The commands for running the LibParam test suite, reported in Listing~\ref{lst:run_param} must be included in the \emph{\DAMA\_run\_test.sh} script in the appropriate section as described in the SUM (Chapter 13).


  \begin{lstlisting}[language=bash, label={lst:run_param}]

  tmp_log="$results_dir"/tmp_log

    TEST_FOLDER="/home/csp/libparam/tst"

    pushd $TEST_FOLDER

    pushd $tst

    touch $tmp_log

    timeout $TIMEOUT ./waf --mutation-opt=$mutant_id  --singleton=TRUE 2>&1 | tee $tmp_log

    EXEC_RET_CODE=$?

    mutant_end_time=$(($(date +%s%N)/1000000))
    mutant_elapsed="$(($mutant_end_time-$mutant_start_time))"

    if [ $EXEC_RET_CODE -ne 124 ]; then
        if grep "successfully" $tmp_log
        then
            EXEC_RET_CODE=0
            echo "PASSED"
        else
            EXEC_RET_CODE=1
            echo "FAILED"
        fi
    fi
    popd
    rm $tmp_log
    popd

  \end{lstlisting}

Then the commands represented in Listing~\ref{lst:run_param_cmds} must be run.

% mutant_id=$1
% tests_list=$2
% \DAMA_FOLDER=$3

  \begin{lstlisting}[language=bash, label={lst:run_param_cmds}]

  bash \DAMA_compile.sh "-1" "TRUE" && bash \DAMA_run_tests.sh "-1" "./test_list.csv" "./"

  \end{lstlisting}

The file \texttt{test\_list.csv} must contain the list of all test cases, as exposed in Listings~\ref{lst:test_param}.

  \begin{lstlisting}[label={lst:test_param}]

  bindings,4427.200000
  example,4145.400000
  file_store,2954.000000
  i2c,2793.400000
  log,2881.200000
  param3,3007.800000
  param4,3273.600000
  rparam3,5099.400000
  rparam4,8003.200000
  serialize,2656.400000
  spi,2825.800000
  store,2774.400000
  store_load,3182.000000
  vmem_store_checksum_first,3005.800000
  vmem_store_checksum_last,2963.400000

  \end{lstlisting}

  \subsubsection{Pass-Fail criteria}

  The following criteria must be respected in order to declare this task a pass:

  \begin{itemize}
    \item All test cases in Listing~\ref{lst:test_param} (for LibParam) and Listing~\ref{lst:test_esail} (for ESAIL) shall be executed.
    \item All test cases listed in Listing~\ref{lst:test_param} (for LibParam) and Listing~\ref{lst:test_esail} (for ESAIL) shall pass.
  \end{itemize}

\subsection{Configuring and running \emph{\DAMA\_obtain\_coverage.sh}}

A prerequisite for this task is having successfully performed \EMPH{Instrumenting the source code - \DAMA}, as described in Section~\ref{subsec:instrumenting}.

\subsubsection{ESAIL}

The variables for running the \DAMA-pipeline on ESAIL must be set in the \emph{\DAMA\_configure.sh} file, as reported in Listing~\ref{lst:configure_esail}. The significance of these variables is described in the SUM (Chapter 13).

  \begin{lstlisting}[language=bash, label={lst:configure_esail}]

  # the location of the csv with all the test identifiers and the execution time
  tests_list=$\DAMA_FOLDER/tests.csv

  # the location of the csv containing the definitions of the mutation operators
  fault_model=$\DAMA_FOLDER/fault_model.csv

  # the datatype of the elements of the target buffer
  buffer_type="unsigned long long int"

  # padding: can be used to skip the first n bit of a buffer, normally set to 0
  padding=0

  # singleton: can set to true to load the fault model into a singleton   variable, normally set to "TRUE", can also  be set to "FALSE"
  singleton="TRUE"

  \end{lstlisting}

Then the commands represented in Listing~\ref{lst:coverage_esail_cmds} must be run.

  \begin{lstlisting}[language=bash, label={lst:coverage_esail_cmds}]

  bash \DAMA_obtain_coverage.sh

  \end{lstlisting}

\subsubsection{LibParam}

The variables for running the \DAMA-pipeline on ESAIL must be set in the \emph{\DAMA\_configure.sh} file, as reported in Listing~\ref{lst:configure_param}. The significance of these variables is described in the SUM (Chapter 13).

  \begin{lstlisting}[language=bash, label={lst:configure_param}]

  # the location of the csv with all the test identifiers and the execution time
  tests_list=$\DAMA_FOLDER/tests_param.csv

  # the location of the csv containing the definitions of the mutation operators
  fault_model=$\DAMA_FOLDER/LIBP-FM.csv

  # the datatype of the elements of the target buffer
  buffer_type="unsigned long long int"

  # padding: can be used to skip the first n bit of a buffer, normally set to 0
  padding=0

  # singleton: can set to true to load the fault model into a singleton variable, normally set to "TRUE", can also  be set to "FALSE"
  singleton="TRUE"

  \end{lstlisting}

Then the commands represented in Listing~\ref{lst:coverage_param_cmds} must be run.

  \begin{lstlisting}[language=bash, label={lst:coverage_param_cmds}]

  bash pipeline_scripts/\DAMA_obtain_coverage.sh ./

  \end{lstlisting}

\subsubsection{Pass-Fail criteria}

These criteria must be respected to consider this task a pass:
\begin{itemize}
  \item The compilation lof shall show no error.
  \item The compilation log shall report the phrase \texttt{compilation OK}.
  \item No test cases shall fail.
  \item A \texttt{testlist} folder shall be generated. Inside this folder there shall be a file called \texttt{test\_<mutant>} for every mutant.
  \item Every file shall contain a subset of the tests listed in Listing~\ref{lst:test_param} (for LibParam) or in Listing~\ref{lst:test_esail} (for ESAIL).
\end{itemize}




\subsection{Configuring and running \emph{\DAMA\_mutants\_launcher.sh}}

A prerequisite for this task is having successfully performed \EMPH{Instrumenting the source code - \DAMA}, as described in Section~\ref{subsec:instrumenting}.

\subsubsection{ESAIL}

The variables for running the \DAMA-pipeline on ESAIL must be set in the \emph{\DAMA\_configure.sh} file, as reported in Listing~\ref{lst:configure_esail}. The significance of these variables is described in the SUM (Chapter 13).

Then the commands represented in Listing~\ref{lst:launcher_esail_cmds} must be run.

  \begin{lstlisting}[language=bash, label={lst:launcher_esail_cmds}]

  bash \DAMA_mutants_launcher.sh ./

  \end{lstlisting}

  The compilation logs shall report the phrase \texttt{compilation OK}.
  A \texttt{testlist} folder shall be generated. Inside this folder there shall be a file called \texttt{test\_<mutationID>} for every mutant containing a subset of the tests listed in Listing~\ref{lst:test_esail}.
  A \texttt{results} folder shall be generated and it shall contain the files described in the SUM (Section 9.2).

\subsubsection{LibParam}

The variables for running the \DAMA-pipeline on ESAIL must be set in the \emph{\DAMA\_configure.sh} file, as reported in Listing~\ref{lst:configure_param}. The significance of these variables is described in the SUM (Chapter 13).

Then the commands represented in Listing~\ref{lst:launcher_param_cmds} must be run.

  \begin{lstlisting}[language=bash, label={lst:launcher_param_cmds}]

  bash \DAMA_mutants_launcher.sh ./

  \end{lstlisting}

\subsubsection{Pass-Fail criteria}

The following criteria shall be respected in order to consider this task a pass:

\begin{itemize}
  \item The compilation logs shall report the phrase \texttt{configuration OK}.
  \item A \texttt{testlist} folder shall be generated.
  \item Inside this folder there shall be a file called \texttt{test\_<mutationID>} for every mutant containing a subset of the tests listed in Listing~\ref{lst:test_param} for LibParam or in in Listing~\ref{lst:test_esail} for ESAIL.
  \item A \texttt{results} folder shall be generated and it shall contain the files described in the SUM (Section 9.2).
\end{itemize}

\ENDCHANGEDFINAL

\section{Features to be tested}
The scope of the SVS includes all baseline requirements expressed in the SSS.

% \section{Features not to be tested}
% The SVS w.r.t. TS or RB shall describe all the features and significant
% combinations not to be tested.

\section{Test pass and fail criteria}
The pass-fail criteria for the tasks belonging to the \emph{unit testing} validation approach are detailed in the SUTP.

Regarding the tasks belonging to the \emph{application to the case studies} validation approach, the pass criteria are described in Section~\ref{sec:case_studies}.


% \section{Items that cannot be validated by test}
% a. The SVS w.r.t. TS or RB shall list the tasks and items under tests that
% cannot be validated by a test.
% b. Each of them shall be properly justified
% c. For each of them, an analysis, inspection, or review of design shall be
% proposed.
%
% % \section{Manually and automatically generated code}
% % a. The SVS shall address separately the activities to be performed for
% % manually and automatically generated code, although they have the
% % same objective (ECSS‐Q‐ST‐80 clause 6.2.8.2 and 6.2.8.7).
%
% \chapter{Software validation testing specification design}
%
% \section{General}
% a. The SVS w.r.t. TS or RB shall define software
% validation testing specification design, giving the design grouping
% criteria such as function, component, or equipment management.
% b. For each identified test design, the SVS w.r.t. TS or RB shall provide the
% information given in <6.2>.
%
% \section{Organization of each identified test design}
%
% NOTE The SVS w.r.t. TS or RB defines each validation

\clearpage

\input{damat/related}
% !TEX root = MAIN.tex

\chapter{MASS - Software Unit Testing and Integration Testing Approach}


\section{Unit/Integration Testing Strategy}

Integration testing is out of scope because of the motivations discussed in Section~\ref{sec:SUTSIT:org}.

Unit testing aims to verify that the functional requirements of MASS units are correctly implemented; test inputs are identified through the category-partition method.

%\OSCAR{Possibly integration testing might be the use of \FAQAS with the case studies?}

\section{Tasks and Items under Test}

Testing concerns the source code mutation component (hereafter, \emph{SRCMutation}) of MASS.
SRCMutation is the component with the most complicate implementation logic and thus require detailed unit testing.
All the other components either filter or join data, their implementation is simpler than SRCMutation and thus their test automation is performed through system tests (described in SVS).

\section{Feature to be tested}

Testing concerns verifying the correct implementation of the mutation operators implemented by \emph{SRCMutation}.

\section{Feature not to be tested}

Testing does not concern the verification of the capability of \emph{SRCMutation} to parse source files valid according to C/C++ language grammar. Since \emph{SRCMutation} is implemented on top of CLANG, we assume parsing capabilities are inherited from CLANG.



\section{Test Pass - Fail Criteria}

Unit testing pass if all the following are true:
\begin{itemize}
	\item Every test case is executed
	\item All test cases pass
	\item Exceptions and unexpected messages do not appear on screen and logs.
\end{itemize}



\section{Manually and Automatically Generated Code}

The \FAQAS does not contain any automatically generated code.


\clearpage

\subsection{Data-driven mutation not based on buffers}



\MREVISION{C-P-31}{The communication between loosely coupled software components is performed by relying on APIs of a dedicated communication layer; which is typical in well designed software systems.
The functioning of such communication layer may vary from system to system. The \INDEX{communication layer} works by serializing and deserializing the data that should be transmitted on the communication channel. The goal of serialization/deserialization is to perform a data transformation, i.e, translate the representation of data from the format used inside the program (e.g., a data structure) into a low level format that can be transmitted on the channel (e.g., a stream of bytes or a memory buffer).}

\CHANGED{The differences that we may observe from system to system are related to the input interface of the communication layer. We may observe two cases:}

\CHANGED{\begin{itemize}
\item The communication layer provides serialization (deserialization) primitives that receive (produce) unstructured data (e.g, a memory buffer).
\item The communication layer provides serialization (deserialization) primitives that receive (produce) data structured according to a specific format. 
\end{itemize}}

\CHANGED{In both the two cases, the communication layer performs a \INDEX{data transformation}, i.e, translates the data from the format used inside the program (e.g., a data structure) into a low level format that can be transmitted on the channel (e.g., a memory buffer). However, this commonality does not enable the definition of a single solution to perform data mutation. More precisely, \EMPH{a solution that performs mutation on memory buffers may not be practical in both the two cases}. Indeed, to alter data that is already flattened on a low level representation it is necessary a (potentially complex) data model that describes how to load such data into a more structured representation that should drive the mutation. 
The data modelling effort would thus be redundant in case the data is structured according to a specific format.} 

\CHANGED{To minimize modelling costs, in the presence of a complex data structure, the data model should coincide with the data structure defined in the programming language used to implement the system (or the modelling language from which the program has been derived), while the fault model should be defined as an extension of such data structure (e.g., through annotations).
Modelling of data flattened into a low level representation is feasible only when this is already the input format of the communication layer (in such cases the transmitted data is not expected to follow a complex structure).
Finally, the definition of a generic data loading solution 
that loads data from a memory buffer into a more structured representation and works with any data structure, might be infeasible. 
%structures (e.g, fields of variable size and multiple dependencies among fields), 
%might be particularly expensive.
}

\CHANGED{In FAQAS industrial case studies, the \INDEX{communication layer} is implemented in-house by the company that produced the case study.
The system works by processing a flat structure stored in  a buffer array, which is the reason why the FAQAS \APPR solution focuses on buffers. However, space systems may rely on the ASN.1 compiler architecture; in this case, the data processed by the compiler is highly structured. The definition of the data structure is provided as an ASN.1 grammar that is then translated by the ASN.1 compiler into a C structure. For this reason, \EMPH{in FAQAS, we also design a preliminary solution for highly structured data defined with ASN.1}. Even if the ASN.1 data is translated into a C structure, we believe that a generic solution that rely on data models defined according to C structures might not be feasible. Indeed, data structures may contain elements with complex dependencies. For example, a tree data structure may define the tree depth  in a specific data field and programmers may assume that pointer to child nodes are not read when the max depth is not reached (i.e., pointer child pointers in leaf nodes are not NULL). Specifying such logic into a generic framework is particularly hard if not infeasible.}




\CHANGEDNOV{We believe that the technology implementing \INDEX{data mutation} should depend on the type of system under test. This is mostly due to the need for (1) implementing mutation operations that are fast and (2) reducing the amount of data-modelling to be manually performed (ideally engineers would like to reuse existing models and artefacts). Based on the case studies shared for WP2, we observe that data-driven mutation testing might be performed by modifying either data that is stored in an array or in a data structure defined through the ASN.1 grammar. In our vision, these two solutions differ for the strategy used to model the data and for the algorithms implemented to execute mutation. Despite a more general solution (e.g., based on UML models) that glue together these two strategies might be feasible, its implementation might be the target of an ESA activity. Indeed, when data modelling is based on generic high-level models it is necessary to implement a layer that translate high-level representation into low-level data that can be efficiently processed at runtime. The following subsections describe the two distinct cases; however, in FAQAS, we will focus on the implementation of a solution for buffer arrays. The main reasons are two (1) buffer arrays appear to be a common strategy for implementing data communication (2) in FAQAS we lack case studies based on the ASN.1 grammar (see Section~\ref{sec:caseStudies:ASN:data}). In the following, however, we provide the results of a preliminary study concerning the design of a data-driven method for the ASN.1 grammar.}

\ENDCHANGEDWPT

\clearpage

\subsubsection{Fault Model Specifications for ASN.1 grammar}
\label{subsub:asn1model}

The ASN.1 grammar enables engineers to specify data structures where the types of the items in the data structure are selected from a predefined set.

%When the data to be mutated is stored in a data structure defined through the ASN.1 grammar, the fault model is specified by indicating which operators to apply on the specific fields of the data structure. 

We have identified a set of feasible fault classes for each type supported by the ASN1SCC compiler.
The corresponding mutation operators are automatically configured based on the ASN.1 grammar (e.g., in the case of an attribute of type INTEGER, the min/max values of the VOR operator are derived from the boundaries of the INTEGER type).
Table~\ref{table:faultModel:FAQAS:ASN1} provides, for each of such types, the feasible fault classes and the configurations for the mutation operators.
In the configuration for the mutation operators, we refer to the variables (e.g., MIN and MAX) appearing in the ASN.1 xml file.

Figure~\ref{fig:ASN1ProbesGeneration} provides an overview of the process in place to generate probes including the fault model.
The engineer first export the ASN.1 grammar as XML, then he modifies the generated file by specifying, for each \emph{Asn1Type}, the mutation operator to be used (this is done by adding an xml attribute called \emph{MutationOperator} with a value specifying the name of the operator). 

An example is provided in Listings~\ref{asnXML} and \ref{asnXMLUpdated}. Listing~\ref{asnXML} provides the xml generated by the grammar, which includes two INTEGER types.
Listing~\ref{asnXMLUpdated} provides the xml updated by the engineer, who indicates that the two integers should be mutated with the VAR and the VOR operator. To tune the operators, the engineer updates the MIN and MAX values for those integers to capture only nominal values. 
In the case of the first integer (the one to be mutated with VAR), the engineer sets 5 as MAX.
In the case of the second integer (the one to be mutated with VOR), the engineer sets MIN and MAX to 0 and 50 respectively.
%The engineer can tune the mutation by changing the value ranges associated to the different types. For example, this could be done to restrict the valid range of an INTEGER from (MIN=-100, MAX=100) to a nominal range of (MIN=0,MAX=50).
In case a data type is defined through value range constraints, the FAQAS framework will configure one mutation operator instance for each range.

% !TEX root = ../MAIN.tex
\begin{table}[h]
\begin{center}
\small
\begin{tabular}{|p{2cm}|p{2cm}|p{4cm}|p{4cm}|}
\hline
\textbf{Types}&\textbf{Fault Classes}&\textbf{Parameters}&\textbf{Description}\\
\hline
INTEGER&
VAT&
\begin{minipage}{4cm}
T: MAX\\
D: 1\\
\end{minipage}
&
\begin{minipage}{4cm}
\end{minipage}
\\
\hline
INTEGER&
VBT&
\begin{minipage}{4cm}
T: MIN\\
D: 1\\
\end{minipage}
&
\begin{minipage}{4cm}
\end{minipage}
\\
\hline
INTEGER&
VOR&
\begin{minipage}{4cm}
MIN: MIN\\
MAX: MAX\\
D: 1\\
\end{minipage}
&
\begin{minipage}{4cm}
\end{minipage}
\\
\hline
REAL&
VAT&
\begin{minipage}{4cm}
T: MAX\\
D: 1\\
\end{minipage}
&
\begin{minipage}{4cm}
\end{minipage}
\\
\hline
REAL&
VBT&
\begin{minipage}{4cm}
T: MIN\\
D: 1\\
\end{minipage}
&
\begin{minipage}{4cm}
\end{minipage}
\\
\hline
REAL&
VOR&
\begin{minipage}{4cm}
MIN: MIN\\
MAX: MAX\\
D: 1\\
\end{minipage}
&
\begin{minipage}{4cm}
\end{minipage}
\\
\hline
ENUMERATED&
INV&
\begin{minipage}{4cm}
MIN: MIN\\
MAX: MAX\\
D: 1\\
\end{minipage}
&
\begin{minipage}{4cm}
\end{minipage}
\\
\hline
BOOLEAN&
BF&
\begin{minipage}{4cm}
MIN: 0\\
MAX: 0\\
\end{minipage}
&
\begin{minipage}{4cm}
\end{minipage}
\\
\hline
NULL&
BF&
\begin{minipage}{4cm}
MIN: 0\\
MAX: 0\\
\end{minipage}
&
\begin{minipage}{4cm}
\end{minipage}
\\
\hline
BIT STRING&
BF&
\begin{minipage}{4cm}
MIN: 0\\
MAX: 0\\
\end{minipage}
&
\begin{minipage}{4cm}
\end{minipage}
\\
\hline
OCTET STRING&
BF&
\begin{minipage}{4cm}
MIN: 0\\
MAX: 0\\
\end{minipage}
&
\begin{minipage}{4cm}
\end{minipage}
\\
\hline
IA5STRING&
BF&
\begin{minipage}{4cm}
MIN: 0\\
MAX: 0\\
\end{minipage}
&
\begin{minipage}{4cm}
\end{minipage}
\\
\hline
NUMERIC STRING&
BF&
\begin{minipage}{4cm}
MIN: 0\\
MAX: 0\\
\end{minipage}
&
\begin{minipage}{4cm}
\end{minipage}
\\
\hline
SEQUENCE&
-&
\begin{minipage}{4cm}
\end{minipage}
&
\begin{minipage}{4cm}
No mutation environed for this type.
\end{minipage}
\\
\hline
SET&
-&
\begin{minipage}{4cm}
\end{minipage}
&
\begin{minipage}{4cm}
No mutation environed for this type.
\end{minipage}
\\
\hline
CHOICE&
-&
\begin{minipage}{4cm}
\end{minipage}
&
\begin{minipage}{4cm}
No mutation environed for this type.
\end{minipage}
\\
\hline
SEQUENCE OF&
-&
\begin{minipage}{4cm}
\end{minipage}
&
\begin{minipage}{4cm}
No mutation environed for this type.
\end{minipage}
\\
\hline
SET OF&
-&
\begin{minipage}{4cm}
\end{minipage}
&
\begin{minipage}{4cm}
No mutation environed for this type.
\end{minipage}
\\



%Incorrect Identifier& Several transmission data fields have fixed values, for example fields identifying the transmitting satellite. Hardware/software errors may assign incorrect identifiers.\\
%%Incorrect Checksum& Hardware/software errors may result in an incorrect checksum for a Packet or VCDU.\\
%Incorrect Counter& Counters are used to track Packet or VCDU ordering. Hardware/software errors may assign incorrect counter values.\\
%Flipped Data Bits& Physical channel noise may flip one or more bits in the data transmission.\\
\hline
\end{tabular}
\end{center}
\caption{Data Fault Classes for ASN.1 data types.}
\label{table:faultModel:FAQAS:ASN1}
\end{table}%

Figure~\ref{fig:ASN1ProbesGeneration} shows that, finally, the FAQAS toolset generates a modified version of the ASN1 source code containing the serializer and deserializer functions. The generated source code contains the FAQAS API functions to mutate the data in ASN.1 data types. Examples are shown in the following sections (see Figure~\ref{ASN_mutations}).

\begin{figure}[h]
  \centering
    \includegraphics[width=12cm]{images/ASN1mutationProces}
      \caption{Data-driven probes generation process for ASN1.}
      \label{fig:ASN1ProbesGeneration}
\end{figure}


\input{listings/asnXML.tex}
% !TEX root =  ../MAIN.tex

\begin{minipage}{15cm}
\begin{lstlisting}[language=XML, caption=ASN1 grammar updated to reflect a fault model., label=asnXMLUpdated, mathescape=true]
<TypeAssignment Name="TypeNested" CName="TypeNested" AdaName="TypeNested" Line="16" CharPositionInLine="0">
  <Asn1Type id="MY-MODULE.TypeNested" Line="16" CharPositionInLine="15" ParameterizedTypeInstance="false">
    <SEQUENCE acnMaxSizeInBits="3087" acnMinSizeInBits="2688" uperMaxSizeInBits="3087" uperMinSizeInBits="528">
      <SEQUENCE_COMPONENT Name="intVal" Line="17" CharPositionInLine="4" AdaName="intVal" CName="intVal">
        <Asn1Type id="MY-MODULE.TypeNested.intVal" Line="17" CharPositionInLine="11" ParameterizedTypeInstance="false">
          <INTEGER MutationOperator="VAT" acnMaxSizeInBits="4" acnMinSizeInBits="4" uperMaxSizeInBits="4" uperMinSizeInBits="4">
            <Constraints>
              <Range>
                <Min>
                  <IntegerValue>0</IntegerValue>
                </Min>
                <Max>
                  <IntegerValue>5</IntegerValue>
                </Max>
              </Range>
            </Constraints>
            <WithComponentConstraints />
          </INTEGER>
        </Asn1Type>
      </SEQUENCE_COMPONENT>
      <SEQUENCE_COMPONENT Name="int2Val" Line="18" CharPositionInLine="4" AdaName="int2Val" CName="int2Val">
        <Asn1Type id="MY-MODULE.TypeNested.int2Val" Line="18" CharPositionInLine="12" ParameterizedTypeInstance="false">
          <INTEGER MutationOperator="VOR" acnMaxSizeInBits="5" acnMinSizeInBits="5" uperMaxSizeInBits="5" uperMinSizeInBits="5">
            <Constraints>
              <Range>
                <Min>
                  <IntegerValue>0</IntegerValue>
                </Min>
                <Max>
                  <IntegerValue>50</IntegerValue>
                </Max>
              </Range>
            </Constraints>
            <WithComponentConstraints />
          </INTEGER>
        </Asn1Type>
      </SEQUENCE_COMPONENT>
\end{lstlisting}
\end{minipage}






\clearpage
\subsection{FAQAS Data Mutation API and Probes}
\label{sec:FAQASDataMutationProbes}

In FAQAS, the data-driven mutation testing API is automatically generated from the fault model provided by engineers. \INDEX{Data mutation probes} are either manually implemented by software engineers (in the case data mutation should target an ad-hoc communication layer that works with data buffers) or automatically generated by the toolset (in the case data mutation should target an ASN.1-based communication layer).



\subsubsection{Data Mutation Probes for ASN.1}
\label{sec:FAQASDataMutationProbesASN}

%\DONE{This section still needs to be written. We may put a sequence diagram that show that at the beginning the probe loads the info about the mutation operation instance to execute and execute it if feasible.}

%Fabrizio: I removed the picture because it does not help the reader
%\begin{figure}[tb]
%  \centering
%    \includegraphics[width=\textwidth]{images/DataDrivenASNProcess}
%      \caption{Data-driven mutation process for ASN.1 grammars.}
%      \label{fig:DataDrivenASNProcess}
%\end{figure}


After the generation of the extended ASN.1 source code according to the 
the fault model definition process provided in Figure~\ref{fig:ASN1ProbesGeneration}, 
\MREVISION{C-P-37}{the FAQAS framework will automatically modify the extended ASN.1 deserializer (or serializer) code to insert calls to the FAQAS mutation API.
Particularly, the framework will insert one 
invocation of the automatically generated FAQAS mutation API for each data type to be mutated.}
Listings~\ref{ASN_encode} shows an example of a probe added at the beginning of function \emph{TypeNested\_encode} to mutated the TypeNested data to be encoded by function \emph{TypeNested\_encode}.
Listings~\ref{ASN_decode} shows an example of a probe added 
at the end of function \emph{TypeNested\_decode}
to mutated the TypeNested data decoded by function \emph{TypeNested\_decode}.

The insertion of the probe in the serializer code is useful when the fault model is not modified by the engineer but simply includes the boundary values automatically generated by the FAQAS framework. 
This is useful to generate invalid data to be serialized and thus test the capability of the ASN.1 serializer to detect illegal values.

The insertion of the probe in the deserializer code is useful to simulate the generation of invalid data from a faulty component. This is useful when the fault model had been modified by the engineer to reflect possible non-nominal cases with values belonging to the legal value domain.

%Figure~\ref{fig:DataDrivenASNProcess} provides an overview of the mutation process followed by the ASN.1 data mutation functions.
%At runtime, for each mutation operator instance, a single probe will be enabled. 

Similarly to the case of data mutation for data buffers, each mutant can implement a single mutation operation instance or work as a mutant schemata where the mutation operation instance is selected at runtime, based on a configuration parameter. Each mutation operator instance is identified by a unique identifier. 


%Fabrizio: You never introduced E, its not a good example!
%A MOI represents the mutation to be applied. More specifically, it contains the data type name to be mutated, and an ID that represents the mutation operator to be applied, and under what conditions applies.

%For example Listing~\ref{ASN_mutations} shows two possible MOI probes for the E data type. 
%\texttt{E\_1} exercises a mutation for the E data type when the value of \texttt{pVal} is less than or equal to 255. If the condition is true, then the value is modified by replacing it for the MAX value (e.g., 255).
%Similarly, \texttt{E\_2} exercises the E data type when the value of \texttt{pVal} is equal to 1299. If the condition applies, then the value is replaced by $1299 + 1$ (e.g., VAT mutation operator).
%The mutation is saved after its execution, so it is not performed twice. 

% !TEX root =  ../MAIN.tex
\begin{minipage}{14cm}
\begin{lstlisting}[style=CStyle, caption=Example of data-driven mutation probe for ASN.1 that has been added to the encoding function., label=ASN_encode]
flag TypeNested_Encode(const TypeNested* pVal, BitStream* pBitStrm, int* pErrCode, flag bCheckConstraints)
{
    TypeNested_mutate(pVal);
    flag ret = TRUE;
...
}
\end{lstlisting}
\end{minipage}

\begin{minipage}{14cm}
\begin{lstlisting}[style=CStyle, caption=Example of data-driven mutation probe for ASN.1 that has been added to the decoding function., label=ASN_decode]
flag TypeNested_Decode(TypeNested* pVal, BitStream* pBitStrm, int* pErrCode)
{
    flag ret = TRUE;
...
        // mutation
        TypeNested_mutate(pVal);

        return ret  && TypeNested_IsConstraintValid(pVal, pErrCode);
}
\end{lstlisting}
\end{minipage}

%flag E_Decode(E* pVal, BitStream* pBitStrm, int* pErrCode)
%{
%    flag ret = TRUE;
%    *pErrCode = 0;
%    (void)pVal;
%    (void)pBitStrm;
%
%
%    (*(pVal))=5; ret = TRUE; *pErrCode = 0;
%
%    // Manually added probe 
%    E_mutate(pVal);
%    // Manually added probe END
%    return ret  && E_IsConstraintValid(pVal, pErrCode);
%}



Listing~\ref{ASN_mutations} shows three possible mutation operation instances for the \emph{TypeNested} data type configured according to the fault model shown in Listing~\ref{asnXMLUpdated}.
\texttt{TypeNested\_1} applies the VAR operator,
 \texttt{TypeNested\_2} applies the VOR operator by setting the data value below the lower bound.
 \texttt{TypeNested\_3} applies the VOR operator by setting the data value above the upper bound.
% !TEX root =  ../MAIN.tex

\begin{lstlisting}[style=CStyle, caption=Example of automatically generated ASN.1 data-driven mutation operations., label=ASN_mutations]

// TypeNested type
void _FAQAS_TypeNested_mutate(TypeNested *pVal) {

	// ALREADY_MUTATED is a global variable 
	// that traces if in the current execution we already performed data mutation
	if ( ALREADY_MUTATED ){
		return;
	}
	
	// intVal,VAT,1
	if ( ! has_been_mutated("TypeNested_1") ){
		// check that the value is not already 
		// what we want to generate

		if (pVal->intVal != 6 ){
			pVal->intVal = 6;
			save_mutation("TypeNested_1");

			return;
		}
	}
	
	// int2Val,VOR,1
	if ( ! has_been_mutated("TypeNested_2") ){
		// check that the value is not already 
		// what we want to generate

		if (pVal->intVal != 0 ){
			pVal->intVal = -1;
			save_mutation("TypeNested_2");

			return;
		}
	}

	// int2Val,VOR,2
	if ( ! has_been_mutated("TypeNested_3") ){

        printf("%lu\n", pVal -> intVal);

        if (pVal->intVal != 10){
            pVal->intVal = 51;
            save_mutation("TypeNested_3");

            return;
        }
    }

...

\end{lstlisting}

%// max E OR 1st operand constraint
%if ( strcmp(buf,"E_1") ) {                                                  
%	if ((*pVal) <= 255UL) {                                                                                                               
%	    printf("%lu\n", *pVal);
%
%	    if (*pVal != 255UL) {
%	        *pVal = 255UL;
%	        save_mutation();
%
%	        return;
%	    }           
%	}       
%}
%
%// n+1 E 2nd operand constraint
%if ( strcmp(buf, "E_2") ) {
%    if ((*pVal) == 1299UL) {                                                                                                              
%        printf("%lu\n", *pVal);
%
%        *pVal = 1299UL + 1UL;
%        save_mutation();
%        return;
%    }       
%} 


\clearpage



\clearpage
\section{Test Suite Augmentation} % (fold)
\label{sec:data:test_suite_augmentation}

\STARTCHANGEDWPT

The \INDEX{test suite augmentation process} concerns the definition of additional test cases to increase the mutation score.
It consists of four activities \INDEX{Identify Test Inputs}, \INDEX{Generate Test Oracles}, \INDEX{Execute the SUT}, \INDEX{Fix the SUT}. 
Despite these activities match the ones performed in the case of code-driven mutation testing, they are triggered and implemented in a different manner, as described below.

In the presence of mutants not killed by test cases (i.e., when the  \INDEX{mutation score} is not equal to 100\%), engineers are expected to manually investigate the underlying problems. Indeed, as reported in Section~\ref{sec:mutationAnalysisResults}, two might be the reasons for a low MS: poor oracle quality and missing test input sequences (i.e., the software does not reach the state in which it could kill the mutant).
For the first case (poor oracle quality), manual work is needed because automated approaches to automatically generate test oracles in the presence of system or integration test suites are not available. For the second case, existing test generation approaches (e.g., KLEE) might suffer from scalability problem that prevent bringing the system into a desired state ; also, they cannot deal with systems whose components communicate through channels. For this reason, generating test oracles and fixing the SUT (in case a fault is discovered after test suite augmentation) shall be performed manually.

When mutation operators are not applied because of the lack of appropriate data to mutate (i.e., in the presence of fault model coverage and mutation operation coverage below 100\%), engineers are expected to generate new test inputs for the SUT that enable the application of all the mutation operators. 
However, the methodology to adopt may vary based on the test objective and the system architecture. 
We discuss the case of the producer-consumer and client-server architecture, two common software architectures. We leave the discussion of other architectures (e.g., broker architecture and event-bus architecture) to future work.

In Figures~\ref{fig:dataDrivenTestSuiteAugmentationC} to~\ref{fig:dataDrivenTestSuiteAugmentationE}, we exemplify the two architectures. In both the two cases, data-driven mutation may concern the generated data and occur either on the component that generates the data (Figure~\ref{fig:dataDrivenTestSuiteAugmentationC}), or on the component that receives the data (Figure~\ref{fig:dataDrivenTestSuiteAugmentationD}).
For the client-server case, instead, data mutation may concern also the request for data and be performed either on the client or the server (Figure~\ref{fig:dataDrivenTestSuiteAugmentationE}). For the producer-consumer case, static program analysis may be employed to automatically generate the missing data; to this end, we aim to rely on an \INDEX{extended data mutation probe}. For the client-server case, the \INDEX{extended data mutation probe} may still be used but only to generate message requests; therefore, it would be useful only when data-driven analysis is performed on the  request message. We exemplify the two cases below.

\begin{figure}[h]
  \centering
    \includegraphics[width=14cm]{images/dataDrivenTestSuiteAugmentationC}
      \caption{Data-driven mutation analysis for different architectures.}
      \label{fig:dataDrivenTestSuiteAugmentationC}
\end{figure}

\begin{figure}[h]
  \centering
    \includegraphics[width=14cm]{images/dataDrivenTestSuiteAugmentationD}
      \caption{Data-driven mutation analysis for different architectures.}
      \label{fig:dataDrivenTestSuiteAugmentationD}
\end{figure}

\begin{figure}[h]
  \centering
    \includegraphics[width=14cm]{images/dataDrivenTestSuiteAugmentationE}
      \caption{Data-driven mutation analysis for different architectures.}
      \label{fig:dataDrivenTestSuiteAugmentationE}
\end{figure}

\begin{figure}[h]
  \centering
    \includegraphics[width=14cm]{images/dataDrivenTestSuiteAugmentationB}
      \caption{Data-driven mutation analysis for different architectures.}
      \label{fig:dataDrivenTestSuiteAugmentationB}
\end{figure}

\ENDCHANGEDWPT

\clearpage
\subsection{Producer-consumer}


We assume to have a system that exchanges data of type TypeNested defined by relying on the ASN.1 grammar (see Listings~\ref{asnXMLUpdated} and ~\ref{asnXMLUpdated}). Also, we assume that the objective of data-driven mutation testing is to assess the quality of the test cases implemented to verify the consumer component. 
Such test cases may consist of sending predefined data through a producer component and verify that the consumer generates the expected output. 
To perform data-driven mutation, we may rely on a probe installed on the deserializer component. 


To enforce the generation of the required data types, we can augment the producer component with an extended mutation probe (called \emph{\_FAQAS\_TypeNested\_cover} in Listing~\ref{ASN_encode}). In this case the probe should not be used to mutate the data but it should include assertions that enable reachability analysis. Listing~\ref{ASN_encodeReachable} shows an example where, for each mutation operation implemented in the probe, we introduce an \emph{assert(false)} statement. Static analysis tools (e.g., KLEE)  can then be used to find inputs that enable reaching any of these assertions from the entry point of the producer component. For each assertion, the static analysis component will look for an input of the entry point (e.g., the main function) that enables reaching the assertion, i.e., generate data that can be mutated according to the provided mutation operation. The identified inputs can then be used to augment the test suite.

% !TEX root =  ../MAIN.tex
\begin{lstlisting}[style=CStyle, caption=Example of data-driven mutation probe for ASN.1 that has been added to the encoding function., label=ASN_encodeReachable]
flag TypeNested_Encode(const TypeNested* pVal, BitStream* pBitStrm, int* pErrCode, flag bCheckConstraints)
{
    TypeNested_mutate(pVal);
    flag ret = TRUE;
...
}


void _FAQAS_TypeNested_mutate(TypeNested *pVal) {

	// ALREADY_MUTATED is a global variable 
	// that traces if in the current execution we already performed data mutation
	if ( ALREADY_MUTATED ){
		return;
	}
	
	// intVal,VAT,1
	if ( ! has_been_mutated("TypeNested_1") ){
		// check that the value is not already 
		// what we want to generate

		if (pVal->intVal != 6 ){
			pVal->intVal = 6;
			assert(false);
			save_mutation("TypeNested_1");

			return;
		}
	}
	
	// int2Val,VOR,1
	if ( ! has_been_mutated("TypeNested_2") ){
		// check that the value is not already 
		// what we want to generate

		if (pVal->intVal != -1 ){
			pVal->intVal = -1;
			assert(false);
			save_mutation("TypeNested_2");

			return;
		}
	}

	// int2Val,VOR,2
	if ( ! has_been_mutated("TypeNested_3") ){

        printf("%lu\n", pVal -> intVal);

        if (pVal->intVal != 51){
        	    assert(false);	
            pVal->intVal = 51;
            save_mutation("TypeNested_3");

            return;
        }
    }


\end{lstlisting}



%flag E_Decode(E* pVal, BitStream* pBitStrm, int* pErrCode)
%{
%    flag ret = TRUE;
%    *pErrCode = 0;
%    (void)pVal;
%    (void)pBitStrm;
%
%
%    (*(pVal))=5; ret = TRUE; *pErrCode = 0;
%
%    // Manually added probe 
%    E_mutate(pVal);
%    // Manually added probe END
%    return ret  && E_IsConstraintValid(pVal, pErrCode);
%}


\clearpage
\subsection{Client-server}

\STARTCHANGEDWPT

For the client-server case, we rely on the libParam case study provided by GSL. Listing~\ref{GSLmutate} shows the mutation probe, which is inserted into function \emph{gs\_rparam\_process\_packet}, on the server side. The probe mutates the buffer \emph{v\_General}, which contains a copy of a message request (i.e., \emph{request}). In the case of GSL, the FVAT operator configured to mutate \emph{request-$\>$table\_id} cannot be applied (i.e., MOC is not equal to 100\%); this indicates that the test cases do not cover a scenario in which the client passes a \emph{table\_id} above the threshold. To generate such a test case we may rely on the extended probe combined with \INDEX{static program analysis}. 

Listing~\ref{GSLcover} shows how the INDEX{extended mutation probe} might be inserted into the code of libParam. In practice, it requires the engineer to know the portion of code that handles the generation of a request message. Unfortunately, injecting the mutation probe is not sufficient to enable test generation but engineers need also to prepare a test template to enable test generation with KLEE. Listing~\ref{GSLtest} shows an example of such template based on existing libParam test cases; such test case requires the initialization of a number of state variables, which limits the possibility to automate its definition. For this reason, within FAQAS we did not find it feasible to automate data-driven mutation analysis with a tool but we aim to evaluate its manual feasibility in WP4.

Finally, when data-driven mutation is applied to the data generated by the server, test automation is made unfeasible by the fact that KLEE cannot work in the presence of a communication channel within the code to be analyzed. Such shortcoming is not observed when we mutate request data because the extended mutation probe is installed only on the client; the producer-consumer case is not affected by such shortcoming because, in this case, the probe is installed on the producer. Alternative test generation tools or extensions of KLEE shall be considered to overcome such limitations.

% !TEX root =  ../MAIN.tex
\begin{lstlisting}[style=CStyle, caption=Example of data-driven mutation probe for libParam, label=GSLmutate]

static void gs_rparam_process_packet(csp_conn_t * conn, csp_packet_t * request_packet)
{
    csp_packet_t * reply_packet = NULL;
    gs_rparam_query_t * reply;


    /* Handle endian */
    gs_rparam_query_t * request = (gs_rparam_query_t *) request_packet->data;


    request->length = csp_ntoh16(request->length);
    request->checksum = csp_ntoh16(request->checksum);


    FaultModel *fm_General = _FAQAS_General_FM();
    unsigned long long int v_General[6];

    v_General[0] = (unsigned long long int) request->action;
    v_General[1] = (unsigned long long int) request->table_id;
    v_General[2] = (unsigned long long int) request->length;
    v_General[3] = (unsigned long long int) request->checksum;
    v_General[4] = (unsigned long long int) request->seq;
    v_General[5] = (unsigned long long int) request->total;


    _FAQAS_mutate(v_General,fm_General);
    
\end{lstlisting}



%flag E_Decode(E* pVal, BitStream* pBitStrm, int* pErrCode)
%{
%    flag ret = TRUE;
%    *pErrCode = 0;
%    (void)pVal;
%    (void)pBitStrm;
%
%
%    (*(pVal))=5; ret = TRUE; *pErrCode = 0;
%
%    // Manually added probe 
%    E_mutate(pVal);
%    // Manually added probe END
%    return ret  && E_IsConstraintValid(pVal, pErrCode);
%}


% !TEX root =  ../MAIN.tex
\begin{lstlisting}[style=CStyle, caption=Example of extended data-driven mutation probe for libParam, label=GSLcover]

/**
   Get string.
   @note If the returned string is max length, the value buffer will not be 0 terminated.
   @param[in] node CSP address
   @param[in] table_id remote table id.
   @param[in] addr parameter address (remote table).
   @param[in] checksum checksum
   @param[in] timeout_ms timeout
   @param[out] value returned value (user allocated)
   @param[in] value_size size of \a value, i.e. size of parameter type in bytes.
   @return_gs_error_t
*/
static inline gs_error_t gs_rparam_get_string(uint8_t node, gs_param_table_id_t table_id, uint16_t addr,
                                              uint16_t checksum, uint32_t timeout_ms, char * value, size_t value_size)
{
    return gs_rparam_get(node, table_id, addr, GS_PARAM_STRING, checksum, timeout_ms, value, value_size);
}


gs_error_t gs_rparam_get(uint8_t node,
                         gs_param_table_id_t table_id,
                         uint16_t addr,
                         gs_param_type_t type,
                         uint16_t checksum,
                         uint32_t timeout_ms,
                         void * value,
                         size_t value_element_size)
{
    return gs_rparam_get_array(node, table_id, addr, type, checksum, timeout_ms, value, value_element_size, 1);
}


gs_error_t gs_rparam_get_array(uint8_t node,
                               gs_param_table_id_t table_id,
                               uint16_t addr,
                               gs_param_type_t type,
                               uint16_t checksum,
                               uint32_t timeout_ms,
                               void * value,
                               size_t value_element_size,
                               size_t array_size)
{
    /* Calculate length */
    gs_rparam_query_t * query;
    const size_t query_payload_size = sizeof(query->payload.addr[0]) * array_size;
    const size_t query_size = RPARAM_QUERY_LENGTH(query, query_payload_size);
    const size_t reply_payload_element_size = value_element_size + sizeof(query->payload.addr[0]);
    const size_t reply_payload_size = reply_payload_element_size * array_size;
    const size_t reply_size = RPARAM_QUERY_LENGTH(query, reply_payload_size);

    query = alloca(reply_size);
    query->action = RPARAM_GET;
    query->table_id = table_id;
    query->checksum = csp_hton16(checksum);
    query->seq = 0;
    query->total = 0;
    for(unsigned int i = 0; i < array_size; i++) {
        query->payload.addr[i] = csp_hton16(addr + (value_element_size * i));
    }
    query->length = csp_hton16(query_payload_size);

    FaultModel *fm_General = _FAQAS_General_FM();
    unsigned long long int v_General[6];

    v_General[0] = (unsigned long long int) query->action;
    v_General[1] = (unsigned long long int) query->table_id;
    v_General[2] = (unsigned long long int) query->length;
    v_General[3] = (unsigned long long int) query->checksum;
    v_General[4] = (unsigned long long int) query->seq;
    v_General[5] = (unsigned long long int) query->total;


    _FAQAS_cover(v_General,fm_General);


    /* Run single packet transaction */
    if (csp_transaction2(CSP_PRIO_HIGH, node, GS_CSP_PORT_RPARAM, timeout_ms, query, query_size, query, reply_size, CSP_O_CRC32) <= 0) {
        return GS_ERROR_IO;
    }
 ... 
 
 }

\end{lstlisting}



%flag E_Decode(E* pVal, BitStream* pBitStrm, int* pErrCode)
%{
%    flag ret = TRUE;
%    *pErrCode = 0;
%    (void)pVal;
%    (void)pBitStrm;
%
%
%    (*(pVal))=5; ret = TRUE; *pErrCode = 0;
%
%    // Manually added probe 
%    E_mutate(pVal);
%    // Manually added probe END
%    return ret  && E_IsConstraintValid(pVal, pErrCode);
%}


% !TEX root =  ../MAIN.tex
\begin{lstlisting}[style=CStyle, caption=Test template to enable data-driven mutation testing for libParam, label=GSLtest]

    // a little hack - this is next element, we use it check for overwrite and missing 0 termiation
    memset(alltypes_mem.string_A, 'Z', sizeof(alltypes_mem.string_A));
    alltypes_mem.string_A[0][1] = 0;

    char buf[GS_TEST_ALLTYPES_STRING_LENGTH + 10];

    // get max size - no 0 termination
    memset(alltypes_mem.string, 'B', sizeof(alltypes_mem.string));
    memset(buf, 'A', sizeof(buf));
    buf[GS_TEST_ALLTYPES_STRING_LENGTH + 1] = 0;
    
    csp_node CSP_NODE;
    unsigned long long int tableID;
    klee_make_symbolic(&CSP_NODE, sizeof(CSP_NODE), ”CSP_NODE”);
    klee_make_symbolic(&tableID, sizeof(tableID), ”tableID”);
    gs_rparam_get_string(&CSP_NODE, tableID, GS_TEST_ALLTYPES_STRING, GS_RPARAM_MAGIC_CHECKSUM, 1000, buf, GS_TEST_ALLTYPES_STRING_LENGTH);

    
\end{lstlisting}



%flag E_Decode(E* pVal, BitStream* pBitStrm, int* pErrCode)
%{
%    flag ret = TRUE;
%    *pErrCode = 0;
%    (void)pVal;
%    (void)pBitStrm;
%
%
%    (*(pVal))=5; ret = TRUE; *pErrCode = 0;
%
%    // Manually added probe 
%    E_mutate(pVal);
%    // Manually added probe END
%    return ret  && E_IsConstraintValid(pVal, pErrCode);
%}


\ENDCHANGEDWPT

%For example, in the case of the example in Figure~\ref{fig:DataDrivenSimpleExample}, engineers would need to implement test cases that trigger the exchange of \emph{DataMessages}.
%Fully automated approaches to generate test cases for data-driven mutation testing are unavailable; however, techniques that generate input data from scratch~\cite{gligoric2010test} or augment input data~\cite{DiNardo:TOSEM:2017} can be adopted. 
%Also, when the data used by test cases is generated by simulators, meta-heuristic search can be used to drive the generation of input data~\cite{Abdessalem:ICSE:2018}. 
%
%The execution of the SUT and the repair of the SUT are performed manually as in the case of code-driven data mutation.
%
%
%\TODO{Clarify if we generate test cases or not}
%
%Section~\ref{sec:testGenerationData} provides details about the existing solutions to  \emph{Identify Test Inputs} and \emph{Generate Test Oracles}.


% !TEX root = MAIN.tex
\clearpage
\section{Evaluation of Data-driven Mutation Testing Toolsets}
\label{sec:toolsComparisonDataDriven}

This section describes an evaluation we conducted to identify a data-driven mutation testing tool applicable to space context. In particular, we assessed the Peach Fuzzer toolset.

% description of the toolset

% !TEX root = ../MAIN.tex

\begin{table}[h]
\begin{center}
\footnotesize
\begin{tabular}{|p{5cm}|p{9cm}|}
\hline
\textbf{Operator Name}&\textbf{Description}\\
\hline
ArrayVarianceMutator&Change the length of arrays. Given L the original length of the array, the length is changed in range L-N to L+N.\\
ArrayReverseOrderMutator&Reverse the order of an array.\\
ArrayRandomizeOrderMutator&Put array elements in random order.\\
DWORDSliderMutator&Slides a DWORD through the blob.\\
BitFlipperMutator&Flips a given \% of bits in blob. Default is 20\%.\\
BlobMutator&Randomly grows a Blob block or shrinks it.\\
DataTreeRemoveMutator&Remove nodes from data tree.\\
DataTreeDuplicateMutator&Duplicate a node's value starting at 2x through 50x.\\
DataTreeSwapNearNodesMutator&Swap the data of two nodes that are near each other in the data model.\\
NumericalVarianceMutator&Produce numbers that are defaultValue - N to defaultValue + N.\\
NumericalEdgeCaseMutator&Replace with random numbers of appropriate correct size.\\
FiniteRandomNumbersMutator&Produce a finite number of random numbers for each \emph{Number} element.\\
NumericalEvenDistributionMutator&Generate numbers evenly distributed through the total numerical space of the number range.\\
NullMutator&Does nothing, just test the data produced by the fuzzer.\\
PathValidationMutator&Does not mutate. Used to trace path of each test for path validation.\\
SizedVarianceMutator&Change the length of sizes to count - N to count + N.\\
SizedNumericalEdgeCasesMutator&Change the length of sizes to numerical edge cases.\\
SizedDataVarianceMutator& Change the length of sized data to count - N to count + N. Size indicator will stay the same.\\
SizedDataNumericalEdgeCasesMutator&Change the length of sizes to numerical edge cases.\\
StringCaseMutator&Change the case of a string.\\
UnicodeStringsMutator&Generate unicode strings.\\
ValidValuesMutator&Replace with random values other than the legal ones.\\
UnicodeBomMutator&Injects BOM markers into default value and longer strings.\\
UnicodeBadUtf8Mutator&Generate bad UTF-8 strings.\\
UnicodeUtf8ThreeCharMutator&Generate long UTF-8 three byte strings.\\
StringMutator&Generate a random unicode string, for each string node, one Node at a time.\\
XmlW3CMutator&Replace XML trees with invalid, non-well former, and valid (but random) XML trees.\\
PathMutator&Replace a path with an erroneous path generated according to 20 different rules.\\
HostnameMutator&Replace a hostname with an erroneous hostname generated according to 20 different rules.\\
IpAddressMutator&Replace an IP address with an erroneous IP address generated according to 20 different rules.\\
TimeMutator&Replace a time value with an erroneous value generated according to 3 different rules.\\
DateMutator&Replace a date with 60 predefined erroneous dates.\\ 
FilenameMutator&Replace a file name with an file name generated according to 10 different rules.\\
ArrayNumericalEdgeCasesMutator&This operator is not well documented in the source code of Peach.\\
BlobSpread&This operator is not well documented in the source code of Peach.\\
\hline
\end{tabular}
\end{center}
\caption{Mutation Operators for the opensource version of Peach~\cite{PeachMozilla}}
\label{table:PeachOperators}
\end{table}%

% !TEX root =  ../MAIN.tex

\begin{minipage}{15cm}
\begin{lstlisting}[language=XML, caption=Portion of a Peach data model., label=peach, mathescape=true]
<Number name="lfh_CompSize" size="32" endian="little" signed="false"/>
<Number name="lfh_DecompSize" size="32" endian="little" signed="false"/>
<Number name="lfh_FileNameLen" size="16" endian="little" signed="false">
    <Relation type="size" of="lfh_FileName"/>
</Number>
<Number name="lfh_ExtraFldLen" size="16" endian="little" signed="false">
    <Relation type="size" of="lfh_FldName"/>
</Number>
<String name="lfh_FileName"/>
<String name="lfh_ExtraField"/>
\end{lstlisting}
\end{minipage}



Peach~\cite{PeachMozilla,PeachFuzzer} is a fuzzing tool that relies on block models~\cite{pham2016model,spike} to perform data mutations. In other words, Peach perform mutations by altering the data of an input 􏰘according to a large, predefi􏰘ned set of rules. For example, Listing~\ref{peach} introduces a portion of a data model describing the properties of the Zip data format~\cite{zipformat}. 

Even though Peach is currently a proprietary software~\cite{PeachFuzzer}, the Mozilla Foundation maintains a community edition of the toolset~\cite{PeachMozilla}, the community edition implements basic features such as the fuzzing capabilities. The proprietary version of Peach instead provides features for automatic generation of test cases and detailed reports about the potential security threats of a software~\cite{PeachFuzzer}. The version we evaluated in this activity was the community edition provided by the Mozilla Foundation. We provide an overview of the mutation operators implemented by the Peach community edition in Table~\ref{table:PeachOperators}.

% what we did

In the assessment of Peach, we defined three criteria to understand its applicability to the space context software. The first criteria concerns assessing if the community edition of Peach does work and if it can be installed properly. The second criteria concerns its portability. Finally, the third criteria concerns assessing its compatibility with FAQAS case study systems.

Regarding the first criteria, we tested Peach by applying it to the \texttt{unzip} program, and zip file mutants with the fuzzing capabilities of Peach. For this objective we reproduced the steps indicated in~\cite{zipexample}. So, first, we generated a Peach Data Model for Zip data files, and then we specified a launcher that enables the complete mutation process.

Peach provides a monitoring infrastructure that enables the execution of the whole mutation process. The process consists of the following steps:
\begin{enumerate}
	\item Specifying the data model for the a data type into an XML file.
	\item Loading the data model into Peach.
	\item Generating a new mutant (i.e, a mutated input).
	\item Running the program taking as an input the generated mutant, and with the monitoring infrastructure enabled.
	\item If the program crashes the process is stopped.
	\item If the program does not crashes, the process goes back to step 3, performing a new mutation.
\end{enumerate}

In particular, we were able to generate mutants for the Zip data format, but we could not run the monitoring infrastructure since it had dependencies with graphical environments that prevent us to execute it properly.

Regarding the second criteria, and specifically its portability. We seek to integrate it into the case studies as a component of their software to mutate data once it is sent, basically the idea would be to intercept the methods that exchange data, and apply Peach directly to the variable containing the data.
During the evaluation, we discovered that Peach -mainly implemented in Python- can be used as a Python library, and that this library can be invoked to generate multiple mutants in a off-line mode.

% why it was discarded

Regarding the third criteria and its compatibility with our case studies, we conclude that its integration with embedded systems is unlikely to work, mainly because of the characteristics of our case study systems.
For example, the ESAIL case study system runs within a real-time operative system (i.e., RTEMS by Edisoft) that does not possess a filesystem. Therefore, integrating Peach into ESAIL is not feasible because it might affect the real-time performance of the application, and also because it will be necessary to implement a solution to port the Peach toolset into the ESAIL infrastructure.





% !TEX root = MutationTestingSurvey.tex

\chapter{Mutation Testing Benchmarks}
\label{chapter:industry}

In Section~\ref{sec:limitations} we have provided an overview of state-of the-art solutions to perform mutation testing and address limitations of the mutation testing process.
Each of these solutions had been evaluated against a set of case study systems deemed representative for the usage context. Such case study systems consist of a software under test (SUT) and a test suite for the software.
Hereafter, we'll use the term \INDEX{benchmark} to indicate the set of case study systems used in the empirical evaluation presented in a research paper.
%introduce an empirical evaluation aimed to prove or disprove a specific theory proposed by the authors. 
These benchmarks can be used as reference for future mutation testing developments. 
Also, a deep understanding of the characteristics of the benchmarks considered in the literature, may provide insights on the generalizability of the results to space software (e.g, results evaluated against a real time application are more likely generalizable to the case of space software).


% !TEX root =  ../MutationTestingSurvey.tex

\setlength\LTleft{0pt}
\setlength\LTright{0pt}
\scriptsize 
\begin{longtable}{@{\extracolsep{\fill}}|p{1.2cm}|p{6cm}|p{4.3cm}|p{1.2cm}|@{}}
\caption{\normalsize List of conferences and journals considered in our survey.}
\label{table:papers} \\
\hline

	\textbf{Acronym} & \textbf{Name}	&	\textbf{\begin{tabular}[c]{@{}l@{}}Type of Venue\\(Conference Proceedings, Journal)\end{tabular}}	&	\textbf{Publisher}\\

\hline
	ICST & International Conference on Software Testing, Validation and Verification &	Conference Proceedings	&	IEEE\\
	ICSTW & International Conference on Software Testing, Verification and Validation Workshops &	Conference Proceedings	&	IEEE\\
	ICSE & International Conference on Software Engineering &	Conference Proceedings	&	IEEE/ACM\\
	ASE & International Conference on Automated Software Engineering & Conference Proceedings	& IEEE/ACM\\
	ISSTA & International Symposium on Software Testing and Analysis & Conference Proceedings	&	ACM\\
	TSE & Transactions on Software Engineering & Journal	&	IEEE\\
	IST & Information and Software Technology & Journal	&	Elsevier\\
	STVR & Software Testing, Verification and Reliability & Journal	&	Wiley\\
	SCP & Science of Computer Programming & Journal	&	Elsevier\\
\hline                                                           
\end{longtable}
\normalsize


In this chapter we provide a survey of the benchmarks considered in empirical evaluations described in research work presented in top software engineering conferences and journals, between 2013 and 2019. Table~\ref{table:papers} provides the list of conferences and journals considered in our survey. For every venue, we have considered all the empirical evaluations considering C and C++ case study systems. Table~\ref{table:papers} provides the acronym and name of the venue, the type of venue, i.e., whether if it is a journal of a conference proceedings, and finally the publisher of the venue. 

%\DONE{Add the table of conferences, describe the columns in the text. Columns should be: name, type of venue (conference proceedings,journal),publisher.}

%\DONE{I rewrote, I do not expect you to introduce a table. You may have writte someting like "In this section we describe of the most relevant benchmarks targeting C software systems, they are summarized in Table..". I rewrote differently.}

\REVTWO{C40}{In the following, we introduce Sections~\ref{section:industry:code} and~\ref{section:industry:data}, which present detailed information about benchmarks on code-driven and data-driven mutation testing, respectively.}

%\DONE{check the date}

\section{Code-Driven Mutation Testing Benchmarks}
\label{section:industry:code}

% !TEX root =  ../MutationTestingSurvey.tex


\setlength\LTleft{0pt}
\setlength\LTright{0pt}
\scriptsize 
\begin{longtable}{@{\extracolsep{\fill}}|p{3.5cm}|p{2cm}|p{2.2cm}|p{1.2cm}|p{2.2cm}|@{}}
\caption{\normalsize List of the Case Study Systems considered in the literature.}
\label{table:case_studies} \\

\hline

\textbf{Case Study (SUT)}	&	\textbf{Size of SUT}	&	\textbf{\begin{tabular}[c]{@{}l@{}}Size of Test Suite\\(test cases)\end{tabular}}	&	\textbf{\begin{tabular}[c]{@{}l@{}}Number\\of Papers\end{tabular}}	&	\textbf{Reference}	\\
\hline

Coreutils & 8 - 83 KLOC & 1022 - 18\,719 & 3 & \cite{hariri2019comparing,papadakis2018mutation,chekam2017empirical}\\
Findutils & 18 KLOC & 4\,931 & 2 & \cite{papadakis2018mutation,chekam2017empirical}\\
Grep & 9 KLOC & 5\,899 & 2 & \cite{papadakis2018mutation,chekam2017empirical}\\
Make & 7 - 35 KLOC & 691 & 5 & \cite{papadakis2018mutation,chekam2017empirical,kintis2017detecting,papadakis2015trivial,yao2014study}\\
Gzip & 2\,819 LOC & N/A & 2 & \cite{kintis2017detecting,papadakis2015trivial}\\
MSMTP & 6\,010 LOC & N/A & 2 & \cite{kintis2017detecting,papadakis2015trivial}\\
Git & 8\,750 LOC & N/A & 2 & \cite{kintis2017detecting,papadakis2015trivial}\\
Vim & 42 - 39 KLOC & 98 & 3 & \cite{wang2017faster,kintis2017detecting,papadakis2015trivial}\\
Siemens & 1 - 69 KLOC & 1\,531 - 22\,138 & 6 & \cite{phan2018music,wang2017faster,papadakis2016threats,papadakis2014mitigating,yao2014study,clark2013semantic}\\
RODOS & 125 KLOC & 48 & 1 & \cite{denisov2018mull}\\
OpenSSL & 311 KLOC & 77 & 1 & \cite{denisov2018mull}\\
LLVM Framework & 1.3 MLOC & 625 & 1 & \cite{denisov2018mull}\\
Codeflaws & 266 KLOC & 122\,261 & 1 & \cite{papadakis2018mutant}\\
Curl & 12\,753 LOC & N/A & 1 & \cite{phan2018music}\\
WuMemoryBenchmark & 32\,537 LOC & 503 & 1 & \cite{wu2017memory}\\
DelamaroDeletionBenchmark & 2\,853 LOC & 814 & 2 & \cite{delamaro2014designing,delamaro2014experimental}\\
Matrix TCL Pro & 3\,228 LOC & 24 & 1 & \cite{delgado2017assessment}\\
XmlRPC++ & 2\,194 LOC & 34 & 2 & \cite{delgado2017assessment,delgado2015class}\\
Dolphin & 3\,667 LOC & 70 & 1 & \cite{delgado2017assessment}\\
TinyXML2 & 2\,620 LOC & 62 & 2 & \cite{delgado2017assessment,delgado2015class}\\
KmyMoney & 13\,709 LOC & 248 & 1 & \cite{delgado2017assessment}\\
QtDom & 2\,117 LOC & 56 & 1 & \cite{delgado2017assessment}\\
NequivackBenchmark & 403 LOC & N/A & 1 & \cite{holling2016nequivack}\\
MuVMBenchmark & 302 LOC & 2\,256 & 1 & \cite{tokumoto2016muvm}\\
Space & 5 - 9 KLOC & 100 - 13\,585 & 4 & \cite{tokumoto2016muvm,papadakis2014mitigating,yao2014study,clark2013semantic}\\
Flex & 10 - 14 KLOC & 567 & 2 & \cite{papadakis2014mitigating,yao2014study}\\
YaoBenchmark & 1\,208 & N/A & 1 & \cite{yao2014study}\\
\hline                                                         
\end{longtable}

\normalsize

Table~\ref{table:case_studies} provides the list of the case study systems considered in the literature. For each case study we report (1) the case study (i.e., the software under test - SUT), (2) the size of the SUT (the size may vary according to the specific study), (3) the size of the test suite for the SUT\footnote{The size may vary from an empirical evaluation to another; also, in some cases the test suite details were not available (N/A).} (4) the number of papers that report using the case study, and (5) the references to the papers that report the case study.

%\DONE{You have to describe what types of programs it contains, when it was developed.}

From Table~\ref{table:case_studies} can be seen that the Siemens, Make and Space are the most common case studies with 6, 5 and 4 uses, respectively. 
The most used case study is the \INDEX{Siemens suite}. The programs belonging to this suite are commonly used to evaluate state-of-the-art solutions because they it includes faulty versions affected by faults introduced by engineers during development. The suite is available through the Subject Infrastructure Repository (SIR) from the University of Nebraska-Lincoln\footnote{https://sir.csc.ncsu.edu/portal/index.php}, in particular the suite contains a diverse collection of C programs that include code involving integer and floating-point operations, pointers, memory allocation, loops and complex conditional expressions.The Siemens suite was introduced in 1994 by Hutchins et al.~\cite{hutchins1994experiments}, the programs from the suite come with a large pool of test cases written initially by Hutchins et al. and then augmented by Rothermel et al.~\cite{rothermel1998empirical}.

Table~\ref{table:benchmarks} provides the list of benchmarks identified in the literature. We report (1) the case studies of the benchmark (i.e., the software under test - SUT), (2) the size of SUT in terms of lines of code (LOC), (3) the size of the SUT test suite in terms of number of test cases, (4) the original goal of the evaluation, and (5) the actual reference to the paper in which the benchmark was originally presented. 



%\DONE{"Appreciate" is a little too much :)}
%From Table~\ref{table:benchmarks}, it also can be appreciated the wide variety of case studies selected in the literature, while some authors decide to assess their techniques using very simple programs such as \textit{abs}~\cite{tokumoto2016muvm}. Other authors preferred using Unix applications such as the Coreutils package~\cite{hariri2019comparing,papadakis2018mutation,chekam2017empirical}. 
%On the other hand, some authors selected very complex programs such as OpenSSL and LLVM~\cite{denisov2018mull}, but usually the experiments are exercised only on small components of these large applications.

The size of the benchmark case studies varies a lot. They include simple algorithms such as \textit{abs} (6 LOC)~\cite{tokumoto2016muvm}, large Unix utilities such as the Coreutils package~\cite{hariri2019comparing,papadakis2018mutation,chekam2017empirical}, and programs implementing complex functions such as OpenSSL and LLVM~\cite{denisov2018mull}.
However, when large and complex software systems are considered in the empirical evaluation, the evaluation concerns only a subset of the components of these large applications.
For example, in the study performed by Kintis et al.~\cite{kintis2017detecting} they considered the assessment of Vim, a Unix text editor of 362 KLOC, however, because of the size of the program, the authors decided to restrict the analysis only to a couple of components such as \texttt{spell} and \texttt{eval}, 16 and 22 KLOC, respectively. 

%\DONE{Provide an example. Describe what they did in a paper where they selected a component, otherwise the sentence above might not be understood}

Concerning the size of the selected software under test, the most relevant studies have been presented by Papadakis and Chekam~\cite{papadakis2018mutation,chekam2017empirical,papadakis2018mutant}. Specifically, these authors considered the case studies Coreutils, Findutils, Grep, Make and Codeflaws (83 KLOC, 18 KLOC, 9 KLOC, 35 KLOC and 266 KLOC respectively). 
Concerning the size of the test suite employed to assess their adequacy, again Papadakis and Chekam~\cite{papadakis2018mutation,chekam2017empirical,papadakis2018mutant} present the most comprehensive studies with test suite sizes ranging from 58\,131, to 122\,261 test cases.
Despite scalability remains an open problem for mutation testing, the work of Papadakis and Chekam~\cite{papadakis2018mutation,chekam2017empirical,papadakis2018mutant} shows that optimization techniques are scaling up to large software systems.

Concerning the adoption of \INDEX{industrial case studies}, the most recent work is that of \cite{delgado2018evaluation} where mutation testing has been applied to 15 functions of a Commercial Off The Shelf Component used in nuclear systems. 
Another paper evaluating the applicability of mutation testing to safety critical systems is that of Daran and Thavenod-Fosse~\cite{daran1996software}, who conducted a study to identify if mutations are correlated with real faults; the experimentation was carried out on a critical software from the civil nuclear field. Andrews et al.~\cite{andrews2005mutation}, who explored the relation between hand-seeded and real faults in the Space software. Space is a software developed at the European Space Agency that it has been used as case study in software engineering papers since 1998 \cite{frankl1998further}.
Baker and Habli~\cite{baker2012empirical} conducted experiments on two safety-critical airborne systems, C and Ada, that had satisfied the coverage requirements for certification. In their experiments, they found an effective subset of mutation operators able to detect multiple deficiencies in test suites already assessed by experts. 

\REVTWO{C41}{Even though, many efforts has been done to make code-driven mutation testing a scalable solution, we conclude from this benchmark section, that unfortunately no study has yet applied mutation testing to complex and large real industrial software. Furthermore, in the context of space software, the applicability of mutation testing on large-scale satellite systems has not been empirically evaluated yet.}

\REVTWO{C42}{}


% carried out an empirical evaluation based on two safety-critical airborne systems that had satisfied the coverage requirements for certification. Those systems were developed using high-integrity subsets for C (MISRA C [33]) and Ada. In their experiments, they found an effective subset of mutation operators that was able to detect different deficiencies in tests suites which had already met statement and MC/DC coverage and had been manually peer-reviewed.

% !TEX root = MutationTestingSurvey.tex

\chapter{Mutation Testing Benchmarks}
\label{chapter:industry}

In Section~\ref{sec:limitations} we have provided an overview of state-of the-art solutions to perform mutation testing and address limitations of the mutation testing process.
Each of these solutions had been evaluated against a set of case study systems deemed representative for the usage context. Such case study systems consist of a software under test (SUT) and a test suite for the software.
Hereafter, we'll use the term \INDEX{benchmark} to indicate the set of case study systems used in the empirical evaluation presented in a research paper.
%introduce an empirical evaluation aimed to prove or disprove a specific theory proposed by the authors. 
These benchmarks can be used as reference for future mutation testing developments. 
Also, a deep understanding of the characteristics of the benchmarks considered in the literature, may provide insights on the generalizability of the results to space software (e.g, results evaluated against a real time application are more likely generalizable to the case of space software).


% !TEX root =  ../MutationTestingSurvey.tex

\setlength\LTleft{0pt}
\setlength\LTright{0pt}
\scriptsize 
\begin{longtable}{@{\extracolsep{\fill}}|p{1.2cm}|p{6cm}|p{4.3cm}|p{1.2cm}|@{}}
\caption{\normalsize List of conferences and journals considered in our survey.}
\label{table:papers} \\
\hline

	\textbf{Acronym} & \textbf{Name}	&	\textbf{\begin{tabular}[c]{@{}l@{}}Type of Venue\\(Conference Proceedings, Journal)\end{tabular}}	&	\textbf{Publisher}\\

\hline
	ICST & International Conference on Software Testing, Validation and Verification &	Conference Proceedings	&	IEEE\\
	ICSTW & International Conference on Software Testing, Verification and Validation Workshops &	Conference Proceedings	&	IEEE\\
	ICSE & International Conference on Software Engineering &	Conference Proceedings	&	IEEE/ACM\\
	ASE & International Conference on Automated Software Engineering & Conference Proceedings	& IEEE/ACM\\
	ISSTA & International Symposium on Software Testing and Analysis & Conference Proceedings	&	ACM\\
	TSE & Transactions on Software Engineering & Journal	&	IEEE\\
	IST & Information and Software Technology & Journal	&	Elsevier\\
	STVR & Software Testing, Verification and Reliability & Journal	&	Wiley\\
	SCP & Science of Computer Programming & Journal	&	Elsevier\\
\hline                                                           
\end{longtable}
\normalsize


In this chapter we provide a survey of the benchmarks considered in empirical evaluations described in research work presented in top software engineering conferences and journals, between 2013 and 2019. Table~\ref{table:papers} provides the list of conferences and journals considered in our survey. For every venue, we have considered all the empirical evaluations considering C and C++ case study systems. Table~\ref{table:papers} provides the acronym and name of the venue, the type of venue, i.e., whether if it is a journal of a conference proceedings, and finally the publisher of the venue. 

%\DONE{Add the table of conferences, describe the columns in the text. Columns should be: name, type of venue (conference proceedings,journal),publisher.}

%\DONE{I rewrote, I do not expect you to introduce a table. You may have writte someting like "In this section we describe of the most relevant benchmarks targeting C software systems, they are summarized in Table..". I rewrote differently.}

\REVTWO{C40}{In the following, we introduce Sections~\ref{section:industry:code} and~\ref{section:industry:data}, which present detailed information about benchmarks on code-driven and data-driven mutation testing, respectively.}

%\DONE{check the date}

\section{Code-Driven Mutation Testing Benchmarks}
\label{section:industry:code}

% !TEX root =  ../MutationTestingSurvey.tex


\setlength\LTleft{0pt}
\setlength\LTright{0pt}
\scriptsize 
\begin{longtable}{@{\extracolsep{\fill}}|p{3.5cm}|p{2cm}|p{2.2cm}|p{1.2cm}|p{2.2cm}|@{}}
\caption{\normalsize List of the Case Study Systems considered in the literature.}
\label{table:case_studies} \\

\hline

\textbf{Case Study (SUT)}	&	\textbf{Size of SUT}	&	\textbf{\begin{tabular}[c]{@{}l@{}}Size of Test Suite\\(test cases)\end{tabular}}	&	\textbf{\begin{tabular}[c]{@{}l@{}}Number\\of Papers\end{tabular}}	&	\textbf{Reference}	\\
\hline

Coreutils & 8 - 83 KLOC & 1022 - 18\,719 & 3 & \cite{hariri2019comparing,papadakis2018mutation,chekam2017empirical}\\
Findutils & 18 KLOC & 4\,931 & 2 & \cite{papadakis2018mutation,chekam2017empirical}\\
Grep & 9 KLOC & 5\,899 & 2 & \cite{papadakis2018mutation,chekam2017empirical}\\
Make & 7 - 35 KLOC & 691 & 5 & \cite{papadakis2018mutation,chekam2017empirical,kintis2017detecting,papadakis2015trivial,yao2014study}\\
Gzip & 2\,819 LOC & N/A & 2 & \cite{kintis2017detecting,papadakis2015trivial}\\
MSMTP & 6\,010 LOC & N/A & 2 & \cite{kintis2017detecting,papadakis2015trivial}\\
Git & 8\,750 LOC & N/A & 2 & \cite{kintis2017detecting,papadakis2015trivial}\\
Vim & 42 - 39 KLOC & 98 & 3 & \cite{wang2017faster,kintis2017detecting,papadakis2015trivial}\\
Siemens & 1 - 69 KLOC & 1\,531 - 22\,138 & 6 & \cite{phan2018music,wang2017faster,papadakis2016threats,papadakis2014mitigating,yao2014study,clark2013semantic}\\
RODOS & 125 KLOC & 48 & 1 & \cite{denisov2018mull}\\
OpenSSL & 311 KLOC & 77 & 1 & \cite{denisov2018mull}\\
LLVM Framework & 1.3 MLOC & 625 & 1 & \cite{denisov2018mull}\\
Codeflaws & 266 KLOC & 122\,261 & 1 & \cite{papadakis2018mutant}\\
Curl & 12\,753 LOC & N/A & 1 & \cite{phan2018music}\\
WuMemoryBenchmark & 32\,537 LOC & 503 & 1 & \cite{wu2017memory}\\
DelamaroDeletionBenchmark & 2\,853 LOC & 814 & 2 & \cite{delamaro2014designing,delamaro2014experimental}\\
Matrix TCL Pro & 3\,228 LOC & 24 & 1 & \cite{delgado2017assessment}\\
XmlRPC++ & 2\,194 LOC & 34 & 2 & \cite{delgado2017assessment,delgado2015class}\\
Dolphin & 3\,667 LOC & 70 & 1 & \cite{delgado2017assessment}\\
TinyXML2 & 2\,620 LOC & 62 & 2 & \cite{delgado2017assessment,delgado2015class}\\
KmyMoney & 13\,709 LOC & 248 & 1 & \cite{delgado2017assessment}\\
QtDom & 2\,117 LOC & 56 & 1 & \cite{delgado2017assessment}\\
NequivackBenchmark & 403 LOC & N/A & 1 & \cite{holling2016nequivack}\\
MuVMBenchmark & 302 LOC & 2\,256 & 1 & \cite{tokumoto2016muvm}\\
Space & 5 - 9 KLOC & 100 - 13\,585 & 4 & \cite{tokumoto2016muvm,papadakis2014mitigating,yao2014study,clark2013semantic}\\
Flex & 10 - 14 KLOC & 567 & 2 & \cite{papadakis2014mitigating,yao2014study}\\
YaoBenchmark & 1\,208 & N/A & 1 & \cite{yao2014study}\\
\hline                                                         
\end{longtable}

\normalsize

Table~\ref{table:case_studies} provides the list of the case study systems considered in the literature. For each case study we report (1) the case study (i.e., the software under test - SUT), (2) the size of the SUT (the size may vary according to the specific study), (3) the size of the test suite for the SUT\footnote{The size may vary from an empirical evaluation to another; also, in some cases the test suite details were not available (N/A).} (4) the number of papers that report using the case study, and (5) the references to the papers that report the case study.

%\DONE{You have to describe what types of programs it contains, when it was developed.}

From Table~\ref{table:case_studies} can be seen that the Siemens, Make and Space are the most common case studies with 6, 5 and 4 uses, respectively. 
The most used case study is the \INDEX{Siemens suite}. The programs belonging to this suite are commonly used to evaluate state-of-the-art solutions because they it includes faulty versions affected by faults introduced by engineers during development. The suite is available through the Subject Infrastructure Repository (SIR) from the University of Nebraska-Lincoln\footnote{https://sir.csc.ncsu.edu/portal/index.php}, in particular the suite contains a diverse collection of C programs that include code involving integer and floating-point operations, pointers, memory allocation, loops and complex conditional expressions.The Siemens suite was introduced in 1994 by Hutchins et al.~\cite{hutchins1994experiments}, the programs from the suite come with a large pool of test cases written initially by Hutchins et al. and then augmented by Rothermel et al.~\cite{rothermel1998empirical}.

Table~\ref{table:benchmarks} provides the list of benchmarks identified in the literature. We report (1) the case studies of the benchmark (i.e., the software under test - SUT), (2) the size of SUT in terms of lines of code (LOC), (3) the size of the SUT test suite in terms of number of test cases, (4) the original goal of the evaluation, and (5) the actual reference to the paper in which the benchmark was originally presented. 



%\DONE{"Appreciate" is a little too much :)}
%From Table~\ref{table:benchmarks}, it also can be appreciated the wide variety of case studies selected in the literature, while some authors decide to assess their techniques using very simple programs such as \textit{abs}~\cite{tokumoto2016muvm}. Other authors preferred using Unix applications such as the Coreutils package~\cite{hariri2019comparing,papadakis2018mutation,chekam2017empirical}. 
%On the other hand, some authors selected very complex programs such as OpenSSL and LLVM~\cite{denisov2018mull}, but usually the experiments are exercised only on small components of these large applications.

The size of the benchmark case studies varies a lot. They include simple algorithms such as \textit{abs} (6 LOC)~\cite{tokumoto2016muvm}, large Unix utilities such as the Coreutils package~\cite{hariri2019comparing,papadakis2018mutation,chekam2017empirical}, and programs implementing complex functions such as OpenSSL and LLVM~\cite{denisov2018mull}.
However, when large and complex software systems are considered in the empirical evaluation, the evaluation concerns only a subset of the components of these large applications.
For example, in the study performed by Kintis et al.~\cite{kintis2017detecting} they considered the assessment of Vim, a Unix text editor of 362 KLOC, however, because of the size of the program, the authors decided to restrict the analysis only to a couple of components such as \texttt{spell} and \texttt{eval}, 16 and 22 KLOC, respectively. 

%\DONE{Provide an example. Describe what they did in a paper where they selected a component, otherwise the sentence above might not be understood}

Concerning the size of the selected software under test, the most relevant studies have been presented by Papadakis and Chekam~\cite{papadakis2018mutation,chekam2017empirical,papadakis2018mutant}. Specifically, these authors considered the case studies Coreutils, Findutils, Grep, Make and Codeflaws (83 KLOC, 18 KLOC, 9 KLOC, 35 KLOC and 266 KLOC respectively). 
Concerning the size of the test suite employed to assess their adequacy, again Papadakis and Chekam~\cite{papadakis2018mutation,chekam2017empirical,papadakis2018mutant} present the most comprehensive studies with test suite sizes ranging from 58\,131, to 122\,261 test cases.
Despite scalability remains an open problem for mutation testing, the work of Papadakis and Chekam~\cite{papadakis2018mutation,chekam2017empirical,papadakis2018mutant} shows that optimization techniques are scaling up to large software systems.

Concerning the adoption of \INDEX{industrial case studies}, the most recent work is that of \cite{delgado2018evaluation} where mutation testing has been applied to 15 functions of a Commercial Off The Shelf Component used in nuclear systems. 
Another paper evaluating the applicability of mutation testing to safety critical systems is that of Daran and Thavenod-Fosse~\cite{daran1996software}, who conducted a study to identify if mutations are correlated with real faults; the experimentation was carried out on a critical software from the civil nuclear field. Andrews et al.~\cite{andrews2005mutation}, who explored the relation between hand-seeded and real faults in the Space software. Space is a software developed at the European Space Agency that it has been used as case study in software engineering papers since 1998 \cite{frankl1998further}.
Baker and Habli~\cite{baker2012empirical} conducted experiments on two safety-critical airborne systems, C and Ada, that had satisfied the coverage requirements for certification. In their experiments, they found an effective subset of mutation operators able to detect multiple deficiencies in test suites already assessed by experts. 

\REVTWO{C41}{Even though, many efforts has been done to make code-driven mutation testing a scalable solution, we conclude from this benchmark section, that unfortunately no study has yet applied mutation testing to complex and large real industrial software. Furthermore, in the context of space software, the applicability of mutation testing on large-scale satellite systems has not been empirically evaluated yet.}

\REVTWO{C42}{}


% carried out an empirical evaluation based on two safety-critical airborne systems that had satisfied the coverage requirements for certification. Those systems were developed using high-integrity subsets for C (MISRA C [33]) and Ada. In their experiments, they found an effective subset of mutation operators that was able to detect different deficiencies in tests suites which had already met statement and MC/DC coverage and had been manually peer-reviewed.

% !TEX root = MutationTestingSurvey.tex

\chapter{Mutation Testing Benchmarks}
\label{chapter:industry}

In Section~\ref{sec:limitations} we have provided an overview of state-of the-art solutions to perform mutation testing and address limitations of the mutation testing process.
Each of these solutions had been evaluated against a set of case study systems deemed representative for the usage context. Such case study systems consist of a software under test (SUT) and a test suite for the software.
Hereafter, we'll use the term \INDEX{benchmark} to indicate the set of case study systems used in the empirical evaluation presented in a research paper.
%introduce an empirical evaluation aimed to prove or disprove a specific theory proposed by the authors. 
These benchmarks can be used as reference for future mutation testing developments. 
Also, a deep understanding of the characteristics of the benchmarks considered in the literature, may provide insights on the generalizability of the results to space software (e.g, results evaluated against a real time application are more likely generalizable to the case of space software).


\input{tables/papers}

In this chapter we provide a survey of the benchmarks considered in empirical evaluations described in research work presented in top software engineering conferences and journals, between 2013 and 2019. Table~\ref{table:papers} provides the list of conferences and journals considered in our survey. For every venue, we have considered all the empirical evaluations considering C and C++ case study systems. Table~\ref{table:papers} provides the acronym and name of the venue, the type of venue, i.e., whether if it is a journal of a conference proceedings, and finally the publisher of the venue. 

%\DONE{Add the table of conferences, describe the columns in the text. Columns should be: name, type of venue (conference proceedings,journal),publisher.}

%\DONE{I rewrote, I do not expect you to introduce a table. You may have writte someting like "In this section we describe of the most relevant benchmarks targeting C software systems, they are summarized in Table..". I rewrote differently.}

\REVTWO{C40}{In the following, we introduce Sections~\ref{section:industry:code} and~\ref{section:industry:data}, which present detailed information about benchmarks on code-driven and data-driven mutation testing, respectively.}

%\DONE{check the date}

\section{Code-Driven Mutation Testing Benchmarks}
\label{section:industry:code}

\input{tables/case_studies}

Table~\ref{table:case_studies} provides the list of the case study systems considered in the literature. For each case study we report (1) the case study (i.e., the software under test - SUT), (2) the size of the SUT (the size may vary according to the specific study), (3) the size of the test suite for the SUT\footnote{The size may vary from an empirical evaluation to another; also, in some cases the test suite details were not available (N/A).} (4) the number of papers that report using the case study, and (5) the references to the papers that report the case study.

%\DONE{You have to describe what types of programs it contains, when it was developed.}

From Table~\ref{table:case_studies} can be seen that the Siemens, Make and Space are the most common case studies with 6, 5 and 4 uses, respectively. 
The most used case study is the \INDEX{Siemens suite}. The programs belonging to this suite are commonly used to evaluate state-of-the-art solutions because they it includes faulty versions affected by faults introduced by engineers during development. The suite is available through the Subject Infrastructure Repository (SIR) from the University of Nebraska-Lincoln\footnote{https://sir.csc.ncsu.edu/portal/index.php}, in particular the suite contains a diverse collection of C programs that include code involving integer and floating-point operations, pointers, memory allocation, loops and complex conditional expressions.The Siemens suite was introduced in 1994 by Hutchins et al.~\cite{hutchins1994experiments}, the programs from the suite come with a large pool of test cases written initially by Hutchins et al. and then augmented by Rothermel et al.~\cite{rothermel1998empirical}.

Table~\ref{table:benchmarks} provides the list of benchmarks identified in the literature. We report (1) the case studies of the benchmark (i.e., the software under test - SUT), (2) the size of SUT in terms of lines of code (LOC), (3) the size of the SUT test suite in terms of number of test cases, (4) the original goal of the evaluation, and (5) the actual reference to the paper in which the benchmark was originally presented. 



%\DONE{"Appreciate" is a little too much :)}
%From Table~\ref{table:benchmarks}, it also can be appreciated the wide variety of case studies selected in the literature, while some authors decide to assess their techniques using very simple programs such as \textit{abs}~\cite{tokumoto2016muvm}. Other authors preferred using Unix applications such as the Coreutils package~\cite{hariri2019comparing,papadakis2018mutation,chekam2017empirical}. 
%On the other hand, some authors selected very complex programs such as OpenSSL and LLVM~\cite{denisov2018mull}, but usually the experiments are exercised only on small components of these large applications.

The size of the benchmark case studies varies a lot. They include simple algorithms such as \textit{abs} (6 LOC)~\cite{tokumoto2016muvm}, large Unix utilities such as the Coreutils package~\cite{hariri2019comparing,papadakis2018mutation,chekam2017empirical}, and programs implementing complex functions such as OpenSSL and LLVM~\cite{denisov2018mull}.
However, when large and complex software systems are considered in the empirical evaluation, the evaluation concerns only a subset of the components of these large applications.
For example, in the study performed by Kintis et al.~\cite{kintis2017detecting} they considered the assessment of Vim, a Unix text editor of 362 KLOC, however, because of the size of the program, the authors decided to restrict the analysis only to a couple of components such as \texttt{spell} and \texttt{eval}, 16 and 22 KLOC, respectively. 

%\DONE{Provide an example. Describe what they did in a paper where they selected a component, otherwise the sentence above might not be understood}

Concerning the size of the selected software under test, the most relevant studies have been presented by Papadakis and Chekam~\cite{papadakis2018mutation,chekam2017empirical,papadakis2018mutant}. Specifically, these authors considered the case studies Coreutils, Findutils, Grep, Make and Codeflaws (83 KLOC, 18 KLOC, 9 KLOC, 35 KLOC and 266 KLOC respectively). 
Concerning the size of the test suite employed to assess their adequacy, again Papadakis and Chekam~\cite{papadakis2018mutation,chekam2017empirical,papadakis2018mutant} present the most comprehensive studies with test suite sizes ranging from 58\,131, to 122\,261 test cases.
Despite scalability remains an open problem for mutation testing, the work of Papadakis and Chekam~\cite{papadakis2018mutation,chekam2017empirical,papadakis2018mutant} shows that optimization techniques are scaling up to large software systems.

Concerning the adoption of \INDEX{industrial case studies}, the most recent work is that of \cite{delgado2018evaluation} where mutation testing has been applied to 15 functions of a Commercial Off The Shelf Component used in nuclear systems. 
Another paper evaluating the applicability of mutation testing to safety critical systems is that of Daran and Thavenod-Fosse~\cite{daran1996software}, who conducted a study to identify if mutations are correlated with real faults; the experimentation was carried out on a critical software from the civil nuclear field. Andrews et al.~\cite{andrews2005mutation}, who explored the relation between hand-seeded and real faults in the Space software. Space is a software developed at the European Space Agency that it has been used as case study in software engineering papers since 1998 \cite{frankl1998further}.
Baker and Habli~\cite{baker2012empirical} conducted experiments on two safety-critical airborne systems, C and Ada, that had satisfied the coverage requirements for certification. In their experiments, they found an effective subset of mutation operators able to detect multiple deficiencies in test suites already assessed by experts. 

\REVTWO{C41}{Even though, many efforts has been done to make code-driven mutation testing a scalable solution, we conclude from this benchmark section, that unfortunately no study has yet applied mutation testing to complex and large real industrial software. Furthermore, in the context of space software, the applicability of mutation testing on large-scale satellite systems has not been empirically evaluated yet.}

\REVTWO{C42}{}


% carried out an empirical evaluation based on two safety-critical airborne systems that had satisfied the coverage requirements for certification. Those systems were developed using high-integrity subsets for C (MISRA C [33]) and Ada. In their experiments, they found an effective subset of mutation operators that was able to detect different deficiencies in tests suites which had already met statement and MC/DC coverage and had been manually peer-reviewed.

\input{tables/industry}

\clearpage

\section{Data-Driven Mutation Testing Benchmarks}
\label{section:industry:data}

In this section we provide a survey of the benchmarks considered both in empirical evaluations described in research work and in industrial cases for data-driven mutation testing.

Table~\ref{table:benchmarks_datadriven} provides the list of the case study systems considered in the literature and industry. Similar to the previous section, for each case study we report (1) the case study (i.e., the software under test - SUT), (2) the size of the SUT (the size can be expressed in terms of bytecode instructions, lines of code (LOC), or executable size (KB or MB)), (3) the size of the test suite for the SUT (unfortunately, in most cases the test suite details were not available (N/A)), and (4) the references to the papers that report the case study.

Table~\ref{table:benchmarks_datadriven} presents case studies that were applied to UML models~\cite{di2017augmenting}, block models~\cite{pham2016model} and grammars~\cite{AFL:industrialcases}.

Concerning the size and typology of SUT, the case studies reported in Table~\ref{table:benchmarks_datadriven} varies a lot.
For instance, experimentation on block models has been applied to different types of desktop applications such as VLC, Adobe Reader, Real Player, and Windows Media Player~\cite{pham2016model} which sizes goes from 60 KB to 2.32 MB. 

Experimentation on grammars has been widely studied, but for sake of simplicity we only reported the most important case studies for AFL tool~\cite{AFL:industrialcases}. Particularly, AFL has been applied to a very wide range of software typology, for example it has been applied on programming languages such as PHP, web browsers such as Firefox, libraries such as OpenSSL, and even application server such as MySQL Server. 

Regarding case studies applied to space context software, we highlight SES-DAQ, a DAQ system developed by SES that processes bytestreams of transmitted satellite data. The version of SES-DAQ used for the experimentation performed by Di Nardo et al.~\cite{di2017augmenting} had a size of 32\,469 bytecode instructions.

\input{tables/industry_datadriven}




\clearpage

\section{Data-Driven Mutation Testing Benchmarks}
\label{section:industry:data}

In this section we provide a survey of the benchmarks considered both in empirical evaluations described in research work and in industrial cases for data-driven mutation testing.

Table~\ref{table:benchmarks_datadriven} provides the list of the case study systems considered in the literature and industry. Similar to the previous section, for each case study we report (1) the case study (i.e., the software under test - SUT), (2) the size of the SUT (the size can be expressed in terms of bytecode instructions, lines of code (LOC), or executable size (KB or MB)), (3) the size of the test suite for the SUT (unfortunately, in most cases the test suite details were not available (N/A)), and (4) the references to the papers that report the case study.

Table~\ref{table:benchmarks_datadriven} presents case studies that were applied to UML models~\cite{di2017augmenting}, block models~\cite{pham2016model} and grammars~\cite{AFL:industrialcases}.

Concerning the size and typology of SUT, the case studies reported in Table~\ref{table:benchmarks_datadriven} varies a lot.
For instance, experimentation on block models has been applied to different types of desktop applications such as VLC, Adobe Reader, Real Player, and Windows Media Player~\cite{pham2016model} which sizes goes from 60 KB to 2.32 MB. 

Experimentation on grammars has been widely studied, but for sake of simplicity we only reported the most important case studies for AFL tool~\cite{AFL:industrialcases}. Particularly, AFL has been applied to a very wide range of software typology, for example it has been applied on programming languages such as PHP, web browsers such as Firefox, libraries such as OpenSSL, and even application server such as MySQL Server. 

Regarding case studies applied to space context software, we highlight SES-DAQ, a DAQ system developed by SES that processes bytestreams of transmitted satellite data. The version of SES-DAQ used for the experimentation performed by Di Nardo et al.~\cite{di2017augmenting} had a size of 32\,469 bytecode instructions.

% !TEX root =  ../MutationTestingSurvey.tex


\setlength\LTleft{0pt}
\setlength\LTright{0pt}
\small 
\begin{longtable}{@{\extracolsep{\fill}}|p{4cm}|p{3.5cm}|p{3cm}|p{1.8cm}|@{}}
\caption{\normalsize Summary of Data-Driven Mutation Testing Benchmarks.}
\label{table:benchmarks_datadriven} \\

\hline

\textbf{Case Study (SUT)}	&	\textbf{Size of SUT}	&	\textbf{\begin{tabular}[c]{@{}l@{}}Size of Test Suite\\(test cases)\end{tabular}}	 & \textbf{Reference}	\\
\hline

SES-DAQ & 32\,469 bytecode instructions & 32 & \cite{di2017augmenting,di2015evolutionary,di2015generating} \\

VLC 2.0.7 & 184 KB & N/A & \cite{pham2016model} \\
Libpng 1.5.4 & 176 KB & N/A & \cite{pham2016model} \\
XnView 1.98 & 4.46 MB & N/A & \cite{pham2016model} \\
Adobe Reader 9.2& 2.32 MB & N/A & \cite{pham2016model} \\
Windows Media Player 9.0 & 1.22 MB & N/A & \cite{pham2016model} \\
Real Player SP 1.0 & 60 KB & N/A & \cite{pham2016model} \\
MIDI Player 0.35& 336 KB & N/A & \cite{pham2016model} \\
Orbital Viewer 1.04 & 538 KB & N/A & \cite{pham2016model} \\

PHP 5.4.36 & 735 KLOC & N/A & \cite{AFL:industrialcases}\\
Firefox 32.0.1 & 62 MB & N/A & \cite{AFL:industrialcases}\\
OpenSSL 1.0.2 & 264 KLOC & N/A & \cite{AFL:industrialcases}\\
PuTTY 0.54 & 68 KLOC & N/A & \cite{AFL:industrialcases}\\
MySQL Server 5.7 & 373 MB & N/A & \cite{AFL:industrialcases}\\
indent 2.2.11 & 24 KLOC & N/A & \cite{AFL:industrialcases}\\

HotelRS & 1.5 KLOC & 33  & \cite{Appelt:SQLI:ISSTA:2014}\\
Sugar-CRM & 352 KLOC & 33 & \cite{Appelt:SQLI:ISSTA:2014}\\

JavaScript Interpreter (IE7) & 113\,562 machine instructions& 2\,800 & \cite{Godefroid:GrammarBasedFuzzying:2008}\\

\textbf{Windows NT utilities}
\begin{minipage}[t]{2.5cm}
attrib\\
chkdsk\\
comp\\
expand\\
fc\\
find\\
help\\
label\\
replace\\
\end{minipage}
 & 
 \begin{minipage}[t]{2.5cm}
 \hfill\\
21 KB\\
25 KB\\
25 KB\\
65 KB\\
24 KB\\
17 KB\\
12 KB\\
17 KB\\
21 KB\\
\end{minipage}
 &  \begin{minipage}[t]{2.5cm}
 \hfill\\
 N/A
\end{minipage} &  \begin{minipage}[t]{2.5cm}
 \hfill\\
 \cite{ghosh1998testing}
\end{minipage}\\
 

\hline                                                           
\end{longtable}


\normalsize




\clearpage

\section{Data-Driven Mutation Testing Benchmarks}
\label{section:industry:data}

In this section we provide a survey of the benchmarks considered both in empirical evaluations described in research work and in industrial cases for data-driven mutation testing.

Table~\ref{table:benchmarks_datadriven} provides the list of the case study systems considered in the literature and industry. Similar to the previous section, for each case study we report (1) the case study (i.e., the software under test - SUT), (2) the size of the SUT (the size can be expressed in terms of bytecode instructions, lines of code (LOC), or executable size (KB or MB)), (3) the size of the test suite for the SUT (unfortunately, in most cases the test suite details were not available (N/A)), and (4) the references to the papers that report the case study.

Table~\ref{table:benchmarks_datadriven} presents case studies that were applied to UML models~\cite{di2017augmenting}, block models~\cite{pham2016model} and grammars~\cite{AFL:industrialcases}.

Concerning the size and typology of SUT, the case studies reported in Table~\ref{table:benchmarks_datadriven} varies a lot.
For instance, experimentation on block models has been applied to different types of desktop applications such as VLC, Adobe Reader, Real Player, and Windows Media Player~\cite{pham2016model} which sizes goes from 60 KB to 2.32 MB. 

Experimentation on grammars has been widely studied, but for sake of simplicity we only reported the most important case studies for AFL tool~\cite{AFL:industrialcases}. Particularly, AFL has been applied to a very wide range of software typology, for example it has been applied on programming languages such as PHP, web browsers such as Firefox, libraries such as OpenSSL, and even application server such as MySQL Server. 

Regarding case studies applied to space context software, we highlight SES-DAQ, a DAQ system developed by SES that processes bytestreams of transmitted satellite data. The version of SES-DAQ used for the experimentation performed by Di Nardo et al.~\cite{di2017augmenting} had a size of 32\,469 bytecode instructions.

% !TEX root =  ../MutationTestingSurvey.tex


\setlength\LTleft{0pt}
\setlength\LTright{0pt}
\small 
\begin{longtable}{@{\extracolsep{\fill}}|p{4cm}|p{3.5cm}|p{3cm}|p{1.8cm}|@{}}
\caption{\normalsize Summary of Data-Driven Mutation Testing Benchmarks.}
\label{table:benchmarks_datadriven} \\

\hline

\textbf{Case Study (SUT)}	&	\textbf{Size of SUT}	&	\textbf{\begin{tabular}[c]{@{}l@{}}Size of Test Suite\\(test cases)\end{tabular}}	 & \textbf{Reference}	\\
\hline

SES-DAQ & 32\,469 bytecode instructions & 32 & \cite{di2017augmenting,di2015evolutionary,di2015generating} \\

VLC 2.0.7 & 184 KB & N/A & \cite{pham2016model} \\
Libpng 1.5.4 & 176 KB & N/A & \cite{pham2016model} \\
XnView 1.98 & 4.46 MB & N/A & \cite{pham2016model} \\
Adobe Reader 9.2& 2.32 MB & N/A & \cite{pham2016model} \\
Windows Media Player 9.0 & 1.22 MB & N/A & \cite{pham2016model} \\
Real Player SP 1.0 & 60 KB & N/A & \cite{pham2016model} \\
MIDI Player 0.35& 336 KB & N/A & \cite{pham2016model} \\
Orbital Viewer 1.04 & 538 KB & N/A & \cite{pham2016model} \\

PHP 5.4.36 & 735 KLOC & N/A & \cite{AFL:industrialcases}\\
Firefox 32.0.1 & 62 MB & N/A & \cite{AFL:industrialcases}\\
OpenSSL 1.0.2 & 264 KLOC & N/A & \cite{AFL:industrialcases}\\
PuTTY 0.54 & 68 KLOC & N/A & \cite{AFL:industrialcases}\\
MySQL Server 5.7 & 373 MB & N/A & \cite{AFL:industrialcases}\\
indent 2.2.11 & 24 KLOC & N/A & \cite{AFL:industrialcases}\\

HotelRS & 1.5 KLOC & 33  & \cite{Appelt:SQLI:ISSTA:2014}\\
Sugar-CRM & 352 KLOC & 33 & \cite{Appelt:SQLI:ISSTA:2014}\\

JavaScript Interpreter (IE7) & 113\,562 machine instructions& 2\,800 & \cite{Godefroid:GrammarBasedFuzzying:2008}\\

\textbf{Windows NT utilities}
\begin{minipage}[t]{2.5cm}
attrib\\
chkdsk\\
comp\\
expand\\
fc\\
find\\
help\\
label\\
replace\\
\end{minipage}
 & 
 \begin{minipage}[t]{2.5cm}
 \hfill\\
21 KB\\
25 KB\\
25 KB\\
65 KB\\
24 KB\\
17 KB\\
12 KB\\
17 KB\\
21 KB\\
\end{minipage}
 &  \begin{minipage}[t]{2.5cm}
 \hfill\\
 N/A
\end{minipage} &  \begin{minipage}[t]{2.5cm}
 \hfill\\
 \cite{ghosh1998testing}
\end{minipage}\\
 

\hline                                                           
\end{longtable}


\normalsize




\newpage

\bibliographystyle{IEEEtran}
\bibliography{bibliography}


\end{document}  