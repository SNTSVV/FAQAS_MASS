% !TEX root = MutationTestingSurvey.tex

\subsection{Equivalent Mutants}
\label{sec:opt:equivalent}

Equivalent mutants are mutants that behave as the original program, they are semantically equivalent to the original version despite being syntactically different. The process of identifying equivalent mutants has been defined as an undecidable problem \cite{madeyski2013overcoming}. However, research has cope with heuristics to tackle the problem of equivalent mutants.

According to the literature \cite{madeyski2013overcoming}, the equivalent mutant heuristics can be classified in two different groups: (1) detecting equivalent mutant techniques, and (2) reducing equivalent mutant techniques.

\subsubsection{Detecting Equivalent Mutant Techniques}

The first group aims to detect and discard the equivalent mutants during the mutation process. In this direction, Papadakis et al. \cite{papadakis2015trivial, kintis2017detecting,papadakis2019mutation} has developed the Trivial Compiler Optimisation technique, the approach relies on the idea that compiler optimisations transforms mutants to the optimised version, thus when the original program can be transformed by an optimisation to one of its mutants, then the mutant is defined equivalent.
Recent studies \cite{papadakis2015trivial} show that the Trivial Compiler Optimisation is able to detect approximately up to 30\% of the equivalent mutants.

A different approach for detecting equivalent mutants is to formulate the problem as a constraint satisfaction problem. The main idea is to analyze the path condition of a mutant, in which the mutant is defined equivalent if and only if the input constraint is unsatisfiable. Offutt et al. \cite{offutt1996detecting,offutt1997automatically} carried out an experimental evaluation on 11 Fortran subject programs and detected 47\% of the existing equivalent mutants by applying this heuristic.
Similarly, Holling et al. \cite{holling2016nequivack,papadakis2012mutation} presented an approach for identifying non-equivalent mutants and improving the confidence of the mutation score. By using static analysis and symbolic execution they defined a six-steps procedure to determine which mutants are \textit{non-equivalent}. \textit{Non-equivalent} mutants are identified every time they find a counter-example input for which the outputs of a pair of functions (the original function and the mutant one) is different. In the case no counter-example is found, then the mutant is classified as \textit{unknown}. 
In like manner, Riener et al. \cite{riener2011test} proposed the Symbolic Bounded Model Checking (SymBMC) procedure for the automated generation of test cases from a set of mutants. The approach examines the original program and its mutants on the same input and seeks for executions resulting in different observable output for the program and mutants, if the procedure finds a failing execution, the input data is saved as an effective new test case. Every time a new test case is found, the mutant is defined as non-equivalent, in a similar way to Holling's approach \cite{holling2016nequivack}.

Other approaches consider the use of software clones (i.e., similar code fragments) to detect equivalent mutants \cite{kintis2013identifying}, in this work Kintis et al. propose that mirrored mutants (i.e., mutants that belong to the same software clone) present the same behavior with respect to each other. So, for a set of mirrored mutants, is enough to prove equivalence of one mutant, instead of trying to detect equivalence for the whole set.

With a different approach, Adamopoulos et al. \cite{adamopoulos2004overcome} introduced a co-evolutionary technique for detecting equivalent mutants. The technique defines a fitness function that sets a poor fitness value to an equivalent mutant. Through the fitness function equivalent mutants are removed during the co-evolutionary process, and only mutants that are hard to kill and test cases that are good at detecting mutants are kept for future iterations of the algorithm. On the other hand, Maldonado et al. \cite{maldonado2005bayesian} developed a Bayesian Learning-Based technique for helping tester to detect equivalent mutants using an inference algorithm.

% 	\item Margrave's change-impact analysis \cite{martin2007fault} (READ)
% 	\item Using Lesar model-checker for eliminating equivalent mutants \cite{du2008towards} (READ)

% \textbf{Avoiding equivalent mutant generation techniques}

% \begin{itemize}
% 	\item \textbf{Selective mutation} \cite{mresa1999efficiency}:
% 	Randomly selecting 10\%, 20\%, 30\%, 40\%, 50\% and 60\% of the mutants results in a fault loss of approximately 26\%, 16\%, 13\%, 10\%, 7\% and 6\% respectively \cite{papadakis2010empirical}.

% 	Rothermel et al. \cite{rothermel1996experimental} and then Andrews et al. \cite{andrews2005mutation} proposed a small set of operators that is a sufficiently accurate approximation of the results obtained by using all possible operators (e.g., Replace numerical constant, negate jump condition, replace arithmetic operator, omit method calls).

% 	Namin et al. \cite{siami2008sufficient} proposed a statistical analysis procedure for identifying a subset of operators that could predict mutation score (reduce mutants in a 93\%), the developed tool aims C programs.

% 	Delamaro et al. \cite{delamaro2014designing} designed new deletion operators, and found that they form a cost-effective alternative to other operators, i.e., they produce less equivalent mutants.


\subsubsection{Reducing Equivalent Mutant Techniques}

The second group of techniques aims to reduce the amount equivalent mutants produced during mutation process.
In this direction, Gr\"{u}n et al. \cite{grun2009impact} proposed that mutants that do not alter the control-flow with respect to the original program, should be avoided since they have a greater chance of being equivalent.
In the same research line, Schuler et al. \cite{schuler2009efficient} demonstrated that mutants that violate dynamic invariants are less likely to be equivalent and should be preferred over those that do not alter invariants with respect to the original program version.
A different perspective is to focus in code coverage of mutants, for example, Schuler et al. \cite{schuler2010covering,schuler2013covering} discovered that mutants that change code coverage with respect to the original program, has a likelihood of 68\%-70\% to be non-equivalent.

Program slicing has been presented as a possible mechanism for reducing equivalent mutants \cite{voas1997software, hierons1999using, harman2001relationship}: in particular, Harman et al. \cite{harman2001relationship} presented a technique for reducing equivalent mutation generation using program dependence analysis. The idea is to avoid mutants that fail to propagate corrupted data into the inspection set at the probe point, a mutant that fails to propagate specific data means that no semantic change is being introduced on the software behavior. To carry on this analysis, the authors used a method called \textit{JR-dependence}, the method allows to relate variable and node pairs rather than simply considering nodes. With \textit{JR-dependence} is possible to know the set of variables that can and cannot be used to kill a mutant, which is beneficial for mutation testing. 
Consequently, Offutt et al. \cite{offutt2006class} used the guidelines by Harman et al. \cite{harman2001relationship} to eliminate equivalent mutants by applying the optimizations directly in the implementation of object oriented operators developed on the MuJava mutation testing tool. 

On the other hand, Kintis et al. \cite{kintis2014using,kintis2015medic} discovered that equivalent mutants have specific data-flow patterns which form data-flow anomalies, the main idea is that through static data-flow analysis is possible to reveal code locations that are sensitive to produce equivalent mutants together with the mutation operators being applied.

Research has proved that higher order mutation testing can be helpful in reducing equivalent mutants \cite{jia2009higher,kintis2010evaluating,offutt1992investigations,papadakis2010empirical}, since two or more mutations are applied simultaneously, the chances of producing equivalent mutants decreases consistently. For instance, Papadakis and Malevris \cite{papadakis2010empirical}, worked on a approach for higher order mutants for the C programming language that lead to a reduction of approximately 80-90\% of the generated equivalent mutants, with a fault detection ability loss only of 11-15\%. 
% For instance, Offutt demonstrated that the set of test data developed for first order mutants (FOMs) actually killed a higher percentage of mutants when applied to second order mutants (SOMs) \cite{offutt1992investigations}. 
% Jia and Harman identified six different types of HOMs \cite{jia2009higher} and presented a categorization of HOMs. They introduced the concept of subsuming and strongly subsuming HOMs.
% Polo et al. \cite{polo2009decreasing} studied three strategies to combine FOMs and generate mutants, and found that they can achieve significant cost reductions without losing any effectiveness (they reduced the number of mutants in a approximately 50\%, without much decrease in the quality of the test suite).
%Instead, Kintis et al. \cite{kintis2010evaluating} developed a solution for the Java language, they state that SOMs achieve higher collateral coverage for strong mutation as compared with third or higher order mutants. With their approach they obtained a mutant reduction of between 65-87\% and a loss of test effectiveness from 1.75-4.2\%.
% Mateo et al. \cite{mateo2012validating,madeyski2013overcoming} found that second order mutants (SOM) are significantly more efficient that first order mutants (FOM).



\begin{table*}[ht]
\centering
\scriptsize
\begin{tabular}{lllllllp{4cm}}
\toprule
Author(s)          & Year   & Language & \begin{tabular}[c]{@{}l@{}}Largest\\Subject\end{tabular} & \begin{tabular}[c]{@{}l@{}}\#Eq. \\ Mutants\end{tabular} & \begin{tabular}[c]{@{}l@{}}Available \\ Tool\end{tabular} & Category                                                 & Findings                                                                                      \\
\midrule
Baldwin \& Sayward \cite{baldwin1979heuristics} & 1979   &          &                                                           &                                                          &                                                           & Detect                                                   & Compiler optimization can be used to detect equivalent mutants                                \\
Acree  \cite{acree1980mutation}       & 1980   & Fortran  &                                                           & 25                                                       &                                                           & Detect                                                   & Human make mistakes when they identify equivalent mutants                                     \\
Offutt \& Craft \cite{offutt1994using}   & 1994   & Fortran  & 52                                                        & 255                                                      &                                                           & Detect                                                   & Compiler optimisation can detect on average 45\% of equivalent mutants                        \\
Offutt \& Pan \cite{offutt1996detecting,offutt1997automatically}     & 1996-7 & Fortran  & 29                                                        & 695                                                      & Yes                                                       & Detect                                                   & Constraint-based testing can detect on average 47\% of equivalent mutants                     \\
Voas \& McGraw \cite{voas1997software}    & 1997   &          &                                                           &                                                          &                                                           & Detect                                                   & Slicing may be helpful in detecting equivalent mutants                                        \\
Hierons et al. \cite{hierons1999using}      & 1999   &          &                                                           &                                                          &                                                           & \begin{tabular}[c]{@{}l@{}}Detect/\\ Reduce\end{tabular} & Program slicing can be used to detect and assist the identification of equivalent mutants     \\
Harman et al.  \cite{harman2001relationship}    & 2001   &          &                                                           &                                                          &                                                           & \begin{tabular}[c]{@{}l@{}}Detect/\\ Reduce\end{tabular} & Dependence analysis can be used to detect and assist the identification of equivalent mutants \\
Adamopoulos et al  & 2004 \cite{adamopoulos2004overcome}  &          &                                                           &                                                          &                                                           & Reduce                                                   & Co-evolution can help in reducing the effects of equivalent mutants                           \\
Grun et al. \cite{grun2009impact}       & 2009   & Java     & 12,449                                                     & 8                                                        & Yes                                                       & Reduce                                                   & Coverage Impact can be used to classify killable mutants                                      \\
Schuler et al. \cite{schuler2009efficient}    & 2009   & Java     & 94,902                                                     & 10                                                       & Yes                                                       & Reduce                                                   & Invariants violations can be used to classify killable mutants                                \\
Schuler \& Zeller \cite{schuler2010covering,schuler2013covering} & 2010-2 & Java     & 94,902                                                     & 63                                                       & Yes                                                       & Reduce                                                   & Coverage impact can be used to classify killable mutants                                      \\
Nica \& Wotawa \cite{nica2012using}    & 2012   & Java     & 380                                                       & 1,424                                                     &                                                           & Detect                                                   & Constraint-based testing can detect equivalent mutants                                        \\
Kintis et al. \cite{kintis2012isolating,kintis2015employing}     & 2012-4 & Java     & 94,902                                                     & 89                                                       &                                                           & Reduce                                                   & Higher order mutants can be used to classify killable mutants                                 \\
Kintis \& Malevris \cite{kintis2014using} & 2014   & Java     & 25,909                                                     & 84                                                       &                                                           & Detect                                                   & Data-flow patterns can detect 69\% of the equivalent mutants introduced by the AOIS operator  \\
Papadakis et al. \cite{papadakis2014mitigating}    & 2014   & C        & 513                                                       & 5,589                                                     &                                                           & Reduce                                                   & Coverage impact can be used to classify killable mutants                                      \\
Papadakis et al. \cite{papadakis2015trivial}    & 2014   & C        & 362,769                                                    & 9,551                                                     & Yes                                                       & Detect                                                   & Compilers can be used to effectively automate the mutant equivalence detection               \\
\bottomrule
\end{tabular}
\end{table*}

