% !TEX root = MutationTestingSurvey.tex

\section{Responses to ESA comments provided on 03.04.2020}
\label{sec:ESA:comments:2}


\setlength\LTleft{0pt}
\setlength\LTright{0pt}
\tiny 
\begin{longtable}{|p{1.5cm}|p{12cm}|@{}}
\label{table:comments:responses} 
\textbf{Comment ID}&\textbf{Comment and Response (below)}\\
\\
\midrule
C6 \& C7
&
Have you seen numbers for this mutation score and threshold in the literature? Is this something to be checked during the use case evaluation?
\\
\cmidrule{2-2}
&
We have addressed the comments above.
\DONE{OScar: please check if the survey of Papadakis say something aboth teh threshold (C7)}
\DONE{I checked, unfortunately I did not find anything on the Papadakis' survey or google scholar.}
\\
\hline
C8
&
Elaborate a bit more on C8 (pros and cons of doing mutation at source code / IR/ Assembly/ Executable);
\\
\cmidrule{2-2}
&
\DONE{Oscar: you may refer to taht paper of Darko Marinov and Co. to say IR is not good}
We provide a discussion on the pros and cons of doing mutation at different levels in Section~\ref{sub:compileTime}.
\\
\hline
C31
&
What is this sufficient set of operators?
\\
\cmidrule{2-2}
&
\TODO{Oscar}
\\
\hline
C32
&
Can you please add the solution for this example? i.e. do we need two different test cases of isPalindrome to detect both mutants?
\\
\cmidrule{2-2}
&
\TODO{Oscar}
\\
\hline
C33
&
Even if the objectives are complementary, both of them should be pursued for a data mutation testing approach?
\\
\cmidrule{2-2}
&
We have addressed the comment above.
\\
\hline
C34
&
The sentence sounds weird... To automate?? Is this activity something that can be automated?
\\
\cmidrule{2-2}
&
We have addressed the comment above by clarifying our text.
\\
\hline
C35
&
Is it possible to add an example of equivalent and redundant mutants?
\\
\cmidrule{2-2}
&
We have added the requested examples.
\\
\hline
C36
&
\begin{minipage}{12cm}
Related to automation, in my opinion, what it is key is that the test assessment process (for both data and code mutation) is as much automated as possible.\\

Automated generation of test cases is a very nice to have. In an industrial environment, let's say that we could afford spending some time to manually augment the test suite.\\

You may consider this to prioritize tasks within this activity.
\end{minipage}
\\
\cmidrule{2-2}
&
We agree on the comment. No need to change the text in this deliverable.
\\
\hline
C37
&
Are we missing a chapter to address the Generation of Test Oracles?\\
\cmidrule{2-2}
&
We have added a section concerning generation of test oracles for code-driven mutation testing (Section~\ref{sec:oraclesGeneration:codeDriven}) and data-driven mutation testing 
(Section~\ref{sec:oracles:dataMutation}).
\\
\hline
C38
&
\begin{minipage}{12cm}
a. From these Case Studies, is there any that you would like to try out within FAQAS?\\

b. One thing that we may need for FAQAS framework is to have kind of a test suite allowing to test the tool, and also to test the tool when new versions will be produced. Would any of these case studies fulfill that?
\end{minipage}
\\
\cmidrule{2-2}
&
We have discussed these topics by voice.
\\
\hline
C39
&
Do you have any information on the kind of test suite? (e.g. is it unit testing, system testing, ...)
\\
\cmidrule{2-2}
&
\TODO{}
\\
\hline
C40
&
Are these case studies focused on Code-Mutation, Data-Mutation, or both?\\
\cmidrule{2-2}
&
We created two sections on Chapter~\ref{chapter:industry}, each of them dedicated to code-driven and data-driven mutation testing, respectively.
\\
\hline
C41
&
Is there any meaningful conclusion (positive or negative) from those industrial case studies?\\
\cmidrule{2-2}
&
We added a concluding paragraph on Section~\ref{section:industry:code}. 
\\
\hline
C42
&
\begin{minipage}{12cm}
Can we make a conclusion paragraph on this?\\

e.g. No tool based on mutation testing is known to be used within an industrial software development environment\\
e.g. Mutation testing is seen applied mainly within research environments\\
etc, etc
\end{minipage}
\\
\cmidrule{2-2}
&
We added a concluding paragraph on Section~\ref{section:industry:code}. 
\\
\hline
C43
&
Is there any of these trends that could be meaningful to explore?
\\
\cmidrule{2-2}
&
\TODO{}
\\
\hline
C44
&
\begin{minipage}{12cm}
	\begin{itemize}
		\item Is there any particular trend for Code-Based mutation testing? (e.g. research is on-going or vanishing, the way to apply it, the type of operators used, the tools supporting it, ...)
		\item Any particular trend for Data-Based mutation testing?
	\end{itemize}
\end{minipage}
\\
\cmidrule{2-2}
&
\TODO{}
\\
\hline
C45&
\begin{minipage}{12cm}
D1 is fulfilling well requirement R1-1 as in the SoW. There is only one exception, on the red sentence below:\\

[R1-1.c] The applications of mutation testing (e.g. code and data mutation, test-suite evaluation, test cases generation, test-data generation, \textcolor{red}{code quality improvement}, ...)\\

The evaluation of code quality improvement is to be looked at. Indeed, this would be a secondary objective of applying mutation testing on space systems, but we would like to understand if mutation testing could help improve the code quality or not.\\
\end{minipage}
\\
\cmidrule{2-2}
&
\begin{minipage}{12cm}
We added a paragraph on Chapter~\ref{chapter:trends} explaining that there are no works in literature about quality code improvement based on code-driven mutation testing.
\end{minipage}
\\

\bottomrule                                                             
\end{longtable}
\normalsize

\clearpage