% !TEX root = MutationTestingSurvey.tex

\section{Data-driven Mutation Operators}
\label{sec:data_operators}

As mentioned in Section~\ref{sec:dataProcess}, the software engineering literature does not include any study on data-driven mutation testing but only testing approaches based on the injection of data faults.
For this reason, 
in this section we provide an overview of the fault injection techniques that have originally been developed to support software testing and can be used in the context of data-driven mutation testing. 
More precisely, we focus on the techniques for the modification of data that are presented in software testing research papers.
We refer to these techniques as \emph{data-driven mutation operators}.

Table~\ref{table:dataOperators} provides an overview of the data-driven mutation operators that can be applied to the case of space software and embedded systems. 
We selected the set of data-driven mutation operators in Table~\ref{table:dataOperators} based on the different type of faults commonly affecting space and embedded systems. 
Each operator aims to create data faults that might be observed in real systems either because of hardware errors or software errors.

In Table~\ref{table:dataOperators}, column \emph{Name} provides the name of the specific operator described, column Model-based indicates if the operator requires a model of the structure of the data to be mutated, 
 column \emph{Type of fault} indicates the type of faults each operator aims to simulate,
 column \emph{Definition} provides a brief description of the mutation operator, column \emph{Target} indicates the type of data targeted by the approach,
 column \emph{Reference} provides a reference to a research paper or tool describing the operator more in detail.
 
%Fabrizio: "First" and "Then" read like a story, which is not good in a Tech Report
Concerning the type of faults considered, we focus both on hardware and software errors.
For hardware errors, we include the categories \emph{CPU Faults}, \emph{Memory Faults}, and \emph{Signal Faults}. 
For software errors, we include the category of \emph{Data Processing Faults}.
Category \emph{Communication Faults} simulates problems that can be caused either by software or hardware errors.

In the following, we provide a brief overview of the operators generating each type of fault:
\begin{itemize}
	%Note: always put a comma at the end of a list before "or" "and"
	\item Category \emph{CPU Faults} includes operators that perform mutations in the contents of an individual bit, byte, or word in a CPU register. The mutations in this category can target saved, floating-point, program-counter, global and stack-pointer register locations. 
	\item Category \emph{Memory Faults} includes operators that perform mutations in the contents of an individual bit, byte, or word in a memory register. The mutations in this category can target stack, heap, global-data and user-defined memory locations.
	\item Category \emph{Communication Faults} includes operators that simulate packet corruption faults. The mutations in this category can target channels between components, single messages, and the addresses of the messages to be exchanged.
	\item Category \emph{Data Processing Faults} includes techniques that mutate the data being processed by the system. Some of these techniques are white-box, they process the the source code of the application to generate inputs more efficiently. Other techniques require a model of the data to be mutated. The mutations in this category can target both the input and the output parameters of the interface of a software component.
\end{itemize}

% !TEX root =  ../MutationTestingSurvey.tex

\newcommand{\FTAPE}{\cite{tsai1999stress}}
\newcommand{\FIAT}{\cite{barton1990fault}}
\newcommand{\GOOFI}{\cite{aidemark2001goofi}}
\newcommand{\DOCTOR}{\cite{han1995doctor}}
\newcommand{\ORCHESTRA}{\cite{dawson1996testing}}
\newcommand{\Fuzz}{\cite{miller1995fuzz}}
\newcommand{\Ballista}{\cite{koopman2000exception}}
\newcommand{\RIDDLE}{\cite{ghosh1998testing}}
\newcommand{\AFL}{\cite{gutmann2016fuzzing}}
\newcommand{\SAGE}{\cite{godefroid2012sage}}
\newcommand{\pFuzzer}{\cite{mathis2019parser}} 
\newcommand{\MoWF}{\cite{pham2016model}}
\newcommand{\DiNardoICST}{\cite{di2015generating}}
\newcommand{\DiNardoASE}{\cite{di2015evolutionary}}
\newcommand{\Matinnejad}{\cite{Matinnejad19}}

\begin{itemize}
	\item FTAPE tool: \cite{tsai1999stress}
	\item FIAT tool: \cite{barton1990fault}
	\item GOOFI tool: \cite{aidemark2001goofi}
	\item DOCTOR tool: \cite{han1995doctor}
	\item ORCHESTRA tool: \cite{dawson1996testing}
	\item Fuzz tool:\cite{miller1995fuzz}
	\item Ballista tool: \cite{koopman2000exception}
	\item RIDDLE tool: \cite{ghosh1998testing}
	\item American Fuzzy Lop tool: \cite{gutmann2016fuzzing}
	\item SAGE tool: \cite{godefroid2012sage}
	\item pFuzzer tool: \cite{mathis2019parser}
	\item MoWF tool: \cite{pham2016model}
	\item Di Nardo et al. Model-based Mutation tool: \cite{di2015generating}
	\item Di Nardo et al. Search-based Mutation tool: \cite{di2015evolutionary}
	\item Matinnejad et al. Signal Mutation tool: \cite{Matinnejad19}
\end{itemize}

\tiny
\setlength\LTleft{0pt}
\setlength\LTright{0pt}
\begin{longtable}{@{\extracolsep{\fill}}|p{1.5cm}|p{2cm}|p{5cm}|p{3cm}|p{1cm}|@{}}
\toprule
	\textbf{Name}	&	\textbf{Type of Fault}	&	\textbf{Definition}	&	\textbf{Target}	&	\textbf{Reference} \\
	\midrule
stuck-at-0 & CPU Faults; Memory Faults & This operator corrupts data by replacing a bit/byte/word with zero. & CPU registers; Memory registers & \FTAPE, \FIAT, \GOOFI \\
stuck-at-1 & CPU Faults; Memory Faults & This operator corrupts data by replacing a bit/byte/word with one. & CPU registers; Memory registers & \FTAPE, \FIAT, \GOOFI \\
bit flips & CPU Faults; Memory Faults; Data Processing Faults & This operator mutates data by inverting the value of each bit (i.e., replacing 0 with 1 and 1 with 0). & CPU registers: saved registers, floating point registers, the program counter, global pointer, stack pointer; Local memory; Input or output parameters of a software interface. & \FTAPE, \FIAT, \GOOFI \\
disk driver codes & Communication Faults & This operator performs data mutation by modifying the error flags of a disk driver. & I/O disk driver codes. & \FTAPE \\
set & CPU Faults; Communication Faults; Memory Faults & This operator sets the value of a bit/byte by replacing the current value with one or a user-defined bitmap. & Memory: stack, heap, global data, user code, user-defined memory location; Registers: data, stack, address, program counter, status register & \FTAPE, \FIAT, \GOOFI \\
clear & CPU Faults; Communication Faults; Memory Faults & This operator clears the value of a bit/byte by replacing the current value "v" with zero. & Memory: stack, heap, global data, user code, kernel code, user-defined memory location; Registers: data, stack, address, program counter, status register & \FTAPE, \FIAT, \GOOFI \\
toggle & CPU Faults; Communication Faults; Memory Faults & This operator toggles (i.e., sets a bit to its complement state) a bit/byte. & Memory: stack, heap, global data, user code, kernel code, user-defined memory location; Registers: data, stack, address, program counter, status register & \FTAPE, \FIAT, \GOOFI \\
lose message & Communication Faults & This operator causes a message to be lost in between two communicating components. Messages can be lost intermittently, with a probability distribution specified by the users, or alternatively, every message can be lost during a certain period. & Faulty link, Faulty direction, Single message & \DOCTOR, \ORCHESTRA \\
duplicate message & Communication Faults & This operator causes a message to be duplicated in between two communicating components. & Faulty link, Faulty direction, Single message & \DOCTOR, \ORCHESTRA \\
alter message header & Communication Faults & This operator alters a message. The change is performed in a similar manner as for memory faults, i.e., by performing bit flips. The mutation is performed in the message header. & Faulty link, Faulty direction, Single message & \DOCTOR, \ORCHESTRA \\
alter message body & Communication Faults & This operator alters a message. The change is performed in a similar manner as for memory faults, i.e., by performing bit flips. The mutation is performed in the message body. & Faulty link, Faulty direction, Single message & \DOCTOR, \ORCHESTRA \\
delay message & Communication Faults & This operator causes a message to be delayed in between two communicating components. The delay time can either be constant or follow a probability distribution. & Faulty link, Faulty direction, Single message & \DOCTOR, \ORCHESTRA \\
fuzzing & Data Processing Faults & Fuzzing techniques replace values with values (i.e., fuzz values) that are different and, usually, randomly generated. & Structured input data (e.g., files, streams) & \Fuzz, \Ballista \\
grammar-based fuzzing & Data Processing Faults & This fuzzing approach relies on a grammar that describes the format of the data to be altered; new values are generated according to the grammar. The use of the grammar decreases the chances of generating invalid inputs. & Structured input data (e.g., files, streams) & \RIDDLE \\
evolutionary fuzzing & Data Processing Faults & This fuzzing approach relies on evolutionary algorithms (e.g., genetic algorithms) to generate fuzz values. It relies on a fitness function to maximize structural coverage (e.g., branches executed during testing). The fitness is implemented by instrumenting the code of the application to collect code coverage. & Structured input data (e.g., files, streams) & \AFL \\
whitebox fuzzing & Data Processing Faults & This fuzzing approach relies on symbolic execution techniques to cover corners cases on the different execution paths. & Structured input data (e.g., files, streams) & \SAGE \\
parser-directed fuzzing & Data Processing Faults & Fuzzing techniques replace a correct value with a randomly generated one. This fuzzing operator is applied to programs that integrate a parser component to extract information from the provided inputs. It is implemented through a test generator technique that aims to produce valid inputs for the parser, the inputs are produced randomly and then checked using constraint solving techniques. & Structured input data (e.g., files, streams) & \pFuzzer \\
model-based whitebox fuzzing & Data Processing Faults & Fuzzing techniques replace a correct value with a randomly generated one, in this case, the replacement is done using a model-based whitebox fuzzing technique, the technique prunes from the search space those paths that are exercised by invalid, malformed inputs. The model should specify the format of the data chunks and integrity constraints. & Structured input data (e.g., files, streams) & \MoWF \\
data-type based injection & Data Processing Faults & This operator replaces a value with an invalid one, selected on the basis of the type of the parameter being corrupted. & Structured input data (e.g., files, streams) & \Fuzz, \Ballista, \RIDDLE \\
model-based mutation & Data Processing Faults & This approach alters existing inputs by relying on a data model that capture the structure of the data and the constraints among data fields. A predefined set of operators indicating how to alter data based on its structure are provided. & Structured input data (e.g., files, streams) & \DiNardoICST \\
search-based data mutation & Data Processing Faults & This approach alters existing inputs by relying on a data model that captures the structure of the data and the constraints between data fields. A search-based, evolutionary algorithm is used to maximize a number of mutation objectives including the number of constraints being violated and code coverage. & Structured input data (e.g., files, streams) & \DiNardoASE \\
signal mutation & Signal faults & This approach alters signal by either shifting the signal based on a randomly selected value or shifting the signal and increasing the number of signal pieces (i.e., segments).  & Input signals & \Matinnejad \\

	\bottomrule                                                             
\caption{Data-driven Mutation Operators.}
\label{table:dataOperators}
\end{longtable}
\normalsize

Among all the data mutation operators reported in Table~\ref{table:dataOperators}, the ones targeting \emph{Data Processing Faults} are more powerful since they concern the modification of complex data structures. They are briefly overviewed in the following.

\emph{AFL} introduces instrumentation-guided fuzzers, it uses a form of edge coverage to pick up subtle changes to program control-flow~\cite{gutmann2016fuzzing}. This way, it ensures to generate inputs that exercise all the different code paths.
Although AFL tackles the redundant mutants problem (i.e., does not generate different test inputs for the same execution path), it may produce equivalent mutants.
%Fabrizio: you had this definition of equivalent: "(i.e., correct fuzzed data, but semantically meaningless)." it is difficult to understand

\emph{Parser-Directed Fuzzing} aims at producing valid inputs for input parsers~\cite{mathis2019parser}. The challenge is to cover all the lexical and syntactical features of a certain language. The approach systematically produces inputs for the parser and tracks all the comparisons made; after every rejection, it satisfies the comparisons leading to rejections, effectively covering the input space. 
Evaluated on five subjects, from CSV files to JavaScript, the \emph{pFuzzer} prototype covers more tokens than both lexical-based (AFL) and constraint-based approaches (KLEE).
%Fabrizio: I cannot understand the following sentences
%Although it tackles the problem of redundant mutants by well covering the space of possible inputs, it does not tackle the redundant mutant problem.
%Cannot handle semantic restrictions imposed by certain nontrivial input languages. The approach has difficulties to handle complex sequences recursion.

\emph{RIDDLE}~\cite{ghosh1998testing} adopts a grammar to describe the format of inputs; random and boundary values are generated for tokens representing input parameters.
One of its drawbacks is the cost of setting up a grammar to describe the space of program inputs.
%, also, it requires more knowledge about the target than purely random ones.

\emph{SAGE} adopts symbolic execution to systematically generate malformed inputs~\cite{godefroid2012sage}. SAGE performs fuzzing on file- and packet-parsing applications. 
The program is first executed with concrete inputs; in order to identify a set of constraints on inputs, then, one of the constraints in the set is negated, and new malformed inputs are generated to satisfy the new set of constraints. 
The main benefit of SAGE is that it forces the program to execute corner cases not covered by the initial inputs; for example, 
%Fabrizio: you wrote the following, which I could not understand please check if my sentence is correct
%(e.g., reached one-third of all bugs found by fuzz testing in Microsoft projects \cite{bounimova2013billions}). 
one-third of all the bugs found by means of fuzz testing are detected thanks to SAGE \cite{bounimova2013billions}.
Unfortunately the main limitation of SAGE and symbolic execution-based fuzzers is limited scalability, due to the high execution time required by symbolic execution.

%Fabrizio: not sure what the following is about, ignoring
%\emph{Testing of Fault-Tolerant and Real-Time Distributed Systems via Protocol Fault Injection \cite{dawson1996testing}}: The paper introduces a portable fault injection environment for testing implementations of distributed protocols.

%Fabrizio: we miss the pro/cons from the following
\emph{Model-Based Whitebox Fuzzing} targets program binaries that process structured inputs~\cite{pham2016model}. It efficiently generates valid inputs that exercise critical target locations. This is done through a directed path exploration technique that prunes from the search space those paths that are exercised by invalid, malformed inputs.

%\emph{Generating complex and faulty test data through model-based mutation analysis (Research paper) \cite{di2015generating}}: 
\emph{Model-based data mutation} concerns the automated generation of invalid input data through the mutation of existing data based on a predefined set of mutation operators~\cite{di2015generating}.
The technique relies upon six generic mutation operators to automatically generate faulty data. The technique receives two inputs: field data and a data model, i.e., a UML class diagram annotated with stereotypes and OCL constraints. Stereotypes are used to tailor the behaviour of the generic mutation operators to the fault model that is assumed for the system under test and the environment in which it is deployed. 
OCL constraints are used to capture the characteristics of valid inputs and to indicate which error messages should be produced by the system in the presence of violations of input constraints.
The use of stereotypes and OCL constraints enable the approach to partially addresses the problem of equivalent mutants. Indeed stereotypes indicate which data fields to mutate, while OCL constraints indicates which violations should be reported by the SUT.

\emph{Search-based data mutation} relies on an evolutionary algorithm to perform model-based data mutation and optimize multiple objectives:  
cover all the classes of the data-model, cover all the possible faults of the fault model, cover all the clauses of the input/output constraints,
maximise code coverage~\cite{di2015evolutionary}.
The coverage of each objective is encoded by means of boolean arrays; this information is used to select a minimal set of inputs that maximize the coverage of the different objectives.
At every iteration, the evolutionary algorithm keeps only inputs that contribute to increase the coverage of at least one of the objectives (e.g., inputs that cover one instruction not covered by other inputs).
The paper indirectly addresses the problem of redundant mutants thanks to the inclusion of a coverage objective in the fitness; indeed only mutants that cover different portions of the code are kept, by definition these mutants are distinct because they trigger distinct software behaviours.

