% !TEX root = MutationTestingSurvey.tex

\subsection{Data-driven Mutation Operators}
\label{sec:data_operators}

As mentioned in Section~\ref{sec:dataProcess}, the software engineering literature does not include any study on data-driven mutation testing but only testing approaches based on the injection of data faults.
For this reason, 
in this section, we provide an overview of the fault injection techniques that have originally been developed to support software testing and can be used in the context of data-driven mutation testing. 
\CHANGEDTWO{More precisely, we focus on the techniques for the modification of data \CHANGEDTWO{(hereafter, \INDEX{data mutation techniques})} that are presented in software testing research papers.}


\CHANGEDTWO{Table~\ref{table:dataOperators} provides an overview of groups of data mutation techniques that can be applied to space software and embedded systems. 
We selected the set of data-driven mutation techniques in Table~\ref{table:dataOperators} based on the different types of faults commonly affecting space and embedded systems.} 
Each technique aims to create data faults that might be observed in real systems either because of hardware errors or software errors.
The \INDEX{data-driven mutation operators}, i.e., the specific functions implemented by each technique to alter the type of data they target, are described in separate tables, in the following paragraphs. 
%Since, for each family of technique, multiple implementations of data-driven mutation operators are available

In Table~\ref{table:dataOperators}, column \emph{Name} provides the name of the specific technique described,  
 column \emph{Type of fault} indicates the type of faults each operator aims to simulate,
 column \emph{Data model} indicates if the technique requires a model of the structure of the data to be mutated,
 column \emph{Definition} provides a brief description of the technique, column \emph{Target} indicates the type of data targeted,
 column \emph{Reference} provides a reference to a research paper or tool describing the technique more in detail.
 
%Fabrizio: "First" and "Then" read like a story, which is not good in a Tech Report
Concerning the type of faults considered, we focus both on hardware and software errors.
Hardware errors are captured by the categories \INDEX{CPU Faults}, \INDEX{Memory Faults}, and \INDEX{Signal Faults}. 
Software errors are captured by the category \INDEX{Data Processing Faults}.
Category \INDEX{Communication Faults} simulates problems that can be caused either by software or hardware errors.



\newcommand{\FTAPE}{\cite{tsai1999stress}}
\newcommand{\FIAT}{\cite{barton1990fault}}
\newcommand{\GOOFI}{\cite{aidemark2001goofi}}
\newcommand{\DOCTOR}{\cite{han1995doctor}}
\newcommand{\ORCHESTRA}{\cite{dawson1996testing}}
\newcommand{\Fuzz}{\cite{miller1995fuzz}}
\newcommand{\Ballista}{\cite{koopman2000exception}}
\newcommand{\RIDDLE}{\cite{ghosh1998testing}}
\newcommand{\Superion}{\cite{Wang:GrammarAwareFuzzying:ICSE:2019}}
\newcommand{\AFL}{\cite{gutmann2016fuzzing}}
\newcommand{\SAGE}{\cite{godefroid2012sage}}
\newcommand{\pFuzzer}{\cite{mathis2019parser}} 
\newcommand{\MoWF}{\cite{pham2016model}}
\newcommand{\DiNardoICST}{\cite{di2015generating}}
\newcommand{\DiNardoASE}{\cite{di2015evolutionary}}
\newcommand{\Matinnejad}{\cite{Matinnejad19}}
\newcommand{\MongoDB}{\cite{Guo:MongoDBFuzzer:CACM:2017}}
\newcommand{\SOLMI}{\cite{Jan:ISSTA:2016}}
\newcommand{\MUSQL}{\cite{Appelt:SQLI:ISSTA:2014}}

Table~\ref{table:dataMutation:references} provides the names of the tools and approaches referenced in Table~\ref{table:dataOperators}, along with a URL for the download of the tool, if available.

% !TEX root =  ../MutationTestingSurvey.tex

\begin{table}[ht]
\tiny
\caption{Description of approaches and tools appearing in Table~\ref{table:dataOperators}.}
\begin{center}
\begin{tabular}{|p{1cm}|p{4cm}|p{8cm}|}
\hline
\textbf{Reference} & \textbf{Approach and Tool name} & \textbf{Available} \\
\hline

\cite{di2015evolutionary}	& Di Nardo et al. Search-based Mutation & No \\
\cite{di2015generating} & Di Nardo et al. Model-based Mutation & No \\
\cite{Matinnejad19} & Matinnejad et al. Signal Mutation & No \\

\cite{tsai1999stress} & FTAPE & No\\
\cite{barton1990fault} & FIAT & No \\
\cite{han1995doctor} & DOCTOR & No \\
\cite{dawson1996testing} & ORCHESTRA & No \\

\cite{miller1995fuzz} & Fuzz & Yes, \url{http://pages.cs.wisc.edu/~bart/fuzz/fuzz.html} \\


\cite{koopman2000exception}	&  Ballista & Yes, \url{http://users.ece.cmu.edu/~koopman/ballista/} \\

%\cite{ghosh1998testing}	& RIDDLE & No \\
\cite{Wang:GrammarAwareFuzzying:ICSE:2019} & Superion & \url{https://github.com/zhunki/Superion} \\
\MongoDB	& MongoDB Fuzzer & No \\
\SOLMI	& SOLMI &No \\
\MUSQL	& \emph{$\mu$4SQLi} &No\\
\cite{gutmann2016fuzzing} & American Fuzzy Lop & Yes, \url{https://github.com/google/AFL}\\
\cite{godefroid2012sage} & SAGE & \begin{tabular}[c]{@{}l@{}}Yes, now under Microsoft Security Risk Detection service\\\url{https://www.microsoft.com/en-us/security-risk-detection/}\end{tabular} \\
%\cite{mathis2019parser}	& pFuzzer & Yes, \url{https://github.com/uds-se/pFuzzer}\\
\cite{pham2016model} &	MoWF & No \\
\cite{spike}&SPIKE&\url{https://github.com/guilhermeferreira/spikepp}\\
\cite{PeachFuzzer}&PEACH&\url{https://www.peach.tech}\\
\cite{BooFuzz}&BooFuzz&\url{https://github.com/jtpereyda/boofuzz}\\
\hline
\end{tabular}
\end{center}
\label{table:dataMutation:references}
\end{table}%



% !TEX root =  ../MutationTestingSurvey.tex

\newcommand{\FTAPE}{\cite{tsai1999stress}}
\newcommand{\FIAT}{\cite{barton1990fault}}
\newcommand{\GOOFI}{\cite{aidemark2001goofi}}
\newcommand{\DOCTOR}{\cite{han1995doctor}}
\newcommand{\ORCHESTRA}{\cite{dawson1996testing}}
\newcommand{\Fuzz}{\cite{miller1995fuzz}}
\newcommand{\Ballista}{\cite{koopman2000exception}}
\newcommand{\RIDDLE}{\cite{ghosh1998testing}}
\newcommand{\AFL}{\cite{gutmann2016fuzzing}}
\newcommand{\SAGE}{\cite{godefroid2012sage}}
\newcommand{\pFuzzer}{\cite{mathis2019parser}} 
\newcommand{\MoWF}{\cite{pham2016model}}
\newcommand{\DiNardoICST}{\cite{di2015generating}}
\newcommand{\DiNardoASE}{\cite{di2015evolutionary}}
\newcommand{\Matinnejad}{\cite{Matinnejad19}}

\begin{itemize}
	\item FTAPE tool: \cite{tsai1999stress}
	\item FIAT tool: \cite{barton1990fault}
	\item GOOFI tool: \cite{aidemark2001goofi}
	\item DOCTOR tool: \cite{han1995doctor}
	\item ORCHESTRA tool: \cite{dawson1996testing}
	\item Fuzz tool:\cite{miller1995fuzz}
	\item Ballista tool: \cite{koopman2000exception}
	\item RIDDLE tool: \cite{ghosh1998testing}
	\item American Fuzzy Lop tool: \cite{gutmann2016fuzzing}
	\item SAGE tool: \cite{godefroid2012sage}
	\item pFuzzer tool: \cite{mathis2019parser}
	\item MoWF tool: \cite{pham2016model}
	\item Di Nardo et al. Model-based Mutation tool: \cite{di2015generating}
	\item Di Nardo et al. Search-based Mutation tool: \cite{di2015evolutionary}
	\item Matinnejad et al. Signal Mutation tool: \cite{Matinnejad19}
\end{itemize}

\tiny
\setlength\LTleft{0pt}
\setlength\LTright{0pt}
\begin{longtable}{@{\extracolsep{\fill}}|p{1.5cm}|p{2cm}|p{5cm}|p{3cm}|p{1cm}|@{}}
\toprule
	\textbf{Name}	&	\textbf{Type of Fault}	&	\textbf{Definition}	&	\textbf{Target}	&	\textbf{Reference} \\
	\midrule
stuck-at-0 & CPU Faults; Memory Faults & This operator corrupts data by replacing a bit/byte/word with zero. & CPU registers; Memory registers & \FTAPE, \FIAT, \GOOFI \\
stuck-at-1 & CPU Faults; Memory Faults & This operator corrupts data by replacing a bit/byte/word with one. & CPU registers; Memory registers & \FTAPE, \FIAT, \GOOFI \\
bit flips & CPU Faults; Memory Faults; Data Processing Faults & This operator mutates data by inverting the value of each bit (i.e., replacing 0 with 1 and 1 with 0). & CPU registers: saved registers, floating point registers, the program counter, global pointer, stack pointer; Local memory; Input or output parameters of a software interface. & \FTAPE, \FIAT, \GOOFI \\
disk driver codes & Communication Faults & This operator performs data mutation by modifying the error flags of a disk driver. & I/O disk driver codes. & \FTAPE \\
set & CPU Faults; Communication Faults; Memory Faults & This operator sets the value of a bit/byte by replacing the current value with one or a user-defined bitmap. & Memory: stack, heap, global data, user code, user-defined memory location; Registers: data, stack, address, program counter, status register & \FTAPE, \FIAT, \GOOFI \\
clear & CPU Faults; Communication Faults; Memory Faults & This operator clears the value of a bit/byte by replacing the current value "v" with zero. & Memory: stack, heap, global data, user code, kernel code, user-defined memory location; Registers: data, stack, address, program counter, status register & \FTAPE, \FIAT, \GOOFI \\
toggle & CPU Faults; Communication Faults; Memory Faults & This operator toggles (i.e., sets a bit to its complement state) a bit/byte. & Memory: stack, heap, global data, user code, kernel code, user-defined memory location; Registers: data, stack, address, program counter, status register & \FTAPE, \FIAT, \GOOFI \\
lose message & Communication Faults & This operator causes a message to be lost in between two communicating components. Messages can be lost intermittently, with a probability distribution specified by the users, or alternatively, every message can be lost during a certain period. & Faulty link, Faulty direction, Single message & \DOCTOR, \ORCHESTRA \\
duplicate message & Communication Faults & This operator causes a message to be duplicated in between two communicating components. & Faulty link, Faulty direction, Single message & \DOCTOR, \ORCHESTRA \\
alter message header & Communication Faults & This operator alters a message. The change is performed in a similar manner as for memory faults, i.e., by performing bit flips. The mutation is performed in the message header. & Faulty link, Faulty direction, Single message & \DOCTOR, \ORCHESTRA \\
alter message body & Communication Faults & This operator alters a message. The change is performed in a similar manner as for memory faults, i.e., by performing bit flips. The mutation is performed in the message body. & Faulty link, Faulty direction, Single message & \DOCTOR, \ORCHESTRA \\
delay message & Communication Faults & This operator causes a message to be delayed in between two communicating components. The delay time can either be constant or follow a probability distribution. & Faulty link, Faulty direction, Single message & \DOCTOR, \ORCHESTRA \\
fuzzing & Data Processing Faults & Fuzzing techniques replace values with values (i.e., fuzz values) that are different and, usually, randomly generated. & Structured input data (e.g., files, streams) & \Fuzz, \Ballista \\
grammar-based fuzzing & Data Processing Faults & This fuzzing approach relies on a grammar that describes the format of the data to be altered; new values are generated according to the grammar. The use of the grammar decreases the chances of generating invalid inputs. & Structured input data (e.g., files, streams) & \RIDDLE \\
evolutionary fuzzing & Data Processing Faults & This fuzzing approach relies on evolutionary algorithms (e.g., genetic algorithms) to generate fuzz values. It relies on a fitness function to maximize structural coverage (e.g., branches executed during testing). The fitness is implemented by instrumenting the code of the application to collect code coverage. & Structured input data (e.g., files, streams) & \AFL \\
whitebox fuzzing & Data Processing Faults & This fuzzing approach relies on symbolic execution techniques to cover corners cases on the different execution paths. & Structured input data (e.g., files, streams) & \SAGE \\
parser-directed fuzzing & Data Processing Faults & Fuzzing techniques replace a correct value with a randomly generated one. This fuzzing operator is applied to programs that integrate a parser component to extract information from the provided inputs. It is implemented through a test generator technique that aims to produce valid inputs for the parser, the inputs are produced randomly and then checked using constraint solving techniques. & Structured input data (e.g., files, streams) & \pFuzzer \\
model-based whitebox fuzzing & Data Processing Faults & Fuzzing techniques replace a correct value with a randomly generated one, in this case, the replacement is done using a model-based whitebox fuzzing technique, the technique prunes from the search space those paths that are exercised by invalid, malformed inputs. The model should specify the format of the data chunks and integrity constraints. & Structured input data (e.g., files, streams) & \MoWF \\
data-type based injection & Data Processing Faults & This operator replaces a value with an invalid one, selected on the basis of the type of the parameter being corrupted. & Structured input data (e.g., files, streams) & \Fuzz, \Ballista, \RIDDLE \\
model-based mutation & Data Processing Faults & This approach alters existing inputs by relying on a data model that capture the structure of the data and the constraints among data fields. A predefined set of operators indicating how to alter data based on its structure are provided. & Structured input data (e.g., files, streams) & \DiNardoICST \\
search-based data mutation & Data Processing Faults & This approach alters existing inputs by relying on a data model that captures the structure of the data and the constraints between data fields. A search-based, evolutionary algorithm is used to maximize a number of mutation objectives including the number of constraints being violated and code coverage. & Structured input data (e.g., files, streams) & \DiNardoASE \\
signal mutation & Signal faults & This approach alters signal by either shifting the signal based on a randomly selected value or shifting the signal and increasing the number of signal pieces (i.e., segments).  & Input signals & \Matinnejad \\

	\bottomrule                                                             
\caption{Data-driven Mutation Operators.}
\label{table:dataOperators}
\end{longtable}
\normalsize

To summarize the data in Table~\ref{table:dataOperators}, we observe that:
\begin{itemize}
	%Note: always put a comma at the end of a list before "or" "and"
	\item Category \INDEX{CPU Faults} includes techniques that perform mutations in the contents of an individual bit, byte, or word in a CPU register. The mutations in this category can target saved, floating-point, program-counter, global and stack-pointer register locations. 
	\item Category \INDEX{Memory Faults} includes techniques that perform mutations in the contents of an individual bit, byte, or word in a memory register. The mutations in this category can target stack, heap, global-data and user-defined memory locations.
	\item Category \INDEX{Communication Faults} includes techniques that simulate packet corruption faults. The mutations in this category can target channels between components, single messages, and the addresses of the messages to be exchanged.
	\item Category \INDEX{Data Processing Faults} includes techniques that mutate the data being processed by the system. Some of these techniques are white-box, they process the source code of the application to generate inputs more efficiently. Other techniques require a model of the data to be mutated. The mutations in this category can target both the input and the output parameters of the interface of a software component.
	\item Category \INDEX{Signal Faults} includes techniques that alter the shape of the signals received by the software under test.
\end{itemize}

\CHANGEDTWO{In the following, we provide an overview of the mutation operators implemented by the most representative techniques.
We do not present simple \INDEX{fuzzing} approaches because they simply perform random mutations and they include a subset of the operators implemented in the more advanced \INDEX{evolutionary fuzzing} approaches.}

\subsubsection{Blind memory corruption and blind transmission corruption}

\CHANGEDTWO{The mutation operators implemented by techniques performing \INDEX{blind memory corruption} and \INDEX{blind transmission corruption} are presented in Tables~\ref{table:operators:blindTransmissions} and ~\ref{table:operators:blindTransmissions}, respectively. These techniques do not require a data model. Unfortunately, they are of little usefulness to simulate subtle errors that affect relations among data (e.g., replacing an identifier with another legal one).}

%Among all the data mutation operators reported in Table~\ref{table:dataOperators}, the ones targeting \emph{Data Processing Faults} are more powerful since they concern the modification of complex data structures. %Oscar: not sure that "overviewed" is spelled right %They are briefly overviewed in the following.
%We provide an overview in the following.

\CHANGEDTWO{\subsubsection{Evolutionary Fuzzing}}


%\subsubsection{Approaches generating data from scratch}
\INDEX{AFL} is an \INDEX{instrumentation-guided fuzzer}~\cite{gutmann2016fuzzing}. It integrates \INDEX{genetic algorithms} with a form of branch coverage called state transition coverage,
which captures the number of times each branch is taken in an execution. 
AFL aims to generate inputs that exercise all the different code paths.
%It is implemented by executing for every instruction of the program under test the following lines
%cur_location = <COMPILE_TIME_RANDOM>;
%coverage[cur_location ^ prev_location]++; 
%prev_location = cur_location >> 1;
It requires one or more starting files that contain input data normally processed by the targeted application. 
Each provided input is processed in a queue. For each input in the queue, AFL trims the test case to the smallest size that doesn't alter the coverage of the program. Then it repeatedly mutates the file using a set of fuzzing strategies, which are presented in Table~\ref{table:AFL:operators}.
If any of the generated mutations result in a change of the coverage for the software under test, the mutated input is added to the input queue.

% !TEX root =  ../MutationTestingSurvey.tex

%
%\setlength\LTleft{0pt}
%\setlength\LTright{0pt}
%\begin{longtable}{@{\extracolsep{\fill}}|p{2.5cm}|p{5cm}|p{5cm}|@{}}
%\toprule


\begin{table}[h]
\caption{Mutation Operations performed by AFL~\cite{gutmann2016fuzzing}}
\label{table:AFL:operators}


\tiny
\begin{tabular}{|p{2.5cm}|p{10cm}|}

\hline

\textbf{Walking bit flips}& Sequential, ordered bit flips. The number of bits flipped in a row varies from one to four in steps of two. This strategy is applied till the coverage of the program under test does not change.\\

\hline
\textbf{Walking byte flips}& This method relies on 8-, 16-, or 32-bit wide bitflips with a constant stepover of one byte. \\

\hline
\textbf{Simple arithmetics}& Increment and decrement existing integer values in the input file; this is done considering every byte of the file. The chosen range for the operation is -35 to +35. It is performed in three steps. First, the fuzzer attempts to perform subtraction and addition on individual bytes. The second steps involves looking at 16-bit values, using both endians - but incrementing or decrementing them only if the operation has affected the most significant byte. The final stage follows the same logic, but for 32-bit integers.\\

\hline
\textbf{Known integers}& 
AFL overwrites every byte in the input file with a set of interesting values (e.g., -1, 256, 1024, MAX\_INT-1, MAX\_INT), using both endians (the writes are 8-, 16-, and 32-bit wide).\\

\hline
\textbf{Stacked tweaks}& Sequence of randomly selected mutations among the following: Single-bit flips, Attempts to set "interesting" bytes, words, or dwords (both endians), Addition or subtraction of small integers considering bytes, words, or double words (both endians), Random single-byte sets, Block deletion, Block duplication via overwrite or insertion, Block memset. \\

\hline
\textbf{Test case splicing}& Consists of taking two distinct input files that differ in at least two locations, splicing them at a random location in the middle, and then applying stacked tweaks.\\

\hline
%\bottomrule                                                             

\end{tabular}
\end{table}
%\normalsize




\CHANGEDTWO{\subsubsection{Grammar-based fuzzing}}

\INDEX{Superion}~\Superion~ extends \emph{AFL} to drive mutations based on grammars. It takes as input an ANTLR~\cite{ANTLR} grammar. 
It performs mutations that consist of \INDEX{grammar-aware trimming}, \INDEX{dictionary-based mutations}, and \INDEX{tree-based mutation}.
Grammar-aware trimming is performed by generating an AST tree from the input and by deleting a randomly-selected subtree. 
It aims to generate trimmed inputs that do not alter code coverage in the SUT.
Dictionary-based mutations are performed by building a dictionary using every token in the input file and by inserting the elements of the dictionary in the boundaries of every token in the input file.
Tree-based mutation is performed by processing two input files and replacing the subtrees of one file with the ones of another.
Listing~\ref{mutatedJSONfile} shows a JSON file generated by deleting a randomly-selected subtree from Listing~\ref{JSONfile}.


\begin{minipage}{14cm}
\footnotesize
\begin{lstlisting}[caption=JSON file generated by deleting a subtree from Listing~\ref{JSONfile}., label=mutatedJSONfile]
{
 [
 {
  "firstName": "Fabrizio",
  "lastName": "Pastore",
  "isAlive": true,
  },
  {
  "firstName": "Oscar",
  "lastName": "Cornejo",
  "isAlive": true,
  "address": {
    "streetAddress": "29, Avenue J.F Kennedy",
    "city": "Luxembourg",
    "state": "Luxembourg",
    "postalCode": "L-1855"
  },
  }
  ]
}
\end{lstlisting}
\end{minipage}

%Fabrizio: you had this definition of equivalent: "(i.e., correct fuzzed data, but semantically meaningless)." it is difficult to understand


%Fabrizio: I cannot understand the following sentences
%Although it tackles the problem of redundant mutants by well covering the space of possible inputs, it does not tackle the redundant mutant problem.
%Cannot handle semantic restrictions imposed by certain nontrivial input languages. The approach has difficulties to handle complex sequences recursion.

%\subsubsection{Approaches altering existing data}


\emph{$\mu$4SQLi}~\cite{Appelt:SQLI:ISSTA:2014} relies on a SQL grammar to generate \INDEX{SQL injections}. 
SQL injections are generated by means of a set of mutation operators, shown in Table~\ref{table:Mu4SQLI}, that alter the values of an input according to predefined patterns.
Similarly, SOLMI~\cite{Jan:ISSTA:2016} relies on the XML grammar to alter XML messages based on a predefined set of mutation operators and introduce potential XML injections. 
SOLMI implements four operators that (1) introduce XML meta-characters, (2) introduce closing tags, (3) replicate XML elements, (4) replace XML content.


\begin{table}[h]
\caption{Mutation operators implemented by $\mu$4SQLi}
\label{table:Mu4SQLI}
\begin{tabular}{|p{2cm}|p{11.5cm}|}
\hline
\multicolumn{2}{|c|}{Behaviour-Changing Operators}\\
\hline
MO\_or&Adds an OR-clause to the input\\
MO\_and&Adds an AND-clause to the input\\
MO\_semi&Adds a semicolon followed by an additional SQL statement\\
\hline
\multicolumn{2}{|c|}{Syntax-Repairing Operators}\\
\hline
MO\_par&Appends a parenthesis to a valid input\\
MO\_cmt&Adds a comment command (- - or \#) to an input\\
MO\_qot&Adds a single or double quote to an input\\
\hline
\multicolumn{2}{|c|}{Obfuscation Operators}\\
\hline
MO\_wsp&Changes the encoding of whitespaces \\
MO\_chr&Changes the encoding of a character literal enclosed in quotes\\
MO\_html&Changes the encoding of an input to HTML entity encoding\\
MO\_per&Changes the encoding of an input to percentage encoding\\
MO\_bool&Rewrites a boolean expression while preserving it's truth value\\
MO\_keyw&Obfuscates SQL keywords by randomising the capitalisation and inserting comments\\
\hline
\end{tabular}
\end{table}



One of the major limitations of grammar-based approaches is the cost of defining an input grammar. MongoDB's javascript fuzzer addresses the problem of grammar-based mutation when a grammar is not available. Instead of mutating input files for the SUT based on the input grammar of  the SUT, it mutates test cases for the SUT (in this case the Javascript test cases)~\MongoDB. It replaces subtrees in the AST tree generated from the test cases either other subtrees belonging to the same input file or with subtrees generated following encoded production rules.


\CHANGEDTWO{\subsubsection{Whitebox fuzzing}}

%, also, it requires more knowledge about the target than purely random ones.

\INDEX{SAGE} adopts \INDEX{symbolic execution} to systematically generate malformed inputs~\cite{godefroid2012sage}. SAGE performs fuzzing on file- and packet-parsing applications. 
The program is first executed with concrete inputs; in order to identify a set of constraints on inputs, then, one of the constraints in the set is negated, and new malformed inputs are generated to satisfy the new set of constraints. 
The main benefit of SAGE is that it forces the program to execute corner cases not covered by the initial inputs; for example, 
%Fabrizio: you wrote the following, which I could not understand please check if my sentence is correct
%(e.g., reached one-third of all bugs found by fuzz testing in Microsoft projects \cite{bounimova2013billions}).
%Oscar: yes, what you wrote is correct 
one-third of all the bugs found by means of fuzz testing are detected thanks to SAGE \cite{bounimova2013billions}.
Unfortunately, the main limitation of SAGE and symbolic execution-based fuzzers is its limited scalability, due to the high execution time required by symbolic execution.

%Fabrizio: not sure what the following is about, ignoring
%\emph{Testing of Fault-Tolerant and Real-Time Distributed Systems via Protocol Fault Injection \cite{dawson1996testing}}: The paper introduces a portable fault injection environment for testing implementations of distributed protocols.

\CHANGEDTWO{\subsubsection{Model-based fuzzing}}

%Fabrizio: we miss the pro/cons from the following
\CHANGEDTWO{\INDEX{Model-based fuzzing} targets program binaries that process structured inputs.
\INDEX{Model-Based Blackbox Fuzzing} (MoBF) is performed using block models that capture the structure of the input data for the SUT. A solution to perform MoBF is Peach~\cite{PeachFuzzer,PeachMozilla}, which alters data of an input files according to a large, predefined set of rules. We provide an overview of the mutation operators implemented by Peach in Table~\ref{table:PeachOperators}.
\INDEX{Model-Based Whitebox Fuzzing}, instead, relies on additional information about code coverage to generates valid inputs that exercise critical target locations~\cite{pham2016model}.} This is done through a directed path exploration technique that prunes from the search space those paths that are exercised by invalid, malformed inputs.
Compared with a Model-Based Blackbox Fuzzer (MoBF) approach, the \emph{MoWF} prototype was able to expose all of 13 vulnerabilities on an empirical evaluation carried on nine subject programs, while the MoBF approach only detected 6 out of 13 vulnerabilities. 

% !TEX root = ../MutationTestingSurvey.tex

\begin{table}[h]
\begin{center}
\footnotesize
\CHANGEDTWO{
\begin{tabular}{|p{5cm}|p{9cm}|}
\hline
\textbf{Operator Name}&\textbf{Description}\\
\hline
ArrayVarianceMutator&Change the length of arrays. Given L the original length of the array, the length is changed in range L-N to L+N.\\
ArrayReverseOrderMutator&Reverse the order of an array.\\
ArrayRandomizeOrderMutator&Put array elements in random order.\\
DWORDSliderMutator&Slides a DWORD through the blob.\\
BitFlipperMutator&Flips a given \% of bits in blob. Default is 20\%.\\
BlobMutator&Randomly grows a Blob block or shrinks it.\\
DataTreeRemoveMutator&Remove nodes from data tree.\\
DataTreeDuplicateMutator&Duplicate a node's value starting at 2x through 50x.\\
DataTreeSwapNearNodesMutator&Swap the data of two nodes that are near each other in the data model.\\
NumericalVarianceMutator&Produce numbers that are defaultValue - N to defaultValue + N.\\
NumericalEdgeCaseMutator&Replace with random numbers of appropriate correct size.\\
FiniteRandomNumbersMutator&Produce a finite number of random numbers for each \emph{Number} element.\\
NumericalEvenDistributionMutator&Generate numbers evenly distributed through the total numerical space of the number range.\\
NullMutator&Does nothing, just test the data produced by the fuzzer.\\
PathValidationMutator&Does not mutate. Used to trace path of each test for path validation.\\
SizedVarianceMutator&Change the length of sizes to count - N to count + N.\\
SizedNumericalEdgeCasesMutator&Change the length of sizes to numerical edge cases.\\
SizedDataVarianceMutator& Change the length of sized data to count - N to count + N. Size indicator will stay the same.\\
SizedDataNumericalEdgeCasesMutator&Change the length of sizes to numerical edge cases.\\
StringCaseMutator&Change the case of a string.\\
UnicodeStringsMutator&Generate unicode strings.\\
ValidValuesMutator&Replace with random values other than the legal ones.\\
UnicodeBomMutator&Injects BOM markers into default value and longer strings.\\
UnicodeBadUtf8Mutator&Generate bad UTF-8 strings.\\
UnicodeUtf8ThreeCharMutator&Generate long UTF-8 three byte strings.\\
StringMutator&Generate a random unicode string, for each string node, one Node at a time.\\
XmlW3CMutator&Replace XML trees with invalid, non-well former, and valid (but random) XML trees.\\
PathMutator&Replace a path with an erroneous path generated according to 20 different rules.\\
HostnameMutator&Replace a hostname with an erroneous hostname generated according to 20 different rules.\\
IpAddressMutator&Replace an IP address with an erroneous IP address generated according to 20 different rules.\\
TimeMutator&Replace a time value with an erroneous value generated according to 3 different rules.\\
DateMutator&Replace a date with 60 predefined erroneous dates.\\ 
FilenameMutator&Replace a file name with an file name generated according to 10 different rules.\\
ArrayNumericalEdgeCasesMutator&This operator is not well documented in the source code of Peach.\\
BlobSpread&This operator is not well documented in the source code of Peach.\\
\hline
\end{tabular}
}
\end{center}
\caption{Mutation Operators for the opensource version of Peach~\cite{PeachMozilla}}
\label{table:PeachOperators}
\end{table}%

\CHANGEDTWO{\subsubsection{Model-based data mutation}}

%\emph{Generating complex and faulty test data through model-based mutation analysis (Research paper) \cite{di2015generating}}: 
\INDEX{Model-based data mutation} concerns the automated generation of invalid input data through the mutation of existing data based on a predefined set of mutation operators~\cite{di2015generating}.
The technique receives two inputs: field data and a data model, i.e., a UML class diagram annotated with stereotypes and OCL constraints. 
An example data model has been shown in Figure~\ref{fig:dataModel} while an example OCL constraint appears in Figure~\ref{fig:costraint:firstHeader}. 
The technique relies upon six generic mutation operators to automatically generate faulty data. 
Table~\ref{table:dataModelMutationOperators} provides an overview of the mutation operators proposed in ~\cite{di2015generating}.
In model-based data mutation~\cite{di2015generating} stereotypes are used to tailor the behaviour of the generic mutation operators to the fault model for the system under test and the environment in which it is deployed. 
Table~\ref{table:faultModel:SES} shows a fault model for a satellite system that processes the data presented in Figure~\ref{fig:dataModel}.
Mutation operators are applied to the data according to the stereotypes used in the data model.
Table~\ref{table:mapping} shows the mutation operators and the corresponding stereotypes. In~\cite{di2015generating}, the mutation operator \emph{Attribute Bit Flipping} is applied on all the attributes not tagged with other stereotypes. 

\begin{table}[h]
\begin{center}
\begin{tabular}{|p{5cm}|p{5cm}|p{2.5cm}|}
\hline
\textbf{Fault}&\textbf{Mutation Operator}&\textbf{Stereotype}\\
\hline
Duplicate VCDU/Packet& Class Instance Duplication.&InputData\\
Missing VCDU/Packet& Class Instance Removal.&InputData\\
Wrong Sequence& Class Instances Swapping.&InputData\\
Incorrect Identifier& Attribute Replacement with Random.&Identifier\\
Incorrect Checksum& Attribute Replacement with Random.&Identifier\\
Incorrect Counter& Attribute Replacement using Boundary Condition.&Measure\\
Flipped Data Bits& Attribute Bit Flipping.&\\
\hline
\end{tabular}
\end{center}
\caption{Mapping between Fault Data and Mutation Operators in \cite{di2015generating}.}
\label{table:mapping}
\end{table}%

% !TEX root =  ../MutationTestingSurvey.tex

%
%\setlength\LTleft{0pt}
%\setlength\LTright{0pt}
%\begin{longtable}{@{\extracolsep{\fill}}|p{2.5cm}|p{5cm}|p{5cm}|@{}}
%\toprule


\begin{table}[h]
\caption{Mutation Operators for Model-based Data-driven Mutation Testing. Based on~\cite{di2015generating}}
\label{table:dataModelMutationOperators}


\tiny
\begin{tabular}{|p{2.5cm}|p{5cm}|p{7cm}|}

\hline
\textbf{Operator}&\textbf{Description}&\textbf{Example}\\
\hline
\textbf{Class Instance Duplication (CID)}&
The operator \emph{Class Instance Duplication} duplicates an instance of a class belonging to a collection of elements. This operator copies a randomly chosen instance of a class in a collection and then inserts it at a random position in the collection. This operator simulates unexpected data in a collection.
&In Figure~\ref{fig:dataModel}, this operator can be applied to the associations between the classes \emph{Transmission} and \emph{Vcdu}, and between the classes \emph{VirtualChannel} and \emph{Packet}. In both cases the duplicated data generated by this operator simulates a transmission error.\\
\hline
\textbf{Class Instance Removal (CIR)}
&This mutation operator deletes a randomly selected instance of a class from a collection of elements. 
&In Figure~\ref{fig:dataModel}, this operator can be applied to the associations between the classes \emph{Transmission} and \emph{Vcdu}, and between the classes \emph{VirtualChannel} and \emph{Packet}. The removal of an instance of class \emph{Vcdu}, for example, simulates a transmission error that may lead to either missing or broken Packets. When processing erroneous data created with this mutation operator, SES-DAQ should report a \emph{COUNTER\_JUMP} error as indicated by the constraint in Figure~\ref{fig:costraint:firstHeader}. \\
\hline
\textbf{Class Instances Swapping (CIS)}
&Swaps the positions of two randomly chosen instances of a class in a collection of elements. 
&In Figure~\ref{fig:dataModel}, this operator can be applied to the associations between the classes \emph{Transmission} and \emph{Vcdu}, and between the classes \emph{VirtualChannel} and \emph{Packet}. The effect of swapping two packets belonging to the association between the classes \emph{VirtualChannel} and \emph{Packet} simulates the presence of transmission data sequence errors.\\
\hline

\textbf{Attribute Replacement with Random (ARR)}
&This mutation operator replaces the value of an identifier attribute in an instance of a class with a randomly chosen value. In principle all the attributes of a class can be replaced with randomly chosen values, but in the general case a randomly generated value is not necessarily erroneous.
We are interested in mutations that lead to errors, 
for this reason we introduced the UML stereotype $Identifier$ that allows software engineers to indicate which attributes are used as identifiers, and thus can be mutated according to the ARR operator. The $Identifier$ stereotype enables software engineers to specify a numeric range for the random value to generate.
&In Figure~\ref{fig:dataModel}, this mutation operator can be applied to all the attributes tagged with the stereotype \emph{Identifier}. For example a random mutation of the attribute \emph{versionNumber} belonging to an instance of class \emph{Header} simulates an invalid frame version, which should be reported by the software. 
\\
\hline
\textbf{Attribute Replacement using Boundary Condition (ARBC)}
& This mutation operator changes the value of an attribute according to a boundary condition criterion. This operator is particularly useful for mutating attributes that should be bound within a range, these attributes are usually measures. We thus introduced the UML stereotype \emph{Measure} to tag the attributes that belong to this category. This stereotype enables software engineers to indicate the minimum and maximum values allowed for the tagged attribute. The mutation operator generates four values out of range according to traditional boundary testing strategies: minimum value, minimum value minus one, maximum value, and maximum value plus one. The operator ensures that the generated value is in the range representable with the data type (e.g. unsigned bytes cannot represent negative values).
&In Figure~\ref{fig:dataModel}, this operator can be applied to all the attributes tagged with the UML stereotype \emph{Measure}. In the running example this operator can be applied to the attribute \emph{vcFrameCount} of class \emph{Header}. 
\\
\hline
\textbf{Attribute Bit Flipping (ABF)}
&This operator randomly selects an attribute that corresponds to transmitted data and alters the value of a randomly selected bit. This mutation operator is particularly effective for introducing errors in attributes that cannot be tagged as Identifiers or Measures.
The operator works by flipping a single bit of an attribute. 
&In Figure~\ref{fig:dataModel}, this mutation operator can be applied to the attribute \emph{packetData} of class \emph{Packet} of the running example.
The attribute \emph{packetData} is a byte array: the mutation of one of its bits
simulates the presence of a realistic transmission error that should be identified thanks to the presence of a redundancy check code.
\\
\hline



%\bottomrule                                                             

\end{tabular}
\end{table}
%\normalsize

% !TEX root = ../MAIN.tex
\begin{table}[h]
\begin{center}
\scriptsize
\begin{tabular}{|p{2cm}|p{2cm}|p{4cm}|p{6cm}|}
\hline
\textbf{Fault Class}&\textbf{Types}&\textbf{Parameters}&\textbf{Description}\\
\hline
Value above threshold (VAT)&
\begin{minipage}{6cm}
INT\\
LONG INT\\
FLOAT\\
DOUBLE
\end{minipage}
&
\begin{minipage}{6cm}
T: threshold\\
D: delta with respect to threshold\\
\end{minipage}
&
\begin{minipage}{6cm}
The value is above a threshold T for a delta D. 

\EMPH{Data mutation operation:} The mutation is performed by replacing the current value (a number) with a value of the same type that is equal to $(T+D)$.
\end{minipage}
\\

\hline
Value below threshold (VBT)&
\begin{minipage}{6cm}
INT\\
LONG INT\\
FLOAT\\
DOUBLE
\end{minipage}
&
\begin{minipage}{6cm}
T: threshold\\
D: delta with respect to threshold\\
\end{minipage}
&
\begin{minipage}{6cm}
The value is below a threshold T for a delta D. 

\EMPH{Data mutation operation:} The mutation is performed by replacing the current value (a number) with a value of the same type that is equal to $(T-D)$.
\end{minipage}
\\



\hline
Value out of range (VOR)&
\begin{minipage}{4cm}
INT\\
LONG INT\\
FLOAT\\
DOUBLE
\end{minipage}
&
\begin{minipage}{4cm}
MIN: minimum valid value\\
MAX: maximum valid value\\
D: delta with respect to minimum/maximum valid value
\end{minipage}
&
\begin{minipage}{6cm}
The value is out of the valid range MIN-MAX. 

\EMPH{Data mutation operations (2):}  The mutation is performed by replacing the current value (a number) with 
\begin{itemize}
\item a value of the same type that is equal to $(MIN-D)$
\item a value of the same type that is equal to $(MAX+D)$
\end{itemize}
\end{minipage}
\\

\hline
Bit flip (BF)&
BIN
&
\begin{minipage}{4cm}
MIN: lower bit\\
MAX: higher bit\\
STATE: mutate only if the bit is in the given state\\
\TRFOUR{VALUE: integer specifying the number of bits to mutate}\\
\end{minipage}
&
\begin{minipage}{6cm}
A number of bits randomly chosen in the positions between MIN and MAX (included) are flipped.

\EMPH{Data mutation operation:} the operator flips N randomly selected bit.
If STATE is specified, the mutation is applied only if  the bit is in the specified state. Parameter VALUE specifies the number of bits to mutate.
\end{minipage}
\\

\hline
Invalid numeric value (INV)&
\begin{minipage}{6cm}
INT\\
LONG INT\\
FLOAT\\
DOUBLE
\end{minipage}
&
\begin{minipage}{4cm}
MIN: lower valid value\\
MAX: higher valid value\\
\TRFOUR{D: distribution to follow}\\
\TRFOUR{VALUE: mean value for normal distribution}\\
\end{minipage}
&
\begin{minipage}{6cm}
The value is legal (i.e., in the specified range) but different than the current one, which, in this case, is expected to be consistent with the status of the system.

\EMPH{Data mutation operation:} Mutation is performed by replacing the current value with a different value randomly sampled in the specified range. The parameter D specified the distribution to follow when performing the mutation\footnote{In our implementation 0 indicates uniform, 1 indicates normal around the specified value (but in range).}
\end{minipage}
\\

\hline
Illegal Value (IV)
&
\begin{minipage}{6cm}
INT\\
LONG INT\\
FLOAT\\
DOUBLE
\end{minipage}
&
\begin{minipage}{6cm}
VALUE: illegal value that is observed\\
\end{minipage}
&
\begin{minipage}{6cm}
The value is illegal and equal to the provided one (i.e., parameter \emph{VALUE}).

\EMPH{Data mutation operation:} Mutation is performed by replacing the current value with the value \emph{VALUE}, if different than the current one.
\end{minipage}
\\

\hline
\TRFOUR{Anomalous Signal Amplitude (ASA)}
&
\begin{minipage}{6cm}
INT\\
LONG INT\\
FLOAT\\
DOUBLE
\end{minipage}
&
\begin{minipage}{6cm}
T: change point\\
D: delta to add/remove\\
V: value to multiply\\
\end{minipage}
&
\begin{minipage}{6cm}
The value is modified by amplifying/reducing it by a factor V and adding or removing D from the observed value. It is used to "amplify" a signal in a constant manner to simulate unusual signal. T indicates the observed value below which instead of adding  we subtract .

\EMPH{Data mutation operation:} Mutation is performed by replacing the current value ($v$) with the value ($v'$) computed as follows:

\[
v' =  
    \begin{cases}
      T+(  (v-T)*V  ) + D   & \mathit{if}\ v \ge T\\
      T - (  (T-v)*V  ) - D   & \mathit{if}\ v < T
    \end{cases}       
\]
\end{minipage}
\\


\hline
\TRFOUR{Signal Shift (SS)}
&
\begin{minipage}{6cm}
INT\\
LONG INT\\
FLOAT\\
DOUBLE
\end{minipage}
&
\begin{minipage}{6cm}
D: delta by which the signal should be shifted\\
\end{minipage}
&
\begin{minipage}{6cm}
The value is modified by adding a value D. It simulate an anomalous shift in the signal.
\end{minipage}
\\





\hline
\TRFOUR{Hold Value (HV)}
&
\begin{minipage}{6cm}
BIN\\
INT\\
LONG INT\\
FLOAT\\
DOUBLE
\end{minipage}
&
\begin{minipage}{6cm}
V: number of times to repeat the same value\\
\end{minipage}
&
\begin{minipage}{6cm}
This operator keeps repeating an observed value for $V$ times. It emulates a constant signal replacing a signal supposed to vary.
\end{minipage}
\\



\hline
\TRFOUR{Array Swap (AS)}
&
\begin{minipage}{6cm}
ARRAY\_*\\
\end{minipage}
&
\begin{minipage}{6cm}
MIN: position of element A\\
MAX: position of element B\\
VALUE: number of elements to move\\
\end{minipage}
&
\begin{minipage}{6cm}
Replace a number of elements (number specified by VALUE) located starting from position MIN, with a number of elements located starting from position MAX, and viceversa.
\EMPH{Data mutation operation:} Mutation is performed by replacing the two set of elements with each other.
\end{minipage}
\\


\hline
\TRFOUR{Array Random Swap (ARS)}
&
\begin{minipage}{6cm}
ARRAY\_*\\
\end{minipage}
&
\begin{minipage}{6cm}
MIN: min position of element A/B\\
MAX: max position of element A/B\\
VALUE: number of elements to move\\
\end{minipage}
&
\begin{minipage}{6cm}
Replace a number of elements (number specified by VALUE) located in a position between MIN and MAX, with a number of elements located in a position between MIN and MAX. MIN and MAX specify a position with respect to the beginning and end of the array.  For example, MIN=0 indicates the first element of teh array, MIN=-2 indicates the second element of the array.
\EMPH{Data mutation operation:} Mutation is performed by replacing the two set of elements with each other.
\end{minipage}
\\



%Incorrect Identifier& Several transmission data fields have fixed values, for example fields identifying the transmitting satellite. Hardware/software errors may assign incorrect identifiers.\\
%%Incorrect Checksum& Hardware/software errors may result in an incorrect checksum for a Packet or VCDU.\\
%Incorrect Counter& Counters are used to track Packet or VCDU ordering. Hardware/software errors may assign incorrect counter values.\\
%Flipped Data Bits& Physical channel noise may flip one or more bits in the data transmission.\\
\hline
\end{tabular}
\end{center}
\caption{Data Fault Classes}
\label{table:faultModel:FAQAS}
\end{table}%

Data mutation may lead to the generation of inconsistent data containing trivial faults that do not comply with the given fault model (e.g., checksum errors). 
Inconsistent data might also be caused by mutation operators that target classes. For example the swapping of packets that belong to two different virtual channels may lead to the generation of VCDUs that contain packets with a same id, i.e. inconsistent data. To preserve data consistency the approach in \cite{di2015generating} enables software engineers to configure the behaviour of mutation operators by means of OCL queries and UML stereotypes. OCL queries are used to enable software engineers to further restrict the characteristics of the object instances on which the mutation operators can be applied.   The UML stereotype, \emph{Derived}, instead, enables software engineers to specify which attributes need to be updated after a mutation in order to prevent trivial errors. The stereotype requires that software engineers specify the name of a method that is invoked at runtime by the mutation framework to regenerate the value of the tagged attribute. The implementation of this function should be provided by the software engineer (e.g., a utility function named that recalculates the checksum of a packet). 

\CHANGEDTWO{\subsubsection{Search-based data mutation}}

\INDEX{Search-based data mutation} relies on an evolutionary algorithm to perform model-based data mutation and optimize multiple objectives:  
cover all the classes of the data-model, cover all the possible faults of the fault model, cover all the clauses of the input/output constraints,
maximise code coverage~\cite{di2015evolutionary}.
The coverage of each objective is encoded by means of boolean arrays; this information is used to select a minimal set of inputs that maximize the coverage of the different objectives.
At every iteration, the evolutionary algorithm keeps only inputs that contribute to increase the coverage of at least one of the objectives (e.g., inputs that cover one instruction not covered by other inputs).

An additional source of information concerning the adoption of fuzzing and grammar-based approaches to perform test input generation is the \emph{Fuzzing Book}~\cite{fuzzingbook2019:GrammarFuzzer}.


\clearpage