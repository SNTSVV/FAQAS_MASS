% !TEX root = MutationTestingSurvey.tex

\chapter{Mutation Testing Trends}
\label{chapter:trends}


\section{Mutation Testing Trends}

This section introduces the trends on how mutation testing has been used beyond test assessment.

Mutation testing has been used since it was introduced in the early 1970s, as an indicator of test effectiveness \cite{DeMillo78}. Actually, mutation testing is being used increasingly as baseline methodology for assessing different testing techniques \cite{chekam2017empirical}.

Contrarily to this focus, now researchers are shifting their interest into the exploration of the mutant behaviors \cite{papadakis2019mutation}. The idea nowadays is to focus on the functional and non-functional characteristics that each mutant can introduce into the software under test.
Researchers have exploited this concept to develop methods that automatically localize faults \cite{papadakis2015metallaxis}, repair software \cite{le2011genprog}, improve security properties \cite{loise2017towards} and improves memory consumption \cite{langdon2017genetic}.


Beyond typical code-drive mutation testing process, the mutation concept has been also applied to model artifacts, this line of research is called model-based mutation testing. The objective of these techniques is to identify defects related to missing functionalities and misinterpreted specifications \cite{devroey2016featured,belli2016model}.

Lately, the mutation testing approach has been also applied to security testing, specifically to applications on testing security policies. For example, Mouelhi et al. \cite{mouelhiv2008generic} proposed a meta-model that captures security policy rules, then they mutate the model by applying a set of operators that will simulate faults in the instantiations of the model. 

On a different perspective, Patrick and Jia \cite{patrick2015kernel} proposed a technique to support adaptive random testing (i.e., adaptive random testing seeks to distribute test cases more evenly within the input space). In particular, this technique guides the test selection process based on the killed mutants to improve the fault revelation ability of the approach.

Additionally, mutation testing has been used to automatically detect loop invariants by applying mutations to postconditions clauses \cite{galeotti2015inferring}. It has been used also for measuring code coverage in time-sensitive systems \cite{pankumhang2015iterative}. It has been used for supporting software verification tasks \cite{groce2015verified}. Mutation analysis has been also used for creating software clones \cite{roy2009mutation}. In other contexts, Zhang et al. \cite{zhang2016isomorphic} used mutants to improve the fault detection ability of regression test suites, the approach mutates both the old and new version of the program under analysis, then both versions are executed against the same test suite, if there are differences between the output of both execution then a problem has been found.
The mutation testing approach has been also applied to relational database schemas \cite{wright2013efficient}. More recent approaches such as Predictive Mutation Testing \cite{zhang2018predictive} make use of machine learning techniques for predicting if a certain mutant will be killed against a certain test suite.



