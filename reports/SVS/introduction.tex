% !TEX root = MAIN.tex

\chapter{Introduction}

The purpose of this  Software Validation Specification is to describe the testing, analysis, inspection and
review of design specifications, and is used to document the software validation specification concerning the requirements baseline.

\chapter{Applicable and reference documents}

\begin{itemize}
\item{D1 - Mutation testing survey}
\item{D2 - Study of mutation testing applicability to space software}
\item{D4 - Validation of the toolset}
\item{SSS - Software System Specification}
\item{SUM - Software User Manual}
\item{SUTP - Software Unit Test Plan}
\end{itemize}

\chapter{Terms, definitions, and abbreviated terms}

\begin{itemize}
\item{FAQAS}: activity ITT-1-9873-ESA
\item{FAQAS-framework}: software system to be released at the end of WP4 of FAQAS
\item{D2}: Deliverable D2 of FAQAS, \emph{Study of mutation testing applicability to space software}
\item{D4}: Deliverable D4 of FAQAS, \emph{Validation of the toolset}
\item{SSS}: Software System Specification.
\item{KLEE}: Third-party test generation tool, details are provided in D2.
\item{MLFS}: Mathematical Library for Flight Software.
\item{SUT}: Software under test, i.e, the software that should be mutated employing mutation testing.
\item{SUM}: Software User Manual.
\item{SUTP}: Software Unit Test Plan
\item{WP}: Work package.
\end{itemize}

\chapter{Software Overview}

This documents concern the components of the FAQAS framework:
\begin{itemize}
  \item MASS
  \item \DAMA
  \item SEMuS
%  \item DAMTE
\end{itemize}

For detailed information on their structure and usage see D2 and SUM. Since DAMTE has not been implemented as a fully automated tool but only a methodology has been proposed (see D2 and D4) this deliverable does not cover the validation of DAMTE.

\chapter{Software validation specification task identification}

\STARTCHANGEDFINAL

\section{Task and criteria}
\label{sec:taskCrit}

The validation tasks are meant to ensure that the requirement expressed in the SSS are met.
They are accomplished by following validation approaches:
\begin{itemize}
  \item veryfing the requirements for the most critical and complex software components througt \EMPH{unit testing} as detailed in the SUTP.
  \emph{unit testing} is composed of three tasks:
  \begin{itemize}
    \item performing \emph{\DAMA-TD-DDMutation-1}
    \item performing \emph{\DAMA-TD-DDMutation-2}
    \item performing \emph{\MASS-TD-SRCMutation-1}
    \item performing \emph{SEMuS-TD-TGMutation-1}
  \end{itemize}
  \item verifying the requirements for the rest of the components through \EMPH{application to the case studies} as detailed in the D4 and the SUM.
  Application to the case studies is composed of the following tasks, detailed in \ref{sec:case_studies}
  \begin{itemize}
    \item \emph{configuring \MASS and running Launcher.sh - \MASS}
    \item \emph{configuring \MASS and running PrepareSUT.sh - \MASS}
    \item \emph{configuring \MASS and running GenerateMutants.sh - \MASS}
    \item \emph{configuring \MASS and running CompileOptimizedMutants.sh - \MASS}
    \item \emph{configuring \MASS and running OptimizedPostProcessing.sh - \MASS}
    \item \emph{configuring \MASS and running GeneratePTS.sh - \MASS}
    \item \emph{configuring generate\_template\_config.json and running generate\_direct.py - SEMuS}
    \item \emph{manually editing of the generated test templates - SEMuS} 
    \item \emph{instrumenting the source code - \DAMA}
    \item \emph{configuring and running} a given component, for example \texttt{\DAMA\_compile.sh}.
  \end{itemize}
  \item verifying general requirements through \EMPH{manual inspection}. For this approach no further specific task can be identified.
\end{itemize}



\section{Application to the case studies}
\label{sec:case_studies}

In the following sections the validation task that fall under the \EMPH{application to the case studies} category are described, along with their pass-fail criteria.


\subsection{Configuring \MASS and running Launcher.sh - \MASS}
\label{sec:configuring:mass}

The objective of this task is to validate that \MASS can properly prepare the SUT, collect code coverage information, and correctly apply the eight steps of the methodology for assessing the quality of the SUT test suite.

% !TEX root =  ../MAIN.tex

\begin{table}[tb]
\footnotesize
\centering
\caption{\MASS minimal set of parameters to be configured.}
\label{table:to_configure}
\begin{tabular}{llp{7.5cm}}
\hline
\textbf{Script Name}  & \textbf{Parameter} &  \textbf{Description} \\
\hline
mass\_conf.sh & BUILD\_SYSTEM &  Specifies the building system type.\\
& PROJ &  Path of the SUT root  directory.\\
& PROJ\_SRC &  Path of the SUT source directory.\\
& PROJ\_TST &  Path of the SUT test  directory.\\
& PROJ\_COV &  Path of the directory with SUT coverage information.\\
& PROJ\_BUILD &  Path of the directory where the compiled binary is stored.\\
& ORIGINAL\_MAKEFILE &  Path to the original build script.\\
& COMPILATION\_CMD &  Compilation command of the SUT.\\
\hline
PrepareSUT.sh & None &  Commands shall be provided manually.\\
\hline
mutation\_additional\_functions.sh & run\_tst\_case  &  Implementation of the Bash function run\_tst\_case that executes the test case passed as a parameter.\\
\hline
\end{tabular}
\end{table}

The files and variables to be configured for \MASS are specified in Table~\ref{table:to_configure}. As detailed in the SUM document, these files (i.e., mass\_conf.sh, PrepareSUT.sh, mutation\_additional\_functions.sh) support \MASS to correctly process the paths of the SUT, SUT compilation commands, SUT test suite execution commands, and enable the mutation analysis process.

After configuring the three files, the following command shall be executed for launching the mutation analysis process. 

\begin{lstlisting}[language=bash]
  $ ./Launcher.sh
\end{lstlisting}

Note that this command (1) shall be executed from \MASS workspace, (2) shall automatically perform the following eight steps:

\begin{enumerate}
  \item PrepareSUT
  \item GenerateMutants
  \item CompileOptimizedMutants
  \item OptimizedPostProcessing
  \item GeneratePTS
  \item ExecuteMutants
  \item IdentifyEquivalents
  \item MutationScore
\end{enumerate}

A typical \MASS output is presented in Listing~\ref{mass:output}.

\begin{lstlisting}[language=bash, label=mass:output, caption=\MASS output.]
##### MASS Output #####
## Total mutants generated: 28071
## Total mutants filtered by TCE: 6918
## Sampling type: fsci
## Total mutants analyzed: 461
## Total killed mutants: 369
## Total live mutants: 92
## Total likely equivalent mutants: 53
## MASS mutation score (%): 90.44
## List A of useful undetected mutants: /opt/MLFS/RESULTS/useful_list_a
## List B of useful undetected mutants: /opt/MLFS/RESULTS/useful_list_b
## Number of statements covered: 1973
## Statement coverage (%): 100
## Minimum lines covered per source file: 2
## Maximum lines covered per source file: 138
\end{lstlisting}

\subsubsection{Pass-Fail criteria}

The following criteria must be respected in order to declare this task a pass:

\begin{itemize}
  \item The eight steps of the methodology shall be performed. In turn, the following shall be produced for each steps:
    \begin{itemize}
      \item PrepareSUT: this step shall produce code coverage files, and list of test cases to be executed at \texttt{COV\_FILES} folder.
      \item GenerateMutants: this step shall produce mutant files at \texttt{src-mutants} folder.
      \item CompileOptimizedMutants: inside \texttt{COMPILED} there should be one folder for each optimization level containing (1) list of SHA512 hashes for every compiled mutant, (2) list of non-compiled mutants.
      \item OptimizedPostProcessing: inside \texttt{COMPILED} there should be (1) list of equivalent mutants, (2) list of redundant mutants, (3) list of unique mutants.
      \item GeneratePTS: this step shall produce reduced and prioritized test suites for the SUT at folder \texttt{PRIORITIZED}.
      \item ExecuteMutants: inside \texttt{MUTATION} folder there should be: (1) code coverage for each mutant, (2) traces of killed and live mutants, (3) list of executed mutants.
      \item IdentifyEquivalents: inside folder \texttt{DETECTION} there should be: (1) list of equivalent mutants, (2) mutation traces without equivalent mutants.
      \item MutationScore: inside folder \texttt{RESULTS} there should be: (1) final MASS report, (2) list of killed and live mutants, (3) code coverage report, (4) list of useful mutants. 
    \end{itemize}
  \item A final report shall be reported at the end of the mutation analysis process.
  \item The MASS log shall report no error.
\end{itemize}

\subsection{Configuring \MASS and running PrepareSUT.sh - \MASS}

This validation task consists of configuring \MASS, as specified in the Section~\ref{sec:configuring:mass}, and then executing the PrepareSUT.sh command of \MASS, this command prepare the SUT and collect information about the SUT test suite. Note that this script shall be prepared by the SUT engineer. 
The objective of this task is to validate that \MASS can properly prepare the SUT and collect code coverage information about the SUT test suite, which is necessary for enabling \MASS optimizations.

\begin{lstlisting}[language=bash]
  $ ./PrepareSUT.sh
\end{lstlisting}

The command shall create a \texttt{COV\_FILES} folder inside the \MASS workspace; such folder will contain GCOV files for each test case, and a list of test cases to be executed.

\subsubsection{Pass-Fail criteria}

\begin{itemize}
  \item The execution of PrepareSUT shall be successful, that is, the script shall terminate with a 0 error code.
  \item The required output shall be observed inside \texttt{COV\_FILES} folder.
\end{itemize}

\subsection{Configuring \MASS and running GenerateMutants.sh - \MASS}

This validation task consists of configuring \MASS, as specified in the Section~\ref{sec:configuring:mass}, and then executing the GenerateMutants.sh command of \MASS, this command generates the code coverage matrices and launches the generation of mutants. A prerequisite to this task is to have executed the PrepareSUT command.
The objective of this task is to validate that \MASS can properly process code coverage information and generate code-driven mutants of the SUT.

\begin{lstlisting}[language=bash]
  $ ./GenerateMutants.sh
\end{lstlisting}

The command shall create a \texttt{src-mutants} folder inside the \MASS workspace; such folder will contain one folder for each source under test, and each of this folder shall contain a set of mutants for the source code.

A typical mutant source file will be named using (1) the name of the source file, (2) the mutated line, (3) the instance of the mutation (e.g., the AOR operator can replace a $+$ operator into four instances (a) $-$, (b) $*$, (c) $/$, and (d) $\%$) and the position in the source file, (4) the operator being used, (5) the affected function. An example of mutant would be \url{source.mut.126.1_6_19.ICR.function_1.c}

\subsubsection{Pass-Fail criteria}

\begin{itemize}
  \item The execution of GenerateMutants shall be successful, that is, the script shall terminate with a 0 error code.
  \item The required output shall be observed inside \texttt{src-mutants} folder.
\end{itemize}

\subsection{Configuring \MASS and running CompileOptimizedMutants.sh - \MASS}

This validation task consists of configuring \MASS, as specified in the Section~\ref{sec:configuring:mass}, and then executing the CompileOptimizedMutants.sh command of \MASS, this command compiles the mutants with multiple optimisation levels and generate intermediate files to be processed then by OptimizedPostProcessing. A prerequisite to this task is to have executed the GenerateMutants command.
The objective of this task is to validate that \MASS can properly compile the mutants and generate a list of SHA512 hashes for each mutant, in order to properly identify equivalent and redundant mutants during the OptimizedPostProcessing step.

\begin{lstlisting}[language=bash]
  $ ./CompileOptimizedMutants.sh
\end{lstlisting}

The command shall create a \texttt{COMPILED} folder inside the \MASS workspace; such folder will contain one folder for each optimization level containing (1) list of SHA512 hashes for every compiled mutant, (2) list of non-compiled mutants.

\subsubsection{Pass-Fail criteria}

\begin{itemize}
  \item The execution of CompileOptimizedMutants shall be successful, that is, the script shall terminate with a 0 error code.
  \item The required output shall be observed inside \texttt{COMPILED} folder.
\end{itemize}

\subsection{Configuring \MASS and running OptimizedPostProcessing.sh - \MASS}

This validation task consists of configuring \MASS, as specified in the Section~\ref{sec:configuring:mass}, and then executing the CompileOptimizedMutants.sh command of \MASS, this command iterates over the SHA512 hashes generated by the CompileOptimizedMutants command, and produce a list of unique mutants. A prerequisite to this task is to have executed the CompileOptimizedMutants command.
The objective of this task is to validate that \MASS can properly identify equivalent and redundant mutants to be then analyzed during ExecuteMutants.

\begin{lstlisting}[language=bash]
  $ ./OptimizedPostProcessing.sh
\end{lstlisting}

The command shall create (1) a list of equivalent mutants, (2) a list of redundant mutants, and (3) a list of unique mutants inside the \texttt{COMPILED} folder within the \MASS workspace.

\subsubsection{Pass-Fail criteria}

\begin{itemize}
  \item The execution of OptimizedPostProcessing shall be successful, that is, the script shall terminate with a 0 error code.
  \item The required output shall be observed inside \texttt{COMPILED} folder.
\end{itemize}

\subsection{Configuring \MASS and running GeneratePTS.sh - \MASS}

This validation task consists of configuring \MASS, as specified in the Section~\ref{sec:configuring:mass}, and then executing the CompileOptimizedMutants.sh command of \MASS, this command generates a reduced and prioritized test suite for the SUT. A prerequisite to this task is to have executed the OptimizedPostProcessing command.
The objective of this task is to validate that \MASS can properly generate a reduced and prioritized set of test cases to be used during the ExecuteMutants step.

\begin{lstlisting}[language=bash]
  $ ./GeneratePTS.sh
\end{lstlisting}

The command shall create two files, (1) a reduced, and (2) prioritized test suite for the SUT, at folder \texttt{PRIORITIZED}.

\subsubsection{Pass-Fail criteria}

\begin{itemize}
  \item The execution of GeneratePTS shall be successful, that is, the script shall terminate with a 0 error code.
  \item The required output shall be observed inside \texttt{PRIORITIZED} folder.
\end{itemize}



\subsection{Configuring generate\_template\_config.json and running generate\_direct.py - SEMuS}

% !TEX root =  ../MAIN.tex

\begin{table}[t]
\tiny
\centering
\caption{SEMuS parameters to be configured.}
\label{table:to_conf_semus}
\begin{tabular}{lp{8.5cm}}
\hline
\textbf{Parameter}  &  \textbf{Description} \\
\hline
FAQAS\_SEMU\_CASE\_STUDY\_TOPDIR &  Root folder of the case study \\
FAQAS\_SEMU\_CASE\_STUDY\_WORKSPACE &  SEMuS workspace for the case study \\
FAQAS\_SEMU\_OUTPUT\_TOPDIR & SEMuS output folder, to be placed inside the workspace \\
FAQAS\_SEMU\_GENERATED\_MUTANTS\_TOPDIR & Root folder for storing the generated mutants \\
FAQAS\_SEMU\_REPO\_ROOTDIR &  Root folder of the case study source code\\
FAQAS\_SEMU\_ORIGINAL\_SOURCE\_FILE & Path of the source file under analysis \\
FAQAS\_SEMU\_COMPILE\_COMMAND\_SPECIFIED\_SOURCE\_FILE & Name of the source file under analysis \\
FAQAS\_SEMU\_GENERATED\_MUTANTS\_DIR & Folder for storing the generated mutants for the specified source file\\
FAQAS\_SEMU\_BUILD\_CODE\_FUNC\_STR & Bash function for building the source file under analysis, to be specified in string format\\
FAQAS\_SEMU\_BUILD\_LLVM\_BC & Bash function for building the source file to LLVM bitcode \\
FAQAS\_SEMU\_META\_MU\_TOPDIR &  Root folder for the meta mutant \\
FAQAS\_SEMU\_GENERATED\_META\_MU\_SRC\_FILE &  Path of the source file (i.e., C file) of the meta mutant \\
FAQAS\_SEMU\_GENERATED\_META\_MU\_BC\_FILE &  Path of the source file (i.e., LLVM bitcode file) of the meta mutant \\
FAQAS\_SEMU\_GENERATED\_META\_MU\_MAKE\_SYM\_TOP\_DIR &  Folder for storing intermediate files for the generated inputs \\
FAQAS\_SEMU\_GENERATED\_TESTS\_TOPDIR &  Folder for storing the generated inputs \\
FAQAS\_SEMU\_TEST\_GEN\_TIMEOUT & Timeout in seconds for the test generation process \\
FAQAS\_SEMU\_HEURISTICS\_CONFIG & Configuration array for SEMu heuristics \\
FAQAS\_SEMU\_TEST\_GEN\_MAX\_MEMORY & Maximum test generation memory in MB \\
FAQAS\_SEMU\_STOP\_TG\_ON\_MEMORY\_LIMIT & Parameter to stop test generation when the memory limit is reached \\
FAQAS\_SEMU\_TG\_MAX\_MEMORY\_INHIBIT & Parameter to stop forking states when the memory limit is reached \\
\hline
\end{tabular}
\end{table}



% !TEX root =  ../MAIN.tex

\begin{table}[t]
\tiny
\centering
\caption{Test template generator parameters to be configured.}
\label{table:ttg_semus}
\begin{tabular}{lp{8.5cm}}
\hline
\textbf{Parameter}  &  \textbf{Description} \\
\hline
TYPES\_TO\_INTCONVERT &  Specify how to convert a type to int. \\
TYPES\_TO\_PRINTCODE &  Specify how to print a type.  \\
OUT\_ARGS\_NAMES & Specify the names of function arguments that are used as function output. \\
IN\_OUT\_ARGS\_NAMES & Specify the names of function arguments that are used both as function input and output \\
TYPE\_TO\_INITIALIZATIONCODE &  Specify the initialization statement of a type.\\
VOID\_ARG\_SUBSTITUTE\_TYPE & Specify the underlying type for a void pointer. \\
TYPE\_TO\_SYMBOLIC\_FIELDS\_ACCESS & Specify, for pointer parameters, the number of elements it points to. \\
\hline
\end{tabular}
\end{table}




The objective of this task is to validate that \SEMUS can properly generate test templates that can be processed by the underlying test generation tool (i.e., KLEE).

The files and variables to be configured for \SEMUS are specified in Tables~\ref{table:to_conf_semus} and~\ref{table:ttg_semus}. As detailed in the SUM document, these files (i.e., generate\_template\_config.json and running generate\_direct.py) support \SEMUS to correctly process the paths of the SUT, SUT compilation commands, the configuration of \SEMUS itself, and how to print the values of the functions under test, so \SEMUS can determine if a mutant has been killed.

After configuring the two files, the following command shall be executed for launching the test template generation. 

\begin{lstlisting}[language=bash]
 $ case_studies/$SUT/util_codes/generate_direct.py ../WORKSPACE/DOWNLOADED/casestudy/test.c direct \
                    " -I../WORKSPACE/DOWNLOADED/casestudy/" -c generate_template_config.json
\end{lstlisting}

Note that this command shall generate inside the directory \texttt{case\_studies/\$SUT/util\_codes} one folder for each source under analysis, and inside of these folders, one template for each function under test.

\subsubsection{Pass-Fail criteria}

\begin{itemize}
  \item The test templates shall be generated.
  \item The generate\_direct.py shall terminate with a 0 error code.
\end{itemize}

\subsection{Instrumenting the source code - \DAMA}
\label{subsec:instrumenting}

The variables for running the \DAMA pipeline on ESAIL must be set in the \emph{\DAMA\_configure.sh} file, as reported in Listing~\ref{lst:configure_esail}. The significance of these variables is described in the SUM.

Then the commands represented in Listing~\ref{lst:instrument_esail_cmds} must be run to generate the mutation probes.

\begin{lstlisting}[language=bash, label={lst:instrument_esail_cmds}]
bash DAMA_probe_generation.sh
\end{lstlisting}

The generated probes must be inserted in the target function, as reported in the SUM (Chapter 13).
The Software Under Test (LibParam or ESAIL) shall be compiled with the macro \texttt{-DMUTATIONOPT=-1} enabled.

\subsubsection{Pass-Fail criteria}

The following criteria must be respected in order to declare this task a pass:
\begin{itemize}
  \item The compilation shall be successful.
  \item The compilation log shall report no error.
\end{itemize}

\subsection{Configuring and running \emph{\DAMA\_compile.sh}}

A prerequisite for this task is having successfully performed \EMPH{Instrumenting the source code - \DAMA}, as described in Section~\ref{subsec:instrumenting}.

\subsubsection{ESAIL}

The commands for compiling the SVF, reported in Listing~\ref{lst:compile_esail} must be included in the \emph{\DAMA\_compile.sh} script in the appropriate section as described in the SUM (Chapter 13).


  \begin{lstlisting}[language=bash, label={lst:compile_esail}]

  compilation_folder="/home/svf/Svf"

  pushd $compilation_folder

  make install-debug

      if [ $? -eq 0 ]; then
          echo $x " compilation OK"
      else
          echo $x " compilation FAILED"
      fi

  popd

  \end{lstlisting}

Then the commands represented in Listing~\ref{lst:compile_esail_cmds} must be run.

  \begin{lstlisting}[language=bash, label={lst:compile_esail_cmds}]

  bash \DAMA_compile.sh "0" "TRUE"

  \end{lstlisting}

\subsubsection{LibParam}

The commands for configuring the LibParam test suite, reported in Listing~\ref{lst:compile_param} must be included in the \emph{\DAMA\_compile.sh} script in the appropriate section as described in the SUM (Chapter 13


\begin{lstlisting}[language=bash, label={lst:compile_param}]

TEST_FOLDER="/home/csp/libparam/tst"

pushd $TEST_FOLDER

for f in *; do
    if [ -d "$f" ] && [ "$f" != "include" ]; then

        pushd $f
        echo "cleaning..."
        ./waf clean

        echo "configuring..."
        if [ $singleton == "TRUE" ]; then
        ./waf configure --mutation-opt $mutant_id --singleton $singleton
        else
        ./waf configure --mutation-opt $mutant_id
        fi

        if [ $? -eq 0 ]; then
            echo $x " configuration OK"
        else
            echo $x " configuration FAILED"
        fi
        popd
    fi
done

popd

\end{lstlisting}

Then the commands represented in Listing~\ref{lst:compile_param_cmds} must be run.

  \begin{lstlisting}[language=bash, label={lst:compile_param_cmds}]

  bash \DAMA_compile.sh "0" "TRUE"

  \end{lstlisting}

\subsubsection{Pass-Fail criteria}

The following criteria must be respected in order to declare this task a pass:
\begin{itemize}
  \item The compilation log shall report no error.
  \item The compilation log shall report the phrase \texttt{compilation OK}.
\end{itemize}


\subsection{Configuring and running \emph{\DAMA\_run\_test.sh}}

A prerequisite for this task is having successfully performed \EMPH{Instrumenting the source code - \DAMA}, as described in Section~\ref{subsec:instrumenting}.

\subsubsection{ESAIL}

The commands for running ESAIL's test suite, reported in Listing~\ref{lst:run_esail} must be included in the \emph{\DAMA\_run\_test.sh} script in the appropriate section as described in the SUM (Chapter 13).


  \begin{lstlisting}[language=bash, label={lst:run_esail}]

  ESAIL=/home/svf/Obsw/Test/lib/esail.sh
  PARSE_RESULTS=/home/svf/Obsw/Test/lib/parse_results.sh
  echo -n "${mutant_id};COMPILED;${tst};" >> $results_file

  timeout $TIMEOUT $ESAIL --obsw /home/svf/Obsw/Source/_binaries/OBSW.exe --fast -n -c --source /home/svf/Obsw/Source --version 04010000 -t $tst &
  ESAIL_PID=$!

  wait $ESAIL_PID
  EXEC_RET_CODE=$?

  mutant_end_time=$(($(date +%s%N)/1000000))
  mutant_elapsed="$(($mutant_end_time-$mutant_start_time))"

  \end{lstlisting}

Then the commands represented in Listing~\ref{lst:run_esail_cmds} must be run.

  \begin{lstlisting}[language=bash, label={lst:run_esail_cmds}]

  bash \DAMA_compile.sh "-1" "TRUE" && bash \DAMA_run_tests.sh "-1" "./test_list.csv" "./"

  \end{lstlisting}

The file \texttt{test\_list.csv} must contain the list of all test cases, as exposed in Listings~\ref{lst:test_esail}.

  \begin{lstlisting}[label={lst:test_esail}]

  731;46287
  6343;6161251
  6000;535195
  612;50098
  791;27125
  767;67364
  743;71586
  683;66213
  1841;102287
  3045;60261
  2561;164595
  719;74033
  5702;280961
  2819;203036
  695;62058
  707;44825
  1222;94890
  2435;134397
  755;53084
  779;156284
  543;71224
  6616;35423
  524;32621
  561;27344
  497;31588
  943;25435
  3394;63879
  5176;30804
  5656;34471
  5218;38539
  4361;63238
  2733;127361
  6161;562902
  6262;100839
  3077;83077
  6117;61088
  2848;421200
  4497;156531
  3590;190508
  2985;265498
  5917;127059
  3004;101740
  3917;200336
  2890;522174
  3034;507824
  5374;276736
  2867;441252
  2753;187527
  2928;311267
  3023;79797
  3610;1283052
  2947;995408
  2909;290593
  2966;133739
  3651;293401
  3853;132605
  6138;760249
  3621;46376
  6600;51881
  5190;196980
  4294;46522
  4270;35237
  6583;144017
  6275;143658
  5393;47192
  2604;45503
  2411;83456
  3950;39675
  146;26984
  140;23420
  3969;39893
  113;28140
  74;26170
  154;22556
  126;24476
  1465;68582
  1377;633878
  1364;47647
  2327;254152
  2339;83202
  2395;284150
  6630;228215
  2254;36753
  6362;102271
  2277;53371
  5570;32536
  160;27916
  659;98164
  6315;25372
  1242;645102
  647;27085
  225;27339
  6401;58932
  1807;56712
  4931;46262
  6038;63676
  1505;2975291
  671;84703
  990;34764
  2007;38421
  1702;86591
  5791;90984
  3996;28060
  3358;113929
  5606;31639
  1545;28645
  1572;135209
  1742;26743
  3340;136863
  197;98851
  2375;50526
  4815;74190
  1293;27614
  3313;66150
  5645;26901
  3376;1068552
  3291;584423
  2093;189396
  953;63629
  6330;25552
  6736;2446525

  \end{lstlisting}

\subsubsection{LibParam}

The commands for running the LibParam test suite, reported in Listing~\ref{lst:run_param} must be included in the \emph{\DAMA\_run\_test.sh} script in the appropriate section as described in the SUM (Chapter 13).


  \begin{lstlisting}[language=bash, label={lst:run_param}]

  tmp_log="$results_dir"/tmp_log

    TEST_FOLDER="/home/csp/libparam/tst"

    pushd $TEST_FOLDER

    pushd $tst

    touch $tmp_log

    timeout $TIMEOUT ./waf --mutation-opt=$mutant_id  --singleton=TRUE 2>&1 | tee $tmp_log

    EXEC_RET_CODE=$?

    mutant_end_time=$(($(date +%s%N)/1000000))
    mutant_elapsed="$(($mutant_end_time-$mutant_start_time))"

    if [ $EXEC_RET_CODE -ne 124 ]; then
        if grep "successfully" $tmp_log
        then
            EXEC_RET_CODE=0
            echo "PASSED"
        else
            EXEC_RET_CODE=1
            echo "FAILED"
        fi
    fi
    popd
    rm $tmp_log
    popd

  \end{lstlisting}

Then the commands represented in Listing~\ref{lst:run_param_cmds} must be run.

% mutant_id=$1
% tests_list=$2
% \DAMA_FOLDER=$3

  \begin{lstlisting}[language=bash, label={lst:run_param_cmds}]

  bash \DAMA_compile.sh "-1" "TRUE" && bash \DAMA_run_tests.sh "-1" "./test_list.csv" "./"

  \end{lstlisting}

The file \texttt{test\_list.csv} must contain the list of all test cases, as exposed in Listings~\ref{lst:test_param}.

  \begin{lstlisting}[label={lst:test_param}]

  bindings,4427.200000
  example,4145.400000
  file_store,2954.000000
  i2c,2793.400000
  log,2881.200000
  param3,3007.800000
  param4,3273.600000
  rparam3,5099.400000
  rparam4,8003.200000
  serialize,2656.400000
  spi,2825.800000
  store,2774.400000
  store_load,3182.000000
  vmem_store_checksum_first,3005.800000
  vmem_store_checksum_last,2963.400000

  \end{lstlisting}

  \subsubsection{Pass-Fail criteria}

  The following criteria must be respected in order to declare this task a pass:

  \begin{itemize}
    \item All test cases in Listing~\ref{lst:test_param} (for LibParam) and Listing~\ref{lst:test_esail} (for ESAIL) shall be executed.
    \item All test cases listed in Listing~\ref{lst:test_param} (for LibParam) and Listing~\ref{lst:test_esail} (for ESAIL) shall pass.
  \end{itemize}

\subsection{Configuring and running \emph{\DAMA\_obtain\_coverage.sh}}

A prerequisite for this task is having successfully performed \EMPH{Instrumenting the source code - \DAMA}, as described in Section~\ref{subsec:instrumenting}.

\subsubsection{ESAIL}

The variables for running the \DAMA-pipeline on ESAIL must be set in the \emph{\DAMA\_configure.sh} file, as reported in Listing~\ref{lst:configure_esail}. The significance of these variables is described in the SUM (Chapter 13).

  \begin{lstlisting}[language=bash, label={lst:configure_esail}]

  # the location of the csv with all the test identifiers and the execution time
  tests_list=$\DAMA_FOLDER/tests.csv

  # the location of the csv containing the definitions of the mutation operators
  fault_model=$\DAMA_FOLDER/fault_model.csv

  # the datatype of the elements of the target buffer
  buffer_type="unsigned long long int"

  # padding: can be used to skip the first n bit of a buffer, normally set to 0
  padding=0

  # singleton: can set to true to load the fault model into a singleton   variable, normally set to "TRUE", can also  be set to "FALSE"
  singleton="TRUE"

  \end{lstlisting}

Then the commands represented in Listing~\ref{lst:coverage_esail_cmds} must be run.

  \begin{lstlisting}[language=bash, label={lst:coverage_esail_cmds}]

  bash \DAMA_obtain_coverage.sh

  \end{lstlisting}

\subsubsection{LibParam}

The variables for running the \DAMA-pipeline on ESAIL must be set in the \emph{\DAMA\_configure.sh} file, as reported in Listing~\ref{lst:configure_param}. The significance of these variables is described in the SUM (Chapter 13).

  \begin{lstlisting}[language=bash, label={lst:configure_param}]

  # the location of the csv with all the test identifiers and the execution time
  tests_list=$\DAMA_FOLDER/tests_param.csv

  # the location of the csv containing the definitions of the mutation operators
  fault_model=$\DAMA_FOLDER/LIBP-FM.csv

  # the datatype of the elements of the target buffer
  buffer_type="unsigned long long int"

  # padding: can be used to skip the first n bit of a buffer, normally set to 0
  padding=0

  # singleton: can set to true to load the fault model into a singleton variable, normally set to "TRUE", can also  be set to "FALSE"
  singleton="TRUE"

  \end{lstlisting}

Then the commands represented in Listing~\ref{lst:coverage_param_cmds} must be run.

  \begin{lstlisting}[language=bash, label={lst:coverage_param_cmds}]

  bash pipeline_scripts/\DAMA_obtain_coverage.sh ./

  \end{lstlisting}

\subsubsection{Pass-Fail criteria}

These criteria must be respected to consider this task a pass:
\begin{itemize}
  \item The compilation lof shall show no error.
  \item The compilation log shall report the phrase \texttt{compilation OK}.
  \item No test cases shall fail.
  \item A \texttt{testlist} folder shall be generated. Inside this folder there shall be a file called \texttt{test\_<mutant>} for every mutant.
  \item Every file shall contain a subset of the tests listed in Listing~\ref{lst:test_param} (for LibParam) or in Listing~\ref{lst:test_esail} (for ESAIL).
\end{itemize}




\subsection{Configuring and running \emph{\DAMA\_mutants\_launcher.sh}}

A prerequisite for this task is having successfully performed \EMPH{Instrumenting the source code - \DAMA}, as described in Section~\ref{subsec:instrumenting}.

\subsubsection{ESAIL}

The variables for running the \DAMA-pipeline on ESAIL must be set in the \emph{\DAMA\_configure.sh} file, as reported in Listing~\ref{lst:configure_esail}. The significance of these variables is described in the SUM (Chapter 13).

Then the commands represented in Listing~\ref{lst:launcher_esail_cmds} must be run.

  \begin{lstlisting}[language=bash, label={lst:launcher_esail_cmds}]

  bash \DAMA_mutants_launcher.sh ./

  \end{lstlisting}

  The compilation logs shall report the phrase \texttt{compilation OK}.
  A \texttt{testlist} folder shall be generated. Inside this folder there shall be a file called \texttt{test\_<mutationID>} for every mutant containing a subset of the tests listed in Listing~\ref{lst:test_esail}.
  A \texttt{results} folder shall be generated and it shall contain the files described in the SUM (Section 9.2).

\subsubsection{LibParam}

The variables for running the \DAMA-pipeline on ESAIL must be set in the \emph{\DAMA\_configure.sh} file, as reported in Listing~\ref{lst:configure_param}. The significance of these variables is described in the SUM (Chapter 13).

Then the commands represented in Listing~\ref{lst:launcher_param_cmds} must be run.

  \begin{lstlisting}[language=bash, label={lst:launcher_param_cmds}]

  bash \DAMA_mutants_launcher.sh ./

  \end{lstlisting}

\subsubsection{Pass-Fail criteria}

The following criteria shall be respected in order to consider this task a pass:

\begin{itemize}
  \item The compilation logs shall report the phrase \texttt{configuration OK}.
  \item A \texttt{testlist} folder shall be generated.
  \item Inside this folder there shall be a file called \texttt{test\_<mutationID>} for every mutant containing a subset of the tests listed in Listing~\ref{lst:test_param} for LibParam or in in Listing~\ref{lst:test_esail} for ESAIL.
  \item A \texttt{results} folder shall be generated and it shall contain the files described in the SUM (Section 9.2).
\end{itemize}

\ENDCHANGEDFINAL

\section{Features to be tested}
The scope of the SVS includes all baseline requirements expressed in the SSS.

% \section{Features not to be tested}
% The SVS w.r.t. TS or RB shall describe all the features and significant
% combinations not to be tested.

\section{Test pass and fail criteria}
The pass-fail criteria for the tasks belonging to the \emph{unit testing} validation approach are detailed in the SUTP.

Regarding the tasks belonging to the \emph{application to the case studies} validation approach, the pass criteria are described in Section~\ref{sec:case_studies}.


% \section{Items that cannot be validated by test}
% a. The SVS w.r.t. TS or RB shall list the tasks and items under tests that
% cannot be validated by a test.
% b. Each of them shall be properly justified
% c. For each of them, an analysis, inspection, or review of design shall be
% proposed.
%
% % \section{Manually and automatically generated code}
% % a. The SVS shall address separately the activities to be performed for
% % manually and automatically generated code, although they have the
% % same objective (ECSS‐Q‐ST‐80 clause 6.2.8.2 and 6.2.8.7).
%
% \chapter{Software validation testing specification design}
%
% \section{General}
% a. The SVS w.r.t. TS or RB shall define software
% validation testing specification design, giving the design grouping
% criteria such as function, component, or equipment management.
% b. For each identified test design, the SVS w.r.t. TS or RB shall provide the
% information given in <6.2>.
%
% \section{Organization of each identified test design}
%
% NOTE The SVS w.r.t. TS or RB defines each validation

\clearpage
