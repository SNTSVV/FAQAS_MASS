% !TEX root = MAIN.tex

\chapter{Introduction}

The purpose of this  Software Validation Specification is to describe the testing, analysis, inspection and
review of design specifications, and is used to document the software validation specification concerning the requirements baseline.

\chapter{Applicable and reference documents}

\begin{itemize}
\item{D1 - Mutation testing survey}
\item{D2 - Study of mutation testing applicability to space software}
\item{D4 - Validation of the toolset}
\item{SSS - Software System Specification}
\item{SUM - Software User Manual}
\item{SUTP - Software Unit Test Plan}

\end{itemize}

\chapter{Terms, definitions, and abbreviated terms}

\begin{itemize}
\item{FAQAS}: activity ITT-1-9873-ESA
\item{FAQAS-framework}: software system to be released at the end of WP4 of FAQAS
\item{D2}: Deliverable D2 of FAQAS, \emph{Study of mutation testing applicability to space software}
\item{D4}: Deliverable D4 of FAQAS, \emph{Validation of the toolset}
\item{SSS}: Software System Specification.
\item{KLEE}: Third-party test generation tool, details are provided in D2.
\item{MLFS}: Mathematical Library for Flight Software.
\item{SUT}: Software under test, i.e, the software that should be mutated employing mutation testing.
\item{SUM}: Software User Manual.
\item{SUTP}: Software Unit Test Plan
\item{WP}: Work package.

\end{itemize}

\chapter{Software Overview}

This documents concern the components of the FAQAS framework:
\begin{itemize}
  \item MASS
  \item DAMAt
  \item SEMuS
%  \item DAMTE
\end{itemize}

For detailed information on their structure and usage see D2 and SUM.

\chapter{Software validation specification task identification}

\section{Task and criteria}
\label{sec:taskCrit}

The validation tasks are meant to ensure that the requirement expressed in the SSS are met.
They are accomplished by following validation approaches:
\begin{itemize}
  \item veryfing the requirements for the most critical and complex software components througt \EMPH{unit testing} as detailed in the SUTP.
  \emph{unit testing} is composed of three tasks:
  \begin{itemize}
    \item performing \emph{DAMAt-TD-DDMutation-1}
    \item performing \emph{DAMAt-TD-DDMutation-2}
    \item performing \emph{MASS-TD-SRCMutation-1}
  \end{itemize}
  \item verifying the requirements for the rest of the components through \EMPH{application to the case studies} as detailed in the D4 and the SUM.
  Application to the case studies is composed of the following tasks, detailed in \ref{sec:case_studies}
  \begin{itemize}
    \item \emph{instrumenting the source code}
    \item \emph{configuring and running} a given component, for example \texttt{DAMAt\_compile.sh}.
  \end{itemize}
  \item verifying general requirements through \EMPH{manual inspection}. For this approach no further specific task can be identified.
\end{itemize}

\section{Application to the case studies}
\label{sec:case_studies}

% \subsection{Instrumenting the source code - DAMAt}
%
% \subsubsection{ESAIL}
%
% The variables for running the DAMAt-pipeline on ESAIL must be set in the \emph{DAMAt\_configure.sh} file, as reported in Listing~\ref{lst:configure_esail}. The significance of these variables is described in the SUM.
%
%   \begin{lstlisting}[language=bash, label={lst:configure_esail}]
%
%   # the location of the csv with all the test identifiers and the execution time
%   tests_list=$DAMAt_FOLDER/tests.csv
%
%   # the location of the csv containing the definitions of the mutation operators
%   fault_model=$DAMAt_FOLDER/fault_model.csv
%
%   # the datatype of the elements of the target buffer
%   buffer_type="unsigned long long int"
%
%   # padding: can be used to skip the first n bit of a buffer, normally set to 0
%   padding=0
%
%   # singleton: can set to true to load the fault model into a singleton variable, normally set to "TRUE", can also  be set to "FALSE"
%   singleton="TRUE"
%
%   \end{lstlisting}
%
% Then the commands represented in Listing~\ref{lst:instrument_esail_cmds} must be run to generate the mutation probes.
%
%   \begin{lstlisting}[language=bash, label={lst:compile_esail_cmds}]
%
%   bash DAMAt_probe_generation.sh
%
%   \end{lstlisting}
%
%
%
% \subsubsection{LibParam}

\subsection{Configuring and running \emph{DAMAt\_compile.sh}}

\subsubsection{ESAIL}

The commands for compiling the SVF, reported in Listing~\ref{lst:instrument_esail_cmds} must be included in the \emph{DAMAt\_compile.sh} script in the appropriate section as described in the SUM (Chapter 13).


  \begin{lstlisting}[language=bash, label={lst:compile_esail}]

  compilation_folder="/home/svf/Svf"

  pushd $compilation_folder

  make install-debug

      if [ $? -eq 0 ]; then
          echo $x " compilation OK"
      else
          echo $x " compilation FAILED"
      fi

  popd

  \end{lstlisting}

Then the commands represented in Listing~\ref{lst:compile_esail_cmds} must be run.

  \begin{lstlisting}[language=bash, label={lst:compile_esail_cmds}]

  bash DAMAt_compile.sh "0" "TRUE"

  \end{lstlisting}

\subsubsection{LibParam}

The commands for configuring the LibParam test suite, reported in Listing~\ref{lst:compile_param} must be included in the \emph{DAMAt\_compile.sh} script in the appropriate section as described in the SUM (Chapter 13


\begin{lstlisting}[language=bash, label={lst:compile_param}]

TEST_FOLDER="/home/csp/libparam/tst"

pushd $TEST_FOLDER

for f in *; do
    if [ -d "$f" ] && [ "$f" != "include" ]; then

        pushd $f
        echo "cleaning..."
        ./waf clean

        echo "configuring..."
        if [ $singleton == "TRUE" ]; then
        ./waf configure --mutation-opt $mutant_id --singleton $singleton
        else
        ./waf configure --mutation-opt $mutant_id
        fi

        if [ $? -eq 0 ]; then
            echo $x " configuration OK"
        else
            echo $x " configuration FAILED"
        fi
        popd
    fi
done

popd

\end{lstlisting}

Then the commands represented in Listing~\ref{lst:compile_param_cmds} must be run.

  \begin{lstlisting}[language=bash, label={lst:compile_param_cmds}]

  bash DAMAt_compile.sh "0" "TRUE"

  \end{lstlisting}

\subsubsection{Pass-Fail criteria}

The compilation log shall report no error.
The compilation log shall report the phrase \texttt{compilation OK}.

\subsection{Configuring and running \emph{DAMAt\_run\_test.sh}}

\subsubsection{ESAIL}

The commands for running ESAIL's test suite, reported in Listing~\ref{lst:run_esail} must be included in the \emph{DAMAt\_run\_test.sh} script in the appropriate section as described in the SUM (Chapter 13).


  \begin{lstlisting}[language=bash, label={lst:run_esail}]

  ESAIL=/home/svf/Obsw/Test/lib/esail.sh
  PARSE_RESULTS=/home/svf/Obsw/Test/lib/parse_results.sh
  echo -n "${mutant_id};COMPILED;${tst};" >> $results_file

  timeout $TIMEOUT $ESAIL --obsw /home/svf/Obsw/Source/_binaries/OBSW.exe --fast -n -c --source /home/svf/Obsw/Source --version 04010000 -t $tst &
  ESAIL_PID=$!

  wait $ESAIL_PID
  EXEC_RET_CODE=$?

  mutant_end_time=$(($(date +%s%N)/1000000))
  mutant_elapsed="$(($mutant_end_time-$mutant_start_time))"

  \end{lstlisting}

Then the commands represented in Listing~\ref{lst:run_esail_cmds} must be run.

  \begin{lstlisting}[language=bash, label={lst:run_esail_cmds}]

  bash DAMAt_compile.sh "-1" "TRUE" && bash DAMAt_run_tests.sh "-1" "./test_list.csv" "./"

  \end{lstlisting}

The file \texttt{test\_list.csv} must contain the list of all test cases, as exposed in Listings~\ref{lst:test_esail}.

  \begin{lstlisting}[label={lst:test_esail}]

  731;46287
  6343;6161251
  6000;535195
  612;50098
  791;27125
  767;67364
  743;71586
  683;66213
  1841;102287
  3045;60261
  2561;164595
  719;74033
  5702;280961
  2819;203036
  695;62058
  707;44825
  1222;94890
  2435;134397
  755;53084
  779;156284
  543;71224
  6616;35423
  524;32621
  561;27344
  497;31588
  943;25435
  3394;63879
  5176;30804
  5656;34471
  5218;38539
  4361;63238
  2733;127361
  6161;562902
  6262;100839
  3077;83077
  6117;61088
  2848;421200
  4497;156531
  3590;190508
  2985;265498
  5917;127059
  3004;101740
  3917;200336
  2890;522174
  3034;507824
  5374;276736
  2867;441252
  2753;187527
  2928;311267
  3023;79797
  3610;1283052
  2947;995408
  2909;290593
  2966;133739
  3651;293401
  3853;132605
  6138;760249
  3621;46376
  6600;51881
  5190;196980
  4294;46522
  4270;35237
  6583;144017
  6275;143658
  5393;47192
  2604;45503
  2411;83456
  3950;39675
  146;26984
  140;23420
  3969;39893
  113;28140
  74;26170
  154;22556
  126;24476
  1465;68582
  1377;633878
  1364;47647
  2327;254152
  2339;83202
  2395;284150
  6630;228215
  2254;36753
  6362;102271
  2277;53371
  5570;32536
  160;27916
  659;98164
  6315;25372
  1242;645102
  647;27085
  225;27339
  6401;58932
  1807;56712
  4931;46262
  6038;63676
  1505;2975291
  671;84703
  990;34764
  2007;38421
  1702;86591
  5791;90984
  3996;28060
  3358;113929
  5606;31639
  1545;28645
  1572;135209
  1742;26743
  3340;136863
  197;98851
  2375;50526
  4815;74190
  1293;27614
  3313;66150
  5645;26901
  3376;1068552
  3291;584423
  2093;189396
  953;63629
  6330;25552
  6736;2446525

  \end{lstlisting}

\subsubsection{LibParam}

The commands for running the LibParam test suite, reported in Listing~\ref{lst:run_param} must be included in the \emph{DAMAt\_run\_test.sh} script in the appropriate section as described in the SUM (Chapter 13).


  \begin{lstlisting}[language=bash, label={lst:run_param}]

  tmp_log="$results_dir"/tmp_log

    TEST_FOLDER="/home/csp/libparam/tst"

    pushd $TEST_FOLDER

    pushd $tst

    touch $tmp_log

    timeout $TIMEOUT ./waf --mutation-opt=$mutant_id  --singleton=TRUE 2>&1 | tee $tmp_log

    EXEC_RET_CODE=$?

    mutant_end_time=$(($(date +%s%N)/1000000))
    mutant_elapsed="$(($mutant_end_time-$mutant_start_time))"

    if [ $EXEC_RET_CODE -ne 124 ]; then
        if grep "successfully" $tmp_log
        then
            EXEC_RET_CODE=0
            echo "PASSED"
        else
            EXEC_RET_CODE=1
            echo "FAILED"
        fi
    fi
    popd
    rm $tmp_log
    popd

  \end{lstlisting}

Then the commands represented in Listing~\ref{lst:run_param_cmds} must be run.

% mutant_id=$1
% tests_list=$2
% DAMAt_FOLDER=$3

  \begin{lstlisting}[language=bash, label={lst:run_param_cmds}]

  bash DAMAt_compile.sh "-1" "TRUE" && bash DAMAt_run_tests.sh "-1" "./test_list.csv" "./"

  \end{lstlisting}

The file \texttt{test\_list.csv} must contain the list of all test cases, as exposed in Listings~\ref{lst:test_param}.

  \begin{lstlisting}[label={lst:test_param}]

  bindings,4427.200000
  example,4145.400000
  file_store,2954.000000
  i2c,2793.400000
  log,2881.200000
  param3,3007.800000
  param4,3273.600000
  rparam3,5099.400000
  rparam4,8003.200000
  serialize,2656.400000
  spi,2825.800000
  store,2774.400000
  store_load,3182.000000
  vmem_store_checksum_first,3005.800000
  vmem_store_checksum_last,2963.400000

  \end{lstlisting}

  \subsubsection{Pass-Fail criteria}

  All test cases in Listing~\ref{lst:test_param} (for LibParam) and Listing~\ref{lst:test_esail} shall be executed.
  All test cases listed in Listing~\ref{lst:test_param} (for LibParam) and Listing~\ref{lst:test_esail} (for ESAIL) shall pass.

\subsection{Configuring and running \emph{DAMAt\_obtain\_coverage.sh}}

\subsubsection{ESAIL}

The variables for running the DAMAt-pipeline on ESAIL must be set in the \emph{DAMAt\_configure.sh} file, as reported in Listing~\ref{lst:configure_esail}. The significance of these variables is described in the SUM (Chapter 13).

  \begin{lstlisting}[language=bash, label={lst:configure_esail}]

  # the location of the csv with all the test identifiers and the execution time
  tests_list=$DAMAt_FOLDER/tests.csv

  # the location of the csv containing the definitions of the mutation operators
  fault_model=$DAMAt_FOLDER/fault_model.csv

  # the datatype of the elements of the target buffer
  buffer_type="unsigned long long int"

  # padding: can be used to skip the first n bit of a buffer, normally set to 0
  padding=0

  # singleton: can set to true to load the fault model into a singleton   variable, normally set to "TRUE", can also  be set to "FALSE"
  singleton="TRUE"

  \end{lstlisting}

Then the commands represented in Listing~\ref{lst:coverage_esail_cmds} must be run.

  \begin{lstlisting}[language=bash, label={lst:coverage_esail_cmds}]

  bash DAMAt_obtain_coverage.sh

  \end{lstlisting}

\subsubsection{LibParam}

The variables for running the DAMAt-pipeline on ESAIL must be set in the \emph{DAMAt\_configure.sh} file, as reported in Listing~\ref{lst:configure_param}. The significance of these variables is described in the SUM (Chapter 13).

  \begin{lstlisting}[language=bash, label={lst:configure_param}]

  # the location of the csv with all the test identifiers and the execution time
  tests_list=$DAMAt_FOLDER/tests_param.csv

  # the location of the csv containing the definitions of the mutation operators
  fault_model=$DAMAt_FOLDER/LIBP-FM.csv

  # the datatype of the elements of the target buffer
  buffer_type="unsigned long long int"

  # padding: can be used to skip the first n bit of a buffer, normally set to 0
  padding=0

  # singleton: can set to true to load the fault model into a singleton variable, normally set to "TRUE", can also  be set to "FALSE"
  singleton="TRUE"

  \end{lstlisting}

Then the commands represented in Listing~\ref{lst:coverage_param_cmds} must be run.

  \begin{lstlisting}[language=bash, label={lst:coverage_param_cmds}]

  bash pipeline_scripts/DAMAt_obtain_coverage.sh ./

  \end{lstlisting}

\subsubsection{Pass-Fail criteria}

The compilation lof shall show no error.
The compilation log shall report the phrase \texttt{compilation OK}.
No test cases shall fail.
A \texttt{testlist} folder shall be generated. Inside this folder there shall be a file called \texttt{test\_<mutant>} for every mutant.
Every file shall contain a subset of the tests listed in Listing~\ref{lst:test_param} (for LibParam) or in Listing~\ref{lst:test_esail} (for ESAIL).

\subsection{Configuring and running \emph{DAMAt\_mutants\_launcher.sh}}

\subsubsection{ESAIL}

The variables for running the DAMAt-pipeline on ESAIL must be set in the \emph{DAMAt\_configure.sh} file, as reported in Listing~\ref{lst:configure_esail}. The significance of these variables is described in the SUM (Chapter 13).

Then the commands represented in Listing~\ref{lst:launcher_esail_cmds} must be run.

  \begin{lstlisting}[language=bash, label={lst:launcher_esail_cmds}]

  bash DAMAt_mutants_launcher.sh ./

  \end{lstlisting}

  The compilation logs shall report the phrase \texttt{compilation OK}.
  A \texttt{testlist} folder shall be generated. Inside this folder there shall be a file called \texttt{test\_<mutationID>} for every mutant containing a subset of the tests listed in Listing~\ref{lst:test_esail}.
  A \texttt{results} folder shall be generated and it shall contain the files described in the SUM (Section 9.2).

\subsubsection{LibParam}

The variables for running the DAMAt-pipeline on ESAIL must be set in the \emph{DAMAt\_configure.sh} file, as reported in Listing~\ref{lst:configure_param}. The significance of these variables is described in the SUM (Chapter 13).

Then the commands represented in Listing~\ref{lst:launcher_param_cmds} must be run.

  \begin{lstlisting}[language=bash, label={lst:launcher_param_cmds}]

  bash DAMAt_mutants_launcher.sh ./

  \end{lstlisting}

\subsubsection{Pass-Fail criteria}

The compilation logs shall report the phrase \texttt{configuration OK}.
A \texttt{testlist} folder shall be generated. Inside this folder there shall be a file called \texttt{test\_<mutationID>} for every mutant containing a subset of the tests listed in Listing~\ref{lst:test_param} for LibParam or in in Listing~\ref{lst:test_esail} for ESAIL.
A \texttt{results} folder shall be generated and it shall contain the files described in the SUM (Section 9.2).

\section{Features to be tested}
The scope of the SVS includes all baseline requirements expressed in the SSS.

% \section{Features not to be tested}
% The SVS w.r.t. TS or RB shall describe all the features and significant
% combinations not to be tested.

\section{Test pass and fail criteria}
The pass-fail criteria for the tasks belonging to the \emph{unit testing} validation approach are detailed in the SUTP.

Regarding the tasks belonging to the \emph{application to the case studies} validation approach, the pass criteria are described in Section~\ref{sec:case_studies}.


% \section{Items that cannot be validated by test}
% a. The SVS w.r.t. TS or RB shall list the tasks and items under tests that
% cannot be validated by a test.
% b. Each of them shall be properly justified
% c. For each of them, an analysis, inspection, or review of design shall be
% proposed.
%
% % \section{Manually and automatically generated code}
% % a. The SVS shall address separately the activities to be performed for
% % manually and automatically generated code, although they have the
% % same objective (ECSS‐Q‐ST‐80 clause 6.2.8.2 and 6.2.8.7).
%
% \chapter{Software validation testing specification design}
%
% \section{General}
% a. The SVS w.r.t. TS or RB shall define software
% validation testing specification design, giving the design grouping
% criteria such as function, component, or equipment management.
% b. For each identified test design, the SVS w.r.t. TS or RB shall provide the
% information given in <6.2>.
%
% \section{Organization of each identified test design}
%
% NOTE The SVS w.r.t. TS or RB defines each validation

\clearpage
