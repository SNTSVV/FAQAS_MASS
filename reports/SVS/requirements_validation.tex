% !TEX root =  MAIN.tex
\chapter{Software validation testing specification design}

\section{General}

In the following, we specify the validation procedures adopted for every requirement. Each of the following Sections has a name that matches the name of a Chapter in the SSS. In each Section we thus specify how the requirements described in the SSS Chapter had been validated.
Every requirement is univocally identified. In general, we provide tables with the list of validation procedures adopted.
The entries written inside columns \emph{validation approach} and \emph{task} match the ones described in Section~\ref{sec:taskCrit} and Section~\ref{sec:case_studies}.

\section{Software Overview}
The Chapter \emph{Software Overview} in SSS (Chapter 3) describes requirements concerning the general capabilities of the software, the general constraints, and the software architecture.
The testing procedures are detailed in Table~\ref{tables:overview}.

% !TEX root =  ../MAIN.tex

\begin{table}[h!]
\resizebox{\textwidth}{!}{%
\begin{tabular}{|l|l|l|}
  \hline
  \textbf{Subsection} &  \textbf{Requirement} &  \textbf{Validation Approach} \\
 General Capabilities & FAQAS-SSS-REQ-001 & requirement validation \\
 General Capabilities & FAQAS-SSS-REQ-002 & requirement validation \\
 General Capabilities & FAQAS-SSS-REQ-003 & requirement validation \\
 General Capabilities & FAQAS-SSS-REQ-004 & requirement validation \\
 General Capabilities & FAQAS-SSS-REQ-005 & requirement validation \\
 General Capabilities & FAQAS-SSS-REQ-006 & requirement validation \\
 General Constraints & FAQAS-SSS-REQ-007 & requirement validation \\
 General Constraints & FAQAS-SSS-REQ-008 & requirement validation \\
 General Constraints & FAQAS-SSS-REQ-009 & requirement validation \\
 Software Architecture & FAQAS-SSS-REQ-010 & requirement validation \\
 Software Architecture & FAQAS-SSS-REQ-011 & requirement validation \\
 Operational Environment & FAQAS-SSS-REQ-012 & requirement validation \\
 Operational Environment & FAQAS-SSS-REQ-013 & requirement validation \\
 Operational Environment & FAQAS-SSS-REQ-014 & requirement validation \\
 Operational Environment & FAQAS-SSS-REQ-015 & requirement validation \\
 Assumptions and dependencies & FAQAS-SSS-REQ-016 & requirement validation \\
 Assumptions and dependencies & FAQAS-SSS-REQ-017 & requirement validation \\
 Assumptions and dependencies & FAQAS-SSS-REQ-018 & requirement validation \\
  \hline
\end{tabular}
  }
\caption{Software overview requirements verification}
\label{tables:overview}
\end{table}


\clearpage
\section{Requirements}

The requirements in the SSS Chapter \emph{Requirements} comprise the following categories, one for every section of Chapter 4 of the SSS.

\begin{itemize}
  \item Functional requirements
  \item System interface requirements
  \item Adaptation and missionization requirements
  \item Computer resource requirements
  \item Security requirements
  \item Safety requirements
  \item Reliability and availability requirements
  \item Quality requirements
  \item Design requirements and constraints
  \item Adaptation and installation requirements
  \item Software operations requirements
  \item Software maintenance requirements
  \item System and software observability requirements
\end{itemize}

Below, we discuss each of them.

\subsection{Functional Requirements}

The functional requirements are relative to four procedures belonging to the FAQAS framework:
\begin{itemize}
  \item \emph{Test Suite Evaluation Based on Code-driven Mutation} here reported for readability as \emph{Code-driven Evaluation}
  \item \emph{Test Suite Augmentation Based on Code-driven Mutation} here reported for readability as \emph{Code-driven Augmentation}
  \item \emph{Test Suite Evaluation Based on Data-driven Mutation} here reported for readability as \emph{Data-driven Evaluation}
  \item \emph{Test Suite Augmentation Based on Data-driven Mutation} here reported for readability as \emph{Data-driven Augmentation}
\end{itemize}

The validation procedures adopted for \emph{Code-driven Evaluation} and \emph{Code-driven Augmentation} are detailed in Table~\ref{tables:code}.

% !TEX root =  ../MAIN.tex

\begin{table}[H]
\resizebox{\textwidth}{!}{%
\begin{tabular}{|l|l|l|l|}
\hline
\textbf{Subsection} &  \textbf{Requirement} &  \textbf{Validation Approach} &  \textbf{Task} \\
Code-driven Evaluation & FAQAS-SSS-REQ-019 & approach & case \\
Code-driven Evaluation & FAQAS-SSS-REQ-020 & approach & case \\
Code-driven Evaluation & FAQAS-SSS-REQ-021 & approach & case \\
Code-driven Evaluation & FAQAS-SSS-REQ-022 & approach & case \\
Code-driven Evaluation & FAQAS-SSS-REQ-023 & approach & case \\
Code-driven Evaluation & FAQAS-SSS-REQ-024 & approach & case \\
Code-driven Evaluation & FAQAS-SSS-REQ-025 & approach & case \\
Code-driven Evaluation & FAQAS-SSS-REQ-026 & approach & case \\
Code-driven Evaluation & FAQAS-SSS-REQ-027 & approach & case \\
Code-driven Evaluation & FAQAS-SSS-REQ-028 & approach & case \\
Code-driven Evaluation & FAQAS-SSS-REQ-029 & approach & case \\
Code-driven Evaluation & FAQAS-SSS-REQ-030 & approach & case \\
Code-driven Evaluation & FAQAS-SSS-REQ-031 & approach & case \\
Code-driven Evaluation & FAQAS-SSS-REQ-032 & approach & case \\
Code-driven Evaluation & FAQAS-SSS-REQ-033 & approach & case \\
Code-driven Evaluation & FAQAS-SSS-REQ-034 & approach & case \\
Code-driven Evaluation & FAQAS-SSS-REQ-035 & approach & case \\
Code-driven Evaluation & FAQAS-SSS-REQ-036 & approach & case \\
Code-driven Evaluation & FAQAS-SSS-REQ-037 & approach & case \\
Code-driven Evaluation & FAQAS-SSS-REQ-038 & approach & case \\
Code-driven Evaluation & FAQAS-SSS-REQ-039 & approach & case \\
Code-driven Evaluation & FAQAS-SSS-REQ-040 & approach & case \\
Code-driven Evaluation & FAQAS-SSS-REQ-041 & approach & case \\
Code-driven Evaluation & FAQAS-SSS-REQ-042 & approach & case \\
Code-driven Evaluation & FAQAS-SSS-REQ-043 & approach & case \\
Code-driven Evaluation & FAQAS-SSS-REQ-044 & approach & case \\
Code-driven Evaluation & FAQAS-SSS-REQ-045 & approach & case \\
Code-driven Evaluation & FAQAS-SSS-REQ-046 & approach & case \\
Code-driven Evaluation & FAQAS-SSS-REQ-047 & approach & case \\
Code-driven Evaluation & FAQAS-SSS-REQ-048 & approach & case \\
Code-driven Evaluation & FAQAS-SSS-REQ-049 & approach & case \\
Code-driven Evaluation & FAQAS-SSS-REQ-050 & approach & case \\
Code-driven Evaluation & FAQAS-SSS-REQ-051 & approach & case \\
Code-driven Evaluation & FAQAS-SSS-REQ-052 & approach & case \\
Code-driven Evaluation & FAQAS-SSS-REQ-053 & approach & case \\
Code-driven Evaluation & FAQAS-SSS-REQ-054 & approach & case \\
Code-driven Evaluation & FAQAS-SSS-REQ-055 & approach & case \\
Code-driven Evaluation & FAQAS-SSS-REQ-056 & approach & case \\
Code-driven Augmentation & FAQAS-SSS-REQ-057 & approach & case \\
Code-driven Augmentation & FAQAS-SSS-REQ-058 & approach & case \\
Code-driven Augmentation & FAQAS-SSS-REQ-059 & approach & case \\
Code-driven Augmentation & FAQAS-SSS-REQ-060 & approach & case \\
Code-driven Augmentation & FAQAS-SSS-REQ-061 & approach & case \\
\hline
\end{tabular}
  }
  \caption{Functional requirements verification: Code-Driven}
  \label{tables:code}
\end{table}


The validation procedures adopted for \emph{Data-driven Evaluation} and \emph{Data-driven Augmentation} are detailed in Table~\ref{tables:data}.

% !TEX root = ../MAIN.tex

\begin{table}[H]
\resizebox{\textwidth}{!}{%
\begin{tabular}{|l|l|l|l|}
 \hline
 \textbf{Subsection} &  \textbf{Requirement} &  \textbf{Validation Approach} &  \textbf{Task} \\
 \hline
Data-driven Evaluation & FAQAS-SSS-REQ-062 & unit testing &  performing \emph{DAMAt-TD-DDMutation-1} and \emph{-2} \\
Data-driven Evaluation & FAQAS-SSS-REQ-063 & unit testing &  performing \emph{DAMAt-TD-DDMutation-1} and \emph{-2} \\
Data-driven Evaluation & FAQAS-SSS-REQ-064 & unit testing &  performing \emph{DAMAt-TD-DDMutation-1} and \emph{-2} \\
Data-driven Evaluation & FAQAS-SSS-REQ-065 & application to the case studies & instrumenting the source code \\
Data-driven Evaluation & FAQAS-SSS-REQ-066 & unit testing &  performing \emph{DAMAt-TD-DDMutation-1} and \emph{-2} \\
Data-driven Evaluation & FAQAS-SSS-REQ-067 & unit testing & performing \emph{DAMAt-TD-DDMutation\emph{-2}} \\
Data-driven Evaluation & FAQAS-SSS-REQ-068 & unit testing & performing \emph{DAMAt-TD-DDMutation\emph{-2}} \\
Data-driven Evaluation & FAQAS-SSS-REQ-069 & unit testing & performing \emph{DAMAt-TD-DDMutation-1} and \emph{-2} \\
Data-driven Evaluation & FAQAS-SSS-REQ-070 & unit testing & performing \emph{DAMAt-TD-DDMutation-1} \\
Data-driven Evaluation & FAQAS-SSS-REQ-071 & unit testing & performing \emph{DAMAt-TD-DDMutation-1} \\
Data-driven Evaluation & FAQAS-SSS-REQ-072 & unit testing &  performing \emph{DAMAt-TD-DDMutation-1} and \emph{-2} \\
Data-driven Evaluation & FAQAS-SSS-REQ-073 & unit testing &  performing \emph{DAMAt-TD-DDMutation-1} and \emph{-2} \\
Data-driven Evaluation & FAQAS-SSS-REQ-074 & application to the case studies & configuring and running \emph{DAMAt\_compile.sh} \\
Data-driven Evaluation & FAQAS-SSS-REQ-075 & unit testing & performing \emph{DAMAt-TD-DDMutation-1} \\
Data-driven Evaluation & FAQAS-SSS-REQ-076 & unit testing & performing \emph{DAMAt-TD-DDMutation-2} \\
Data-driven Evaluation & FAQAS-SSS-REQ-077 & application to the case studies & configuring and running \emph{DAMAt\_obtain\_coverage.sh} \\
Data-driven Evaluation & FAQAS-SSS-REQ-078 & unit testing &  performing \emph{DAMAt-TD-DDMutation-1} and \emph{-2} \\
Data-driven Evaluation & FAQAS-SSS-REQ-079 & application to the case studies & configuring and running \emph{DAMAt\_mutants\_launcher.sh}\\
Data-driven Evaluation & FAQAS-SSS-REQ-080 & application to the case studies & configuring and running \emph{DAMAt\_mutants\_launcher.sh} \\
Data-driven Evaluation & FAQAS-SSS-REQ-081 & application to the case studies & configuring and running \emph{DAMAt\_run\_test.sh}\\
Data-driven Evaluation & FAQAS-SSS-REQ-082 & application to the case studies & configuring and running \emph{DAMAt\_mutants\_launcher.sh} \\
Data-driven Evaluation & FAQAS-SSS-REQ-083 & application to the case studies & configuring and running \emph{DAMAt\_mutants\_launcher.sh} \\
Data-driven Evaluation & FAQAS-SSS-REQ-084 & application to the case studies & configuring and running \emph{DAMAt\_mutants\_launcher.sh} \\
Data-driven Evaluation & FAQAS-SSS-REQ-085 & application to the case studies & configuring and running \emph{DAMAt\_mutants\_launcher.sh} \\
Data-driven Evaluation & FAQAS-SSS-REQ-086 & application to the case studies & configuring and running \emph{DAMAt\_mutants\_launcher.sh} \\
Data-driven Evaluation & FAQAS-SSS-REQ-087 & application to the case studies & configuring and running \emph{DAMAt\_mutants\_launcher.sh} \\
Data-driven Evaluation & FAQAS-SSS-REQ-088 & application to the case studies & configuring and running \emph{DAMAt\_mutants\_launcher.sh} \\
Data-driven Evaluation & FAQAS-SSS-REQ-089 & application to the case studies & configuring and running \emph{DAMAt\_mutants\_launcher.sh} \\
Data-driven Evaluation & FAQAS-SSS-REQ-090 & application to the case studies & configuring and running \emph{DAMAt\_mutants\_launcher.sh} \\
Data-driven Evaluation & FAQAS-SSS-REQ-091 & application to the case studies & configuring and running \emph{DAMAt\_mutants\_launcher.sh} \\
Data-driven Evaluation & FAQAS-SSS-REQ-092 & application to the case studies & configuring and running \emph{DAMAt\_mutants\_launcher.sh}} \\
Data-driven Evaluation & FAQAS-SSS-REQ-093 & application to the case studies & configuring and running \emph{DAMAt\_mutants\_launcher.sh} \\
Data-driven Evaluation & FAQAS-SSS-REQ-094 & application to the case studies & configuring and running \emph{DAMAt\_mutants\_launcher.sh} \\
Data-driven Evaluation & FAQAS-SSS-REQ-095 & application to the case studies & configuring and running \emph{DAMAt\_mutants\_launcher.sh} \\
Data-driven Evaluation & FAQAS-SSS-REQ-096 & application to the case studies & configuring and running \emph{DAMAt\_mutants\_launcher.sh} \\
Data-driven Evaluation & FAQAS-SSS-REQ-097 & application to the case studies & configuring and running \emph{DAMAt\_mutants\_launcher.sh} \\
Data-driven Evaluation & FAQAS-SSS-REQ-098 & application to the case studies & configuring and running \emph{DAMAt\_mutants\_launcher.sh} \\
Data-driven Augmentation & FAQAS-SSS-REQ-099 & approach & case \\
Data-driven Augmentation & FAQAS-SSS-REQ-100 & approach & case \\
\hline
\end{tabular}
  }
\caption{Functional requirements verification: Data-Driven}
\label{tables:data}
\end{table}


% \TODO{In this preliminary report, inside the tables, we use the keywords \emph{approach} and \emph{case} to indicate that the approach and task remains to be listed in the next version of the deliverable. For now, only DAMAt has been detailed.}
%\TODO{There are missing cases for DAMTE}

\clearpage

\subsection{System Interface requirements}

Table~\ref{tables:interface} details the validation testing procedures for System interface requirements.

% !TEX root = ../MAIN.tex

\begin{table}[H]
\resizebox{\textwidth}{!}{%
\begin{tabular}{|l|l|l|l|}
\hline
\textbf{Subsection} &  \textbf{Requirement} &  \textbf{Validation Approach} &  \textbf{Task} \\
System Interface requirements & FAQAS-SSS-REQ-101 & manual inspection & -- \\
System Interface requirements & FAQAS-SSS-REQ-102 & unit testing & performing \emph{DAMAt-TD-DDMutation-1} \\
System Interface requirements & FAQAS-SSS-REQ-103 & manual inspection & -- \\
System Interface requirements & FAQAS-SSS-REQ-104 & application to the case studies & configuring and running the MASS and DAMAt pipelines \\
System Interface requirements & FAQAS-SSS-REQ-105 & application to the case studies & configuring and running the MASS and DAMAt pipelines \\
System Interface requirements & FAQAS-SSS-REQ-106 & application to the case studies & configuring and running the MASS and DAMAt pipelines \\
System Interface requirements & FAQAS-SSS-REQ-107 & application to the case studies & configuring and running the MASS and DAMAt pipelines \\
System Interface requirements & FAQAS-SSS-REQ-108 & approach & case \\
System Interface requirements & FAQAS-SSS-REQ-109 & approach & case \\
System Interface requirements & FAQAS-SSS-REQ-110 & application to the case studies & configuring and running the MASS and DAMAt pipelines \\
\hline
\end{tabular}
  }
  \caption{System Interface requirements validation}
  \label{tables:interface}
\end{table}


\subsection{Others}

Table~\ref{tables:others} details the validation testing procedures for all the other categories of requirements, which are:
Adaptation and missionization requirements, Computer resource requirements, Security requirements, Safety requirements, Reliability and availability requirements, Quality requirements, Design requirements and constraints, Adaptation and installation requirements, Software operations requirements, Software maintenance requirements, System and software observability requirements.


% !TEX root = ../MAIN.tex

\begin{table}[h!]
\resizebox{\textwidth}{!}{%
\begin{tabular}{|l|l|l|l|}
\hline
\textbf{Section} &  \textbf{Requirement} &  \textbf{Validation Approach} &  \textbf{Task} \\
Adaptation and missionization requirements & FAQAS-SSS-REQ-111 & manual inspection & -- \\
Adaptation and missionization requirements & FAQAS-SSS-REQ-112 & manual inspection & -- \\
Computer resource requirements & FAQAS-SSS-REQ-113 & manual inspection & -- \\
Security &FAQAS-SSS-REQ-114 & manual inspection & -- \\
Safety &FAQAS-SSS-REQ-115 & manual inspection & -- \\
Reliability and availability requirements & FAQAS-SSS-REQ-116 & manual inspection & -- \\
Reliability and availability requirements & FAQAS-SSS-REQ-117 & manual inspection & -- \\
Quality & FAQAS-SSS-REQ-118 & partner validation & -- \\
Quality & FAQAS-SSS-REQ-119 & manual inspection & -- \\
Quality & FAQAS-SSS-REQ-120 & approach & case \\
Design requirements and constraints & FAQAS-SSS-REQ-121 & manual inspection & -- \\
Design requirements and constraints & FAQAS-SSS-REQ-122 & manual inspection & -- \\
Design requirements and constraints & FAQAS-SSS-REQ-123 & manual inspection & -- \\
Adaptation and installation requirements & FAQAS-SSS-REQ-124 & manual inspection & -- \\
Software operations requirements & FAQAS-SSS-REQ-125 & manual inspection & -- \\
Software maintenance requirements & FAQAS-SSS-REQ-126 & manual inspection & -- \\
System and software observability requirements & FAQAS-SSS-REQ-127 & manual inspection & -- \\
\hline
\end{tabular}
  }
  \caption{Validation of other requirements}
  \label{tables:others}
\end{table}


\clearpage

\section{Verification, validation and system integration}

This section covers the validation testing procedures for the requirement contained in Chapter 5 of the SSS.

They are reported in Table~\ref{tables:verification}.

% !TEX root =  ../MAIN.tex

\begin{table}[h!]
\resizebox{\textwidth}{!}{%
\begin{tabular}{|l|l|l|l|}
  \hline
  \textbf{Subsection} &  \textbf{Requirement} &  \textbf{Validation Approach} &  \textbf{Task} \\
 Verification and validation process requirements & FAQAS-SSS-REQ-128 & manual inspection & -- \\
 Verification and validation process requirements & FAQAS-SSS-REQ-129 & manual inspection & -- \\
 Verification and validation process requirements & FAQAS-SSS-REQ-130 & manual inspection & -- \\
 Verification and validation process requirements & FAQAS-SSS-REQ-131 & manual inspection & -- \\
 Verification and validation process requirements & FAQAS-SSS-REQ-132 & manual inspection & -- \\
 Verification and validation process requirements & FAQAS-SSS-REQ-133 & manual inspection & -- \\
 Validation approach & FAQAS-SSS-REQ-134 & manual inspection & -- \\
 Validation approach & FAQAS-SSS-REQ-135 & manual inspection & -- \\
 \hline
\end{tabular}
  }
  \caption{Validation of verification and validation process requirements}
  \label{tables:verification}
\end{table}

