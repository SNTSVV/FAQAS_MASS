% !TEX root = MAIN.tex
\clearpage

\section{Summary of FAQAS Validation Results}
\label{sec:summary:results}

In this section, we report the summary of results for the FAQAS framework. In particular, for each FAQAS approach, we provide a table with the main output of the technique (e.g., mutation score, number of test cases generated).

\subsection{Code-driven mutation analysis}

\begin{table}[htb]
\caption{Code-driven mutation analysis results: MASS mutation score.}
\label{table:results:mass} 
\small
\centering
\begin{tabular}{|
>{\raggedleft\arraybackslash}p{24mm}@{\hspace{1pt}}|
>{\raggedleft\arraybackslash}p{46mm}@{\hspace{1pt}}|
>{\raggedleft\arraybackslash}p{25mm}@{\hspace{1pt}}|
>{\raggedleft\arraybackslash}p{25mm}@{\hspace{1pt}}|
}
\hline
\textbf{Subject}&\textbf{Predicted Equivalent}&\multicolumn{2}{c|}{\textbf{Mutation Score (\%)}}\\
&\textbf{Mutants Removed (\%)}&\textbf{Traditional}&\textbf{MASS}\\ 
\hline
% 1225-32=1193; 2311/(2311+1193)
\SAIL{}$_{S}$&2.61&65.36&65.95\\
%$\mathit{\SAIL{}}$&&\\
% 4198-2281=1917; 10376/(10376+1917)


% 1214-770=444; 2717/(2717+444)

% 1712-371=1341; 3270/(3270+1341)
\GCSP{}&21.67&65.64&70.92\\
\PARAM{}&63.43&69.12&85.95\\
% 3981-2699=1282; 17484/(17484+1282)
\UTIL{}&54.34&71.20&84.41\\
\MLFS{}{}&69.37&81.80&93.49\\
\hline
$\textbf{Average}$&42.28&70.62&81.14\\
\hline
\end{tabular}

\end{table}

Table~\ref{table:results:mass} introduces the code-driven mutation analysis results, and compares, for each subject, the mutation score of a traditional mutation process with the one obtained with MASS. 


\subsection{Code-driven mutation testing (test suite augmentation)}

\begin{table}[htb]
\caption{Test suite augmentation results.}
\label{table:results:test-gen} 
\centering
\footnotesize
\begin{tabular}{|
@{\hspace{1pt}}p{10mm}|
@{\hspace{1pt}}>{\raggedleft\arraybackslash}p{18mm}@{\hspace{1pt}}|
>{\raggedleft\arraybackslash}p{35mm}@{\hspace{1pt}}|
>{\raggedleft\arraybackslash}p{25mm}@{\hspace{1pt}}|
 >{\raggedleft\arraybackslash}p{25mm}@{\hspace{1pt}}|
}
\hline
\textbf{Subject}&\textbf{Live Mutants}&\textbf{Additionally Killed Mutants}&\textbf{Original MS (\%)}&\textbf{Updated MS (\%)}\\ 
\hline
$\mathit{MLFS}$&3\,891&697&81.80&85.06\\
$\mathit{ASN.1}$&2\,219&1\,729&58.31&90.79\\
\hline
\end{tabular}

\end{table}

Table~\ref{table:results:test-gen} introduces the code-driven mutation testing results, and in particular the number of additionally killed mutants by SEMuS, and how it impacts on the mutation score of the case study.


\subsection{Data-driven mutation analysis}

\begin{table}[htb]
\caption{Data-driven mutation analysis results.}
\label{table:results:data-driven} 
\center
\footnotesize
\begin{tabular}{|
@{\hspace{0pt}}>{\raggedleft\arraybackslash}p{24mm}@{\hspace{1pt}}|
@{\hspace{0pt}}>{\raggedleft\arraybackslash}p{12mm}@{\hspace{1pt}}|
@{\hspace{0pt}}>{\raggedleft\arraybackslash}p{12mm}@{\hspace{1pt}}|
@{\hspace{0pt}}>{\raggedleft\arraybackslash}p{18mm}@{\hspace{1pt}}|
@{\hspace{0pt}}>{\raggedleft\arraybackslash}p{12mm}@{\hspace{1pt}}|
@{\hspace{0pt}}>{\raggedleft\arraybackslash}p{12mm}@{\hspace{1pt}}|
@{\hspace{0pt}}>{\raggedleft\arraybackslash}p{12mm}@{\hspace{1pt}}|
@{\hspace{0pt}}>{\raggedleft\arraybackslash}p{12mm}@{\hspace{1pt}}|
@{\hspace{0pt}}>{\raggedleft\arraybackslash}p{12mm}@{\hspace{1pt}}|
}
\hline
\textbf{Subject} & 
\textbf{\# FMs} & 
\textbf{FMC} & 
\textbf{\#MOs-CFM} & 
\textbf{\#CMOs} & 
\textbf{MOC}  
&\textbf{Killed}&\textbf{Live}&\textbf{MS}
\\
\hline

\ADCS &10 &90.00\%   & 135 & 100 & 74.00\%   &    45&55&45.00\%\\
\GPS &1 &100.00\%    &  23  &  22 & 95.65\%    &      21&1&95.45\%\\
\PDHU &3 &100.00\%  &   29 & 24 & 82.76\%   &     24&0&100.00\%\\
\PARAM &6 &100.00\%  &   80 & 73 & 91.25\%  &        28&45&38.36\%\\


\hline

\end{tabular}

CMO=Covered Mutation Operation, MOs-CFM=Mutation Operations in covered FMs.

\end{table}


Table~\ref{table:results:data-driven} shows the mutation analysis results of DAMAT, our data-driven mutation analysis technique. For each subject, we provide information about the fault model coverage, the mutation operation coverage, and the mutation score of the technique.


