% !TEX root =  ../MAIN.tex

\begin{lstlisting}[style=CStyle, caption=Second example of code for the identification of inputs with KLEE., label=GSLaugmentationTwo_klee]
#include <klee/klee.h>

#include <gs/util/base16.h>
#include <stdio.h>
#include <stdlib.h>
#include <limits.h>
#include <ctype.h>
#include <errno.h>
#include <assert.h>


int base16_decode(const char *encoded, uint8_t *raw)
{
    uint8_t *raw_bytes = raw;
    if (encoded) {
        const char *encoded_bytes = encoded;
        char buf[3];
        char *endp;

    while (encoded_bytes[0]) {
            if (!encoded_bytes[1]) {
                return GS_ERROR_ARG;
            }
            memcpy(buf, encoded_bytes, 2);
            buf[2] = '\0';
            *(raw_bytes++) = (uint8_t) strtoul(buf, &endp, 16);
            if (*endp != '\0') {
                return GS_ERROR_ARG;
            }
            encoded_bytes += 2;
    }
    }
    return (int)(raw_bytes - raw);
}

int MUT_base16_decode(const char *encoded, uint8_t *raw)
{
    uint8_t *raw_bytes = raw;
    if (encoded) {
        const char *encoded_bytes = encoded;
        char buf[3];
        char *endp;

    while (encoded_bytes[0]) {
            if (!encoded_bytes[1]) {
                //return GS_ERROR_ARG;
            }
            memcpy(buf, encoded_bytes, 2);
            buf[2] = '\0';
            *(raw_bytes++) = (uint8_t) strtoul(buf, &endp, 16);
            if (*endp != '\0') {
                return GS_ERROR_ARG;
            }
            encoded_bytes += 2;
    }
    }
    return (int)(raw_bytes - raw);
}


#ifdef KLEE
int main() 
{
    char encoded[4];
    char a,b,c;
    klee_make_symbolic(&a, sizeof(a), "a");
    klee_make_symbolic(&b, sizeof(b), "b");
    klee_make_symbolic(&c, sizeof(c), "c");
    uint8_t raw;
    uint8_t mRaw;
    int ret;
    int mret;

    
    encoded[0]=a;
    encoded[1]=b;
    encoded[2]=c;
    encoded[3]=0;

    raw=0;
    ret = base16_decode(encoded,&raw);

    encoded[0]=a;
    encoded[1]=b;
    encoded[2]=c;
    encoded[3]=0;

    mRaw=0;
    mret = MUT_base16_decode(encoded,&mRaw);
    printf ("ret %d mret %d\n", ret, mret);

    klee_assume( ret == mret );
} 
#endif
\end{lstlisting}
\begin{lstlisting}[style=CStyle, caption=Test case derived for the example in Listing~\ref{GSLaugmentationTwo_klee}., label=GSLaugmentationTwo_kleeTest]
int main(){
    char encoded[4];

    uint8_t raw;
    int ret;

    encoded[0]=48;
    encoded[1]=0;
    encoded[2]=0;
    encoded[3]=0;

    raw=0;
    ret = base16_decode(encoded,&raw);


    assert( ret == GS_ERROR_ARG );
    return 0;
}
\end{lstlisting}

