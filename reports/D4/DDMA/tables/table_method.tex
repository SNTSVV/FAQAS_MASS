% !TEX root =  ../MAIN-DataDrivenMutationAnalysis.tex

%
%\setlength\LTleft{0pt}
%\setlength\LTright{0pt}
%\begin{longtable}{@{\extracolsep{\fill}}|p{2.5cm}|p{5cm}|p{5cm}|@{}}
%\toprule


\begin{table}[tb]
\caption{\APPR fault modelling methodology}
\label{table:method}
\scriptsize
\begin{tabular}{|
@{\hspace{1pt}}>{\raggedleft\arraybackslash}p{10mm}@{\hspace{1pt}}|
@{\hspace{1pt}}>{\raggedleft\arraybackslash}p{15mm}@{\hspace{1pt}}|
@{\hspace{1pt}}>{\raggedleft\arraybackslash}p{15mm}@{\hspace{1pt}}|
@{\hspace{1pt}}>{\raggedleft\arraybackslash}p{10mm}@{\hspace{1pt}}|
@{\hspace{1pt}}>{\raggedleft\arraybackslash}p{13mm}@{\hspace{1pt}}|
@{\hspace{1pt}}>{\raggedleft\arraybackslash}p{17mm}@{\hspace{1pt}}|
}
\hline
\textbf{Data} \textbf{nature}&\textbf{Representation} \textbf{type}&\textbf{Dependencies}&\textbf{\# of input} \textbf{partitions}&\textbf{Operators}&\textbf{Comments}\\
\hline
numerical&I, L, F, D&stateless/stateful&2&[VAT,FVAT]&Nominal below T\\
&&&&or [VBT,FVBT]&Nominal above T\\
\cline{4-6}
&&&3 or more&[VOR,FVOR]&\\
\cline{3-6}
&&stateful&&INV&For valid range\\
\cline{4-6}
&& &&[VOR,FVOR]&For out of range\\
\cline{3-6}
&&signal&&ASA, SS, HV&\\
\hline
categorical&I, H&N/A&N/A&IV&\\
\cline{2-6}
%&string&N/A&N/A&BF\\
%\cline{2-5}
&B&N/A&N/A&BF&\\
\hline
ordinal&I, H&N/A&N/A&ASA&\\
\hline
other&B&N/A&N/A&BF&\\
\hline
\end{tabular}
\textbf{Legend:} N/A not applicable.
\end{table}
%\normalsize