% !TEX root = MAIN.tex

\section{Responses to ESA comments provided on 01.10.2021}
\label{sec:ESA:comments:1}

Comments IDs appear also in the main document next to the text modified to address the comment. To save space in the main text, the prefix \emph{TDR-D4-PABG-} has been abbreviated as \emph{C-P-}.

\setlength\LTleft{0pt}
\setlength\LTright{0pt}
\tiny 
\begin{longtable}{|p{2cm}|p{12cm}|@{}} 
\hline
\\
\textbf{Comment ID}&\textbf{Response}\\
\\
\hline
TDR-D4-PABG-1&
\begin{minipage}{12cm}
The purpose of the Chapter~\ref{chapter:caseStudies} is to present an overview of the case study systems of the FAQAS activity, and furthermore to introduce some preliminary results. More importantly, the different approaches presented in FAQAS cannot be evaluated in the same way, and thus, we need dedicated chapters were we present in details the empirical evaluation to be carried on code-driven mutation analysis (see Chapter~\ref{sec:testSuiteEvaluation:codeDriven}), code-driven mutation test suite augmentation (see Chapter~\ref{sec:testGeneration:codeDriven}), and data-driven mutation analysis (see Chapter~\ref{sec:testSuiteEvaluation:dataDriven}).

However, we added Section~\ref{} that reports a summary of the output of the multiple FAQAS approaches (e.g., mutation scores, number of test cases generated).
\end{minipage}\\
\\
\hline  
TDR-D4-PABG-2&
\begin{minipage}{12cm}
See answer to comment \texttt{TDR-D4-PABG-1}.
\end{minipage}\\
\\
\hline  
TDR-D4-PABG-3&
\begin{minipage}{12cm}
See answer to comment \texttt{TDR-D4-PABG-1}.
\end{minipage}\\
\\
\hline  
TDR-D4-PABG-4&
\begin{minipage}{12cm}
Done.
\end{minipage}\\
\\
\hline  
TDR-D4-PABG-5&
\begin{minipage}{12cm}
The following table represents the metrics already reported, and the metrics that could be added to the MASS report.

\begin{tabular}{|
@{\hspace{1pt}}p{50mm}|
@{\hspace{1pt}}>{\raggedleft\arraybackslash}p{30mm}@{\hspace{1pt}}|
 >{\raggedleft\arraybackslash}p{25mm}@{\hspace{1pt}}|
}
\hline
\textbf{Metric}&\textbf{Is it included in the MASS report?}\\ 
\hline
Number of total mutants generated&Yes\\
Mutants generation time&No\\
Number of compiled mutants&No\\
Percentage of compiled mutants&No\\
Mutants compilation time&No\\
Number of total mutants filtered by compiler optimisations&Yes\\
Sampling type&Yes\\
Number of total mutants executed&Yes\\
Number of test cases executed&No\\
Test cases execution time&No\\
Mutation execution traces&Yes\\
Number of killed mutants&Yes\\
Number of live mutants&Yes\\
Number of likely equivalent mutants&Yes\\
MASS mutation score&Yes\\
List of useful mutants&Yes\\
Number of statements covered&Yes\\
Statement coverage&Yes\\
Minimum lines covered per source file&Yes\\
Maximum lines covered per source file&Yes\\
Distribution of test cases exercising each statement&No\\
\hline
\end{tabular}

\end{minipage}\\
\\
\hline  
TDR-D4-PABG-6&
\begin{minipage}{12cm}
\TODO{To be fixed}
We provide the ESAIL XXX test suite statement coverage, which is the information necessary to perform the different MASS optimizations. We do not provide more detailed coverage information (i.e., function and branch coverage); since its measurement in our testing facilities interferes with the real-time requirements of the test suite.
\end{minipage}\\
\\
\hline  
TDR-D4-PABG-7&
\begin{minipage}{12cm}
See answer to comment \texttt{TDR-D4-PABG-1}.
\end{minipage}\\
\\
\hline  
TDR-D4-PABG-8&
\begin{minipage}{12cm}
We did not make a preliminary assessment of the Libparam case study; given that the environmental setup of the library was identical to Libutil and Libgscsp GSL case studies, we proceeded to directly perform a complete evaluation of Libparam, presented in Section~\ref{sec:testSuiteEvaluation:codeDriven}.
\end{minipage}\\
\\
\hline  
TDR-D4-PABG-9&
\begin{minipage}{12cm}
See answer to comment \texttt{TDR-D4-PABG-1}.
\end{minipage}\\
\\
\hline  
TDR-D4-PABG-10&
\begin{minipage}{12cm}
See answer to comment \texttt{TDR-D4-PABG-1}.
\end{minipage}\\
\\
\hline  
TDR-D4-PABG-11&
\begin{minipage}{12cm}
Actually, with the exception of \SAIL{}, we are comparing MASS to the truth, i.e., a mutation analysis process were all the mutants are considered and executed against the complete test suite. In the case of \SAIL, we defined \SAIL{}$_S$ to be our complete set (i.e., 3,535 mutants), since the actual complete set (i.e., 78,203 mutants) was not be possible to analyze in the time frame of the FAQAS project. 

However, we added a phrase to the design paragraph of the research question to avoid wrong interpretations.
\end{minipage}\\
\\
\hline  
TDR-D4-PABG-12&
\begin{minipage}{12cm}
Yes, it is something to be addressed in the FAQAS follow-on activity. We added a phrase in Section~\ref{sec:exp:thr}.
\end{minipage}\\
\\
\hline  
TDR-D4-PABG-13&
\begin{minipage}{12cm}
We will keep the research questions for the FAQAS follow-on activity.
\end{minipage}\\
\\
\hline  
TDR-D4-PABG-14&
\begin{minipage}{12cm}
Done.
\end{minipage}\\
\\
\hline  
TDR-D4-PABG-15&
\begin{minipage}{12cm}
The following table represents the metrics already reported in the DAMAt report.

\begin{tabular}{|
@{\hspace{1pt}}p{50mm}|
@{\hspace{1pt}}>{\raggedleft\arraybackslash}p{30mm}@{\hspace{1pt}}|
 >{\raggedleft\arraybackslash}p{25mm}@{\hspace{1pt}}|
}
\hline
\textbf{Metric}&\textbf{Is it included in the DAMAT report?}\\ 
\hline
Fault model coverage&Yes\\
Mutation Operation coverage&Yes\\
Number of killed mutants&Yes\\
Number of live mutants&Yes\\
Mutation score&Yes\\
Mutation score by fault model&Yes\\
Mutation score by fault class&Yes\\
Mutation score by data item&Yes\\
\hline
\end{tabular}

\end{minipage}\\
\\
\hline  
TDR-D4-PABG-16&
\begin{minipage}{12cm}
As specified in Section~\ref{sec:threats:damat}, our results generalize for systems with architectures similar to \SAIL, which is a state-of-the-art system in terms of CPS software.
\end{minipage}\\
\\
\hline  
TDR-D4-PABG-17&
\begin{minipage}{12cm}
\TODO{To discuss.}
\end{minipage}\\
\\
\hline                                    
TDR-D4-PABG-18&
\begin{minipage}{12cm}
We meant adding oracles that verify data values already produced by the software under test. We added an additional phrase in Section~\ref{sec:rq2:damat} to clarify it.
\end{minipage}\\
\\
\hline  
TDR-D4-PABG-19&
\begin{minipage}{12cm}
\TODO{Probably would be a good idea before submitting this revision, to try again with the updated version of SEMuS, and report in Chapter 1, as a feasibility study the problems we found when applying SEMuS to ESAIL.}
SEMuS relies on the KLEE symbolic execution engine for test generation. KLEE requires that programs passed as input shall be compiled in LLVM bitcode format. In Deliverable D2 section 1.2.3.7, we presented a feasibility study showing that ESAIL cannot be compiled in LLVM bitcode format because of incompatibility of libraries between clang (LLVM compiler) and RTEMS (ESAIL compiler). Such problems will be assessed in the FAQAS follow-on project.

However, we added a phrase explaining the aforementioned problem.
\end{minipage}\\
\\
\hline  
TDR-D4-PABG-20&
\begin{minipage}{12cm}
The quality of the test cases generated by the technique is high. SEMuS enable us to generate test inputs that kill mutants not previously detected by existing test suites. The test cases generated by SEMuS increases the overall quality of a test suite because (1) cover input partitions not previously considered, (2) evidence the lack of oracles in the test suite, and (3) in some cases, it might spot defects in the code (as already presented in Section~\ref{sec:shortcoming:semus}).
\end{minipage}\\
\\
\hline  
TDR-D4-PABG-21&
\begin{minipage}{12cm}
\TODO{ENRICO: please add the specific location of the GPS probes, as requested by Pedro}
\end{minipage}\\
\\
\hline  
TDR-D4-PABG-22&
\begin{minipage}{12cm}
\TODO{ENRICO: please add the specific location of the PDHU probes, as requested by Pedro}
\end{minipage}\\
\\
\hline  
TDR-D4-PABG-23&
\begin{minipage}{12cm}
In the results of research question 2 (see Section~\ref{sec:rq3:damat}), we already report the average time taken to manually configure a single operator of the fault model (i.e., between five and ten minutes).
\end{minipage}\\
\\
\hline  
TDR-D4-PABG-24&
\begin{minipage}{12cm}
\TODO{ENRICO: please fix the name of both tables as indicated by Pedro.}
\end{minipage}\\
\\
\hline  
TDR-D4-PABG-25&
\begin{minipage}{12cm}
\end{minipage}\\
\\
\hline
\end{longtable}
\normalsize

\clearpage