% !TEX root = MAIN.tex

\chapter{Specific Requirements}

\section{System interface requirements}

\subsection{MASS}
\RQ{} MASS shall support the following commands:
	\begin{itemize}
	\item \texttt{PrepareSUT}: command to prepare the SUT and collect information about the SUT test suite.
	\item \texttt{GenerateMutants}: command to generate mutants from the SUT source code.
	\item \texttt{CompileOptimizedMutants}: command to compile the mutants with the multiple optimisation levels.
	\item \texttt{OptimizedPostProcessing}: command to disregard equivalent and redundant mutants based on compiler optimisations.
	\item \texttt{GeneratePTS}: command to generate the prioritized and reduced test suites.
	\item \texttt{ExecuteMutants}: command to execute mutants against the SUT test suite.
	\item \texttt{IdentifyEquivalents}: command to identify equivalent mutants based on code coverage.
	\item \texttt{MutationScore}: command to compute the mutation score and final reporting.
	\item \texttt{PrepareMutants\_HPC}: command to prepare the mutants workspace for the execution on HPCs.
	\item \texttt{ExecuteMutants\_HPC}: command to execute mutants on HPCs.
	\item \texttt{PostMutation\_HPC}: command to assess past mutant executions, and to decide whether more mutant executions are needed.
\end{itemize}

\RQ{} MASS shall generate the following output files:
	\begin{itemize}
		\item output concerning SUT compilation process
		\item output concerning execution of SUT test cases
		\item output concerning the execution of the \texttt{CompileOptimizedMutants} command
		\item output concerning the execution of the \texttt{OptimizedPostProcessing} command
		\item output concerning the execution of the \texttt{GeneratePTS} command
		\item output concerning the execution of the \texttt{ExecuteMutants} command
		\item output concerning the execution of the \texttt{IdentifyEquivalents} command
		\item output concerning the execution of the \texttt{MutationScore} command
		\item output concerning the execution of the \texttt{PrepareMutants\_HPC} command
		\item output concerning the execution of the \texttt{ExecuteMutants\_HPC} command
		\item output concerning the execution of the \texttt{PostMutation\_HPC} command
	\end{itemize}

\RQ{} MASS shall be configured through a text file.

\RQ{} MASS configuration file shall be set with the following parameters:
	\begin{itemize}
		\item MASS installation path
		\item MASS workspace path
		\item SUT relevant paths (e.g., source code folder, test folder, coverage folder)
		\item SUT compilation command
		\item SUT test case execution command
		\item MASS execution environment mode (i.e., single machine, HPC)
		\item MASS trivial compiler optimisation flags
		\item MASS sampling type
		\item MASS sampling rate
		\item MASS test suite prioritization type
	\end{itemize}

\RQ{} MASS shall provide built-in features to process the SUT test harness Google Test\footnote{https://github.com/google/googletest} and the Basic mathematical Library Test Suite (BLTS)\footnote{https://essr.esa.int/project/mlfs-mathematical-library-for-flight-software}.

\RQ{} MASS shall provide means to configure its execution with any test harness infrastructure for C/C++ software.

\RQ{} MASS shall provide the engineer with the ability to configure the environment variables used to tune the mutation testing process.

\subsection{SEMuS}

\RQ{} SEMuS shall support the following commands:
	\begin{itemize}
	\item \texttt{call\_generate\_direct.sh}: command to generate the test templates for guiding the test generation process.
	\item \texttt{docker\_run.sh}: command to launch the test generation process.
\end{itemize}

\RQ{} SEMuS shall generate the following output files and directories:
\begin{itemize}
	\item a folder containing the mutant sources generated by MASS
	\item a folder storing the source files and the compiled objects of the meta mutant files (e.g., \texttt{*.MetaMu.c} and \texttt{*.MetaMu.bc}) to be processed by SEMu (i.e., KLEE)
	\item a folder containing the outputs of SEMuS concerning the test generation step, this directory shall contain one folder for each test template. Furthermore, it also shall contain the following folders:
	\begin{itemize}
		\item a subfolder containing the SEMu output (e.g., KLEE tests files, execution traces)
		\item a subfolder containing the unit test cases converted from the SEMu output
	\end{itemize}

\end{itemize}

\RQ{} SEMuS shall be configured through a text file.

\RQ{} SEMuS configuration file shall be set with the following parameters:
	\begin{itemize}
		\item SEMuS installation path
		\item SEMuS workspace path
		\item SUT relevant paths (e.g., source code folder, test folder, coverage folder)
		\item SUT compilation command
		\item SEMu specific configuration
	\end{itemize}


\RQ{} SEMuS shall provide the engineer with the ability to configure the environment variables used to tune the test augmentation process.

\subsection{DAMAt}
\RQ{} DAMAt shall be configured through a fault model written by the engineer in csv format (see D2 and SUM).

\RQ{} DAMAt shall support the following commands:
	\begin{itemize}
		\item \texttt{DAMAt\_probe\_generation.sh}: which generates the mutation API.
		\item \texttt{DAMAt\_compile\_and\_run\_mutants.sh}: which executes the DAMAt pipeline.
\end{itemize}

\RQ{} DAMAt shall generate the following output files:
	\begin{itemize}
		\item \texttt{mutation\_sum\_up.csv}: a file containing a synthetic evaluation of the test suite.
		\item \texttt{final\_mutants\_table.csv}: a file containing the definition and status of every mutant.
		\item \texttt{mutation\_score\_by\_data\_item.csv}: a file containing the mutation score by data item.
		\item \texttt{mutation\_score\_by\_fault\_class.csv}: a file containing the mutation score by fault class.
		\item \texttt{mutation\_score\_by\_fault\_model.csv}: a file containing the mutation score by fault model.
		\item \texttt{test\_coverage.csv}: a file containing the tests covering the different fault models.
		\item \texttt{readable\_data.csv}: a file containing a more readable version of the execution data.
		\item \texttt{raw\_data.csv} and \texttt{raw\_data\_sorted.csv}: these files contain all the raw execution data.
		\item \texttt{readable\_operator\_coverage.csv}: a file containing a more readable version of the operator coverage data.
		\item \texttt{operator\_coverage.csv}: a file containing a raw version of the operator coverage data.
		\item \texttt{readable\_FM\_coverage.csv}: a file containing a more readable version of the fault model coverage data.
		\item \texttt{FM\_coverage.csv}: a file containing a raw version of the fault model coverage data.

\RQ{} DAMAt shall be configured through a text file.

\RQ{} DAMAt configuration file shall be set with the following parameters:
	\begin{itemize}
		\item \texttt{tests\_list}: the location of the list of available test cases.
		\item \texttt{fault\_model}: the location of the fault model.
		\item \texttt{buffer\_type}: the data type of the elements of the buffer targeted by the mutation.
		\item \texttt{padding}: number of header elements to skip (optional).
		\item \texttt{singleton}: \texttt{"TRUE"} to initialize the fault model as a singleton variable, \texttt{"FALSE"} to avoid this option.
	\end{itemize}

\RQ{} DAMAt shall provide the engineer with the ability to configure the environment variables used to tune the mutation testing process.


\chapter{Validation requirements}

A complete validation of the \FAQAS requirements is provided in document ITT-1-9873-ESA-FAQAS-D4. Please refer to Chapter 2.
