% !TEX root = MAIN.tex

\chapter{Software Overview}
\label{chapter:overview}

\section{Function and Purpose}

The FAQAS activity concerns the investigation of mutation testing as a method to evaluate the quality of software test suites and mutation testing as a method to derive new software test cases.

The \FAQAS shall support a code-driven and data-driven mutation testing methodology that integrates and extends state-of-art solutions. The reference methodology has been described in D2. The methodology enables a scalable and accurate mutation testing process even when the software under test is large and characterized by test suites requiring long execution time.

The end-user of the \FAQAS is a software engineer (i.e., a professional with a master degree in informatics or related fields)  who aims to evaluate the quality of the test suite developed for a flight software component or system. Hereafter, the end-user of \FAQAS is referred to as \emph{the engineer}.

Since \FAQAS implements two distinct features, code-driven mutation testing and data-driven mutation testing, this document contains two separate sections, each one concerning one of the two features: Section~\ref{sec:codeDriven} concerns code-driven mutation testing, Section~\ref{sec:dataDriven} concerns data-driven mutation testing.

The requirements defined in this chapter are univocally identified by the paragraph id appearing on the left.

\section{General Capabilities}

\RQ{} \FAQAS shall support test suite evaluation based on code mutation.

\RQ{} \FAQAS shall support test suite augmentation based on code mutation.

\RQ{} \FAQAS shall support test suite evaluation based on data mutation.

\RQ{} \FAQAS should support test suite augmentation based on data mutation.

\REVISION{P-8 P-13}{\RQ{} \FAQAS shall implement the code-driven test suite evaluation and test suite augmentation steps depicted in Figure~\ref{fig:code:process}. Detailed explanation is provided in D2 Chapter 1.}

\begin{figure}
	\centering
		\includegraphics[width=12cm]{images/process}
		\caption{Code-driven Mutation Testing Process.}
		\label{fig:code:process}
	\end{figure}
	
	
\REVISION{P-8}{\RQ{} \FAQAS shall implement the data-driven test suite evaluation and test suite augmentation steps depicted in Figure~\ref{fig:data:process}. Detailed explanation is provided in D2 Chapter 2.}

\begin{figure}
	\centering
		\includegraphics[width=12cm]{images/dataProcess}
		\caption{Data-driven Mutation Testing Process.}
		\label{fig:data:process}
	\end{figure}	

\clearpage

\section{General Constraints}

\REVISION{P-3}{\RQ{} Test suite evaluation for code-driven mutation testing shall be based on SRCIror\footnote{https://github.com/TestingResearchIllinois/srciror} (detailed motivation is provided in D2).}

\RQ{} The automated generation of test cases (i.e., the objective of test suite augmentation) shall rely on existing tools because it is an open, complex, research problem.

\RQ{} The automated generation of test cases for code-mutation may rely on KLEE, which is the most stable test case generation tool, based on WP2 evaluation.


\section{Software Architecture}

\RQ{} \REVISION{P-4}{\FAQAS shall follow a modular architecture. The FAQAS Architecture is depicted in Figure~\ref{fig:architecture}. The FAQAS system shall consists of four components 
\begin{itemize}
\item MASS (Mutation Analysis for Space Software), which implements test suite evaluation based on code-driven mutation.
\item SEMuS (Symbolic Execution-based Mutation testing for Space), which implements test suite augmentation based on code-driven mutation.
\item DAMA (Data-driven Mutation Analysis), which implements test suite evaluation based on data-driven mutation.
\item DAMTE (DAta-driven Mutation TEsting), which implements test suite augmentation based on data-driven mutation.
\end{itemize}}

\RQ{} \REVISION{P-4}{At high-level, the four \FAQAS components shall work as follows. All the components receive as input the software under test (SUT), the test suite to evaluate (SUT Test Suite), and a set of configurations. MASS generates a set of code-driven mutants, the mutation score, and a list of live mutants. SEMuS receive as input the list of live mutants and generate test cases that kill them. DAMA generates a set of data-driven mutants, the mutation score, and a list of live mutants. DAMTE receive as input the list of live data-driven mutants and generate test cases that kill them.}

\begin{figure}[h]
	\centering
		\includegraphics[width=12cm]{images/Architecture}
		\caption{FAQAS Architecture. Arrows represent data-flow.}
		\label{fig:architecture}
	\end{figure}

\clearpage
\section{Operational Environment}

\RQ{} \REVISION{P-5}{The \FAQAS shall run on any operating system provided that a bash shell and Python 3 are installed on it.}

\RQ{} The \FAQAS shall provide support to work on HPC infrastructure.

\RQ{} The \FAQAS shall be executed on the following minimum hardware specifications: 4 GB RAM and 10 GB of free disk space.

%\RQ{} The \FAQAS shall be executed on a Linux operating system with a Bash shell.

\RQ{} The \FAQAS shall be tested to work on the following Linux distributions: (i) Debian 9, (ii) Ubuntu 16, and (iii) CentOS 7.



\section{Assumptions and dependencies}

\RQ{} The \FAQAS shall process SUT built using either GCC Make~\footnote{https://gcc.gnu.org/onlinedocs/gccint/Makefile.html} or WAF\footnote{https://waf.io/}.

\REVISION{P-7}{\RQ{} The \FAQAS shall process SUT compiled with GCC~\footnote{https://gcc.gnu.org} versions above 4. }
%4.2.1, 5.4.0, and 6.3.0.

\RQ{} The \FAQAS shall process SUT test suites implemented as scripts (hereafter, \emph{test suite script}) that trigger the execution of test cases (each test case is executed in a separate process). Test suite scripts could be implemented either as Makefiles or a Bash scripts. Each test case is executed through the invocation of a command in the script.



\chapter{Requirements}

\section{Functional requirements}
\label{sec:requirements}

% !TEX root = MAIN.tex

\section{Applicable and reference documents}

\begin{itemize}
\item{D1 - Mutation testing survey}
\item{D2 - Study of mutation testing applicability to space software}
\end{itemize}
\clearpage

\section{Terms, definitions and abbreviated terms}

\begin{itemize}
\item{SUT}: Software under test
\end{itemize}

\clearpage


\chapter{Code-driven Mutation Testing}
\label{chapter:codeDriven}

\section{General description}




\subsection{Product perspective}

\RQ{} The code-driven mutation testing component (in Section~\ref{codeDriven} referred to as \emph{the system}) implements the Mutation Testing Process for code-driven mutation testing described in D2.

\clearpage
\subsection{General capabilities}

\RQ{} The code-driven mutation testing component shall implement the process for the evaluation of test suite effectiveness that is drafted in Figure~\ref{fig:process:codeDriven:evaluation}. Figure~\ref{fig:process:codeDriven:evaluation} relies on UML activity diagram notation. In Figure~\ref{fig:process:codeDriven:evaluation} the execution of specific software artefacts from the end user is made explicit. Also, we use black arrows to draw control-flow, red arrows for data-flow. Each activity is described in Section~\ref{sec:rquirements:capabilities}.

\begin{figure}[h]
  \centering
	\includegraphics[width=15cm]{images/png/Activity1!CodeDrivenTestSuiteEvaluation_1.png}
      \caption{Overview of the code-driven mutation testing process to evaluate test suite effectiveness.}
      \label{fig:process:codeDriven:evaluation}
\end{figure}


\RQ{} The code-driven mutation testing component shall implement the process for the augmentation of test suites effectiveness that is drafted in Figure~\ref{fig:process:codeDriven:augmentation}. Figure~\ref{fig:process:codeDriven:augmentation} relies on UML activity diagram notation. In Figure~\ref{fig:process:codeDriven:augmentation} the execution of specific software artefacts from the end user is made explicit. Also, we use black arrows to draw control-flow, red arrows for data-flow. Each activity is described in Section~\ref{sec:rquirements:capabilities}.

\begin{figure}[h]
  \centering
	\includegraphics[width=15cm]{images/png/Activity1!CodeDrivenTestSuiteAugmentation_2.png}
      \caption{Overview of the code-driven test suite augmentation process.}
      \label{fig:process:codeDriven:augmentation}
\end{figure}


\clearpage

\subsection{General constraints}

\RQ{} The automated generation of test cases (i.e., the objective of test suite augmentation) is an open, complex, research problem. For this reason, it is necessary to rely on existing tools.

\RQ{} The automated generation of test cases shall rely on KLEE, which is the most stable test case generation tool, based on WP2 evaluation.

\clearpage

\subsection{Operational environment}

% cosi generale?
\RQ{} The system works with a Linux operating system and Bash shell.



\subsection{Assumptions and dependencies}

\RQ{} The system targets SUT built using either GCC Make~\footnote{https://gcc.gnu.org/onlinedocs/gccint/Makefile.html} or WAF\footnote{https://waf.io/}.

\RQ{} The system targets SUT compiled with GCC~\footnote{https://gcc.gnu.org}.

\section{Specific requirements}
%\section{General}
\subsection{Capabilities requirements}
\label{sec:rquirements:capabilities}

%\RQ{} The gcov coverage information associated to each test case shall be stored in a separate directory.

\RQ{} The activity \emph{Compile SUT} in Figure~\ref{fig:process:codeDriven:evaluation} concerns the compilation of  the SUT with coverage options enabled.

\RQ{} The activity \emph{Prepare test scripts} in Figure~\ref{fig:process:codeDriven:evaluation} concerns extending the test scripts to store the code coverage of each single test case separately. This is achieved by adding a call to a dedicated bash script provided by FAQAS (\emph{FAQAS-CollectCodeCoverage}).

\RQ{} The activity \emph{Execute test cases} in Figure~\ref{fig:process:codeDriven:evaluation} concerns the execution of the test cases following the practice for the SUT.

\RQ{} The activity \emph{Execute FAQAS-GenerateCodeCoverageMatrix} in Figure~\ref{fig:process:codeDriven:evaluation} concerns the execution of a provided python program delivered with the FAQAS framework.

\RQ{} The activity \emph{Execute FAQAS-GenerateCodeCoverageMatrix} in Figure~\ref{fig:process:codeDriven:evaluation} generates a set of files: 
\begin{itemize}
\item one csv file referred to as \emph{global coverage matrix}, which indicates, for every line of code of the SUT, the ID of the test cases that cover the line of code;
\item a number of files  referred to as \emph{test coverage matrix}, one for each test case of the SUT. Each file indicates, for every line of code of the SUT, the number of times it has been covered during a single execution of the test case;
\item one file with the timeout after which we can consider a test case as non terminated (used in later stages). It is obtained by multiplying the test execution time times three.
\end{itemize}



\RQ{} Activity \emph{Execute FAQAS-GenerateMutants} in Figure~\ref{fig:process:codeDriven:evaluation} concerns the execution of the program \emph{FAQAS-GenerateMutants}. 

\RQ{} \emph{FAQAS-GenerateMutants} automatically generates a number of copies of each source file. Each copy contains one mutant.

% per ora abbiamo solo testato SRCIRor con files .c, dobbiamo valutare se funziona correttamente anche per .cpp
\RQ{} \emph{FAQAS-GenerateMutants} mutates source files with extension .c and .cpp.

\RQ{} \emph{FAQAS-GenerateMutants} generates mutants by applying a set of mutation operators that can be selected by the end-users.

\RQ{} \emph{FAQAS-GenerateMutants} implements the set of operators listed in Table~\ref{table:operators}

% !TEX root =  ../Main.tex

\newcommand{\op}{\mathit{op}}
\newcommand{\ArithmeticSet}{ \texttt{+}, \texttt{-}, \texttt{*}, \texttt{/}, \texttt{\%} }
\newcommand{\LogicalSet}{ \texttt{&&}, \texttt{||} }
\newcommand{\RelationalSet}{ \texttt{>}, \texttt{>=}, \texttt{<}, \texttt{<=}, \texttt{==}, \texttt{!=} }
\newcommand{\BitWiseSet}{ \texttt{\&}, \texttt{|}, \land }
\newcommand{\ShiftSet}{ \texttt{>>}, \texttt{<<} }


\begin{table}[h]
\caption{Implemented set of mutation operators.}
\label{table:operators} 
\centering
\scriptsize
\begin{tabular}{|@{}p{4mm}@{}|@{}p{2cm}@{\hspace{1pt}}|@{}p{11.1cm}@{}|}
\hline
&\textbf{Operator} & \textbf{Description$^{*}$} \\
\hline
\multirow{7}{*}{\rotatebox{90}{\emph{Sufficient Set}}}&ABS               & $\{(v, -v)\}$	\\
\cline{2-3}
&AOR               & $\{(\op_1, op_2) \,|\, \op_1, \op_2 \in \{ \ArithmeticSet \} \land \op_1 \neq \op_2 \} $       \\
&    			  & $\{(\op_1, \op_2) \,|\, \op_1, \op_2 \in \{\texttt{+=}, \texttt{-=}, \texttt{*=}, \texttt{/=}, \texttt{\%} \texttt{=}\} \land \op_1 \neq \op_2 \} $       \\
\cline{2-3}
&ICR               & $\{i, x) \,|\, x \in \{1, -1, 0, i + 1, i - 1, -i\}\}$           \\
\cline{2-3}
&LCR               & $\{(\op_1, \op_2) \,|\, \op_1, \op_2 \in \{ \texttt{\&\&}, || \} \land \op_1 \neq \op_2 \}$            \\
&				  & $\{(\op_1, \op_2) \,|\, \op_1, \op_2 \in \{ \texttt{\&=}, \texttt{|=}, \texttt{\&=}\} \land \op_1 \neq \op_2 \}$            \\
&				  & $\{(\op_1, \op_2) \,|\, \op_1, \op_2 \in \{ \texttt{\&}, \texttt{|}, \texttt{\&\&}\} \land \op_1 \neq \op_2 \}$            \\
\cline{2-3}
&ROR               & $\{(\op_1, \op_2) \,|\, \op_1, \op_2 \in \{ \RelationalSet \}\}$            \\
&				  & $\{ (e, !(e)) \,|\, e \in \{\texttt{if(e)}, \texttt{while(e)}\} \}$ \\
\cline{2-3}
&SDL               & $\{(s, \texttt{remove}(s))\}$            \\
\cline{2-3}
&UOI               & $\{ (v, \texttt{--}v), (v, v\texttt{--}), (v, \texttt{++}v), (v, v\texttt{++}) \}$            \\   
\hline
\hline
\multirow{5}{*}{\rotatebox{90}{\emph{OODL}}}&AOD               & $\{((t_1\,op\,t_2), t_1), ((t_1\,op\,t_2), t_2) \,|\, op \in \{ \ArithmeticSet \} $       \\ 
\cline{2-3}
&LOD               & $\{((t_1\,op\,t_2), t_1), ((t_1\,op\,t_2), t_2) \,|\, op \in \{  \} \}$       \\ 
\cline{2-3}
&ROD               & $\{((t_1\,op\,t_2), t_1), ((t_1\,op\,t_2), t_2) \,|\, op \in \{ \RelationalSet \} \}$       \\ 
\cline{2-3}
&BOD               & $\{((t_1\,op\,t_2), t_1), ((t_1\,op\,t_2), t_2) \,|\, op \in \{ \BitWiseSet \} \}$       \\ 
\cline{2-3}
&SOD               & $\{((t_1\,op\,t_2), t_1), ((t_1\,op\,t_2), t_2) \,|\, op \in \{ \ShiftSet \} \}$       \\ 
%\hline
%COR               & $\{(\op_1, \op_2) \,|\, \op_1, \op_2 \in \{ \texttt{\&\&}, \texttt{||}, \land \} \land \op_1 \neq \op_2 \}$            \\
\hline
\hline
\multirow{3}{*}{\rotatebox{90}{\emph{Other}}}&LVR			& $\{(l_1, l_2) \,|\, (l_1, l_2) \in \{(0,-1), (l_1,-l_1), (l_1, 0), (\mathit{true}, \mathit{false}), (\mathit{false}, \mathit{true})\}\}$\\
&&\\
&&\\
\hline
\end{tabular}

$^{*}$Each pair in parenthesis shows how a program element is modified by the mutation operator. Th eleft element of the pair is replaced with the right element. We follow standard syntax~\cite{kintis2018effective}. Program elements are literals ($l$), integer literals ($i$), boolean expressions ($e$), operators ($\op$), statements ($s$), variables ($v$), and terms ( $t_i$, which might be either variables or literals).
\end{table}


% quindi, non produrremo anche una lista coi nomi dei mutanti + la sua location?
\RQ{} \emph{FAQAS-GenerateMutants} generates as output a directory tree (\emph{mutants directory} in Figure~\ref{fig:process:codeDriven:evaluation}) that follows the structure of the source directory tree of the SUT. However, every source file is replaced by a folder having the same name. The folder contains all the mutants generated for that file. Every mutant has a name that univocally identify it. The mutant name results from the conjunction of the following information:
source file name, mutated function name, mutated line, mutation operator name, mutation operation, mutated “column” (i.e., char position from the beginning of the line).








\RQ{} Activity \emph{Prepare compilation scripts} in Figure~\ref{fig:process:codeDriven:evaluation} concern the modification of the main compilation script for the SUT. The engineer is expected to perform the following manual activities:
\begin{itemize}
\item Remove debugging flags
\item Remove coverage flags
\item Add placeholder for compiler optimization option
\item Add a 'sort' command in the source dependency list to ensure that source files are always compiled in the same order
\end{itemize}



\RQ{} Activity \emph{Execute FAQAS-CompileOptimizedMutants} in Figure~\ref{fig:process:codeDriven:evaluation} concerns the execution of the program \emph{FAQAS-CompileOptimizedMutants}.

\RQ{} The program \emph{FAQAS-CompileOptimizedMutants} compiles every mutant multiple times; once for every compiler optimization option selected by the end-user. It implements pseudocode in Figure~\ref{alg:CompileOptimizedMutants}.

\begin{figure}[h]
\begin{algorithmic}[1]

%\footnotesize
\scriptsize


\Require \emph{OPT}, the set of compiler optimization options specified by the end-user
\Require \emph{MutantsDir}, path to the directory tree containing the mutants
\Require \emph{SUTsources}, path of the folder containing the sources of the SUT
\Require \emph{CompilatonCommand}, the command to execute to compile the original software

\Ensure \emph{hashcodes csv}, a csv file containing for every mutant, for every option, the SHA512 hashcode of the generated executable

\Ensure \emph{unique mutants}, a csv file containing the list of unique mutants. Unique mutants are mutants that are not equivalent nor redundant. See D2 for details.
% (source inputs, follow-up inputs, output data).

\For {OPT in OPTS}
\For {Mutant in MutantsDir}
\State Compile \emph{Mutant} with program \emph{FAQAS-CompileAndExecute}
\State Generate a SHA512 hash of the generated executable
\State Put the generated SHA512 hash in the \emph{hashcodes csv} file
\EndFor
\EndFor

\State Process \emph{hashcodes csv} and identify \emph{unique mutants}
\State Save the list of \emph{unique mutants} in the output \emph{unique mutants csv} file

\end{algorithmic}
\vspace{-3mm}
\caption{FAQAS-CompileOptimizedMutants: Algorithm for compiling mutants with multiple optimization options}
%\vspace{-0.2cm}
\label{alg:CompileOptimizedMutants}
\end{figure}




\RQ{} Activity \emph{Execute FAQAS-CompileAndExecuteMutants} in Figure~\ref{fig:process:codeDriven:evaluation} concerns the execution of the program \emph{FAQAS-CompileAndExecuteMutants}.

\RQ{} The program \emph{FAQAS-CompileAndExecuteMutants} iterates over three activities (implemented by separate executable program that are inkoved automatically without user intervention): \emph{FAQAS-GeneratePrioritizedTestSuite}, \emph{FAQAS-CompileAndExecute}, \emph{FAQAS-IdentifyEquivalentAndRedundantMutants}.

\RQ{} The program \emph{FAQAS-CompileAndExecuteMutants} takes as inputs the mutants selection configuration, the unique mutants csv, the path of the SUT source folder, the command to execute test cases, and the path to the folder containing the test coverage matrixes.

\RQ{} The program \emph{FAQAS-CompileAndExecuteMutants} implements the four mutants selection strategies described in D2: \emph{all mutants}, \emph{proportional uniform sampling}, \emph{proportional method-based sampling}, \emph{uniform fixed-size sampling}, and \emph{uniform FSCI sampling}.

\RQ{} The \emph{mutants selection configuration} indicates the mutants selection strategy and a configuration value to specify the number of mutants to consider, which depends on the strategy; the value may indicate the percentage of mutants to sample (for \emph{proportional uniform sampling}, \emph{proportional method-based sampling}), the number of mutants to sample (for \emph{uniform fixed-size sampling}), the size of the confidence interval (for \emph{uniform FSCI sampling}).

\RQ{} The program\emph{FAQAS-GeneratePrioritizedTestSuite} takes as input the test coverage matrices and generate a file that specifies, for every line of the SUT, the prioritized list of test cases to execute (\emph{prioritized test suite csv}). This file indicates the sequence of test cases to execute for every mutants concerning a specific line.

\RQ{} The activity \emph{Randomly sort mutants} indicates that  \emph{FAQAS-CompileAndExecuteMutants} generate a randomly prioritized list of mutants to compile and execute from the \emph{unique mutants csv}.
In the case of \emph{proportional method-based sampling}, the list contains a set of mutants selected by following the stratified sampling strategy.

\RQ{} The activity \emph{Select and remove first mutant} indicates that  \emph{FAQAS-CompileAndExecuteMutants} select the first mutant in \emph{sorted list of mutants} and remove it from the list.

\RQ{} The program \emph{FAQAS-CompileAndExecute} compiles a mutant by running the makefile of the original program; then it executes the SUT test suite. It follows the algorithm in Figure~\ref{alg:compileAndExecute}.


\begin{figure}[h]
\begin{algorithmic}[1]
\scriptsize
\Require \emph{Mutant}, path of the mutant to compile
\Require \emph{SUTsources}, path of the folder containing the sources of the SUT
\Require \emph{CompilatonCommand}, the command to execute to compile the original software
\Require \emph{TestCommand}, the command to execute to execute a single test case
\Require \emph{TestCases}, the prioritized list of test cases for the line of the mutant
\Require \emph{TestTimeout}, the max execution time that can be taken by the test case
\Ensure \emph{Result} KILLED or LIVE, based on test execution result (i.e., all test cases pass or one test case fails)
\State put \emph{Mutant} in place of the file it has been derived (\emph{original file}), keep the original file in a safe place
\State execute  \emph{CompilatonCommand} inside \emph{SUTsources}
\For {TestCase in TestCases}
\State execute the \emph{TestCase} by running \emph{TestCommand} inside \emph{SUTsources}
% succede qualcosa strano quando scrivi "the"
\If {the \emph{TestCase} fails (i.e., \emph{TestCommand} terminates with an error code)}
\State set \emph{Result} as KILLED
\State break the for loop
\EndIf
\If {a the \emph{TestTimeout} expires}
\State set \emph{Result} as KILLED
\State break the for loop
\EndIf
\EndFor
\State move code coverage information in a subfolder of \emph{mutants coverage dir}
\State restore \emph{original file}
\end{algorithmic}
\caption{FAQAS-CompileAndExecute: Algorithm to compile and test mutants}
\label{alg:compileAndExecute}
\end{figure}

\RQ{} The program \emph{FAQAS-CompileAndExecute} collects the mutation results of every mutant in a file, \emph{mutation results csv}. It contains for every mutant the indication of the mutation result (KILLED/LIVE).

\RQ{} The program \emph{FAQAS-CompileAndExecute} compiles and execute mutants till a termination criteria is met. The termination criteria depends on the mutants selection strategy:
\begin{itemize}
\item \emph{all mutants}: the list \emph{sorted list of mutants} is empty
\item \emph{proportional uniform sampling}: a number of mutants matching the selected percentage has been executed
\item \emph{proportional method-based sampling}: the list \emph{sorted list of mutants} is empty
\item \emph{uniform fixed-size sampling}: a number of mutants matching the selected value has been executed
\item \emph{uniform FSCI sampling}: the confidence interval computed from \emph{mutation results csv} is smaller than the lenght specified by the user.
\end{itemize}

\RQ{} The program \emph{FAQAS-IdentifyEquivalentAndRedundantMutants} relies on code coverage information stored in \emph{mutants coverage dir} to identify equivalent and redundant mutants using the distance criterion $D_C$ (see D2).

\RQ{} The program \emph{FAQAS-IdentifyEquivalentAndRedundantMutants} generates a copy of \emph{mutation results csv} (i.e., \emph{univocal mutation results csv}) where only mutants that are considered non-equivalent and non-redundant are reported.

\RQ{} The activity \emph{Compile mutation score} concerns the computation of the mutation score based on the mutation results reported in \emph{univocal mutation results csv}.

\RQ{} The activity \emph{Execute FAQAS-CompileAndExecuteMutants} in Figure~\ref{fig:process:codeDriven:augmentation} concerns the execution of the program \emph{FAQAS-GenerateTestGenerationScaffolding}.

\RQ{} The program \emph{FAQAS-GenerateTestGenerationScaffolding} takes as input the path of the \emph{SUT source code} and the file \emph{mutation results csv}. It generates a number of files named \emph{MutatantId\_AnalysisMain.c}, one for each live mutant, where MutantId is the ID of a mutant. The file \emph{MutatantId\_AnalysisMain.c} contains a main function that should be used for the analysis with KLEE. 

\RQ{} The content of file \emph{MutatantId\_AnalysisMain.c} should resemble Listing 1.7 and Listing 1.9 of D2 to enable the analysis with KLEE. For example, it should import the source file with the original function targeted by the mutation and the source code of the mutated function. Also, it should contain the definition of all the variables used for the execution of KLEE and a tentative set of required assertions.

\RQ{} The activity \emph{Update scaffolding for mutant} in Figure~\ref{fig:process:codeDriven:augmentation} indicates that the engineer should modify the file  \emph{MutatantId\_AnalysisMain.c} if necessary. In partcular, it might be necessary to refine the assertions produced by \emph{FAQAS-GenerateTestGenerationScaffolding}. More precisely, since assertions should concern output variables, it is necessary to verify that all the necessary output variables had been reported. Indeed, with pointers and pointers to pointers, it is not possible to have a precise identification of output variables.

\RQ{} The activities in the expansion region \emph{generateTestCase} are repeated for every live mutant.

\RQ{} The activity \emph{Execute Execute FAQAS-GenerateTestCase} in Figure~\ref{fig:process:codeDriven:augmentation} concerns the execution of the program \emph{FAQAS-GenerateTestCase}.

\RQ{} The program \emph{FAQAS-GenerateTestCase} generates a tentative unit test case (i.e., a source file in C) that kills the mutant. It executes the KLEE program and the produce a test case file after processing the KLEE output.

\RQ{} The test case generated by \emph{FAQAS-GenerateTestCase} contains an invocation of the function under test (i.e., the function targeted  by the mutation) along with assigned arguments and an assertion that verifies results. The values for the assigned arguments and the verification of results are derived from KLEE output.

\RQ{} If the program \emph{FAQAS-GenerateTestCase} successfully generate a test case the engineer proceeds with inspecting it (activity \emph{Update Test  Case}), otherwise he can consider the mutant as equivalent (activity \emph{Update mutation results}).

\RQ{} The activity \emph{Update Test  Case} in Figure~\ref{fig:process:codeDriven:augmentation} is performed by the engineer. He may need to execute the generated test case to verify that to correctly execute (in case KLEE has generated invalid inputs). The engineer also verify that the assertion with the expected value is correct (i.e., if it matches the specifications). If the expected value is not correct, the SUT might be faulty and should be fixed.

\RQ{} The activity \emph{Add Test Case to SUT Test Suite} in Figure~\ref{fig:process:codeDriven:augmentation} is performed by the engineer, who may add the new test case to the test suite.

\RQ{} The activity \emph{Update mutation results} in Figure~\ref{fig:process:codeDriven:augmentation} is performed when a test case is not generated. This generally happens when the mutant cannot be killed (i.e., is equivalent). The engineer is expected to manually inspect the mutant to be sure that the mutant is equivalent (otherwise the missing test case is due to a limitation of KLEE). If the mutant is equivalent the engineer removes it from the file \emph{mutation results csv}.

\RQ{} The activity \emph{Execute FAQAS-RecomputeMutationScore} in Figure~\ref{fig:process:codeDriven:augmentation}  concerns the execution of the program \emph{FAQAS-RecomputeMutationScore}. It is performed after generating test cases for all the live mutants. Program \emph{FAQAS-RecomputeMutationScore} recomputes the mutation score after ignoring the equivalent mutants detected by KLEE.

\subsection{System interface requirements}

\RQ{} The main user interface for the system is the command line.

\subsection{Adaptation and missionization requirements }

\RQ{} The system shall not be used in mission. The system is tool aimed at supporting the development of flight software. The system shall be used only to support development, validation, and verification activities. 

\subsection{Computer resource requirements}

\RQ{} The system should be executed on a Linux operating system.

\subsection{Security requirements }

\RQ{} The system should not use ports or use network connections.

\subsection{Safety requirements}

\RQ{} To avoid safety problems, the system shall not be used to assess test cases that are executed with target hardware in the loop.

\RQ{} The system cannot foresee the effect of mutation. If executed on the final hardware, the generated mutants might damage the hardware or cause injuries to surrounding people.

\subsection{Reliability and availability requirements}

\RQ{} The system is expected to work according to is functional specifications every time it is invoked.

\RQ{} Since mutation testing execution time depends on both the number of mutants to be executed and the duration of the test suite execution, it is not possible to provide an upper bound for mutation testing execution.

\subsection{Quality requirements}

\RQ{} Usability. Software engineers (i.e., professionals with a master degree in informatics or related fields) should be able to successfully use the software after reading its documentation.

\RQ{} Reusability. The software shall be used in any environment matching the characteristics indicated in this document.

\RQ{} Software development standards. The software development process shall follow ECSS guidelines as per SoW.

\subsection{Design requirements and constraints}

\RQ{} The system should be released with ESA Software Community Licence Permissive – v2.3”, as defined at https://essr.esa.int/. Any reused component should be compatible with the abovementioned licence.

\RQ{} The system shall be implemented using bash shell script language, Python, and C.

\subsection{Software operations requirements}

None foreseen.

\subsection{Software maintenance requirements}

None foreseen.

\subsection{System and software observability requirements}

\RQ{} To enable post-mortem debugging, all the temporary files generated by the FAQAS executables should be kept.

\section{Verification, validation and system integration}
 \subsection{Verification and validation process requirements}
 
 \RQ{} Every mutation operator should be tested by a dedicated unit test.
 
 \RQ{} A system test suite for the whole software should be provided. It should be based on MLFS case study.
 
   \RQ{} The system should enable the computation of the mutation score for the FAQAS case study systems indicated in deliverable D2.
 
 \subsection{Validation approach}

\RQ{} SnT is expected to perform a preliminary validation of the delivered framework.

 \RQ{} FAQAS industry partners are expected to use the system at their premises to validate it.
 

%  
% \subsection{Validation requirements}
% 
%
% 
% \subsection{Verification requirements}
% 
% \RQ{}

 \section{System models}
 
 None reported.


% !TEX root = MAIN.tex

\subsection{Test Suite Evaluation Based on Data-driven Mutation}
\label{sec:dataDriven}

The following requirements regard the Test Suite Evaluation Based on the Data-driven Mutation Testing functionality of the \FAQAS.

\RQ{} The \FAQAS shall work with a fault model and a data model for the SUT specified according to D2.

\RQ{} The \FAQAS shall provide a mutation analysis API (i.e., predefined functions to perform data mutation for buffers).

\RQ{} The \FAQAS shall automatically generate the data mutation probes to be integrated within the SUT.

\RQ{} The engineer shall manually modify the source code of the SUT to integrate mutation probes into it.

%\RQ{} The \FAQAS shall require the engineer to manually modify the scripts used to execute test cases so they include an invocation to \FAQAS after the execution of every single test case.
%
%\RQ{} The \FAQAS shall receive as inputs the command to compile the SUT, the command to execute the test suite, the path to the extended SUT source code, and the mutants selection configuration.

\RQ{} The \FAQAS shall implement the four mutants selection strategies: \emph{all mutants} (i.e., all the mutants are tested), \emph{proportional uniform sampling} (i.e., a subset of the mutants is tested selected based on a percentage), \emph{uniform fixed-size sampling} (i.e., a subset of the mutants is tested selected based on a fixed number), and \emph{uniform FSCI sampling} (i.e., a subset of the mutants is tested, they are selected according to the FSCI criterion).

\RQ{} The \FAQAS shall provide the engineer with the ability to indicate the mutants selection strategy and a configuration value that specifies the number/percentage of mutants to consider.

\RQ{} The \FAQAS shall enable the compilation of a version of the SUT that traces the data items (targeted by mutation) that are covered by each test case.

\RQ{} The \FAQAS shall derive, from logs generated during the execution of the SUT test suite, the data items exercised by each test case.

\RQ{} The \FAQAS shall compile a version of the SUT with the selected mutation operation instance enabled.

\RQ{} For each mutation operator, the \FAQAS shall execute only the test cases exercising the data item targeted by the mutation operator.

\RQ{} The \FAQAS shall automatically determine if a mutant has been killed. The \FAQAS shall terminate the test suite execution when a mutant is killed.

\RQ{} The \FAQAS shall execute the test suite till a termination criterion is met. The termination criterion depends on the mutants selection strategy:
\begin{itemize}
\item \emph{.a) all mutants}: all mutants has been executed
\item \emph{.b) proportional uniform sampling}: a number of mutants matching the selected percentage has been executed
\item \emph{.c) uniform fixed-size sampling}: a number of mutants matching the selected value has been executed
\item \emph{.d) uniform FSCI sampling}: the confidence interval computed from \emph{mutation results csv} is smaller than 10\%.
\end{itemize}

\RQ{} The \FAQAS shall compute the mutation score based on the mutation results.

\RQ{} The \FAQAS shall provide the engineer with the means to inspect the mutation results. The \FAQAS data-driven mutation results shall include the list of data types not covered by any test case, the list of data types exercised by each test case, the list of live mutants, the list of killed mutants.

%\RQ{} The data-driven mutation testing component from \FAQAS shall not implement any feature to automatically generate test cases.

\subsection{Test Suite Augmentation Based on Data-driven Mutation}
\label{sec:codeDrivenAugmentation}

The following requirements regard the Test Suite Augmentation functionality that aims to generate test cases that kill mutants generated with the Data-driven Mutation functionality of the \FAQAS.

\RQ{} The \FAQAS shall generate a version of the testing API that shall contain reachability assertions that enable KLEE to generate inputs that reach the mutation.

\RQ{} The \FAQAS shall not fully automate the Data-driven Test Suite Augmentation functionality.

\remark The engineer shall execute KLEE, after performing the required scaffolding, if necessary.
Also, the engineer shall implement a test case based on KLEE's output.




\section{System interface requirements}

\DONE{Oscar P-15}

\RQ{} The user interface of the \FAQAS is provided through the command line.

\REVISION{P-15a}{\RQ{} The \FAQAS shall use \texttt{gcov}\footnote{https://gcc.gnu.org/onlinedocs/gcc/Gcov.html} to generate and process the code coverage of the SUT.}


\REVISION{P-15c}{\RQ{} The \FAQAS shall support the following commands:
	\begin{itemize}
	\item \texttt{PrepareSUT}: command to prepare the SUT and collect information about the SUT test suite.
	\item \texttt{GenerateMutants}: command to generate mutants from the SUT source code.
	\item \texttt{CompileOptimizedMutants}: command to compile the mutants with the multiple optimisation levels.
	\item \texttt{OptimizedPostProcessing}: command to disregard equivalent and redundant mutants based on compiler optimisations.
	\item \texttt{GeneratePTS}: command to generate the prioritized and reduced test suites.
	\item \texttt{ExecuteMutants}: command to execute mutants against the SUT test suite.
	\item \texttt{IdentifyEquivalents}: command to identify equivalent mutants based on code coverage.
	\item \texttt{MutationScore}: command to compute the mutation score and final reporting.
	\item \texttt{PrepareMutants\_HPC}: command to prepare the mutants workspace for the execution on HPCs.
	\item \texttt{ExecuteMutants\_HPC}: command to execute mutants on HPCs.
	\item \texttt{PostMutation\_HPC}: command to assess past mutant executions, and to decide whether more mutant executions are needed.
\end{itemize}
}

\REVISION{P-15b}{\RQ{} The \FAQAS shall generate the following log files:
	\begin{itemize}
		\item output concerning SUT compilation process
		\item output concerning execution of SUT test cases
		\item output concerning the execution of the \texttt{CompileOptimizedMutants} command
		\item output concerning the execution of the \texttt{OptimizedPostProcessing} command
		\item output concerning the execution of the \texttt{GeneratePTS} command
		\item output concerning the execution of the \texttt{ExecuteMutants} command
		\item output concerning the execution of the \texttt{IdentifyEquivalents} command
		\item output concerning the execution of the \texttt{MutationScore} command
		\item output concerning the execution of the \texttt{PrepareMutants\_HPC} command
		\item output concerning the execution of the \texttt{ExecuteMutants\_HPC} command
		\item output concerning the execution of the \texttt{PostMutation\_HPC} command
	\end{itemize}
}

\REVISION{P-15d}{
\RQ{} The \FAQAS shall be configured through a text file named \texttt{mass\_conf.sh}.

\RQ{} The \FAQAS configuration file shall be set with the following parameters:
	\begin{itemize}
		\item \FAQAS installation path
		\item \FAQAS workspace path
		\item SUT relevant paths (e.g., source code folder, test folder, coverage folder)
		\item SUT compilation command
		\item SUT test case execution command
		\item \FAQAS execution environment mode (i.e., single machine, HPC)
		\item \FAQAS trivial compiler optimisation flags
		\item \FAQAS sampling type
		\item \FAQAS sampling rate
		\item \FAQAS test suite prioritization type
	\end{itemize}
}

\RQ{} \FAQAS shall provide built-in features to process the SUT test harness Google Test\footnote{https://github.com/google/googletest} and the Basic mathematical Library Test Suite (BLTS)\footnote{https://essr.esa.int/project/mlfs-mathematical-library-for-flight-software}.

\REVISION{P-16}{\RQ{} \FAQAS shall provide means to configure its execution with any test harness infrastructure for C/C++ software.}

\RQ{} The \FAQAS shall provide the engineer with the ability to configure the environment variables used to tune the mutation testing process.



\section{Adaptation and missionization requirements}

\RQ{} The \FAQAS shall not be used in mission.

\RQ{} The \FAQAS shall be used in the development environment.




\section{Computer resource requirements}


\RQ{} The \FAQAS shall run on Linux operating systems.



\section{Security requirements }

\RQ{} The \FAQAS shall not use network connections.

\section{Safety requirements}

\RQ{} The \FAQAS shall not be used to assess test cases that are executed with target hardware in the loop, to avoid safety hazards.

\section{Reliability and availability requirements}

\RQ{} The \FAQAS shall work according to its functional specifications every time it is invoked.

% \RQ{} The system is expected to work according to is functional specifications every time it is invoked.

\RQ{} The \FAQAS shall not provide an upper bound for mutation testing execution time.

\remark Mutation testing execution time depends on both the number of mutants to be executed and the duration of the test suite execution, information that cannot be foreseen before execution.

\section{Quality requirements}

\RQ{} Usability. Software engineers shall be able to successfully use the \FAQAS after reading its documentation.

\RQ{} The \FAQAS shall be used in any environment matching the characteristics indicated in this document.

\RQ{} The \FAQAS shall comply with software development standards. The software development process shall follow ECSS guidelines as per SoW.


\section{Design requirements and constraints}

\RQ{} The \FAQAS shall be released with ESA Software Community Licence Permissive – v2.3, as defined at https://essr.esa.int/. Any reused component should be compatible with the abovementioned licence.

\RQ{} The \FAQAS shall be implemented using the bash shell script language, the Python programming language, and the C/C++ programming language.

\RQ{} \FAQAS shall mutate the source code of the SUT not the intermediate representation.


\section{Adaptation and installation requirements}

% - Deployment methods (it is maybe a bit early, but we have to think on the methodology to deploy the tool e.g. compile it, have it delivered with an installer, on a container, etc)

%
\RQ{} The \FAQAS should be installed using an installer software.


\section{Software operations requirements}

\RQ{} The \FAQAS shall be executed at anytime,
%It should be possible to execute the system anytime,
in an environment matching the characteristics indicated in this document.

\section{Software maintenance requirements}

\RQ{} One year maintenance support shall be provided as per SoW.

\section{System and software observability requirements}

\RQ{} The debugging of the \FAQAS shall be possible if and only if
%To enable post-mortem debugging,
all the temporary files generated by the system are kept.

\chapter{Verification, validation and system integration}
 \section{Verification and validation process requirements}

\RQ{} Every mutation operator implemented by the \FAQAS shall be tested by a dedicated unit test case.

\RQ{} Unit test cases shall be executed on a x86-64 desktop PC.

\RQ{} Unit test cases execution time shall not exceed one day.

\RQ{} A system test suite for the Code-Driven Mutation Testing component shall be provided. The Code-Driven Mutation Testing component shall be tested as a whole based on the MLFS case study (see D2).

 \RQ{} The \FAQAS shall enable the correct computation of the mutation score for the FAQAS case study systems indicated in deliverable D2.

 \subsection{Validation approach}

\RQ{} SnT shall perform a preliminary validation of the delivered framework.

\RQ{} FAQAS industry partners shall use the system at their premises to validate it.
