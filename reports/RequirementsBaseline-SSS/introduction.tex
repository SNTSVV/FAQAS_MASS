% !TEX root = MAIN.tex

% \section{Introduction}
% \label{sec:introduction}
% \addcontentsline{toc}{chapter}{Introduction}


%PTCR-SSS-PABG-04:
%My feeling is that it would be better to use a specific identifier for each requirement, otherwise we will end up having a document with plenty of sub-chapters. It will also be easier when we will have to assign test cases to the requirements.

%e.g. you could use FAQAS-SSS-REQ-xyz; or any other id that you would like to use; or you could have different types of ids per functionality e.g. FAQAS-CodeMut-xxx, FAQAS-DataMut-yyy, FAQAS-TestGen-zzz, etc, etc.



% PTCR-SSS-PABG-05:
% From functional point of view, it makes sense to split the requirements on Code-Driven and Data-Driven. But my feeling is that some other types of requirements will be common to both since applicable to the full FAQAS framework. For this last group, there is no need to duplicate them.
% Perhaps, for the functional requirements, the best is to split them according to the architecture of the tool that we can find in the proposal (pasted in below in this document).


This document is the deliverable SSS of the ESA activity ITT-1-9873-ESA. It concerns requirements specification for the \emph{FAQAS framework} to be delivered by ITT-1-9873-ESA. Following the structure described in the SoW \emph{AO9873-ws00pe\_SOW.pdf}, it provides the structured requirements baseline for the FAQAS framework according to ECSS-E-ST-40C Annex B. Since the \emph{FAQAS framework} implements two distinct functionalities, code-driven mutation testing and data-driven mutation testing, this document contains two separate chapters, each one concerning one of the two features: Chapter~\ref{chapter:codeDriven} concerns code-driven mutation testing, Chapter~\ref{chapter:dataDriven} concerns data-driven mutation testing.


Requirements are univocally identified by the paragraph id appearing on the left.
 
 
 

