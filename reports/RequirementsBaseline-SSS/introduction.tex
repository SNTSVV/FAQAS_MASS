% !TEX root = MAIN.tex

\chapter{Introduction}

This document is the deliverable SSS of the ESA activity ITT-1-9873-ESA. It concerns requirements specification for the \emph{FAQAS framework} to be delivered by ITT-1-9873-ESA. Following the structure described in the SoW \emph{AO9873-ws00pe\_SOW.pdf}, it provides the structured requirements baseline for the FAQAS framework according to ECSS-E-ST-40C Annex B.
 
\section{Applicable and reference documents}

\begin{itemize}
\item{D1 - Mutation testing survey}
\item{D2 - Study of mutation testing applicability to space software}
\end{itemize}

\chapter{Terms, definitions and abbreviated terms}

\begin{itemize}
\item{FAQAS}: activity ITT-1-9873-ESA
\item{FAQAS-framework}: software system to be released at the end of WP4 of FAQAS
\item{D2}: Deliverable D2 of FAQAS, \emph{Study of mutation testing applicability to space software}
\REVISION{P-6}{\item{HPC}: High Performance Computing. Computing infrastructure where end-users can execute tasks in parallel on multiple nodes.}
\item{KLEE}: Third party test generation tool, details are provided in D2.
\item{SUM}: Software User Manual, as per definition in ECSS-E-ST-40C.
\item{SUT}: Software under test, i.e, the software that should be mutated by means of mutation testing.
\item{test suite augmentation}: the process of generating one or more test cases to kill one or more mutants. A test case is said to kill a mutant if it fails when executed with the mutated software. Test suite augmentation can be automated or semi-automated.
\item{WP}: Work package

\end{itemize}

\clearpage
 

