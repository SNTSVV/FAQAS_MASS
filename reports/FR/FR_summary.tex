% !TEX root = MAIN.tex
\clearpage

\section{Summary of FAQAS Validation Results}
\label{sec:summary:results}

In this section, we report the summary of results for the FAQAS framework. In particular, for each FAQAS approach, we provide a table with the main output of the technique (e.g., mutation score, number of test cases generated).

\subsection{Code-driven mutation analysis}

\begin{table}[htb]
\caption{Code-driven mutation analysis results: MASS mutation score.}
\label{table:results:mass} 
\small
\centering
\begin{tabular}{|
>{\arraybackslash}p{54mm}@{\hspace{1pt}}|
>{\raggedleft\arraybackslash}p{40mm}@{\hspace{1pt}}|
}
\hline
\textbf{Subject}&\textbf{MASS Mutation Score (\%)}\\
\hline

\SAIL{}$_{S}$ (System test suite)&65.95\\

\SAIL{}$_{S}$ (Unit+System test suite)&70.56\\

\GCSP{}&70.92\\
\PARAM{}&85.95\\

\UTIL{}&84.41\\
\MLFS{}{}&93.49\\
% K: 3104 L: 2219-1480=739 T: 5323-1480=3843
ASN1SCC&80.77\\
\hline
$\textbf{Average}$&78.86\\
\hline
\end{tabular}

\end{table}

Table~\ref{table:results:mass} introduces the code-driven mutation analysis results, and compares, for each subject, the mutation score of a traditional mutation process with the one obtained with MASS. 


\subsection{Code-driven mutation testing (test suite augmentation)}

\begin{table}[htb]
\caption{Test suite augmentation results.}
\label{table:results:test-gen} 
\centering
\footnotesize
\begin{tabular}{|
@{\hspace{1pt}}p{10mm}|
@{\hspace{1pt}}>{\raggedleft\arraybackslash}p{18mm}@{\hspace{1pt}}|
>{\raggedleft\arraybackslash}p{35mm}@{\hspace{1pt}}|
>{\raggedleft\arraybackslash}p{25mm}@{\hspace{1pt}}|
 >{\raggedleft\arraybackslash}p{25mm}@{\hspace{1pt}}|
}
\hline
\textbf{Subject}&\textbf{Live Mutants}&\textbf{Additionally Killed Mutants}&\textbf{Original MS (\%)}&\textbf{Updated MS (\%)}\\ 
\hline
$\mathit{MLFS}$&3\,891&697&81.80&85.06\\
$\mathit{ASN.1}$&2\,219&1\,729&58.31&90.79\\
$\mathit{ESAIL_S}$&1\,041&NA&70.56&NA\\
% additionally killed: clock 2 error 4 timestamp 6 memory 21
$\mathit{Libutil}$&4\,198&35&81.80&81.96\\
\hline
\end{tabular}

\end{table}

Table~\ref{table:results:test-gen} introduces the code-driven mutation testing results; in particular, the number of mutants additionally killed by SEMuS, and the mutation score of the test suite including the new test cases generated by SEMuS.


\subsection{Data-driven mutation analysis}

\begin{table}[htb]
\caption{Data-driven mutation analysis results.}
\label{table:results:data-driven} 
\center
\footnotesize
\begin{tabular}{|
@{\hspace{0pt}}>{\raggedleft\arraybackslash}p{24mm}@{\hspace{1pt}}|
@{\hspace{0pt}}>{\raggedleft\arraybackslash}p{12mm}@{\hspace{1pt}}|
@{\hspace{0pt}}>{\raggedleft\arraybackslash}p{12mm}@{\hspace{1pt}}|
@{\hspace{0pt}}>{\raggedleft\arraybackslash}p{18mm}@{\hspace{1pt}}|
@{\hspace{0pt}}>{\raggedleft\arraybackslash}p{12mm}@{\hspace{1pt}}|
@{\hspace{0pt}}>{\raggedleft\arraybackslash}p{12mm}@{\hspace{1pt}}|
@{\hspace{0pt}}>{\raggedleft\arraybackslash}p{12mm}@{\hspace{1pt}}|
@{\hspace{0pt}}>{\raggedleft\arraybackslash}p{12mm}@{\hspace{1pt}}|
@{\hspace{0pt}}>{\raggedleft\arraybackslash}p{12mm}@{\hspace{1pt}}|
}
\hline
\textbf{Subject} & 
\textbf{\# FMs} & 
\textbf{FMC} & 
\textbf{\#MOs-CFM} & 
\textbf{\#CMOs} & 
\textbf{MOC}  
&\textbf{Killed}&\textbf{Live}&\textbf{MS}
\\
\hline

\ADCS &10 &90.00\%   & 135 & 100 & 74.00\%   &    45&55&45.00\%\\
\GPS &1 &100.00\%    &  23  &  22 & 95.65\%    &      21&1&95.45\%\\
\PDHU &3 &100.00\%  &   29 & 24 & 82.76\%   &     24&0&100.00\%\\
\PARAM &6 &100.00\%  &   80 & 73 & 91.25\%  &        28&45&38.36\%\\


\hline

\end{tabular}

FM=Fault Model, FMC=Fault Model Coverage, MOs-CFM=Mutation Operations in covered FMs,
CMO=Covered Mutation Operation, MOC=Mutation Operation Coverage, Killed=Number of mutants killed by the test suite, Live=Number of mutants not killed by the test suite, MS=Mutation Score.

\end{table}


Table~\ref{table:results:data-driven} shows the mutation analysis results of DAMAT, our data-driven mutation analysis technique. For each subject, we provide information about the fault model coverage, the mutation operation coverage, and the mutation score of the technique.


