% !TEX root = MAIN.tex

\chapter{FAQAS case studies}
\label{chapter:caseStudies}
 

\STARTCHANGEDFINAL
The FAQAS toolset has been applied to \MREVISION{C-P-40}{six} case study systems:
ESAIL, \GCSP{}, \PARAM{}, \UTIL{}, MLFS, ASN1SCC.

 
\INDEX{ESAIL} is a microsatellite developed by LXS in a Public-Private-Partnership with ESA and ExactEarth. For our empirical evaluation, we considered the onboard control software of \SAIL{} (hereafter, simply \SAIL{}\emph{-CSW}), which consists of 924 source files with a total size of 187,116 LOC. 
\SAIL{}\emph{-CSW} is verified by unit test suites and system test suites that run in different facilities (e.g., Software Validation Facility~\cite{Isasi2019}, FlatSat~\cite{Eickhoff:Simulate}, Protoflight Model~\cite{ecssHB10A}). We focus on the unit test suite and the SVF test suite because the other test suites require dedicated hardware.  
To address some of our research questions, all the mutants must be executed against the test suite, which is not feasible for the case of \SAIL{}\emph{-CSW} due to its large size and test suite. For this reason, we have identified a subsystem of \SAIL{}\emph{-CSW} (hereafter, \emph{\SAIL{}}$_{S}$) that consists of a set of files, selected by \TWO engineers, that are representative of the different functionalities in \SAIL{}\emph{-CSW}: service/protocol layer functions, critical functions of the satellite implemented in high-level drivers, application layer functions.

\INDEX{\GCSP{}}, \INDEX{\PARAM{}}, and \INDEX{\UTIL{}}  are utility libraries developed by \ONE.
\emph{\GCSP{}} is a network protocol library including low-level drivers (e.g., CAN, I2C).
%an extension of \OPENCSP{}; it provides convenience wrapping of \CSP functionality,
%definition of standard \CSP ports, and low-level drivers (e.g., CAN, I2C).
{\PARAM{}} is a light-weight parameter system designed for \ONE satellite subsystems. 
{\UTIL{}} is a utility library providing cross-platform APIs for use in both embedded systems and Linux development environments.



The \INDEX{Mathematical Library for Flight Software}\footnote{https://essr.esa.int/project/mlfs-mathematical-library-for-flight-software} (MLFS) implements mathematical functions ready for qualification. 
%\DONE{Please rewrite the following sentence}
MLFS is born from the need of having a mathematical library ready for qualification for flight software. 
In FAQAS, we considered the unit test suite of MLFS (it achieves branch and MC/DC coverage).

\INDEX{ASN1SCC}\footnote{https://github.com/ttsiodras/asn1scc} is an open source ASN.1 compiler that generates C/C++ and SPARK/Ada code suitable for low resource environments such as space systems. Moreover, the compiler can produce a test harness that provides full statement coverage in the generated code, and therefore significantly improves its quality. In the context of FAQAS, we focus our analysis on the source code automatically generated by the ASN.1 compiler, rather than in ASN1SCC software itself. 

Except for \SAIL{}\emph{-CSW}, all subjects
are compiled to generate executables for the development environment OS (Linux); we rely on the Gnu Compiler Collection (GCC)  for Linux X86~\cite{GCC} versions 5.3 and 6.3 for \MLFS{}{} and ONE, respectively. \SAIL{}\emph{-CSW} is compiled with the
LEON/ERC32 RTEMS Cross Compilation System, which includes the GCC C/C++ compiler version 4.4.6 for  RTEMS-4.8 (Sparc architecture)~\cite{RTEMS}.



Table~\ref{table:caseStudies:figures} provides additional details about each case study.


For the validation of each tool in the FAQAS toolset, we selected case studies presenting characteristics compatible with the requirements of the tool under test. Table~\ref{tab:caseStudies} provides the list of case studies along with an indication of the type of mutation analysis/testing (i.e., code-driven or data-driven) and the tools they are targeted for.  \emph{Code-driven mutation analysis} (implemented by MASS) does not present any specific requirement except the availability of source code; for this reason, it has been validated with all the available case studies systems.
\emph{Code-driven mutation testing} (implemented by SEMuS), instead, requires the software under test to be comprised of components communicating through function calls (e.g., not network channels); for this reason, for SEMuS, we selected unit test suites.
\emph{Data-driven mutation analysis} (implemented by DAMAt) targets components communicating through channels, which are usually tested with integration and system test suites. For this reason we selected systems tested with such types of test suites.
\emph{Data-driven mutation testing} (implemented by DAMTE) aims to generate test cases that improve data-driven mutation analysis; however, since it is performed with symbolic execution tools, it presents the same limitations of SEMuS, that is, the analyzed part of the software under test should be comprised of components communicating through function calls. For this reason, it can target only libGCSP and libParam. We selected libParam because it is a higher-level library and thus more representative for this specific case.

\begin{table}[tb]
\caption{Overview of subject artefacts.}
\label{table:caseStudies:figures} 
\scriptsize
\centering
\begin{tabular}{|
@{\hspace{1pt}}p{13mm}
@{\hspace{2pt}}|
@{\hspace{1pt}}>{\raggedleft\arraybackslash}p{8mm}@{\hspace{1pt}}|
@{\hspace{1pt}}>{\raggedleft\arraybackslash}p{18mm}@{\hspace{1pt}}|
@{\hspace{1pt}}>{\raggedleft\arraybackslash}p{20mm}@{\hspace{1pt}}|
@{\hspace{1pt}}>{\raggedleft\arraybackslash}p{24mm}@{\hspace{1pt}}|
p{20mm}|}
\hline
\textbf{Subject}&\textbf{LOC}&\textbf{Test suite type}&\textbf{\# Test cases}&\textbf{Statement} \textbf{coverage}\\
\hline
\mbox{\SAIL{}\emph{-CSW}}& 74,155 & System& 384 & 90.38\% \\
\SAIL{}$_S$& 2,235 & System& 384 & 95.36\%\\
%ESAIL execution time: 8.3 hours
\GCSP{}& 9,836 & Integration& 89 & 63.10\%\\
%LIBCSP execution time: 107 seconds
\PARAM{}& 3,179 & Integration& 170 & 77.60\%\\
%LIBPARAM execution time: 50 seconds
\UTIL{}& 10,576 & Unit& 201 & 83.20\%\\
%LIBUTIL execution time: 67 seconds
\MLFS{}{}& 5,402 & Unit& 4042 & 100.00\%\\
ASN1CC& 4,338 & Unit& 107 & 99.13\%\\
%MLFS execution time: 313 seconds (5.2 minutes)
%\MLFS{}{}$_V$& 5402 & 4042 & 100.00\%\\
\hline
\end{tabular}

\end{table}

\begin{table}[htp]
\caption{Case studies for the FAQAS activity.}
\label{tab:caseStudies}
\begin{center}
\begin{tabular}{|p{1.2cm}|p{6cm}|p{1.5cm}|p{1.5cm}|p{1.5cm}|p{1.5cm}|}
\hline
\textbf{}&\textbf{}&\multicolumn{2}{c}{\textbf{Code-driven}}&\multicolumn{2}{c|}{\textbf{Data-driven}}\\
\textbf{Partner}&\textbf{Case study}&\textbf{MASS}&\textbf{SEMuS}&\textbf{DAMAt}&\textbf{DaMTe}\\
\hline
LXS&System Test Suite for ESAIL&Y&N&Y&N\\
LXS&Unit Test Suite for ESAIL&Y&Y&N&N\\
GSL&Unit Test Suite for libUtil&Y&Y&N&N\\
GSL&Integration Test Suite for libgscsp&Y&N&Y&N\\
GSL&System Test Suite for libparam&Y&N&Y&Y\\
ESA&MLFS mathematical library&Y&Y&N&N\\
ESA&ASN1 Compiler&Y&Y&N&N\\
\hline
\end{tabular}
\end{center}
%An asterisk (*) is used to indicate validation cases that will be finalized in WP4.
\end{table}

\ENDCHANGEDFINAL

\clearpage

% !TEX root = MAIN.tex

\section{LXS - ESAIL System Test Suite}
\label{chapter:caseStudies:LXS}

\subsection{Overview of the case study}

ESAIL is a microsatellite developed by LXS in a PPP with ESA and ExactEarth. 
The Payload is an AIS Receiver for ship- and vessel-detection from space, and the satellite weight at launch will be approximately 115kg. The satellite payload also enables advanced raw data handling and RF-Spectrum sampling for Ground processing.

\begin{figure}[h]
	\centering
    \includegraphics[width=0.7\textwidth]{images/esail}
    \caption{ESAIL system testing environment.}
    \label{fig:esail_case_study}
\end{figure}
 
The SVF simulator has been used for functional validation of the ESAIL CSW (see Figure~\ref{fig:esail_case_study}). The SVF is indeed one of the main testing tools used in satellite projects. The SVF simulator can be seen as a testing facility that presents also its own dedicated test suite to ensure the correctness of the SVF models and assembly to avoid later misunderstanding of the expected behaviours of the satellite CSW. 
In the context of the FAQAS project we consider the system test suite for the validation of the CSW, which are implemented using the SVF as the driving tool.
%Therefore, the SVF Simulator enables the evaluation of the FAQAS framework against two test suites: (1) the Test Suite of the SVF Simulator that validates the Simulator itself, (2) the system tests for the validation of CSW, which are implemented using the SVF as the driving tool.

Details about ESAIL are provided in the document \emph{FAQAS-LXS-MAN-001\_1- SVF Software Installation and User Manual} uploaded on Alfresco.

ESAIL is the largest case study system in FAQAS, the software consists of 924 source files with a total size of 187\,116 LOC. The system test suite consists of 121 python test scripts with a total of 384 sub-test cases. The system test suite takes up to 10 hours to finish its execution.

\subsection{Code-driven mutation testing}

The code-driven mutation testing process in ESAIL will target all the components of the ESAIL on-board software, these components are:

\begin{itemize}
	\item ADCS
	\item CAN
	\item EPS
	\item FDIR
	\item OPSE
	\item SERVICES
	\item TCS
	\item TMTC
\end{itemize}

\subsection{Data-driven mutation testing}

A detailed description of the application of data-driven mutation testing to ESAIL is provided in APPENDIX~\ref{appendix:esailFM}.



% !TEX root = MAIN.tex

\clearpage

\section{GSL - libgcsp}
\label{sec:caseStudies:GSL:libgcsp}

\subsection{Overview of the case study}

The GomSpace CSP library (libgscsp) is a GomSpace extension to the open source CubeSat Space Protocol library.
The GomSpace CSP library provides:
\begin{itemize}
\item convenience wrapping of CSP functionality, primarily initialization.
\item definition of standard CSP ports (used by other GomSpace products).
\item connecting low-level drivers (e.g. CAN, I2C from Embed library) with CSP interfaces 
\item generic CSP service dispatcher, forwards incoming connections to service handlers.
\end{itemize}

The libgscsp contains a GomSpace branch (https://github.com/GomSpace/libcsp) of the open source libcsp (https://github.com/libcsp/libcsp), located in the subfolder lib/libcsp, and an extension of the CSP library. The two libcsp branches are kept as identical as possible, as features specific to GomSpace are placed in libgscsp. The extension of the CSP library provides utility functions for GSL-specific products.

Details about libgscsp are provided in the document \emph{gs-man-nanosoft-ms100-command-and-management-sdk-3.6.2-1-g67fe6e1.pdf} uploaded on Alfresco.


The size of libgscsp is 1\,497 LOC, %include 306, src 1776+15 = total = 1497
while libcsp (GSL branch) is 8\,339 LOC. % 6789 + 1550

\MREVISION{C-P-45}{According to the definition described in ECSS-Q-ST-80C~\cite{ecss80C}, we consider this test suite an integration test suite. Indeed, every test case verifies the integration of multiple components. If we assume that every component is implemented in a separate file, libgscsp test suite matches this definition. For example, the test case \texttt{TEST\_csp\_port\_bind} test capabilities from components csp\_conn (e.g., csp\_close) and csp\_port (e.g., csp\_port\_get\_socket)} 

\MREVISION{C-P-44}{The libgscsp integration test suite consists of 89 test cases, the test infrastructure is based on the \emph{Google C++ Testing Framework}~\cite{googletest}. The validation environment is compiled and executed through the WAF meta-build system~\cite{waf}.
To perform the experiments in the \INDEX{UL HPC}, we configured an Ubuntu 16.04 Singularity container to be executed with two processors and 8 GB of memory.}


\subsection{Code-driven mutation testing}



In FAQAS, under the assumption that test suites are of high quality standards, we mutate only the statements covered by the test suite. This is also due to the fact that a mutant can be killed only if it is covered by at least one test case. 

In the case of libgscsp, the integration test suite covers basic algorithms such as error handling and routing while hardware specific functions such as the handling of CAN protocol is verified by means of a test suite for hardware in the loop. 

GomSpace relies on a risk-driven approach to integration testing; for this reason GSL code does not reach 100\% code coverage in the integration tests. The code that has not been covered is judged to be low risk, but is covered by other techniques like code inspection and manual testing. The integration tests run are covering the normal use cases for our satellites. libCSP is not specific to GomSpace, although GSL uses it heavily, and not all features are used in GomSpace satellites.

%For this reason, the code-driven mutation testing process in libgscsp will target all the components covered by the libgscsp unit test suite.

% !TEX root = ../MAIN.tex

\begin{table}[h]

\footnotesize
\parbox{.45\linewidth}{
\centering
\begin{tabular}{|l|l|}
\hline
\textbf{Coverage Type} & \textbf{Coverage Rate} \\
\hline
Statement     & 58.4\% (390 of 668 statements)\\
Functions     & 71.4\% (50 of 70 functions)\\
Branches      & 41.2\% (165 of 400 branches)\\
\hline
\end{tabular}
\caption{libgscsp code coverage.}
\label{table:libgscsp_coverage}
}
\hfill
\parbox{.45\linewidth}{
\centering
\begin{tabular}{|l|l|}
\hline
\textbf{Coverage Type} & \textbf{Coverage Rate} \\
\hline
Statement     & 64.1\% (2\,112 of 3\,297 statements)\\
Functions     & 72.5\% (248 of 342 functions)\\
Branches      & 44.9\% (989 of 2\,201 branches)\\
\hline
\end{tabular}
\caption{libcsp code coverage.}
\label{table:libcsp_coverage}
}
\end{table}	

\begin{enumerate}
	\item \textbf{libgscsp extension}

	Table~\ref{table:libgscsp_coverage} provides code coverage information of the libgscsp integration test suite for the GSL extension to the CSP library. Below, we report the files covered by the test suite and targeted by mutation testing; they correspond to 56\% of the source files. %(10/18)
	%we focus our analysis to the following subset of components (i.e., components with code coverage greater than 0\%):
	
	\begin{itemize}
	 	\item src/clock.c
	 	\item src/commands.c
	 	\item src/conn.c
	 	\item src/csp.c
	 	\item src/error.c
	 	\item src/log.c
	 	\item src/router.c
	 	\item src/service\_dispatcher.c
	 	\item src/service\_handler.c
	 	\item src/linux/command\_line.c

	 \end{itemize} 

	\item \textbf{libcsp (GSL branch)}

	Table~\ref{table:libcsp_coverage} presents the code coverage of the libgscsp integration test suite for the GSL branch of the CSP library. 
	% Given the code coverage, we focus our mutation analysis on the following subset of components (i.e., components with code coverage greater than 0\%):
Below, we report the files covered by the test suite and targeted by mutation testing; they correspond to 45\% of the source files. %(32/71)

	\begin{itemize}
		\item src/arch/csp\_time.c
		\item src/arch/csp\_system.c
		\item src/arch/posix/csp\_thread.c
		\item src/arch/posix/csp\_semaphore.c
		\item src/arch/posix/csp\_malloc.c
		\item src/arch/posix/csp\_queue.c
		\item src/arch/posix/csp\_time.c
		\item src/arch/posix/pthread\_queue.c
		\item src/arch/posix/csp\_system.c
		\item src/crypto/csp\_sha1.c
		\item src/crypto/csp\_hmac.c
		\item src/crypto/csp\_xtea.c
		\item src/interfaces/csp\_if\_lo.c
		\item src/rtable/csp\_rtable.c
		\item src/rtable/csp\_rtable\_cidr.c
		\item src/rtable/csp\_rtable\_static.c
		\item src/transport/csp\_rdp.c
		\item src/transport/csp\_udp.c
		\item src/csp\_sfp.c
		\item src/csp\_debug.c
		\item src/csp\_service\_handler.c
		\item src/csp\_crc32.c
		\item src/csp\_io.c
		\item src/csp\_qfifo.c
		\item src/csp\_iflist.c
		\item src/csp\_endian.c
		\item src/csp\_route.c
		\item src/csp\_buffer.c
		\item src/csp\_port.c
		\item src/csp\_conn.c
		\item src/csp\_init.c
		\item src/csp\_services.c
	\end{itemize}


\end{enumerate}

\subsubsection{Mutation Testing Preliminary Results}


\input{tables/libgscsp_preliminary}

In order to evaluate the feasibility of code-driven mutation testing for libgscsp, we conducted a preliminary experiment using the mutation operators AOR, ROR, ICR, LCR, ABS, UOI and SDL we generated 6\,196 mutants. For the experimentation we targeted only the libcsp (GSL branch) source code. Preliminary results can be found in Table~\ref{table:libgscsp_preliminary}.
Particularly, we observe that from the 6\,196 generated mutants, we had 1\,700 mutants that were not compiled by the compilation toolset of libgscsp, most probably because the mutation introduced a syntactical error that was detected by the toolset.
Then, we identified 1\,708 live mutants that were not detected by the test suite. Instead, we had 2\,495 mutants detected by the test suite, and 277 mutants killed by timeout (they led to infinite loop). The final mutation score was of 61.88\%.

The identification of equivalent mutants still needs to be assessed.


\subsection{Data-driven mutation testing}

\begin{figure}[h]
  \centering
    \includegraphics[width=0.9\textwidth]{images/csp_packet}
      \caption{CSP protocol header.}
      \label{fig:csp_packet}
\end{figure}

The data-driven mutation testing process in libgscsp will target the data packet transferred between the server and the client using the CSP protocol. Specifically, the mutations will affect the header of the packet, which contains routing information (see Figure~\ref{fig:csp_packet}) and the payload 
%the mutations will affect the content itself of the packet 
(i.e., the data being transferred).

%\DONE{Does the header contains any infor about the payload? Because the only mutation we can perform on the payload is to cut it.}

%\DONE{No, it seems not. The six fields in the packet header regards: pri (Priority), src (Source address), dst (Destination address)
%dport (Destination port), sport (Source port) and flags (specifies specific byte instructions such as use of fragmentation, HMAC verification, etc)}


\begin{figure}[h]
  \centering
    \includegraphics[width=0.9\textwidth]{images/FaultModelCSP_temp}
      \caption{CSP fault model.}
      \label{fig:csp_faultmodel}
\end{figure}

Figure~\ref{fig:csp_faultmodel} introduces an example of the possible fault models to be applied to the  header of the packet.
% and \emph{Payload} targets the data of the packet.
The \emph{Header} fault model include mutation operators such as bit-flips (BF), insertion of invalid values (INV), insertion of values out of range (VOR), and insertion of values above threshold (VAT).
%Instead, the \emph{Payload} fault model consists of applying a bit-flip to the data packet; it is used to verify that the test suite has at least one test case that controls the correct transmission of the content of the packet.

Concerning the flags data item, it has a size of 8 bits, i.e., 8 flags are available. Only 4 flags are currently used:
\begin{itemize}
\item 00000001              CRC32
\item 00000010             RDP
\item 00000100             XTEA
\item 00001000             HMAC
\end{itemize}
We force each flag to be set to zero by defining a different BF mutation operator with \emph{State=1} for each of the four bits.

% !TEX root =  ../MAIN.tex

\begin{minipage}{16cm}
\begin{lstlisting}[style=CStyle, caption=Data-driven mutation example on libgscsp (csp\_io.c excerpt)., label=csp_integration]
unsigned int v[6] = {packet->id.flags, packet->id.sport, packet->id.dport, 
                    packet->id.dst, packet->id.src, packet->id.pri};

FaultModel *fm = _FAQAS_Identifier_FM();                                                                                          
mutate(v, fm);
_FAQAS_delete_FM(fm);

packet->id.flags = v[0];
packet->id.sport = v[1];
packet->id.dport = v[2]; 
packet->id.dst = v[3]; 
packet->id.src = v[4]; 
packet->id.pri = v[5]; 
\end{lstlisting}
\end{minipage}

Both mutations (i.e, header and payload) will be performed before the message is serialized. 
For example, the mutations to the header will affect the \emph{csp\_send\_direct} function of the \emph{csp\_io component}. 
Listing~\ref{csp_integration} shows an example of our data-driven mutator prototype within the \emph{csp\_io component}. 
Particularly, in Line 1 the packet header is translated into an unsigned int vector. 
Then, in Line 4 an instance of the packet identifier fault model is created, 
the instance of the fault model and the unsigned int vector is then passed to the FAQAS mutate function in line 5.
%In this case, the fault model can target all six items of the header (i.e., Flags, Source Port, Destination Port, Destination Address, Source Address, and Priority). 
After the mutation has been applied, from Line 8 to Line 13, the vector is then translated back to the original representation.
 


\section{GSL - libparam}
\label{sec:caseStudies:GSL:libparam}

\subsection{Overview of the case study}

The Parameter System (i.e., libparam) is a light-weight parameter system designed for GomSpace satellite subsystems. It is based around a logical memory architecture, where every parameter is referenced directly by its logical address. A backend system takes care of translating addresses into physical addresses.
The features of this system include:
\begin{itemize}
\item Direct memory access for quick parameter reads.
\item Simple data types: uint, int, float, double, string.
\item Arrays of simple data types.
\item Supports multiple stores per table, e.g. FRAM, MCU flash, file (binary or text).
\item Remote client with support for most features (rparam).
\item Packed GET, SET queries, supporting multiple parameter set/get in a single request. Data serialization and deserialization.
\item Supports both little and big-endian systems.
\item Commands for both local (param) and remote access (rparam).
\item Parameter server for remote access over CSP.
\item Compile-time configuration of parameter system
\end{itemize}

Details about libparam are provided in the document \emph{gs-man-nanosoft-ms100-command-and-management-sdk-3.6.2-1-g67fe6e1.pdf} uploaded on Alfresco.

Details about code coverage, mutation  score, and fault model for data-driven mutation testing will be provided at the end of WP2.

%\subsection{Code-driven mutation testing}
%
%\TODO{Here we will simply add code coverage information for libparam}
%
%\subsection{Data-driven mutation testing}
%
%\TODO{To populate this section we need the following from GSL: Action 3. (deadline: end of this week is better) GSL should provide a description of the structure of the data exchanged by libparam along with a name of the function that loads the data received from the libcsp layer. GSL suggested looking for documentation inside the libparam folder, but we did not find such doc folder, actually, there is only one doc folder and it concerns the documentation for libcsp not libparam.}


\section{GSL - libutil}
\label{sec:caseStudies:GSL:libutil}

\subsection{Overview of the case study}

The Utility library provides cross-platform APIs for common functionality, for use in both embedded systems and standard PCs running Linux. 

Details about libutil are provided in the document \emph{gs-man-nanosoft-ms100-command-and-management-sdk-3.6.2-1-g67fe6e1.pdf} uploaded on Alfresco.

The size of libutil is 10\,576 LOC, while the unit test suite consists of 201 test cases written in C.

\subsection{Code-driven mutation testing}

The code-driven mutation testing process in libutil will target all the components covered by the libutil unit test suite. 
The libutil unit test suite do not cover hardware-specific functions (e.g., drivers), which are covered at the system level.

% !TEX root = ../MAIN.tex

\begin{table}[h]

\centering
\begin{tabular}{|l|l|}
\hline
\textbf{Coverage Type} & \textbf{Coverage Rate} \\
\hline
Statement     & 83.2\% (8\,817 of 10\,596 statements)\\
Functions     & 82.1\% (725 of 883 functions)\\
Branches      & 56.6\% (2\,618 of 4\,627 branches)\\
\hline
\end{tabular}
\caption{libutil code coverage.}
\label{table:gslibutil_coverage}

\end{table}

Table~\ref{table:gslibutil_coverage} provides the code coverage information of the unit test suite for the GSL libutil library. 
%Given the code coverage, we focus our analysis to the following subset of components (i.e., components with code coverage greater than 0\%):
Below, we report the files covered by the test suite and targeted by mutation testing; they correspond to 82\% of the source files. %(55/67)

\begin{itemize}
	\item src/base16.c
	\item src/bytebuffer.c
	\item src/byteorder.c
	\item src/clock.c
	\item src/crc32.c
	\item src/crc8.c
	\item src/error.c
	\item src/fletcher.c
	\item src/function\_scheduler.c
	\item src/hexdump.c
	\item src/lock.c
	\item src/rtc.c
	\item src/string.c
	\item src/strtoint.c
	\item src/time.c
	\item src/timestamp.c
	\item src/cmd/command.c
	\item src/cmd/log.c
	\item src/cmd/vmem.c
	\item src/drivers/sys/memory.c
	\item src/gosh/command.c
	\item src/gosh/console.c
	\item src/gosh/default\_commands.c
	\item src/linux/clock.c
	\item src/linux/command\_line.c
	\item src/linux/cwd.c
	\item src/linux/delay.c
	\item src/linux/function.c
	\item src/linux/mutex.c
	\item src/linux/queue.c
	\item src/linux/rtc.c
	\item src/linux/sem.c
	\item src/linux/signal.c
	\item src/linux/stdio.c
	\item src/linux/thread.c
	\item src/linux/time.c
	\item src/linux/drivers/gpio/gpio.c
	\item src/linux/drivers/gpio/gpio\_sysfs.c
	\item src/linux/drivers/gpio/gpio\_virtual.c
	\item src/linux/drivers/i2c/i2c.c
	\item src/linux/drivers/spi/spi.c
	\item src/linux/drivers/sys/memory.c
	\item src/log/commands.c
	\item src/log/log.c
	\item src/log/appender/console.c
	\item src/log/appender/simple\_file.c
	\item src/test/cmocka.c
	\item src/test/command.c
	\item src/test/log.c
	\item src/vmem/commands.c
	\item src/vmem/vmem.c
	\item src/watchdog/monitor\_task.c
	\item src/watchdog/watchdog.c
	\item src/zip/zip.c
	\item src/zip/miniz/miniz.c
\end{itemize}

\subsubsection{Mutation Testing Preliminary Results}

% !TEX root = ../MAIN.tex

\begin{table}[h]
\centering
\caption{Code-driven mutation testing preliminary results for the libutil case study.}
\label{table:libutil_preliminary}
\begin{tabular}{|l|l|l|l|l|l|l|}
\hline
        & \multicolumn{5}{c|}{Mutants}                                                                      & \multirow{3}{*}{\begin{tabular}[c]{@{}l@{}}Mutation Score\\ (K/K+L)\end{tabular}} \\ \cline{1-6}
        &     &                                                        &      & \multicolumn{2}{c|}{Killed} &                                                                                   \\ \cline{1-6}
Mutants & All & \begin{tabular}[c]{@{}l@{}}Not\\ Compiled\end{tabular} & Live & Test Failure    & Timeout   &                                                                                   \\ \hline
Total   &  16\,886   &  2\,561                                                      & 3\,402      & 10\,634                & 289          & 76.25\%                                                                           \\ \hline
\end{tabular}
\end{table}    
             

In order to analyze the feasibility of the code-driven mutation testing, we conducted a preliminary experimentation using the mutation operators AOR, ROR, ICR, LCR, and SDL we generated 16\,886 mutants. Preliminary results can be found in Table~\ref{table:libutil_preliminary}.
Particularly, we observe that from the 16\,886 generated mutants, we had 2\,561 mutants that were not compiled by the compilation toolset of libutil, most probably because the mutation introduced a syntactical error that was detected by the toolset.
Then, we identified 3\,402 live mutants that were not detected by the test suite. Instead, we had 10\,634 killed mutants detected by the test suite, and 289 mutants that produced libutil to go into an infinite loop, and thus were killed by timeout. The final mutation score was of 76.25\%.

The identification of equivalent mutants still needs to be performed.


\subsection{Data-driven mutation testing}

We do not plan to apply data drive mutation testing to this case study because is a standalone library; it does not integrate communicating components.



% !TEX root = MAIN.tex
\clearpage

\section{MLFS}
\label{sec:caseStudies:GSL:MLSF}

\subsection{Overview of the case study}

The \INDEX{Mathematical Library for Flight Software} (MLFS) implements mathematical functions ready for qualification. 
%\DONE{Please rewrite the following sentence}
MLFS is born from the need of having a mathematical library ready for qualification for flight software. Well known mathematical libraries such as \texttt{libm}~\cite{libm} and \texttt{newlib}~\cite{newlib} are not completely validated with respect to specific input ranges, errors and performance, and so, they do not comply with ECSS criticality category B.
The set of functions provided by MLFS are limited to the functions typically needed in flight software. 

%\DONE{What is doc?}
Detailed information about the MLFS library and the test suite is provided in the following documents shared on Alfresco.

\begin{itemize}
	\item \emph{E1356-GTD-SUM-01\_I1\_R2.pdf}, installation and execution instructions of the Mathematical Library for Flight Software. 
	\item \emph{E1356-CS-SUM-01\_I1\_R5.pdf}, installation and execution instructions of the MLFS test suite.
\end{itemize}

The source code size is 5\,402 LOC, while the unit test suite consists of 4\,042 tests for 92 functions.

\subsection{Code-driven mutation analysis}

% !TEX root = ../MAIN.tex

\begin{table}[h]

\centering
\begin{tabular}{|l|l|}
\hline
\textbf{Coverage Type} & \textbf{Coverage Rate} \\
\hline
Statement     & 100\% (1\,978 of 1\,978 statements)\\
Functions     & 100\% (90 of 90 functions)\\
Branches      & 100\% (1\,302 of 1\,302 branches)\\
\hline
\end{tabular}
\caption{MLFS code coverage.}
\label{table:mlfs_coverage}

\end{table}

Table~\ref{table:mlfs_coverage} provides the code coverage information of the MLFS unit test suite. The test suite achieves 100\% code coverage (i.e., statement, function and branch coverage).

Given the code coverage, we focus our analysis on all components of the MLFS:

\begin{itemize}
	\item MLFS: implementation of 4 mathematical functions.
	\item Machine: dedicated implementation of 2 mathematical functions for the sparc v8 architecture.
	\item Math: implementation of 66 mathematical functions.
	\item Common: implementation of 20 mathematical functions.
\end{itemize}

%\subsubsection{Mutation Analysis Preliminary Results}
%
%% !TEX root = ../MAIN.tex

\begin{table}[h]
\small
\centering
\caption{Code-driven mutation testing preliminary results for the MLFS case study.}
\label{table:mlfs_preliminary}
\begin{tabular}{|l|l|l|l|l|l|l|}
\hline
        & \multicolumn{5}{c|}{Mutants}                                                                      & \multirow{3}{*}{\begin{tabular}[c]{@{}l@{}}Mutation Score\\ (K/K+L)\end{tabular}} \\ \cline{1-6}
        &     &                                                        &      & \multicolumn{2}{c|}{Killed} &                                                                                   \\ \cline{1-6}
Mutants & All & \begin{tabular}[c]{@{}l@{}}Not\\ Compiled\end{tabular} & Live & Test Failure    & Timeout   &                                                                                   \\ \hline
Total   &  17\,696   &  2\,153   & 22      & 15\,332                & 189          & 99.86\%                                                                           \\ \hline
\end{tabular}
\end{table}    
             
%
%In order to analyze the feasibility of the code-driven mutation analysis for MLFS, we conducted a preliminary experimentation using the mutation operators AOR, ROR, ICR, LCR, and SDL we generated 17\,696 mutants. Preliminary results can be found in Table~\ref{table:mlfs_preliminary}.
%Particularly, we observe that from the 17\,696 generated mutants, we had 2\,153 mutants that were not compiled by the compilation toolset of MLFS, most probably because the mutation introduced a syntactical error that was detected by the toolset.
%Then, we only identified 22 live mutants that were not detected by the test suite. Instead, we had 15\,332 killed mutants detected by the test suite, and 189 mutants that produced libutil to go into an infinite loop, and thus were killed by timeout. The final mutation score was of 99.86\%.
%
%The identification of equivalent mutants still needs to be performed.


\subsection{Data-driven mutation analysis}

We do not plan to apply data drive mutation analysis to this case study because is a standalone library; it does not integrate communicating components.





% !TEX root = MAIN.tex

\clearpage

\section{ASN1SCC}
\label{sec:caseStudies:GSL:ASN1}

\subsection{Overview of the case study}

\INDEX{ASN1SCC} is an open source ASN.1 compiler that generates C/C++ and SPARK/Ada code suitable for low resource environments such as space systems. Moreover, the compiler can produce a test harness that provides full statement coverage in the generated code, and therefore significantly improves its quality.


In the context of FAQAS, we focus our analysis on the auto-generated source code by the ASN.1 compiler, rather than in ASN1SCC software itself.

For the definition of the ASN.1 compiler case study, we introduce the example of a specific grammar. An excerpt of such grammar is shown in Listing~\ref{asn_excerpt}. 
The excerpt of the grammar introduces the definition of six data types, each data type also specifies the expected constraint, for example, the data type \texttt{MyInt} is an INTEGER which can have values from 0 to 20. The full source code of the grammar can be found in the file \emph{test.asn} uploaded on Alfresco.

For the given grammar, the size of the auto-generated source code is 4\,338 LOC. While the unit test suite consists of 107 auto-generated test cases.

% !TEX root =  ../MAIN.tex

\begin{minipage}{15cm}
\begin{lstlisting}[style=CStyle, caption=Excerpt of the tested ASN1 grammar., label=asn_excerpt, mathescape=true]
MyInt ::= INTEGER (0 .. 20)

My2ndInt ::= MyInt ( 1 .. 18)

AType ::= SEQUENCE {
    blArray SEQUENCE (SIZE(10)) OF BOOLEAN
}

My2ndAType ::= AType

TypeNested ::= SEQUENCE {
    intVal  INTEGER(0..10),
    int2Val INTEGER(-10..10),
    int3Val MyInt (10..12),
    intArray    SEQUENCE (SIZE (10)) OF INTEGER (0..3),
    realArray   SEQUENCE (SIZE (10)) OF REAL (0.1 .. 3.14),
    octStrArray SEQUENCE (SIZE (10)) OF OCTET STRING (SIZE(1..10)),
    boolArray   SEQUENCE (SIZE (10)) OF T-BOOL,
    label   OCTET STRING (SIZE(10..40)),
    bAlpha  T-BOOL,
    bBeta   BOOLEAN,
    sString T-STRING,
    arr     T-ARR,
    arr2    T-ARR2
}

E ::= INTEGER (0..255|1299)(5)
\end{lstlisting}
\end{minipage}







