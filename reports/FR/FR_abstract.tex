% !TEX root = MAIN.tex

\chapter*{Abstract}
\label{sec:abstract}
\addcontentsline{toc}{chapter}{Abstract}

This document is the final report of the ESA activity \EMPH{ITT-1-9873-ESA (Applicability of Mutation Testing Method for Flight Software)}, which concerns the development of a framework for the automated assessment and the automated improvement of test suites for space software. In this report, we use the term space software to indicate software to be deployed on hardware that runs on-orbit. 

The activity lead to the development of a toolset, the \EMPH{FAQAS toolset}, which includes three tools \EMPH{MASS}, \EMPH{SEMuS}, and \EMPH{DAMAt}. MASS (Mutation Analysis for Space Software) is a tool for the assessment of test suites based on mutation analysis. Mutation analysis evaluate the quality of the test suite by generating faulty software versions called mutants and by reporting the percentage of mutants detected by the test suite. MASS scales to large software systems because it relies on mutants sampling based on confidence interval estimation.  SEMuS (Symbolic Execution Mutation testing for Space software) is a tool for the automated improvement of test suites; it automatically generates unit test cases that detect the presence of mutants. DAMAt (DAta-driven Mutation Analysis with Tables) is a tool that assesses the quality of test suites by simulating errors in the data exchanged by software components, different from MASS it emulates not algorithmic but higher level faults.

The \EMPH{scalability} and \EMPH{effectiveness} of the FAQAS toolset has been demonstrated through the application of the toolset to case study systems provided by GomSpace, LuxSpace, and ESA. Both MASS and SEMuS enabled the identification of faults affecting the software under analysis. Both MASS and DAMAt enabled the identification of major pitfalls in the test suite. Mutation analysis has been demonstrated to be feasible (i.e., executable in few days even for large systems); however, adequate computational resources (e.g., multiple computation nodes) are necessary. The automated generation of unit test cases, instead can produce useful results in few minutes. Our results show that the FAQAS toolset enables ensuring high-quality in space software.
