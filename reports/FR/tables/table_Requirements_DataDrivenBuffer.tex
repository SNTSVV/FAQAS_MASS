% !TEX root = ../MAIN.tex
\begin{table}[h]
\begin{center}
\small
\begin{tabular}{|p{2cm}|p{12cm}|}
\hline
\textbf{Requirement ID}&\textbf{Description}\\
\hline
R.DDB.1&
The data to mutate is stored in a buffer (i.e., an array of type T).
\\
\hline
R.DDB.2&
The engineer may specify multiple fault models. One for each data buffer to mutate.
\\
\hline
R.DDB.3&
In a data-driven mutation testing session, the mutations to be performed may be driven by multiple fault models.
\\
\hline
R.DDB.4&
\begin{minipage}{12cm}
The type T of the buffer to mutate should be the same for a same data driven mutation testing session. Buffers of different types T should be tested in different sessions.
\end{minipage}
\\
\hline
R.DDB.5&
The data to mutate may span over multiple items of the buffer.
\\


%Incorrect Identifier& Several transmission data fields have fixed values, for example fields identifying the transmitting satellite. Hardware/software errors may assign incorrect identifiers.\\
%%Incorrect Checksum& Hardware/software errors may result in an incorrect checksum for a Packet or VCDU.\\
%Incorrect Counter& Counters are used to track Packet or VCDU ordering. Hardware/software errors may assign incorrect counter values.\\
%Flipped Data Bits& Physical channel noise may flip one or more bits in the data transmission.\\
\hline
\end{tabular}
\end{center}
\caption{Requirements to mutate data stored in buffers}
\label{table:faultModel:FAQAS}
\end{table}%