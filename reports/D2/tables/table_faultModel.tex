% !TEX root = ../MAIN.tex
\begin{table}[h]
\begin{center}
\scriptsize
\begin{tabular}{|p{2cm}|p{2cm}|p{4cm}|p{6cm}|}
\hline
\textbf{Fault Class}&\textbf{Types}&\textbf{Parameters}&\textbf{Description}\\
\hline
Value above threshold (VAT)&
\begin{minipage}{6cm}
INT\\
LONG INT\\
FLOAT\\
DOUBLE
\end{minipage}
&
\begin{minipage}{6cm}
T: threshold\\
D: delta with respect to threshold\\
\end{minipage}
&
\begin{minipage}{6cm}
The value is above a threshold T for a delta D. 

\EMPH{Data mutation operation:} The mutation is performed by replacing the current value (a number) with a value of the same type that is equal to $(T+D)$.
\end{minipage}
\\

\hline
Value below threshold (VBT)&
\begin{minipage}{6cm}
INT\\
LONG INT\\
FLOAT\\
DOUBLE
\end{minipage}
&
\begin{minipage}{6cm}
T: threshold\\
D: delta with respect to threshold\\
\end{minipage}
&
\begin{minipage}{6cm}
The value is below a threshold T for a delta D. 

\EMPH{Data mutation operation:} The mutation is performed by replacing the current value (a number) with a value of the same type that is equal to $(T-D)$.
\end{minipage}
\\



\hline
Value out of range (VOR)&
\begin{minipage}{4cm}
INT\\
LONG INT\\
FLOAT\\
DOUBLE
\end{minipage}
&
\begin{minipage}{4cm}
MIN: minimum valid value\\
MAX: maximum valid value\\
D: delta with respect to minimum/maximum valid value
\end{minipage}
&
\begin{minipage}{6cm}
The value is out of the valid range MIN-MAX. 

\EMPH{Data mutation operations (2):}  The mutation is performed by replacing the current value (a number) with 
\begin{itemize}
\item a value of the same type that is equal to $(MIN-D)$
\item a value of the same type that is equal to $(MAX+D)$
\end{itemize}
\end{minipage}
\\

\hline
Bit flip (BF)&
BIN
&
\begin{minipage}{4cm}
MIN: lower bit\\
MAX: higher bit\\
STATE: mutate only if the bit is in the given state\\
\TRFOUR{VALUE: integer specifying the number of bits to mutate}\\
\end{minipage}
&
\begin{minipage}{6cm}
A number of bits randomly chosen in the positions between MIN and MAX (included) are flipped.

\EMPH{Data mutation operation:} the operator flips N randomly selected bit.
If STATE is specified, the mutation is applied only if  the bit is in the specified state. Parameter VALUE specifies the number of bits to mutate.
\end{minipage}
\\

\hline
Invalid numeric value (INV)&
\begin{minipage}{6cm}
INT\\
LONG INT\\
FLOAT\\
DOUBLE
\end{minipage}
&
\begin{minipage}{4cm}
MIN: lower valid value\\
MAX: higher valid value\\
\TRFOUR{D: distribution to follow}\\
\TRFOUR{VALUE: mean value for normal distribution}\\
\end{minipage}
&
\begin{minipage}{6cm}
The value is legal (i.e., in the specified range) but different than the current one, which, in this case, is expected to be consistent with the status of the system.

\EMPH{Data mutation operation:} Mutation is performed by replacing the current value with a different value randomly sampled in the specified range. The parameter D specified the distribution to follow when performing the mutation\footnote{In our implementation 0 indicates uniform, 1 indicates normal around the specified value (but in range).}
\end{minipage}
\\

\hline
Illegal Value (IV)
&
\begin{minipage}{6cm}
INT\\
LONG INT\\
FLOAT\\
DOUBLE\\
HEX
\end{minipage}
&
\begin{minipage}{6cm}
VALUE: illegal value that is observed\\
\end{minipage}
&
\begin{minipage}{6cm}
The value is illegal and equal to the provided one (i.e., parameter \emph{VALUE}).

\EMPH{Data mutation operation:} Mutation is performed by replacing the current value with the value \emph{VALUE}, if different than the current one.
\end{minipage}
\\

\hline
\TRFOUR{Anomalous Signal Amplitude (ASA)}
&
\begin{minipage}{6cm}
INT\\
LONG INT\\
FLOAT\\
DOUBLE
\end{minipage}
&
\begin{minipage}{6cm}
T: change point\\
D: delta to add/remove\\
V: value to multiply\\
\end{minipage}
&
\begin{minipage}{6cm}
The value is modified by amplifying/reducing it by a factor V and adding or removing D from the observed value. It is used to "amplify" a signal in a constant manner to simulate unusual signal. T indicates the observed value below which instead of adding  we subtract .

\EMPH{Data mutation operation:} Mutation is performed by replacing the current value ($v$) with the value ($v'$) computed as follows:

\[
v' =  
    \begin{cases}
      T+(  (v-T)*V  ) + D   & \mathit{if}\ v \ge T\\
      T - (  (T-v)*V  ) - D   & \mathit{if}\ v < T
    \end{cases}       
\]
\end{minipage}
\\


\hline
\TRFOUR{Signal Shift (SS)}
&
\begin{minipage}{6cm}
INT\\
LONG INT\\
FLOAT\\
DOUBLE
\end{minipage}
&
\begin{minipage}{6cm}
D: delta by which the signal should be shifted\\
\end{minipage}
&
\begin{minipage}{6cm}
The value is modified by adding a value D. It simulate an anomalous shift in the signal.
\end{minipage}
\\





\hline
\TRFOUR{Hold Value (HV)}
&
\begin{minipage}{6cm}
BIN\\
INT\\
LONG INT\\
FLOAT\\
DOUBLE
\end{minipage}
&
\begin{minipage}{6cm}
V: number of times to repeat the same value\\
\end{minipage}
&
\begin{minipage}{6cm}
This operator keeps repeating an observed value for $V$ times. It emulates a constant signal replacing a signal supposed to vary.
\end{minipage}
\\



\hline
\TRFOUR{Array Swap (AS)}
&
\begin{minipage}{6cm}
ARRAY\_*\\
\end{minipage}
&
\begin{minipage}{6cm}
MIN: position of element A\\
MAX: position of element B\\
VALUE: number of elements to move\\
\end{minipage}
&
\begin{minipage}{6cm}
Replace a number of elements (number specified by VALUE) located starting from position MIN, with a number of elements located starting from position MAX, and viceversa.
\EMPH{Data mutation operation:} Mutation is performed by replacing the two set of elements with each other.
\end{minipage}
\\


\hline
\TRFOUR{Array Random Swap (ARS)}
&
\begin{minipage}{6cm}
ARRAY\_*\\
\end{minipage}
&
\begin{minipage}{6cm}
MIN: min position of element A/B\\
MAX: max position of element A/B\\
VALUE: number of elements to move\\
\end{minipage}
&
\begin{minipage}{6cm}
Replace a number of elements (number specified by VALUE) located in a position between MIN and MAX, with a number of elements located in a position between MIN and MAX. MIN and MAX specify a position with respect to the beginning and end of the array.  For example, MIN=0 indicates the first element of teh array, MIN=-2 indicates the second element of the array.
\EMPH{Data mutation operation:} Mutation is performed by replacing the two set of elements with each other.
\end{minipage}
\\



%Incorrect Identifier& Several transmission data fields have fixed values, for example fields identifying the transmitting satellite. Hardware/software errors may assign incorrect identifiers.\\
%%Incorrect Checksum& Hardware/software errors may result in an incorrect checksum for a Packet or VCDU.\\
%Incorrect Counter& Counters are used to track Packet or VCDU ordering. Hardware/software errors may assign incorrect counter values.\\
%Flipped Data Bits& Physical channel noise may flip one or more bits in the data transmission.\\
\hline
\end{tabular}
\end{center}
\caption{Data Fault Classes}
\label{table:faultModel:FAQAS}
\end{table}%