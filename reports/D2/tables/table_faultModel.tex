% !TEX root = ../MAIN.tex
\begin{table}[h]
\begin{center}
\scriptsize
\begin{tabular}{|p{2cm}|p{2cm}|p{4cm}|p{4cm}|}
\hline
\textbf{Fault Class}&\textbf{Types}&\textbf{Parameters}&\textbf{Description}\\
\hline
Value above threshold (VAT)&
\begin{minipage}{4cm}
INT\\
FLOAT\\
DOUBLE
\end{minipage}
&
\begin{minipage}{4cm}
T: threshold\\
D: delta with respect to threshold\\
\end{minipage}
&
\begin{minipage}{4cm}
The value is above a threshold T for a delta D. 

\EMPH{Data mutation operation:} The mutation is performed by replacing the current value (a number) with a value of the same type that is equal to $(T+D)$.
\end{minipage}
\\

\hline
Value below threshold (VBT)&
\begin{minipage}{4cm}
INT\\
FLOAT\\
DOUBLE
\end{minipage}
&
\begin{minipage}{4cm}
T: threshold\\
D: delta with respect to threshold\\
\end{minipage}
&
\begin{minipage}{4cm}
The value is below a threshold T for a delta D. 

\EMPH{Data mutation operation:} The mutation is performed by replacing the current value (a number) with a value of the same type that is equal to $(T-D)$.
\end{minipage}
\\



\hline
Value out of range (VOR)&
\begin{minipage}{4cm}
INT\\
FLOAT\\
DOUBLE
\end{minipage}
&
\begin{minipage}{4cm}
MIN: minimum valid value\\
MAX: maximum valid value\\
D: delta with respect to minimum/maximum valid value
\end{minipage}
&
\begin{minipage}{4cm}
The value is out of the valid range MIN-MAX. 

\EMPH{Data mutation operations (2):}  The mutation is performed by replacing the current value (a number) with 
\begin{itemize}
\item a value of the same type that is equal to $(MIN-D)$
\item a value of the same type that is equal to $(MAX+D)$
\end{itemize}
\end{minipage}
\\

\hline
Bit flip (BF)&
BIN
&
\begin{minipage}{4cm}
MIN: lower bit\\
MAX: higher bit\\
STATE: mutate only if the bit is in the given state\\
\end{minipage}
&
\begin{minipage}{4cm}
One of the bits in the position between MIN and MAX (included) is flipped.

\EMPH{Data mutation operation:} the operator flips one randomly selected bit.
If STATE is specified, the mutation is applied only if  the bit is in the specified state.
\end{minipage}
\\

\hline
Invalid numeric value (INV)&
\begin{minipage}{4cm}
INT\\
FLOAT\\
DOUBLE
\end{minipage}
&
\begin{minipage}{4cm}
MIN: lower valid value\\
MAX: higher valid value\\
\end{minipage}
&
\begin{minipage}{4cm}
The value is legal (i.e., in the specified range) but different than the current one, which, in this case, is expected to be consistent with the status of the system.

\EMPH{Data mutation operation:} Mutation is performed by replacing the current value with a different value randomly sampled in the specified range.
\end{minipage}
\\

\hline
Illegal Value (IV)
&
\begin{minipage}{4cm}
INT\\
FLOAT\\
DOUBLE\\
HEX
\end{minipage}
&
\begin{minipage}{4cm}
VALUE: invalid value that is observed\\
\end{minipage}
&
\begin{minipage}{4cm}
The value is illegal and equal to the provided one (i.e., parameter \emph{VALUE}).

\EMPH{Data mutation operation:} Mutation is performed by replacing the current value with the value \emph{VALUE}, if different than the current one.
\end{minipage}
\\


%Incorrect Identifier& Several transmission data fields have fixed values, for example fields identifying the transmitting satellite. Hardware/software errors may assign incorrect identifiers.\\
%%Incorrect Checksum& Hardware/software errors may result in an incorrect checksum for a Packet or VCDU.\\
%Incorrect Counter& Counters are used to track Packet or VCDU ordering. Hardware/software errors may assign incorrect counter values.\\
%Flipped Data Bits& Physical channel noise may flip one or more bits in the data transmission.\\
\hline
\end{tabular}
\end{center}
\caption{Data Fault Classes}
\label{table:faultModel:FAQAS}
\end{table}%