% !TEX root = ../MAIN.tex
\begin{table}[h]
\begin{center}
\small
\begin{tabular}{|p{1cm}|p{2cm}|p{1cm}|p{1cm}|p{1cm}|p{1cm}|p{1cm}|p{2cm}|p{1cm}|p{1cm}|}
\hline
\textbf{Fault Model Name}&\textbf{DataItem}&\textbf{Span}&\textbf{Type}&\textbf{Fault Class}&\textbf{Min}&\textbf{Max}&\textbf{Threshold}&\textbf{Delta}&\textbf{State}\\
\hline
IfHK&0&1&BIN&BF&0&0&-&-&-\\
IfHK&1&1&INT&VOR&0&5&-&1&-\\
IfHK&2&2&BIN&BF&0&63&-&-&-\\
IfHK&4&1&BIN&BF&0&0&-&-&-\\
\hline
IfStatus&0&1&BIN&BF&0&0&-&-&-\\
\hline
\end{tabular}
\end{center}
\caption{Driven Fault Model Buffer}
\label{table:faultModel:example}
\end{table}%