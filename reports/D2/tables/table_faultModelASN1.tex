% !TEX root = ../MAIN.tex
\begin{table}[h]
\begin{center}
\small
\begin{tabular}{|p{2cm}|p{2cm}|p{4cm}|p{4cm}|}
\hline
\textbf{Types}&\textbf{Fault Classes}&\textbf{Parameters}&\textbf{Description}\\
\hline
INTEGER&
VAT&
\begin{minipage}{4cm}
T: maximum valid value\\
D: 1\\
\end{minipage}
&
\begin{minipage}{4cm}
The value is above a threshold T for a delta D. 

\EMPH{Data mutation operation:} The mutation is performed by replacing the current value (a number) with a value of the same type that is equal to $(T+D)$, with T taking its maximum valid value.
\end{minipage}
\\
\hline
INTEGER&
VBT&
\begin{minipage}{4cm}
T: minimum valid value\\
D: 1\\
\end{minipage}
&
\begin{minipage}{4cm}
The value is below a threshold T for a delta D. 

\EMPH{Data mutation operation:} The mutation is performed by replacing the current value (a number) with a value of the same type that is equal to $(T-D)$, with T taking its minimum valid value.
\end{minipage}
\\
\hline
INTEGER&
VOR&
\begin{minipage}{4cm}
MIN: minimum valid value\\
MAX: maximum valid value\\
D: 1\\
\end{minipage}
&
\begin{minipage}{4cm}
The value is out of the valid range MIN-MAX. 

\EMPH{Data mutation operations (2):}  The mutation is performed by replacing the current value (a number) with 
\begin{itemize}
\item a value of the same type that is equal to $(MIN-D)$
\item a value of the same type that is equal to $(MAX+D)$
\end{itemize}
\end{minipage}
\\
\hline
REAL&
VAT&
\begin{minipage}{4cm}
T: MAX\\
D: 1\\
\end{minipage}
&
\begin{minipage}{4cm}
\end{minipage}
\\
\hline
REAL&
VBT&
\begin{minipage}{4cm}
T: MIN\\
D: 1\\
\end{minipage}
&
\begin{minipage}{4cm}
\end{minipage}
\\
\hline
REAL&
VOR&
\begin{minipage}{4cm}
MIN: MIN\\
MAX: MAX\\
D: 1\\
\end{minipage}
&
\begin{minipage}{4cm}
\end{minipage}
\\
\hline
ENUMERATED&
INV&
\begin{minipage}{4cm}
MIN: MIN\\
MAX: MAX\\
D: 1\\
\end{minipage}
&
\begin{minipage}{4cm}
\end{minipage}
\\
\hline
BOOLEAN&
BF&
\begin{minipage}{4cm}
MIN: 0\\
MAX: 0\\
\end{minipage}
&
\begin{minipage}{4cm}
\end{minipage}
\\
\hline
NULL&
BF&
\begin{minipage}{4cm}
MIN: 0\\
MAX: 0\\
\end{minipage}
&
\begin{minipage}{4cm}
\end{minipage}
\\
\hline
BIT STRING&
BF&
\begin{minipage}{4cm}
MIN: 0\\
MAX: 0\\
\end{minipage}
&
\begin{minipage}{4cm}
\end{minipage}
\\
\hline
OCTET STRING&
BF&
\begin{minipage}{4cm}
MIN: 0\\
MAX: 0\\
\end{minipage}
&
\begin{minipage}{4cm}
\end{minipage}
\\
\hline
IA5STRING&
BF&
\begin{minipage}{4cm}
MIN: 0\\
MAX: 0\\
\end{minipage}
&
\begin{minipage}{4cm}
\end{minipage}
\\
\hline
NUMERIC STRING&
BF&
\begin{minipage}{4cm}
MIN: 0\\
MAX: 0\\
\end{minipage}
&
\begin{minipage}{4cm}
\end{minipage}
\\
\hline
SEQUENCE&
-&
\begin{minipage}{4cm}
\end{minipage}
&
\begin{minipage}{4cm}
No mutation envisioned for this type.
\end{minipage}
\\
\hline
SET&
-&
\begin{minipage}{4cm}
\end{minipage}
&
\begin{minipage}{4cm}
No mutation envisioned for this type.
\end{minipage}
\\
\hline
CHOICE&
-&
\begin{minipage}{4cm}
\end{minipage}
&
\begin{minipage}{4cm}
No mutation envisioned for this type.
\end{minipage}
\\
\hline
SEQUENCE OF&
-&
\begin{minipage}{4cm}
\end{minipage}
&
\begin{minipage}{4cm}
No mutation envisioned for this type.
\end{minipage}
\\
\hline
SET OF&
-&
\begin{minipage}{4cm}
\end{minipage}
&
\begin{minipage}{4cm}
No mutation envisioned for this type.
\end{minipage}
\\



%Incorrect Identifier& Several transmission data fields have fixed values, for example fields identifying the transmitting satellite. Hardware/software errors may assign incorrect identifiers.\\
%%Incorrect Checksum& Hardware/software errors may result in an incorrect checksum for a Packet or VCDU.\\
%Incorrect Counter& Counters are used to track Packet or VCDU ordering. Hardware/software errors may assign incorrect counter values.\\
%Flipped Data Bits& Physical channel noise may flip one or more bits in the data transmission.\\
\hline
\end{tabular}
\end{center}
\caption{Data Fault Classes for ASN.1 data types.}
\label{table:faultModel:FAQAS:ASN1}
\end{table}%