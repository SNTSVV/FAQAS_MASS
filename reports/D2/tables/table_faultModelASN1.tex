% !TEX root = ../MAIN.tex

\setlength\LTleft{0pt}
\setlength\LTright{0pt}
\scriptsize
\begin{longtable}{@{\extracolsep{\fill}}|p{2cm}|p{2cm}|p{3.5cm}|p{4.5cm}|@{}}
\caption{Data Fault Classes for ASN.1 data types.}
\label{table:faultModel:FAQAS:ASN1}\\

\hline
\textbf{Types}&\textbf{Fault Classes}&\textbf{Parameters}&\textbf{Description}\\
\hline
INTEGER&
VAT&
\begin{minipage}{3.5cm}
T: maximum valid value\\
D: 1\\
\end{minipage}
&
\begin{minipage}{4.5cm}
The value is above a threshold T for a delta D. 

\EMPH{Data mutation operation:} The mutation is performed by replacing the current value (a number) with a value of the same type that is equal to $(T+D)$, with T taking its maximum valid value.
\end{minipage}
\\
\hline
INTEGER&
VBT&
\begin{minipage}{3.5cm}
T: minimum valid value\\
D: 1\\
\end{minipage}
&
\begin{minipage}{4.5cm}
The value is below a threshold T for a delta D. 

\EMPH{Data mutation operation:} The mutation is performed by replacing the current value (a number) with a value of the same type that is equal to $(T-D)$, with T taking its minimum valid value.
\end{minipage}
\\
\hline
INTEGER&
VOR&
\begin{minipage}{3.5cm}
MIN: minimum valid value\\
MAX: maximum valid value\\
D: 1\\
\end{minipage}
&
\begin{minipage}{4.5cm}
The value is out of the valid range MIN-MAX. 

\EMPH{Data mutation operations (2):}  The mutation is performed by replacing the current value (a number) with 
\begin{itemize}
\item a value of the same type that is equal to $(MIN-D)$
\item a value of the same type that is equal to $(MAX+D)$
\end{itemize}
\end{minipage}
\\
\hline
REAL&
VAT&
\begin{minipage}{3.5cm}
T: maximum valid value\\
D: 1\\
\end{minipage}
&
\begin{minipage}{4.5cm}
The value is above a threshold T for a delta D. 

\EMPH{Data mutation operation:} The mutation is performed by replacing the current value (a number) with a value of the same type that is equal to $(T+D)$, with T taking its maximum valid value.
\end{minipage}
\\
\hline
REAL&
VBT&
\begin{minipage}{3.5cm}
T: minimum valid value\\
D: 1\\
\end{minipage}
&
\begin{minipage}{4.5cm}
The value is below a threshold T for a delta D. 

\EMPH{Data mutation operation:} The mutation is performed by replacing the current value (a number) with a value of the same type that is equal to $(T-D)$, with T taking its minimum valid value.
\end{minipage}
\\
\hline
REAL&
VOR&
\begin{minipage}{3.5cm}
MIN: minimum valid value\\
MAX: maximum valid value\\
D: 1\\
\end{minipage}
&
\begin{minipage}{3.5cm}
The value is out of the valid range MIN-MAX. 

\EMPH{Data mutation operations (2):}  The mutation is performed by replacing the current value (a number) with 
\begin{itemize}
\item a value of the same type that is equal to $(MIN-D)$
\item a value of the same type that is equal to $(MAX+D)$
\end{itemize}
\end{minipage}
\\
\hline
ENUMERATED&
VAT&
\begin{minipage}{3.5cm}
T: enumerated item value\\
D: 1\\
\end{minipage}
&
\begin{minipage}{4.5cm}
The value is above a threshold T for a delta D. 

\EMPH{Data mutation operation:} The mutation is performed by replacing the current value (a number) with a value of the same type that is equal to $(T+D)$. 
\end{minipage}
\\
\hline
ENUMERATED&
VBT&
\begin{minipage}{3.5cm}
T: enumerated item value\\
D: 1\\
\end{minipage}
&
\begin{minipage}{4.5cm}
The value is below a threshold T for a delta D. 

\EMPH{Data mutation operation:} The mutation is performed by replacing the current value (a number) with a value of the same type that is equal to $(T-D)$. 
\end{minipage}
\\
\hline
BOOLEAN&
BF&
\begin{minipage}{3.5cm}
MIN: lower bit\\
MAX: higher bit\\
STATE: mutate only if the bit is in the given state\\
\end{minipage}
&
\begin{minipage}{4.5cm}
One of the bits in the position between MIN and MAX (included) is flipped.


\EMPH{Data mutation operation:} the operator flips one randomly selected bit of the BOOLEAN variable.
If STATE is specified, the mutation is applied only if the bit is in the specified state.
\end{minipage}
\\
\hline
NULL&
BF&
\begin{minipage}{3.5cm}
MIN: lower bit\\
MAX: higher bit\\
STATE: mutate only if the bit is in the given state\\
\end{minipage}
&
\begin{minipage}{4.5cm}
One of the bits in the position between MIN and MAX (included) is flipped.


\EMPH{Data mutation operation:} the operator flips one randomly selected bit of the NULL variable.
If STATE is specified, the mutation is applied only if the bit is in the specified state.
\end{minipage}
\\
\hline
BIT STRING&
BF&
\begin{minipage}{3.5cm}
MIN: lower bit\\
MAX: higher bit\\
STATE: mutate only if the bit is in the given state\\
\end{minipage}
&
\begin{minipage}{4.5cm}
One of the bits in the position between MIN and MAX (included) is flipped.


\EMPH{Data mutation operation:} the operator flips one randomly selected bit of the BIT STRING variable.
If STATE is specified, the mutation is applied only if the bit is in the specified state.
\end{minipage}
\\
\hline
OCTET STRING&
BF&
\begin{minipage}{3.5cm}
MIN: lower bit\\
MAX: higher bit\\
STATE: mutate only if the bit is in the given state\\
\end{minipage}
&
\begin{minipage}{4.5cm}
One of the bits in the position between MIN and MAX (included) is flipped.


\EMPH{Data mutation operation:} the operator flips one randomly selected bit of the OCTET STRING variable.
If STATE is specified, the mutation is applied only if the bit is in the specified state.
\end{minipage}
\\
\hline
IA5STRING&
BF&
\begin{minipage}{3.5cm}
MIN: lower bit\\
MAX: higher bit\\
STATE: mutate only if the bit is in the given state\\
\end{minipage}
&
\begin{minipage}{4.5cm}
One of the bits in the position between MIN and MAX (included) is flipped.


\EMPH{Data mutation operation:} the operator flips one randomly selected bit of the IA5STRING variable.
If STATE is specified, the mutation is applied only if the bit is in the specified state.
\end{minipage}
\\
\hline
NUMERIC STRING&
BF&
\begin{minipage}{3.5cm}
MIN: lower bit\\
MAX: higher bit\\
STATE: mutate only if the bit is in the given state\\
\end{minipage}
&
\begin{minipage}{4.5cm}
One of the bits in the position between MIN and MAX (included) is flipped.


\EMPH{Data mutation operation:} the operator flips one randomly selected bit of the NUMERIC STRING variable.
If STATE is specified, the mutation is applied only if the bit is in the specified state.
\end{minipage}
\\
\hline
SEQUENCE&
SWAP&
\begin{minipage}{4.5cm}
\end{minipage}
&
\begin{minipage}{4.5cm}
This operator aims to swap two items in a sequence of items of different types. It still requires to be defined, based on ASN.1 specifications.
\end{minipage}
\\
\hline
SEQUENCE&
DEL&
\begin{minipage}{4.5cm}
\end{minipage}
&
\begin{minipage}{4.5cm}
This operator aims to delete one item in a set/sequence of items of the different types. It still requires to be defined, based on ASN.1 specifications.
\end{minipage}
\\
\hline
SEQUENCE&
DUP&
\begin{minipage}{4.5cm}
\end{minipage}
&
\begin{minipage}{4.5cm}
This operator aims to duplicate one item in a set/sequence of items of the different types. It still requires to be defined, based on ASN.1 specifications.
\end{minipage}
\\
\hline
SET&
DEL&
\begin{minipage}{4.5cm}
\end{minipage}
&
\begin{minipage}{4.5cm}
\end{minipage}
\\
\hline
SET&
DUP&
\begin{minipage}{4.5cm}
\end{minipage}
&
\begin{minipage}{4.5cm}
\end{minipage}
\\
\hline
CHOICE&
-&
\begin{minipage}{4.5cm}
\end{minipage}
&
\begin{minipage}{4.5cm}
No mutation envisioned for this type.
\end{minipage}
\\
\hline
SEQUENCE OF&
SWAPOF&
\begin{minipage}{4.5cm}
\end{minipage}
&
\begin{minipage}{4.5cm}
This operator aims to swap two items in a sequence/set of items of the same type. It still requires to be defined, based on ASN.1 specifications.
\end{minipage}
\\
\hline
SEQUENCE OF&
DELOF&
\begin{minipage}{4.5cm}
\end{minipage}
&
\begin{minipage}{4.5cm}
This operator aims to eliminate one item in a sequence/set of items of the same type. It still requires to be defined, based on ASN.1 specifications.
\end{minipage}
\\
\hline
SEQUENCE OF&
DUPOF&
\begin{minipage}{4.5cm}
\end{minipage}
&
\begin{minipage}{4.5cm}
This operator aims to duplicate one item in a sequence/set of items of the same type. It still requires to be defined, based on ASN.1 specifications.
\end{minipage}
\\
\hline
SET OF&
DELOF&
\begin{minipage}{4.5cm}
\end{minipage}
&
\begin{minipage}{4.5cm}
\end{minipage}
\\
\hline
SET OF&
DUPOF&
\begin{minipage}{4.5cm}
\end{minipage}
&
\begin{minipage}{4.5cm}
\end{minipage}
\\



%Incorrect Identifier& Several transmission data fields have fixed values, for example fields identifying the transmitting satellite. Hardware/software errors may assign incorrect identifiers.\\
%%Incorrect Checksum& Hardware/software errors may result in an incorrect checksum for a Packet or VCDU.\\
%Incorrect Counter& Counters are used to track Packet or VCDU ordering. Hardware/software errors may assign incorrect counter values.\\
%Flipped Data Bits& Physical channel noise may flip one or more bits in the data transmission.\\
\hline
\end{longtable}
\normalsize
