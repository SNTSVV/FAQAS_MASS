% !TEX root = MutationTestingSurvey.tex

\subsection{Mutation Operators}
\label{sec:operators}

Mutation testing introduces small syntactical changes into the code of a program 
%\MREVISION{C8}{(source code, intermediate representation, or executable code)} 
through a set of mutation operators.
The goal of the mutation operators is to simulate artificial faults by systematically introducing  simple syntax changes \MREVISION{C9}{based on errors that programmers typically make. 
Each of these operators model a specific type of fault. For example, the use of a wrong variable is a fault; the \emph{variable references} category of operators, introduced later in this section, model this type of fault.}


\TODO{verify which set we are considering}

Table~\ref{table:codeoperatorssummary} provides the  category of the mutation operators that will be implemented in FAQAS along with their relevance. All the relevant operators will be implemented in the FAQAS toolset. The others might be implemented if empirical evaluation, code inspection, and discussion with engineers show that might be necessary.

The relevant set coincides with the \INDEX{sufficient set of operators}~\cite{rothermel1996experimental,andrews2005mutation}.
This set is composed of the following operators: absolute value insertion (ABS), arithmetic operator replacement (AOR), logical connector replacement (LCR), relational operator replacement (ROR) the unary operator insertion (UOI) operator and the \INDEX{statement deletion operator} (SDL).
We refer to the sufficient set of operators identified by Andrews et al.~\cite{andrews2005mutation}. With respect to previous work, this set includes also the the \INDEX{statement deletion operator} (SDL). The SDL operator has been included because it ensures that every pointer-manipulation and field-assignment statement is properly tested, thus targeting faults not simulated by the rest of the sufficient operators. In addition, recent research results show that the SDL operator is the most effective for fault detection~\cite{delamaro2014designing}. 


We selected the \INDEX{sufficient set of operators}, because
there is empirical evidence that \emph{(1) test suites that are \INDEX{mutation adequate} (i.e., achieve 100\% mutation score) with respect to the sufficient set of operators also achieve a very high mutation score if we consider a larger set of mutation operators}~\cite{offutt1996experimental}, \emph{(2) the sufficient set of operators enables an \EMPH{accurate estimation of the mutation score} of a test suite}~\cite{siami2008sufficient}, and (3) the mutation score can estimate the \INDEX{fault detection rate} (i.e., the number of real faults discovered) of a test suite~\cite{andrews2005mutation}.


\TODO{Unclear what is the column number of operators}

% !TEX root =  ../MutationTestingSurvey.tex



\setlength\LTleft{0pt}
\setlength\LTright{0pt}
\small 
\begin{longtable}{@{\extracolsep{\fill}}|l|l|l|l|@{}}
\caption{\normalsize Summary of the code-driven mutation operators applicable to space-context software. On the last column \emph{Relevance for FAQAS} the symbols mean: ** \textit{relevant}, * \textit{potentially relevant}, * \textit{NA} Not Applicable to FAQAS case studies.}
\label{table:codeoperatorssummary} \\
\hline

	\textbf{\begin{tabular}[c]{@{}l@{}}Operator\\Category\end{tabular}}	&	\textbf{\begin{tabular}[c]{@{}l@{}}Number of\\Operators\end{tabular}}	&	\textbf{Context}	&	\textbf{Relevance for FAQAS}\\

\hline
	Arithmetic			&	14	&	C/C++; LLVM-IR; OO; SQL; Simulink	& **\\
	Bitwise				&	15	&	C/C++; LLVM-IR; OO; SQL 			& **\\
	Casts				&	1	&	C/C++ 								& *\\
	Constants			&	8	&	C/C++; LLVM-IR; OO; Simulink 		& **\\
	Control-flow		&	11	&	C/C++ & *\\
	Coverage			&	6	&	C/C++; ADA & *\\
	Deletion			&	5	&	C/C++; LLVM-IR; OO & **\\
	Logical				&	10	&	C/C++; LLVM-IR; OO; Simulink & **\\
	Relational			&	5	&	C/C++; LLVM-IR; OO; SQL; Simulink & **\\
	Variable References	&	12	&	C/C++; LLVM-IR; OO; SQL; Simulink & **\\
	Shift				&	10	&	C/C++; OO & *\\
	Statements			&	21	&	C/C++; ADA; Simulink & *\\
	Structures			&	3	&	C/C++ & *\\
	Strings				&	7	&	C/C++ & *\\
	Function Calls		&	38	&	C/C++; LLVM-IR; SQL & *\\
	Floating Points		&	6	&	C/C++ & *\\
	Memory Operations	&	9	&	C/C++ & *\\
	System-level		&	14	&	OO & *\\
	Object-Oriented		&	36	&	C/C++; OO & *\\
	SQL					&	27	&	SQL & NA\\
	ADA					&	47	&	ADA & NA\\
\bottomrule                                                             
\end{longtable}
\normalsize

