% !TEX root =  ../MAIN.tex
\begin{lstlisting}[style=CStyle, caption=Test template to enable data-driven mutation testing for libParam, label=GSLtest]

    // a little hack - this is next element, we use it check for overwrite and missing 0 termiation
    memset(alltypes_mem.string_A, 'Z', sizeof(alltypes_mem.string_A));
    alltypes_mem.string_A[0][1] = 0;

    char buf[GS_TEST_ALLTYPES_STRING_LENGTH + 10];

    // get max size - no 0 termination
    memset(alltypes_mem.string, 'B', sizeof(alltypes_mem.string));
    memset(buf, 'A', sizeof(buf));
    buf[GS_TEST_ALLTYPES_STRING_LENGTH + 1] = 0;
    
    csp_node CSP_NODE;
    unsigned long long int tableID;
    klee_make_symbolic(&CSP_NODE, sizeof(CSP_NODE), ”CSP_NODE”);
    klee_make_symbolic(&tableID, sizeof(tableID), ”tableID”);
    gs_rparam_get_string(&CSP_NODE, tableID, GS_TEST_ALLTYPES_STRING, GS_RPARAM_MAGIC_CHECKSUM, 1000, buf, GS_TEST_ALLTYPES_STRING_LENGTH);

    
\end{lstlisting}



%flag E_Decode(E* pVal, BitStream* pBitStrm, int* pErrCode)
%{
%    flag ret = TRUE;
%    *pErrCode = 0;
%    (void)pVal;
%    (void)pBitStrm;
%
%
%    (*(pVal))=5; ret = TRUE; *pErrCode = 0;
%
%    // Manually added probe 
%    E_mutate(pVal);
%    // Manually added probe END
%    return ret  && E_IsConstraintValid(pVal, pErrCode);
%}
