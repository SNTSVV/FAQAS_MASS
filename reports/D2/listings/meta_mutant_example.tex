% !TEX root =  ../MAIN.tex



\begin{lstlisting}[style=CStyle, float=h, caption=Function T\_INT\_IsConstraintValid., label=original_meta]
flag T_INT_IsConstraintValid(const T_INT* pVal, int* pErrCode)
{
    flag ret = TRUE;
    (void)pVal;

    ret = ((*(pVal)) <= 50UL);
    *pErrCode = ret ? 0 :  ERR_T_INT;

    return ret;
}
\end{lstlisting}

\begin{lstlisting}[style=CStyle, float=h, caption=Mutant 1 of function T\_INT\_IsConstraintValid., label=meta_mutant_1]
flag T_INT_IsConstraintValid(const T_INT* pVal, int* pErrCode)
{
    flag ret = TRUE;
    (void)pVal;

    ret = ((*(pVal)) <= 50UL);
    *pErrCode = ret ? 1 :  ERR_T_INT;

    return ret;
}
\end{lstlisting}

\begin{lstlisting}[style=CStyle, float=h, caption=Mutant 2 of function T\_INT\_IsConstraintValid., label=meta_mutant_2]
flag T_INT_IsConstraintValid(const T_INT* pVal, int* pErrCode)
{
    flag ret = TRUE;
    (void)pVal;

    ret = ((*(pVal)) <= 50UL);
    *pErrCode = ret ? (-1) :  ERR_T_INT;

    return ret;
}
\end{lstlisting}

\begin{lstlisting}[style=CStyle, float=h, caption=Meta-Mutant for function T\_INT\_IsConstraintValid., label=meta_mutant_example]
flag T_INT_IsConstraintValid(const T_INT* pVal, int* pErrCode)
{
    flag ret = TRUE;
    (void)pVal;

    ret = ((*(pVal)) <= 50UL);

    klee_semu_GenMu_Mutant_ID_Selector_Func(1,2);
    *pErrCode = ret ? 
    	( klee_semu_GenMu_Mutant_ID_Selector==2 ?
    		((-1)):
		    (klee_semu_GenMu_Mutant_ID_Selector==1?
			    (1):
			    (0))) 
			    :  ERR_T_INT;
    klee_semu_GenMu_Post_Mutation_Point_Func(0,0);
    klee_semu_GenMu_Post_Mutation_Point_Func(1,2);

    return ret;
}
\end{lstlisting}