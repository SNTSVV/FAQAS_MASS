% !TEX root = MAIN.tex
\clearpage

\section{MLFS}
\label{sec:caseStudies:GSL:MLSF}

\subsection{Overview of the case study}

The Mathematical Library for Flight Software (MLFS) implements mathematical functions ready for qualification. MLFS reuses libm present in newlib and pre-qualifying the library to ECSS criticality category B. The set of functions are limited to the functions typically needed in flight software. 

Details about MLFS are provided in the document \emph{doc} uploaded on Alfresco.

The source code size is 5\,402 LOC, while the unit test suite consists of 4\,042 tests for 92 functions.

\subsection{Code-driven mutation testing}

% !TEX root = ../MAIN.tex

\begin{table}[h]

\centering
\begin{tabular}{|l|l|}
\hline
\textbf{Coverage Type} & \textbf{Coverage Rate} \\
\hline
Statement     & 100\% (1\,978 of 1\,978 statements)\\
Functions     & 100\% (90 of 90 functions)\\
Branches      & 100\% (1\,302 of 1\,302 branches)\\
\hline
\end{tabular}
\caption{MLFS code coverage.}
\label{table:mlfs_coverage}

\end{table}

Table~\ref{table:mlfs_coverage} provides the code coverage information of the MLFS unit test suite. The test suite achieves 100\% code coverage (i.e., statement, function and branch coverage).

Given the code coverage, we focus our analysis on all components of the MLFS:

\begin{itemize}
	\item MLFS: implementation of 4 mathematical functions.
	\item Machine: dedicated implementation of 2 mathematical functions for the sparc v8 architecture.
	\item Math: implementation of 66 mathematical functions.
	\item Common: implementation of 20 mathematical functions.
\end{itemize}

\subsubsection{Mutation Testing Preliminary Results}

\TODO{We can add some preliminary results}

% % !TEX root = ../MAIN.tex

\begin{table}[h]
\small
\centering
\caption{Code-driven mutation testing preliminary results for the libgscsp case study.}
\label{table:libgscsp_preliminary}
\begin{tabular}{|l|l|l|l|l|l|l|}
\hline
        & \multicolumn{5}{c|}{Mutants}                                                                      & \multirow{3}{*}{\begin{tabular}[c]{@{}l@{}}Mutation Score\\ (K/K+L)\end{tabular}} \\ \cline{1-6}
        &     &                                                        &      & \multicolumn{2}{c|}{Killed} &                                                                                   \\ \cline{1-6}
Mutants & All & \begin{tabular}[c]{@{}l@{}}Not\\ Compiled\end{tabular} & Live & Test Failure    & Timeout   &                                                                                   \\ \hline
Total   &  6\,196   &  1\,700 & 1\,708      & 2\,495                & 277          & 61.88\%                                                                           \\ \hline
\end{tabular}
\end{table}    
             

% In order to analyze the feasibility of the code-driven mutation testing for MLFS, we conducted a preliminary experimentation using the mutation operators 
% AOR, ROR, ICR, LCR, ABS, UOI and SDL we generated 6\,196 mutants. For the experimentation we consider only the libcsp (GSL branch) source code. Preliminary results can be found in Table~\ref{table:libgscsp_preliminary}.
% Particularly, we observe that from the 6\,196 generated mutants, we had 1\,700 mutants that were not compiled by the compilation toolset of libgscsp, most probably because the mutation introduced a syntactical error that was detected by the toolset.
% Then, we identified 1\,708 live mutants that were not detected by the test suite. Instead, we had 2\,495 killed mutants detected by the test suite, and 277 mutants that produced libutil to go into an infinite loop, and thus were killed by timeout. The final mutation score was of 61.88\%.

The identification of equivalent mutants still needs to be assessed.


\subsection{Data-driven mutation testing}

We do not plan to apply data drive mutation testing to this case study because is a standalone library; it does not integrate communicating components.



