% !TEX root = MAIN.tex

\section{Acronyms and Abbreviations}
\label{sec:acronyms}


%@{\extracolsep{\fill}}
\begin{tabular}{|p{1.2cm}|p{12cm}|}
%&\TODO{Add acronyms. OSCAR: I've add some...} \\

%ACO & Ant Colony Optimization\\
D1& Deliverable 1 of FAQAS activity, \emph{Analysis and Survey of Mutation Testing}\\
D2& Deliverable 2 of FAQAS activity, \emph{Study of Mutation Testing Applicability to Space Software}\\
FOM & First Order Mutant\\
HOM & Higher-Order Mutant\\
LOC & Lines of Code\\
%SBST & Search Based Software Testing\\
%Fabrizio: we cannot use SE as acronym
%SE & Symbolic Execution\\
SOM & Second Order Mutant\\
SUT & Software Under Test\\
TS & Test Suite\\
UL HPC & University of Luxembourg High Performance Computing

%Comment #2:
%When generating the pdf file, make sure to have the index of the document on the left, so that the document is easier to navigate.
%Author: Pedro Barrios Subject: Sticky Note Date: 28/02/2020, 09:18:06
%Comment #3:
%Try to make the document a bit more friendly to read (e.g. by adding some sentences in bold (see the highlighted ones for this paragraph), making bigger
%separation between paragraphs, ...)
%Author: Pedro Barrios Subject: Sticky Note Date: 28/02/2020, 09:18:11
%Comment #4:
%It is perhaps interesting to add a chapter with some basic definitions?
%e.g. equivalent mutant, redundant mutant, mutation score, killed mutant, live mutant, weak mutation, ...
%Author: Pedro Barrios Subject: Sticky Note Date: 28/02/2020, 09:18:18
%Comment #5:
%Please, consider to add more examples.

                                                           
\end{tabular}
\normalsize

\clearpage