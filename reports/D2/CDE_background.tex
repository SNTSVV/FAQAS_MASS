% !TEX root =  Main.tex

\subsection{Background and Applicability of state-of-the-art mutation testing techniques to space software}
\label{sec:background}

In this section, we discuss the applicability of state-of-the-art mutation testing optimizations in the context of space software. A detailed overview of mutation testing solutions and optimizations can be found in recent surveys~\cite{jia2010analysis,papadakis2019mutation} and deliverable D1.

\subsubsection{Mutation adequacy and mutation score computation}
\label{background:adequacy}
A mutant is said to be killed if at least one test case in the test suite fails when executed against the mutant.
Mutants that do not lead to the failure of any test case are said to be live.
Three conditions should hold for a test case to kill a mutant: \emph{reachability} (i.e, the test case should execute the mutated statement), \emph{necessity} (i.e., the test case should reach an incorrect intermediate state after executing the mutated statement), and \emph{sufficiency} (i.e., the final state of the mutated program should differ from that of the original program)~\cite{offutt1997automatically}.

%The identification of killed and live mutants enables the definition of a mutation adequacy criterion as follow, a test suite is mutation-adequate if all mutants are killed by at least one test case of the test suite. Also, 
The mutation score, i.e., the percentage of killed mutants, measures the quality of a test suite quantitatively. Recent studies have shown that achieving a high mutation score improves significantly the fault detection capability of a test suite
~\cite{papadakis2018mutation}. 
%However, 
%to ensure better fault detection than a randomly selected subset of test cases, a test suite should achieve a very high mutation score~\cite{Chekam:17}.
%More precisely, they show that among randomly selected test suites, ranked based on mutation score and structural coverage, 
%only  the test suites in the top 5\% rank according to mutation score achieve a better fault detection rate than the ones ranked according to other criteria (e.g., branch coverage)~\cite{Chekam:17}. 
%The literature lacks studies on the identification of a mutation score threshold that guarantees achieving a fault detection rate higher than the one achieved by other adequacy criteria. 
However, a very high mutation score (e.g., above 0.75) is required to achieve a higher fault detection rate than the one obtained with other coverage criteria, such as statement and branch coverage~\cite{Chekam:17}.

The capability of a test case to kill a mutant also depends on the observability of the program state. To overcome the limitations due to observability, different strategies to identify killed mutants can be adopted; they are known as strong, weak, firm, and flexible mutation coverage~\cite{ammann2016introduction}. For space software, we suggest to rely on strong mutation because it is the only criterion that assesses the actual test suite's fault detection capability; indeed, it relies on a mutation score that reflects the percentage of mutants leading to test failures. With the other mutation coverage criteria, a mutant is killed if the state of the mutant after the execution of the mutated statement differs from the one observed with the original code, without any guarantee that either the erroneous values in state variables propagate or the test oracles detect them. 




\subsubsection{Mutation Operators}
\label{sec:related:operators}

%% !TEX root =  ../Main.tex

\newcommand{\op}{\mathit{op}}
\newcommand{\ArithmeticSet}{ \texttt{+}, \texttt{-}, \texttt{*}, \texttt{/}, \texttt{\%} }
\newcommand{\LogicalSet}{ \texttt{&&}, \texttt{||} }
\newcommand{\RelationalSet}{ \texttt{>}, \texttt{>=}, \texttt{<}, \texttt{<=}, \texttt{==}, \texttt{!=} }
\newcommand{\BitWiseSet}{ \texttt{\&}, \texttt{|}, \land }
\newcommand{\ShiftSet}{ \texttt{>>}, \texttt{<<} }


\begin{table}[h]
\caption{Implemented set of mutation operators.}
\label{table:operators} 
\centering
\scriptsize
\begin{tabular}{|@{}p{4mm}@{}|@{}p{2cm}@{\hspace{1pt}}|@{}p{11.1cm}@{}|}
\hline
&\textbf{Operator} & \textbf{Description$^{*}$} \\
\hline
\multirow{7}{*}{\rotatebox{90}{\emph{Sufficient Set}}}&ABS               & $\{(v, -v)\}$	\\
\cline{2-3}
&AOR               & $\{(\op_1, op_2) \,|\, \op_1, \op_2 \in \{ \ArithmeticSet \} \land \op_1 \neq \op_2 \} $       \\
&    			  & $\{(\op_1, \op_2) \,|\, \op_1, \op_2 \in \{\texttt{+=}, \texttt{-=}, \texttt{*=}, \texttt{/=}, \texttt{\%} \texttt{=}\} \land \op_1 \neq \op_2 \} $       \\
\cline{2-3}
&ICR               & $\{i, x) \,|\, x \in \{1, -1, 0, i + 1, i - 1, -i\}\}$           \\
\cline{2-3}
&LCR               & $\{(\op_1, \op_2) \,|\, \op_1, \op_2 \in \{ \texttt{\&\&}, || \} \land \op_1 \neq \op_2 \}$            \\
&				  & $\{(\op_1, \op_2) \,|\, \op_1, \op_2 \in \{ \texttt{\&=}, \texttt{|=}, \texttt{\&=}\} \land \op_1 \neq \op_2 \}$            \\
&				  & $\{(\op_1, \op_2) \,|\, \op_1, \op_2 \in \{ \texttt{\&}, \texttt{|}, \texttt{\&\&}\} \land \op_1 \neq \op_2 \}$            \\
\cline{2-3}
&ROR               & $\{(\op_1, \op_2) \,|\, \op_1, \op_2 \in \{ \RelationalSet \}\}$            \\
&				  & $\{ (e, !(e)) \,|\, e \in \{\texttt{if(e)}, \texttt{while(e)}\} \}$ \\
\cline{2-3}
&SDL               & $\{(s, \texttt{remove}(s))\}$            \\
\cline{2-3}
&UOI               & $\{ (v, \texttt{--}v), (v, v\texttt{--}), (v, \texttt{++}v), (v, v\texttt{++}) \}$            \\   
\hline
\hline
\multirow{5}{*}{\rotatebox{90}{\emph{OODL}}}&AOD               & $\{((t_1\,op\,t_2), t_1), ((t_1\,op\,t_2), t_2) \,|\, op \in \{ \ArithmeticSet \} $       \\ 
\cline{2-3}
&LOD               & $\{((t_1\,op\,t_2), t_1), ((t_1\,op\,t_2), t_2) \,|\, op \in \{  \} \}$       \\ 
\cline{2-3}
&ROD               & $\{((t_1\,op\,t_2), t_1), ((t_1\,op\,t_2), t_2) \,|\, op \in \{ \RelationalSet \} \}$       \\ 
\cline{2-3}
&BOD               & $\{((t_1\,op\,t_2), t_1), ((t_1\,op\,t_2), t_2) \,|\, op \in \{ \BitWiseSet \} \}$       \\ 
\cline{2-3}
&SOD               & $\{((t_1\,op\,t_2), t_1), ((t_1\,op\,t_2), t_2) \,|\, op \in \{ \ShiftSet \} \}$       \\ 
%\hline
%COR               & $\{(\op_1, \op_2) \,|\, \op_1, \op_2 \in \{ \texttt{\&\&}, \texttt{||}, \land \} \land \op_1 \neq \op_2 \}$            \\
\hline
\hline
\multirow{3}{*}{\rotatebox{90}{\emph{Other}}}&LVR			& $\{(l_1, l_2) \,|\, (l_1, l_2) \in \{(0,-1), (l_1,-l_1), (l_1, 0), (\mathit{true}, \mathit{false}), (\mathit{false}, \mathit{true})\}\}$\\
&&\\
&&\\
\hline
\end{tabular}

$^{*}$Each pair in parenthesis shows how a program element is modified by the mutation operator. Th eleft element of the pair is replaced with the right element. We follow standard syntax~\cite{kintis2018effective}. Program elements are literals ($l$), integer literals ($i$), boolean expressions ($e$), operators ($\op$), statements ($s$), variables ($v$), and terms ( $t_i$, which might be either variables or literals).
\end{table}


Mutation testing introduces small syntactical changes into the code (source code or machine code) of a program through a set of mutation operators that simulate programming mistakes. 



The  \emph{sufficient set of operators} is widely used for conducting empirical evaluations ~\cite{offutt1996experimental,rothermel1996experimental,andrews2005mutation,kintis2017detecting}. 
%Initially defined by Offutt et al., the set has been extended to include newly defined operators.
The original sufficient set, defined by Offutt et al., is composed of the following operators: Absolute Value Insertion (ABS), Arithmetic Operator Replacement (AOR), Integer Constraint Replacement (ICR), Logical Connector Replacement (LCR), Relational Operator Replacement (ROR), and Unary Operator Insertion (UOI)~\cite{offutt1996experimental}.
% operator and the \INDEX{statement deletion operator} (SDL).
Andrews et al.~\cite{andrews2005mutation} have also included the 
\emph{statement deletion operator} (SDL)~\cite{delamaro2014designing}, which ensures that every pointer-manipulation and field-assignment statement is tested. 
%Table~\ref{table:sufficient_operators} provides an overview of the operators belonging to the sufficient set.
%, thus targeting faults not simulated by the rest of the sufficient operators. In addition, recent research results show that the SDL operator is the most effective for fault detection~\cite{delamaro2014designing}. 

\CHANGED{The SDL operator has inspired the definition of mutation operators (e.g., \emph{OODL operators}) that delete portions of program statements, with the objective of replacing the sufficient set with a simpler set of mutation operators.
For example, the OODL mutation operators include the delete Arithmetic (AOD), Bitwise (BOD), Logical (LOD), Relational (ROD), and Shift (SOD) operators.}
Deletion operators produce significantly less equivalent mutants~
\cite{delamaro2014designing,delamaro2014experimental} and, furthermore, 
test suites that kill mutants generated with both the SDL and the \emph{OODL operators} kill a very high percentage of all mutants (i.e., 97\%)~\cite{delamaro2014experimental}. 
%However, since space software is different than other types of software systems (e.g., it includes functions to process signals, which are absent in Unix utilities), the pertinence of the mutation score generated with the SSDL operator should be evaluated.

%Operators used in other papers:
%
%Papadakis, M., Shin, D., Yoo, S., & Bae, D.-H. (2018). Are mutation scores correlated with real fault detection? a large scale empirical study on the relationship between mutants and real faults. 2018 IEEE/ACM 40th International Conference on Software Engineering (ICSE), 537?548.
%AOR (Arithmetic Operator Replacement), LOR (Logical Operator Replacement), COR (Conditional Operator Replacement), ROR (Relational Operator Replacement), ORU (Operator Replace- ment Unary), STD (STatement Deletion), and . Additional
%
%Chekam, T. T., Papadakis, M., Le Traon, Y., & Harman, M. (2017). An empirical study on mutation, statement and branch coverage fault revelation that avoids the unreliable clean program assumption. 2017 IEEE/ACM 39th International Conference on Software Engineering (ICSE), 597?608.
%The mutation tool includes the (large and varied) set of mutant operators used in previous research [4], [30], [49]. Specifically, we used mutants related to arithmetic, relational, conditional, logical, bitwise, shift, pointers and unary operators. We also used statement deletion, variable and constant replacement.



%(1) test suites that are mutation adequate (i.e., achieve 100\% mutation score) with respect to the sufficient set of operators also achieve a very high mutation score if we consider a larger set of mutation operators~\cite{offutt1996experimental}, 
The sufficient set of operators enables an accurate estimation of the mutation score of a test suite~\cite{siami2008sufficient}; furthermore, the mutation score computed with the sufficient set is a good estimate the fault detection rate (i.e., the portion of real faults discovered) of a test suite~\cite{andrews2005mutation}. 

Recent work has shown that, to maximize the detection of real faults, a set of operators should be used in addition to the sufficient set and includes: Conditional Operator Replacement (COR),
Literal Value Replacement (LVR), and Arithmetic Operator Deletion (AOD)~\cite{Kintis2018}. 
%AOD is subsumed by OODL.
%However, LVR, which replaces constants with 0 and 1, or true and false is subsumed by 

An alternative to the sufficient set of operators is the generation of \emph{higher order mutants}, which result from the application of multiple mutation operators for each mutation~\cite{jia2009higher,kintis2010evaluating,offutt1992investigations,papadakis2010empirical}. However, higher order mutants are \CHANGED{easier to be killed
%lower strength 
than the first order ones (i.e., less effective to assess test suites limitations)}~\cite{papadakis2010mutation,papadakis2019mutation}, and there is 
\CHANGED{limited empirical evidence regarding which mutation operators should be combined to resemble real faults and minimize the number of redundant mutants~\cite{papadakis2019mutation}.}
%\emph{lack of clear theory on which mutants are of some value}, which may lead to redundant mutants~\cite{papadakis2019mutation}. For this reason, we focus our study on first order mutations.
% generated with the sufficient set.

%decreases consistently. 
%For instance, Papadakis and Malevris~\cite{papadakis2010empirical}, worked on a approach for higher order mutants for the C programming language that lead to a reduction of approximately 80-90\% of the generated equivalent mutants, with a fault detection ability loss only of 11-15\%. 
%
%
%that higher order mutants are of relatively lower strength than the first order ones
%
%Since modern space software is implemented in C and C++, thus sharing a degree of similarity with the case studies considered in the empirical evaluations for the sufficient set, we believe that the empirical findings concerning the sufficient set also hold for space software.

%rely on the sufficient set of operators.
%, to determine if results reported in the literature may generalize to the case of space software.
%For these reasons, we believe the sufficient set of operators to be necessary to be considered also for space software.

%% !TEX root =  ../Main.tex

\begin{table}
\scriptsize
\centering
\label{table:codeoperatorssummary}
\caption{Categories of mutation operators applicable to space software.}
\begin{tabular}{p{1.8cm}|p{5.4cm}}
\toprule
Operator Category & Description\\
\midrule
arithmetic & Includes operators that performs mutations on arithmetic expressions and assignments, usually these mutations replaces an element by an operator $op \in \{+, -, *, /, \%\}$. \\
bitwise & Includes operators that performs bitwise mutations on expressions and assignments, these mutations aims for replacing an element of an expression by an operator $op \in \{|, \&, \wedge, \sim\}$. \\
casts & Includes operators that work by mutating casting expressions.\\
constants & Includes operators that mutate constants in the code. \\
control-flow & Includes operators that modify and alter the execution flow of the running application.\\
coverage & Includes operators that mutate the code in such a way that the static structure of the code (e.g., branches) is likely covered during execution.\\
deletion & Includes operators that mutate the code by deleting statements, operators, constants and variables.  \\
floating points & Includes operators working on floating point comparison (FPC).\\
function calls & Includes operators that mutate function calls and their components (e.g., signature and parameters).\\
logical & Includes operators that performs logical mutations on expressions and assignments; these mutations replace an element of an expression by an operator $op \in \{\&\&, \|, !\}$.\\
variable references & Includes operators mutating different types of variable references in the code (e.g., scalars, arrays and pointers).\\
shift & Includes operators that performs shift mutations on expressions and assignments; usually, these mutations replace an element by an operator $op \in \{<<, <<<, >>, >>>\}$.\\
SQL & Includes operators that modify SQL clauses, SQL \texttt{NULL} operators, and SQL \texttt{WHERE} operators.\\
statements & Includes operators performing syntactical changes at a statement-level granularity.\\
strings & Includes operators that performs mutations on variables of string type.\\
structures & Includes operators that mutate C/C++ structures.\\
higher-order & Includes operators for generating second- and higher-order mutants. These strategies combine first order mutants to simulate more complex faults, motivated by a desire to capture subtle faults~\cite{jia2009higher}. Usually, each operator of this category introduce a strategy on how to combine first order operators to form a higher-order one. For example, the \emph{RMix} operator combines any two randomly selected first order operator to generate a higher-order mutant. \\
system-configuration & Includes operators that perform mutations at a system-configuration level; usually, these mutation operators aim to cause faults concerning the configuration of the system itself.\\
system-level & Includes operators that performs mutations at a system-level; usually, these mutation operators aim to cause faults concerning the integration of system components. \\
ADA expressions & Includes operators that work by modifying expressions in the ADA programming language. \\
ADA operands & Includes operators that perform mutations by replacing operands in ADA expressions (e.g., variables, constants, records, arrays, pointers).\\
ADA tasks & Includes operators that mutate ADA code at the granularity of tasks and aim to trigger problems related to concurrency in the program.\\
\bottomrule
\end{tabular}
\end{table}
%
%In addition to the sufficient set of operators, we can group the mutation operators targeting space software (i.e., software implemented in C/C++) into XX categories. Table~\ref{} provides an overview of these additional categories of operators along with references and a discussion of the reasons why they should be selected or avoided when applying mutation testing to space software.

%% !TEX root =  ../Main.tex

\newcommand{\op}{\mathit{op}}
\newcommand{\ArithmeticSet}{ \texttt{+}, \texttt{-}, \texttt{*}, \texttt{/}, \texttt{\%} }
\newcommand{\LogicalSet}{ \texttt{&&}, \texttt{||} }
\newcommand{\RelationalSet}{ \texttt{>}, \texttt{>=}, \texttt{<}, \texttt{<=}, \texttt{==}, \texttt{!=} }
\newcommand{\BitWiseSet}{ \texttt{\&}, \texttt{|}, \land }
\newcommand{\ShiftSet}{ \texttt{>>}, \texttt{<<} }


\begin{table}[h]
\caption{Implemented set of mutation operators.}
\label{table:operators} 
\centering
\scriptsize
\begin{tabular}{|@{}p{4mm}@{}|@{}p{2cm}@{\hspace{1pt}}|@{}p{11.1cm}@{}|}
\hline
&\textbf{Operator} & \textbf{Description$^{*}$} \\
\hline
\multirow{7}{*}{\rotatebox{90}{\emph{Sufficient Set}}}&ABS               & $\{(v, -v)\}$	\\
\cline{2-3}
&AOR               & $\{(\op_1, op_2) \,|\, \op_1, \op_2 \in \{ \ArithmeticSet \} \land \op_1 \neq \op_2 \} $       \\
&    			  & $\{(\op_1, \op_2) \,|\, \op_1, \op_2 \in \{\texttt{+=}, \texttt{-=}, \texttt{*=}, \texttt{/=}, \texttt{\%} \texttt{=}\} \land \op_1 \neq \op_2 \} $       \\
\cline{2-3}
&ICR               & $\{i, x) \,|\, x \in \{1, -1, 0, i + 1, i - 1, -i\}\}$           \\
\cline{2-3}
&LCR               & $\{(\op_1, \op_2) \,|\, \op_1, \op_2 \in \{ \texttt{\&\&}, || \} \land \op_1 \neq \op_2 \}$            \\
&				  & $\{(\op_1, \op_2) \,|\, \op_1, \op_2 \in \{ \texttt{\&=}, \texttt{|=}, \texttt{\&=}\} \land \op_1 \neq \op_2 \}$            \\
&				  & $\{(\op_1, \op_2) \,|\, \op_1, \op_2 \in \{ \texttt{\&}, \texttt{|}, \texttt{\&\&}\} \land \op_1 \neq \op_2 \}$            \\
\cline{2-3}
&ROR               & $\{(\op_1, \op_2) \,|\, \op_1, \op_2 \in \{ \RelationalSet \}\}$            \\
&				  & $\{ (e, !(e)) \,|\, e \in \{\texttt{if(e)}, \texttt{while(e)}\} \}$ \\
\cline{2-3}
&SDL               & $\{(s, \texttt{remove}(s))\}$            \\
\cline{2-3}
&UOI               & $\{ (v, \texttt{--}v), (v, v\texttt{--}), (v, \texttt{++}v), (v, v\texttt{++}) \}$            \\   
\hline
\hline
\multirow{5}{*}{\rotatebox{90}{\emph{OODL}}}&AOD               & $\{((t_1\,op\,t_2), t_1), ((t_1\,op\,t_2), t_2) \,|\, op \in \{ \ArithmeticSet \} $       \\ 
\cline{2-3}
&LOD               & $\{((t_1\,op\,t_2), t_1), ((t_1\,op\,t_2), t_2) \,|\, op \in \{  \} \}$       \\ 
\cline{2-3}
&ROD               & $\{((t_1\,op\,t_2), t_1), ((t_1\,op\,t_2), t_2) \,|\, op \in \{ \RelationalSet \} \}$       \\ 
\cline{2-3}
&BOD               & $\{((t_1\,op\,t_2), t_1), ((t_1\,op\,t_2), t_2) \,|\, op \in \{ \BitWiseSet \} \}$       \\ 
\cline{2-3}
&SOD               & $\{((t_1\,op\,t_2), t_1), ((t_1\,op\,t_2), t_2) \,|\, op \in \{ \ShiftSet \} \}$       \\ 
%\hline
%COR               & $\{(\op_1, \op_2) \,|\, \op_1, \op_2 \in \{ \texttt{\&\&}, \texttt{||}, \land \} \land \op_1 \neq \op_2 \}$            \\
\hline
\hline
\multirow{3}{*}{\rotatebox{90}{\emph{Other}}}&LVR			& $\{(l_1, l_2) \,|\, (l_1, l_2) \in \{(0,-1), (l_1,-l_1), (l_1, 0), (\mathit{true}, \mathit{false}), (\mathit{false}, \mathit{true})\}\}$\\
&&\\
&&\\
\hline
\end{tabular}

$^{*}$Each pair in parenthesis shows how a program element is modified by the mutation operator. Th eleft element of the pair is replaced with the right element. We follow standard syntax~\cite{kintis2018effective}. Program elements are literals ($l$), integer literals ($i$), boolean expressions ($e$), operators ($\op$), statements ($s$), variables ($v$), and terms ( $t_i$, which might be either variables or literals).
\end{table}

\subsubsection{Compile-time Scalability}
\label{sec:compile:time}

%The time spend in compiling mutants depend on the number of mutants to be compiled. 
%The time spent in compiling mutants depends on the number of invocations of the compiler.
%For this reason, \emph{mutant schemata} include all the mutations into a single executable~\cite{untch1993mutation} thus needing only a single  compiler pass. 

To reduce the number of invocations of the compiler to one, \emph{mutant schemata} include all the mutations into a single executable~\cite{untch1993mutation}. 
With mutant schemata, the mutations to be tested are selected at run-time through configuration parameters. This may lead to a compilation speed-up of 300\% \cite{papadakis2010automatic}. 


Another solution to address compile-time scalability issues consists of \emph{mutating machine code}  (e.g., binary code~\cite{becker2012xemu}, assembly language~\cite{crouzet2006sesame},
Java~\cite{ma2006mujava}, 
 and
.NET~\cite{derezinska2011object} bytecode) thus avoiding the execution of the compilation process after creating a mutant. 
%Empirical results show that the generation of mutants for compiled code requires only 50\% of the time required by a traditional mutation testing process applied to source code~\cite{derezinska2011object,becker2012xemu}.
A common solution consists of mutating the
 LLVM Intermediate Representation (IR) \cite{hariri2016evaluating}, 
which enables the development of mutants that work with multiple programming languages~\cite{hariri2019comparing} and facilitates the integration of optimizations based on dynamic program analysis~\cite{denisov2018mull}.


Unfortunately, the mutation of machine code 
may lead to mutants that are not representative of real faults because impossible to generate with the source code~\cite{schuler2009efficient}.
In the case of IR mutation, a part of these mutants can be automatically identified~\cite{denisov2018mull}; however,
the number of generated mutants tend to be higher at the IR level than at the source code level, which may reduce scalability~\cite{hariri2019comparing}.
 In addition, we have encountered three problems that prevented the application of 
 mutation testing tools based on  LLVM IR to our case study systems.
First, space software relies on compiler pipelines (e.g., RTEMS~\cite{RTEMS}) that include architecture-specific optimizations not supported by LLVM. 
Second, there is no guarantee that the executables generated by LLVM are equivalent to those produced by the original compiler.
%, which may 
%undermine the validity of mutation testing assessment 
% (e.g., mutants may fail because of errors introduced by the compiler). 
 Third, efficient toolsets based on LLVM often  perform mutations dynamically~\cite{denisov2018mull}, which is infeasible when the software under test needs to be executed within a dedicated simulator, a common situation with space software and many other types of embedded software in cyber-physical systems.




\subsubsection{Runtime Scalability}
\label{sec:scalability}

A straightforward mutation testing process consists of executing the full test suite against every mutant; however, it may lead to scalability problems in the case of a large SUT with expensive test executions.
\emph{Simple optimizations} that can be applied to space software consist of (S1) stopping the execution of the test suite when the mutant has been killed, (S2) executing only those test cases that cover the mutated statements~\cite{delamaro1996proteum}, and (S3) rely on timeouts to automatically detect infinite loops introduced by mutation~\cite{papadakis2019mutation}. 

\emph{Split-stream execution} consists of generating a modified version of the SUT that creates multiple processes (one for each mutant) only when the mutated code is reached \cite{king1991fortran,tokumoto2016muvm}, thus saving time and resources. Unfortunately, it cannot be applied in the case of space software that needs to be run with simulators because, in general, the hosting simulator cannot be forked by the hosted SUT.

Another feasible solution consists of  \emph{randomly selecting a portion of the generated mutants}~\cite{zhang2010operator,gopinath2015hard,zhang2013operator}. 
%
%Zhang et al. \cite{zhang2010operator} demonstrated that a test suite that is capable of killing 50\% of the mutants results in killing more than 99\% of all mutants.
%
%Gopinath et al \cite{gopinath2015hard} show that the mutation score obtained for a subset of the mutants is representative of the mutation score obtained by considering all the mutants. Best results were obtained by considering 1 000 mutants. The number of mutants to consider is independent of the total number of available mutants. With 1 000 mutants, the error in the prediction of the mutation score was 7\% with a probability of 95\%.
% 
Zhang et al. \cite{zhang2013operator} empirically demonstrated that a random selection of 5\% of the mutants is sufficient for 
%correctly predicting the mutation score. 
estimating, with high confidence, the mutation score obtained with the complete mutants set.
Further,
they show that sampling mutants uniformly across different program elements (e.g., functions) %i.e., to have a same percentage of mutants selected for every function/methods) 
leads to a more accurate mutation score prediction than sampling mutants globally in a random fashion. That approach also fares better than uniformly distributing the sampled mutants across different mutation operators. In the presence of large software systems that lead to thousands of mutants, random mutation testing is the only viable solution. However, for very large systems common in industry, randomly selecting 5\% of the mutants might still be too expensive.

\CHANGEDOCT{Gopinath et al. estimate the number of mutants required for an accurate mutation score~\cite{gopinath2015hard}.
They base their idea on the intuition that, under the assumption of independence between mutants,
the mutation score should follow a binomial distribution because it estimates a probability (i.e., the probability for a mutant to be killed by the test suite).
They rely on Tchebysheff’s inequality~\cite{Tchebichef1867} to find a theoretical lower bound on the number of mutants required for an accurate mutation score. 
Since mutants might not be independent but positively correlated (i.e., the detection of a mutant may imply the detection of another mutant), they argue that the binomial distribution provides a conservative estimation of the population variance.
They compute a pessimistic lower bound of 1,000 mutants, which enables the computation of a mutation score that approximates the real mutation score with 93\% accuracy (max absolute error 7\%)  for 95\% of the samples.}
\CHANGEDOCT{Also, they indicate the lack of independence between mutants as the reason why empirical results show that the observed accuracy is above 97\% (absolute error $< 2.7\%$) when 1,000 mutants are selected.
Unfortunately, despite we agree that the mutation score, under the assumption of independence between mutants, should follow a binomial distribution, we find that
% we believe the empirical accuracy to be higher than the predicted one because of variance (which depends on the mutation score $p$ itself, $\mathit{variance} = p(1-p)$ ) being lower than the worst case (i.e., $p=0.5$).
%Indeed,  
Gopinath et al. considerations on covariance contrast with work on correlated binomial distribution and related models~\cite{NG:ModifiedBinomialDistributions:1989,DerGeest:BnomialWithDependentBernoulli-JSCS:2005} which shows that 
a distribution becomes more spread out for positive correlation among the Bernoulli variables~\cite{diniz2010,Zhang:CrrelatedFirearm:NIST:2019}. For example, Mingoti has shown that, in the presence of correlation, the sample size increases as the value of the coefficient of correlation increases~\cite{Mingoti:2003}}. 

\CHANGEDOCT{The correlated binomial model~\cite{Bahadur} and its improved versions (e.g., the additive binomial model~\cite{Kupper1978}) are used when variables (i.e., mutants) are not independent~\cite{Zhang:CrrelatedFirearm:NIST:2019}. In the presence of second order interactions, their probability mass function can be expressed as 
$P(Y)=P'(Y){1+\rho*g(Y)}$, with $P'(Y)$ being the probability mass function of the binomial distribution, g(Y) being a function of $Y$ and $p$, and $\rho$ being the correlation coefficient. It reduces to the binomial distribution when the correlation coefficient is zero. Zhang et al., for example, estimate the correlation coefficient using maximum likelihood estimators~\cite{Zhang:CrrelatedFirearm:NIST:2019}.
Bahadur~\cite{Bahadur,Kadane2016} provides an upper bound for $\rho$,
$$\rho \le \frac{2*p(1-p)}{(n-1)*p*(1-p)+0.25-\gamma_0}$$
$$\gamma_0=\min{0\le k \le n} [{k-(n-p)-1/2}^2]$$
with $n$ being the number of samples.
For a large number of samples (e.g., between 300 and 500, which are the samples considered for the best approach in our empirical evaluation in Section~\ref{sec:evaluation}), 
$\rho = 0.0066$, which is effectively zero. This observation enables us to safely model the mutation score using a binomial distribution ignoring the potential covariance. We demonstrate the validity of our assumption in Section~\ref{sec:evaluation}. We leave the estimation of correlation coefficients to future work.}

%However, what we observe is that positive correlation likely to occur between mutants belonging to the same statement or block of code rather than mutants belonging to different code blocks.
%A sample of 9,604 mutants approximate the real mutation score with 99\% accuracy for 95\% of the samples. 

%In the correlated binomial model the probability mass function is expressed as
%
%$$

\CHANGEDNOV{Related to the work Gopinath et al., the statistics literature provides a number of alternatives for the computation of a sample size (i.e., the number of mutants) that enables the estimation of results with a given degree of precision~\cite{Krejcie,Cochran,Bartlett,Krishnamoorthy07}. For example, the Cochran method is widely adopted when the sample estimates are approximately normally distributed~\cite{Bartlett}. 
%However, since the mutation score captures the probability for a mutant to be killed by the test suite, it is likely to follow a binomial distribution instead of a normal distribution~\cite{gopinath2015hard}, and consequently, the Cochran method may under-approximate the sample size.
Concerning sample size estimation for binomial distributions, the most recent work is that of Gonçalves et al.~\cite{Goncalves2012}, who recently studied the sample size problem by relying on different heuristics for the computation of confidence intervals for binomial proportions.  
A confidence interval indicates that the estimated parameter (e.g., the mutation score) has a probability $p_c$ of lying in it ($p_c$ is referred to as the confidence level). 
Based on their study, the largest number of samples required to compute a 95\% confidence interval is 1568.}
%with margin error of 0.05
%Determining the sample size required to achieve a certain confidence interval is recently attracting interest in statistics literature~\cite{Krishnamoorthy07}.}

\CHANGEDNOV{An alternative to computing the sample size in advance is provided by sequential analysis approaches, which determine the sample size while conducting the statistical test~\cite{waldSequential}. For example, the sequential probability ratio test, which can be used to test simple hypotheses, has been used in mutation testing as condition to determine when to stop test case generation (i.e., when the mutation score is above a given threshold)~\cite{Hsu:90}. In our context, we are interested in point estimation, not hypothesis testing; in this case, 
the sample size can be determined through a fixed-width sequential confidence interval (FSCI), i.e., we compute the confidence interval after every new sample and stop sampling when the interval is within the desired bound~\cite{Frey:FixedWidthSequentialConfidenceIntervals:AmericanStatistician:2010,Chen2013,Yaacoub:OptimalStopping}. 
%Fabrizio: PLease note that the results in Fry's paper are based on half confidence inteval of 0.10. Not clear what happens for 0.05 (what we want). Not sure we want to run also that simulation.
Based on Freys' simulations, we observe that the Wald method~\cite{WaldMethodVollset} for estimating confidence intervals minimizes the sample size but requires an appropriate estimation of variance. We therefore rely on the best non-parametric method, which is Clopper-Pearson~\cite{ClopperPearson}.
FSCI has been successfully adopted to define oracles for non-deterministic systems~\cite{Wang:Uncertainty:TOSEM} but has never been applied to determine the number of mutants to consider for mutation testing.}
% it estimates the sample size $n$ according to the formulas}
%%
%$ n = \frac{n_0}{1+\frac{n_0}{\mathit{N}}}, n_0=\frac{Z^2*p*(1-p)}{e^2}$
%%
%\CHANGED{where $\mathit{N}$ is the population size (i.e., the number of mutants), $Z$ the z-value for a selected alpha level ($Z$ is 1.96 for an alpha of 0.05, i.e., when you want to be 95\% confident about the computed mutation score), $e$ is the acceptable margin of error (e.g., 0.05), $p*(1-p)$ estimates the standard deviation by multiplying the probability of the property of interest (e.g., killing a mutant) with its complement $q$. A conservative value for $p$, which maximizes the estimated standard deviation, is 0.5. The factor $n_0$ provides the upper bound for the sample size, which is $384.16$.}
%%The Krejcie and Morgan method, instead, relies on the following formula $n = \frac{Z^2*N*p(1−p)}{e^2*(N−1)}+X^2*p*(1−p)$.
%%The formula abve gives a sample size of 384,16
%\FIXME{ For example, }

Other solutions to address \emph{runtime scalability problems} of mutation testing  aim to \emph{prioritize test cases} to maximize the likelihood of executing the ones that kill the mutants first~\cite{just2012using,papadakis2011automatically,zhang2013faster}. Their main goal is to save time by preventing the execution of a large portion of the test suite, for each mutant.
Solutions that simply prioritize faster test cases~\cite{just2012using}
 may not be sufficient with system-level test suites whose test cases have similar, long execution times.
Approaches that rely on data-flow analysis to identify and prioritize the test cases that likely satisfy the killing conditions~\cite{papadakis2011automatically} are prohibitively expensive and may not scale to large systems.
Solutions that combine multiple coverage measures~\cite{zhang2013faster} should be adapted in order 
to be feasible in the space context.
Coverage-based solutions~\cite{zhang2013faster} combine three criteria: (1) the number of times the mutated statement is exercised by the test case (multiple iterations over a same statement are more likely with a diverse set of variable values and thus have higher probability to kill the mutant), (2) the proximity of the mutated statement to the end of the test case (closer ones have higher chances of satisfying the sufficiency condition), and (3) the percentage of mutants belonging to the same class file of the mutated statement that were already killed by the test case (test cases that kill multiple mutants likely exercise the SUT with a diverse set of inputs). 
Criterion (3) is also used to reduce the test suite size (i.e., to select only the test cases above a given percentage threshold). 
Unfortunately, only criterion (1) seems applicable for space software; indeed, criterion (2) may be ineffective with system test cases whose results are checked after long executions, while criterion (3) may be inaccurate when only a random, small subset of mutants is executed. 






%For \INDEX{test case reduction}, the idea is to remove those test cases that are somehow redundant (e.g., test cases that when removed from the test suites do not change the mutation score).
%Usaola et al. \cite{usaola2012reduction} proposed a greedy algorithm that iteratively selects  the test cases that kill most of the mutants that were not killed by the previously selected test cases. 
%%\DONE{No change to do here. However please keep them in mind for the current work.}
%Shi et al. \cite{shi2014balancing} assessed the effects of reducing the size of test suites with an experiment on 18 projects with a total of 261\,235 test cases. Their results show that \emph{it is possible to maintain constant the mutation score and reduce test suite size without loss in the \emph{fault detection rate}}. 
%On the same line, Zhang et al. \cite{zhang2013faster} suggest to define a subset of tests of the original test suite and to run the mutants against the subset, their assumption is that if the mutants cannot be killed by the subset also the original test suite will not be able to kill the mutants.
%
%
%




\subsubsection{Detection of Equivalent Mutants}
\label{sec:background:equivalent}

Although identifying equivalent mutants is an undecidable problem~\cite{madeyski2013overcoming,Bugg:Correctness:82}, several heuristics have been developed to alleviate it. 

The simplest solution consists of relying on \emph{trivial compiler optimisations}~\cite{papadakis2015trivial, kintis2017detecting,papadakis2019mutation}, i.e., compile both the mutants and the original program with compiler optimisations enabled and determine that the mutant is equivalent when their executable code match. In C programs, compiler optimisations can reduce the total number of mutants by 28\%~\cite{kintis2017detecting}.


%Solutions that identify equivalent mutants based on symbolic execution~\cite{holling2016nequivack}, bounded model checking~\cite{riener2011test}, and program slicing ~\cite{harman2001relationship} are unlikely to scale with large systems because of the limitations of static analysis. Also, their implementation often relies on LLVM, which may prevent their applicability to space software (see Section~\ref{sec:compile:time}).

Solutions that identify equivalent mutants based on \emph{static program analysis} (e.g., concolic execution~\cite{holling2016nequivack}, and bounded model checking~\cite{riener2011test}) require the manual identification of the state variables and program outputs that contain the results of the computation performed by the mutated function.
% and thus should differ after the execution of the original and mutated code. 
Such manual analysis is infeasible in the presence of a large set of mutants, most particularly for large systems. 
%Despite some form of automation might be envisioned in the presence of unit test cases, such solution is not applicable when test cases target the whole system. 
Alternatively, reachability analysis might be employed to determine if two mutants are not equivalent according to weak mutation. 
However, weak mutation does not guarantee that the sufficiency condition holds; consequently, the generated results still need to be manually inspected to determine if a difference in the final state of the system is observable. In addition, existing static analysis solutions cannot work with system-level test cases that require hardware and environment simulators.
Indeed, (1) simulation results cannot be predicted by pure static analysis, and (2) concolic execution tools, which rely on LLVM, cannot be run if the SUT executable should be generated with a specific compiler (see Section~\ref{sec:compile:time}).
%that produce results that cannot be fully predicted by means of static analysis.
% and program slicing ~\cite{harman2001relationship} are unlikely to scale with large systems because of the limitations of static analysis. Also, 
 %Furthermore, their implementation often relies on LLVM, which may prevent their applicability to space software (see Section~\ref{sec:compile:time}).

Alternative solutions rely on \emph{dynamic analysis}; more precisely, they compare data collected when testing the original software and the mutants~\cite{grun2009impact,schuler2010covering,schuler2013covering,schuler2009efficient}.
%A large number of program classes presenting different statement coverage in test suites executions with the original and the mutated code is an indicator for  mutants to be non-equivalent~\cite{grun2009impact}.
The most extensive empirical study on the topic shows that non-equivalent mutants can be detected by counting the number of methods (excluding the mutated method) that, for at least one test case, either (1) have statements that are executed at a different frequency with the mutant, (2) generate at least one different return value, or (3) are invoked at a different frequency~\cite{schuler2013covering}. To determine if a mutant is non-equivalent, it is possible to define a threshold indicating the smallest number of methods with such characteristics. A threshold of one identifies non-equivalent mutants with an average precision above 70\% and an average recall above 60\%. This solution outperforms more sophisticated methods relying on dynamic invariants~\cite{schuler2009efficient}. Also, coverage frequency alone leads to results close to the ones achieved by including all the three criteria~\cite{schuler2013covering}.
However, the identification of an optimal threshold value remains an open issue; indeed, values above one slightly improve precision but significantly undermine recall (e.g., recall is below 50\% for a threshold of 5). 
%Also, the threshold value may depend on the size of the test suite. 

Concerning the applicability of dynamic analysis methods to space software, it is worth noting that, because of real-time constraint, it may only be feasible to collect coverage data. 
The  results reported in the literature concern a small number of mutants (i.e., 140) for Java software; 
additional empirical evaluations with C/C++ space software are thus needed.
%In particular, the identification of coverage differences across the whole software might be ineffective when the SUT is exercised with system test cases that may lead to non-deterministic coverage (e.g., because of interrupt handlers). 
Finally, we notice that although code coverage might be effective to determine that mutants are non-equivalent, it might be inappropriate to identify equivalent mutants. Indeed, 
%although differences in code coverage correspond to semantic differences, 
non-equivalent mutants that are exercised with an inappropriate set of inputs may show the same coverage as the original program. For example, the branching condition $(x >= 0)$ has the same coverage as the mutated condition $(x > 0)$ when the value $0$ is never assigned to $x$ in the test cases. 
An empirical assessment of the portion of non-equivalent mutants that may be discarded by coverage-based approaches is thus necessary.

%For this reason, the mutation score computed after ignoring mutants that   might be higher than the real mutation score, which might be dangerous if mutation testing is used to assess the quality of test suites for safety critical software.

\subsubsection{Detection of Redundant Mutants}

Redundant mutants are either \emph{duplicates}, i.e., mutants that are equivalent with each other but not equivalent to the original program, or \emph{subsumed}, i.e., mutants that are not equivalent with each other but are killed by the same test cases. 

Duplicate mutants can be detected by relying on the same approaches adopted for equivalent mutants. 

To identify subsumed mutants, Shin et al. augment the test suite with additional test cases that fail with one mutant only~\cite{Shin:TSE:DCriterion:2018}. 
The augmented test suite has a higher
 fault detection rate than a test suite that simply satisfies mutation coverage; however, with large software systems the approach becomes infeasible because of the lack of scalable test input generation approaches.


\subsubsection{Summary}

We aim to rely on the sufficient set of operators since it has been successfully used to generate a mutation score that accurately estimates the fault detection rate for software written in C and C++, languages commonly used in embedded software.
%Based on recent results, we should however extend the sufficient set with LVR, and all the operators belonging to OODL.
\CHANGED{Further, since recent results have reported on the usefulness of both LVR and OODL operators to support the generation of test suites with high fault revealing power~\cite{Kintis2018}, the sufficient set may be extended to include these two operators as well.}

To speed up mutation testing by reducing the number of mutants, we should consider the SDL operator alone \CHANGED{or in combinaton with the OODL operators}. However, such heuristic should be carefully evaluated to determine the level of confidence we can expect.

Among compile time optimizations, only mutant schemata appear to be feasible with space software.
\REVNOV{C-P-10}{However, based on preliminary evaluation with the case studies for FAQAS, compile time optimizations appear not to be necessary if a smart compilation process is in place. Basically, the compilation routine that we propose (i.e., copy the mutated file inside a compiled source tree) enables fast compilation of mutants.}
Concerning scalability, simple optimizations (i.e., S1, S2, and S3 in Section~\ref{sec:scalability}) are feasible. Other feasible solutions are the ones relying on mutant sampling and coverage metrics. However, the level of confidence one may expect when mutant sampling rates are below 5\% should be evaluated. 
%Furthermore, code coverage metrics that are feasible for space software need to be defined.

Equivalent mutants can be identified through trivial compiler optimizations and the analysis of coverage differences; however, it is necessary to define appropriate coverage metrics and thresholds that maximize both precision and recall. The same approach can be adopted to identify duplicate mutants. The generation of test cases that distinguish subsumed mutants is out of the scope of this work.
\REVNOV{C-P-08}{However, the percentage of subsumed mutants should be reported by the FAQAS framework.}



