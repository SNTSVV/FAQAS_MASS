% !TEX root = MAIN.tex

\chapter{Data-driven Mutation Testing}
\label{chapter:datamutation}

% !TEX root = MutationTestingSurvey.tex

\section{Overview of the Data-driven Mutation Testing Process}
\label{sec:dataProcess}

	\begin{figure}
	\centering
		\includegraphics[width=\textwidth]{images/dataProcess}
		\caption{Data-driven Mutation Testing Process.}
		\label{fig:data:process}
	\end{figure}



This Chapter defines a test suite assessment process based on the injection of faults in the data processed by software components; we refer to this process as \INDEX{data-driven mutation testing}. 
%The definition of data-driven mutation testing is a unique contribution of this book; it has not been presented in previous literature work.

Data-driven mutation testing aims to assess test suites by simulating faults that affect the data produced, received, or exchanged by the software and its components.
It is based on a \INDEX{fault model} capturing the type of data faults that might affect the system. The fault model is produced by software engineers based on their domain knowledge and experience~\cite{di2015generating}. The considered faults might be due to programming errors, hardware problems, or critical situations in the environment (e.g., noise in the channel). The data is then automatically  mutated (i.e., modified) by a set of operators that aim to replicate the faults in the fault model. For example, the \INDEX{bit flip operator} flips a randomly selected bit of a field of the transmitted data (see Section~\ref{sec:faultModel}). 
%Mutation operators can be applied multiple times, on different data chunks or over repeated executions of a test case, ti


Figure \ref{fig:data:process} shows the reference \INDEX{data-driven mutation testing process} that will be considered in FAQAS. The process is based on two main sub-processes, \INDEX{test suite evaluation} and \INDEX{test suite augmentation}, which are described in Sections~\ref{sec:data:test_suite_evaluation}~and~\ref{sec:data:test_suite_augmentation}, respectively. Differently from the code-driven mutation testing process introduced in Section~\ref{sec:process}, the data-driven mutation testing process has not been formalized by existing software testing literature. An extensive discussion of related work has been presented in deliverable D1.

\MREVISION{C-P-29}{The type of faults that might be simulated by data-driven mutation testing are \INDEX{CPU faults}, \INDEX{memory faults}, \INDEX{data processing faults}, and \INDEX{communication faults}. 
However, it is worth noting that these faults are the means to perform mutation testing (i.e., evaluate test suites), they are not the purpose of data-driven mutation testing. Data-driven mutation testing does not aim at simulating such faults to determine if the system is robust but it aims to determine if, in the presence of such faults, the test suite fails as it would be expected. The choice of the faults to simulate depends on the type of test suite to evaluate.}

\CHANGED{In FAQAS we focus on evaluating the quality of \INDEX{functional test suites}. In the following, we briefly discuss the reasons why we do not evaluate \INDEX{robustness testing} test suites, which, among the different types of test suites, is the closest to mutation testing. Indeed,  it is often performed through mutation.
Space systems are expected to be robust with respect to CPU faults and memory faults; for this reasons such faults are often the means for performing robustness testing. Robustness testing is often performed by relying on ad-hoc fault injection systems (e.g., by corrupting memory through debugger features), which may require the test cases to be manual performed. For these reasons evaluating the quality of robustness test cases with an automated and generic strategy appear infeasible.}

%\CHANGED{To perform data-driven mutation testing of functional test suites all the different types of faults might be considered (e.g., CPU faults, memory faults, data processing faults, and communication faults). However, CPU faults and memory faults simply result into erroneous results, which are simulated by replacing valid values or performing bit flips.}

\CHANGED{With data-driven mutation testing we aim to simulate higher-level problems that typically affect complex data structures. The main reason is that other types of problems (e.g., computation of erroneous results) are already targeted by code-driven mutation. For this reason, we aim to perform data-driven mutation testing at the boundary of software components.
An ideal target for data-driven mutation testing are loosely coupled software components; typically the ones that run on separated pieces of hardware (e.g., the on board controller and the ADCS).
Despite other cases can be envisaged (e.g., pure software components that run on the same hardware), we target components running on separated hardware since we believe they are more likely affected by problems due to an incorrect understanding of requirements specifications or error-handling (e.g., because developed by separate teams).}
%Since data-driven mutation testing alters the data produced, received, or exchanged by the software or its components, 

\CHANGED{For this reason, in FAQAS, we apply data-driven mutation testing to evaluate test suites that trigger the execution and communication between multiple components (e.g., system or integration test cases). In FAQAS, data-driven mutation testing is not meant to be applied to assess unit test suites.}

\clearpage

\renewcommand\APPR{\emph{DAMAt}\xspace}

\section{Test Suite Evaluation} % (fold)
\label{sec:data:test_suite_evaluation}

\STARTCHANGEDWPT

% !TEX root = MAIN.tex

\chapter{Introduction}

The purpose of this  Software Validation Specification is to describe the testing, analysis, inspection and
review of design specifications, and is used to document the software validation specification concerning the requirements baseline.

\chapter{Applicable and reference documents}

\begin{itemize}
\item{D1 - Mutation testing survey}
\item{D2 - Study of mutation testing applicability to space software}
\item{D4 - Validation of the toolset}
\item{SSS - Software System Specification}
\item{SUM - Software User Manual}
\item{SUTP - Software Unit Test Plan}
\end{itemize}

\chapter{Terms, definitions, and abbreviated terms}

\begin{itemize}
\item{FAQAS}: activity ITT-1-9873-ESA
\item{FAQAS-framework}: software system to be released at the end of WP4 of FAQAS
\item{D2}: Deliverable D2 of FAQAS, \emph{Study of mutation testing applicability to space software}
\item{D4}: Deliverable D4 of FAQAS, \emph{Validation of the toolset}
\item{SSS}: Software System Specification.
\item{KLEE}: Third-party test generation tool, details are provided in D2.
\item{MLFS}: Mathematical Library for Flight Software.
\item{SUT}: Software under test, i.e, the software that should be mutated employing mutation testing.
\item{SUM}: Software User Manual.
\item{SUTP}: Software Unit Test Plan
\item{WP}: Work package.
\end{itemize}

\chapter{Software Overview}

This documents concern the components of the FAQAS framework:
\begin{itemize}
  \item MASS
  \item \DAMA
  \item SEMuS
%  \item DAMTE
\end{itemize}

For detailed information on their structure and usage see D2 and SUM.

\chapter{Software validation specification task identification}

\STARTCHANGEDFINAL

\section{Task and criteria}
\label{sec:taskCrit}

The validation tasks are meant to ensure that the requirement expressed in the SSS are met.
They are accomplished by following validation approaches:
\begin{itemize}
  \item veryfing the requirements for the most critical and complex software components througt \EMPH{unit testing} as detailed in the SUTP.
  \emph{unit testing} is composed of three tasks:
  \begin{itemize}
    \item performing \emph{\DAMA-TD-DDMutation-1}
    \item performing \emph{\DAMA-TD-DDMutation-2}
    \item performing \emph{\MASS-TD-SRCMutation-1}
    \item performing \emph{SEMuS-TD-TGMutation-1}
  \end{itemize}
  \item verifying the requirements for the rest of the components through \EMPH{application to the case studies} as detailed in the D4 and the SUM.
  Application to the case studies is composed of the following tasks, detailed in \ref{sec:case_studies}
  \begin{itemize}
    \item \emph{configuring \MASS and running Launcher.sh - \MASS}
    \item \emph{configuring \MASS and running PrepareSUT.sh - \MASS}
    \item \emph{configuring \MASS and running GenerateMutants.sh - \MASS}
    \item \emph{configuring \MASS and running CompileOptimizedMutants.sh - \MASS}
    \item \emph{configuring \MASS and running OptimizedPostProcessing.sh - \MASS}
    \item \emph{configuring \MASS and running GeneratePTS.sh - \MASS}
    \item \emph{configuring generate\_template\_config.json and running generate\_direct.py - SEMuS}
    \item \emph{manually editing of the generated test templates - SEMuS} 
    \item \emph{instrumenting the source code - \DAMA}
    \item \emph{configuring and running} a given component, for example \texttt{\DAMA\_compile.sh}.
  \end{itemize}
  \item verifying general requirements through \EMPH{manual inspection}. For this approach no further specific task can be identified.
\end{itemize}



\section{Application to the case studies}
\label{sec:case_studies}

In the following sections the validation task that fall under the \EMPH{application to the case studies} category are described, along with their pass-fail criteria.


\subsection{Configuring \MASS and running Launcher.sh - \MASS}
\label{sec:configuring:mass}

The objective of this task is to validate that \MASS can properly prepare the SUT, collect code coverage information, and correctly apply the eight steps of the methodology for assessing the quality of the SUT test suite.

% !TEX root =  ../MAIN.tex

\begin{table}[tb]
\footnotesize
\centering
\caption{\MASS minimal set of parameters to be configured.}
\label{table:to_configure}
\begin{tabular}{llp{7.5cm}}
\hline
\textbf{Script Name}  & \textbf{Parameter} &  \textbf{Description} \\
\hline
& BUILD\_SYSTEM &  Specifies the building system type.\\
& PROJ &  Path of the SUT root  directory.\\
& PROJ\_SRC &  Path of the SUT source directory.\\
& PROJ\_TST &  Path of the SUT test  directory.\\
& PROJ\_COV &  Path of the directory with SUT coverage information.\\
& PROJ\_BUILD &  Path of the directory where the compiled binary is stored.\\
& ORIGINAL\_MAKEFILE &  Path to the original build script.\\
& COMPILATION\_CMD &  Compilation command of the SUT.\\
& TCE\_COMPILE\_CMD &  Compilation command for TCE analysis.\\
& PRIORITIZED &  Specifies whether \MASS should be executed with a prioritized test suite.\\
& SAMPLING &  Specifies the mutant sampling technique.\\
\hline
PrepareSUT.sh & None &  Commands shall be provided manually.\\
\hline
mutation\_additional\_functions.sh & run\_tst\_case  &  Implementation of the Bash function run\_tst\_case that executes the test case passed as a parameter.\\
\hline
\end{tabular}
\end{table}

The files and variables to be configured for \MASS are specified in Table~\ref{table:to_configure}. As detailed in the SUM document, these files (i.e., mass\_conf.sh, PrepareSUT.sh, mutation\_additional\_functions.sh) support \MASS to correctly process the paths of the SUT, SUT compilation commands, SUT test suite execution commands, and enable the mutation analysis process.

After configuring the three files, the following command shall be executed for launching the mutation analysis process. 

\begin{lstlisting}[language=bash]
  $ ./Launcher.sh
\end{lstlisting}

Note that this command (1) shall be executed from \MASS workspace, (2) shall automatically perform the following eight steps:

\begin{enumerate}
  \item PrepareSUT
  \item GenerateMutants
  \item CompileOptimizedMutants
  \item OptimizedPostProcessing
  \item GeneratePTS
  \item ExecuteMutants
  \item IdentifyEquivalents
  \item MutationScore
\end{enumerate}

A typical \MASS output is presented in Listing~\ref{mass:output}.

\begin{lstlisting}[language=bash, label=mass:output, caption=\MASS output.]
##### MASS Output #####
## Total mutants generated: 28071
## Total mutants filtered by TCE: 6918
## Sampling type: fsci
## Total mutants analyzed: 461
## Total killed mutants: 369
## Total live mutants: 92
## Total likely equivalent mutants: 53
## MASS mutation score (%): 90.44
## List A of useful undetected mutants: /opt/MLFS/RESULTS/useful_list_a
## List B of useful undetected mutants: /opt/MLFS/RESULTS/useful_list_b
## Number of statements covered: 1973
## Statement coverage (%): 100
## Minimum lines covered per source file: 2
## Maximum lines covered per source file: 138
\end{lstlisting}

\subsubsection{Pass-Fail criteria}

The following criteria must be respected in order to declare this task a pass:

\begin{itemize}
  \item The eight steps of the methodology shall be performed. In turn, the following shall be produced for each steps:
    \begin{itemize}
      \item PrepareSUT: this step shall produce code coverage files, and list of test cases to be executed at \texttt{COV\_FILES} folder.
      \item GenerateMutants: this step shall produce mutant files at \texttt{src-mutants} folder.
      \item CompileOptimizedMutants: inside \texttt{COMPILED} there should be one folder for each optimization level containing (1) list of SHA512 hashes for every compiled mutant, (2) list of non-compiled mutants.
      \item OptimizedPostProcessing: inside \texttt{COMPILED} there should be (1) list of equivalent mutants, (2) list of redundant mutants, (3) list of unique mutants.
      \item GeneratePTS: this step shall produce reduced and prioritized test suites for the SUT at folder \texttt{PRIORITIZED}.
      \item ExecuteMutants: inside \texttt{MUTATION} folder there should be: (1) code coverage for each mutant, (2) traces of killed and live mutants, (3) list of executed mutants.
      \item IdentifyEquivalents: inside folder \texttt{DETECTION} there should be: (1) list of equivalent mutants, (2) mutation traces without equivalent mutants.
      \item MutationScore: inside folder \texttt{RESULTS} there should be: (1) final MASS report, (2) list of killed and live mutants, (3) code coverage report, (4) list of useful mutants. 
    \end{itemize}
  \item A final report shall be reported at the end of the mutation analysis process.
  \item The MASS log shall report no error.
\end{itemize}

\subsection{Configuring \MASS and running PrepareSUT.sh - \MASS}

This validation task consists of configuring \MASS, as specified in the Section~\ref{sec:configuring:mass}, and then executing the PrepareSUT.sh command of \MASS, this command prepare the SUT and collect information about the SUT test suite. Note that this script shall be prepared by the SUT engineer. 
The objective of this task is to validate that \MASS can properly prepare the SUT and collect code coverage information about the SUT test suite, which is necessary for enabling \MASS optimizations.

\begin{lstlisting}[language=bash]
  $ ./PrepareSUT.sh
\end{lstlisting}

The command shall create a \texttt{COV\_FILES} folder inside the \MASS workspace; such folder will contain GCOV files for each test case, and a list of test cases to be executed.

\subsubsection{Pass-Fail criteria}

\begin{itemize}
  \item The execution of PrepareSUT shall be successful, that is, the script shall terminate with a 0 error code.
  \item The required output shall be observed inside \texttt{COV\_FILES} folder.
\end{itemize}

\subsection{Configuring \MASS and running GenerateMutants.sh - \MASS}

This validation task consists of configuring \MASS, as specified in the Section~\ref{sec:configuring:mass}, and then executing the GenerateMutants.sh command of \MASS, this command generates the code coverage matrices and launches the generation of mutants. A prerequisite to this task is to have executed the PrepareSUT command.
The objective of this task is to validate that \MASS can properly process code coverage information and generate code-driven mutants of the SUT.

\begin{lstlisting}[language=bash]
  $ ./GenerateMutants.sh
\end{lstlisting}

The command shall create a \texttt{src-mutants} folder inside the \MASS workspace; such folder will contain one folder for each source under test, and each of this folder shall contain a set of mutants for the source code.

A typical mutant source file will be named using (1) the name of the source file, (2) the mutated line, (3) the instance of the mutation (e.g., the AOR operator can replace a $+$ operator into four instances (a) $-$, (b) $*$, (c) $/$, and (d) $\%$) and the position in the source file, (4) the operator being used, (5) the affected function. An example of mutant would be \url{source.mut.126.1_6_19.ICR.function_1.c}

\subsubsection{Pass-Fail criteria}

\begin{itemize}
  \item The execution of GenerateMutants shall be successful, that is, the script shall terminate with a 0 error code.
  \item The required output shall be observed inside \texttt{src-mutants} folder.
\end{itemize}

\subsection{Configuring \MASS and running CompileOptimizedMutants.sh - \MASS}

This validation task consists of configuring \MASS, as specified in the Section~\ref{sec:configuring:mass}, and then executing the CompileOptimizedMutants.sh command of \MASS, this command compiles the mutants with multiple optimisation levels and generate intermediate files to be processed then by OptimizedPostProcessing. A prerequisite to this task is to have executed the GenerateMutants command.
The objective of this task is to validate that \MASS can properly compile the mutants and generate a list of SHA512 hashes for each mutant, in order to properly identify equivalent and redundant mutants during the OptimizedPostProcessing step.

\begin{lstlisting}[language=bash]
  $ ./CompileOptimizedMutants.sh
\end{lstlisting}

The command shall create a \texttt{COMPILED} folder inside the \MASS workspace; such folder will contain one folder for each optimization level containing (1) list of SHA512 hashes for every compiled mutant, (2) list of non-compiled mutants.

\subsubsection{Pass-Fail criteria}

\begin{itemize}
  \item The execution of CompileOptimizedMutants shall be successful, that is, the script shall terminate with a 0 error code.
  \item The required output shall be observed inside \texttt{COMPILED} folder.
\end{itemize}

\subsection{Configuring \MASS and running OptimizedPostProcessing.sh - \MASS}

This validation task consists of configuring \MASS, as specified in the Section~\ref{sec:configuring:mass}, and then executing the CompileOptimizedMutants.sh command of \MASS, this command iterates over the SHA512 hashes generated by the CompileOptimizedMutants command, and produce a list of unique mutants. A prerequisite to this task is to have executed the CompileOptimizedMutants command.
The objective of this task is to validate that \MASS can properly identify equivalent and redundant mutants to be then analyzed during ExecuteMutants.

\begin{lstlisting}[language=bash]
  $ ./OptimizedPostProcessing.sh
\end{lstlisting}

The command shall create (1) a list of equivalent mutants, (2) a list of redundant mutants, and (3) a list of unique mutants inside the \texttt{COMPILED} folder within the \MASS workspace.

\subsubsection{Pass-Fail criteria}

\begin{itemize}
  \item The execution of OptimizedPostProcessing shall be successful, that is, the script shall terminate with a 0 error code.
  \item The required output shall be observed inside \texttt{COMPILED} folder.
\end{itemize}

\subsection{Configuring \MASS and running GeneratePTS.sh - \MASS}

This validation task consists of configuring \MASS, as specified in the Section~\ref{sec:configuring:mass}, and then executing the CompileOptimizedMutants.sh command of \MASS, this command generates a reduced and prioritized test suite for the SUT. A prerequisite to this task is to have executed the OptimizedPostProcessing command.
The objective of this task is to validate that \MASS can properly generate a reduced and prioritized set of test cases to be used during the ExecuteMutants step.

\begin{lstlisting}[language=bash]
  $ ./GeneratePTS.sh
\end{lstlisting}

The command shall create two files, (1) a reduced, and (2) prioritized test suite for the SUT, at folder \texttt{PRIORITIZED}.

\subsubsection{Pass-Fail criteria}

\begin{itemize}
  \item The execution of GeneratePTS shall be successful, that is, the script shall terminate with a 0 error code.
  \item The required output shall be observed inside \texttt{PRIORITIZED} folder.
\end{itemize}



\subsection{Configuring generate\_template\_config.json and running generate\_direct.py - SEMuS}

% !TEX root =  ../MAIN.tex

\begin{table}[t]
\tiny
\centering
\caption{SEMuS parameters to be configured.}
\label{table:to_conf_semus}
\begin{tabular}{lp{8.5cm}}
\hline
\textbf{Parameter}  &  \textbf{Description} \\
\hline
FAQAS\_SEMU\_CASE\_STUDY\_TOPDIR &  Root folder of the case study \\
FAQAS\_SEMU\_CASE\_STUDY\_WORKSPACE &  SEMuS workspace for the case study \\
FAQAS\_SEMU\_OUTPUT\_TOPDIR & SEMuS output folder, to be placed inside the workspace \\
FAQAS\_SEMU\_GENERATED\_MUTANTS\_TOPDIR & Root folder for storing the generated mutants \\
FAQAS\_SEMU\_REPO\_ROOTDIR &  Root folder of the case study source code\\
FAQAS\_SEMU\_ORIGINAL\_SOURCE\_FILE & Path of the source file under analysis \\
FAQAS\_SEMU\_COMPILE\_COMMAND\_SPECIFIED\_SOURCE\_FILE & Name of the source file under analysis \\
FAQAS\_SEMU\_GENERATED\_MUTANTS\_DIR & Folder for storing the generated mutants for the specified source file\\
FAQAS\_SEMU\_BUILD\_CODE\_FUNC\_STR & Bash function for building the source file under analysis, to be specified in string format\\
FAQAS\_SEMU\_BUILD\_LLVM\_BC & Bash function for building the source file to LLVM bitcode \\
FAQAS\_SEMU\_META\_MU\_TOPDIR &  Root folder for the meta mutant \\
FAQAS\_SEMU\_GENERATED\_META\_MU\_SRC\_FILE &  Path of the source file (i.e., C file) of the meta mutant \\
FAQAS\_SEMU\_GENERATED\_META\_MU\_BC\_FILE &  Path of the source file (i.e., LLVM bitcode file) of the meta mutant \\
FAQAS\_SEMU\_GENERATED\_META\_MU\_MAKE\_SYM\_TOP\_DIR &  Folder for storing intermediate files for the generated inputs \\
FAQAS\_SEMU\_GENERATED\_TESTS\_TOPDIR &  Folder for storing the generated inputs \\
FAQAS\_SEMU\_TEST\_GEN\_TIMEOUT & Timeout in seconds for the test generation process \\
FAQAS\_SEMU\_HEURISTICS\_CONFIG & Configuration array for SEMu heuristics \\
FAQAS\_SEMU\_TEST\_GEN\_MAX\_MEMORY & Maximum test generation memory in MB \\
FAQAS\_SEMU\_STOP\_TG\_ON\_MEMORY\_LIMIT & Parameter to stop test generation when the memory limit is reached \\
FAQAS\_SEMU\_TG\_MAX\_MEMORY\_INHIBIT & Parameter to stop forking states when the memory limit is reached \\
\hline
\end{tabular}
\end{table}



% !TEX root =  ../MAIN.tex

\begin{table}[t]
\tiny
\centering
\caption{Test template generator parameters to be configured.}
\label{table:ttg_semus}
\begin{tabular}{lp{8.5cm}}
\hline
\textbf{Parameter}  &  \textbf{Description} \\
\hline
TYPES\_TO\_INTCONVERT &  Specify how to convert a type to int. \\
TYPES\_TO\_PRINTCODE &  Specify how to print a type.  \\
OUT\_ARGS\_NAMES & Specify the names of function arguments that are used as function output. \\
IN\_OUT\_ARGS\_NAMES & Specify the names of function arguments that are used both as function input and output \\
TYPE\_TO\_INITIALIZATIONCODE &  Specify the initialization statement of a type.\\
VOID\_ARG\_SUBSTITUTE\_TYPE & Specify the underlying type for a void pointer. \\
TYPE\_TO\_SYMBOLIC\_FIELDS\_ACCESS & Specify, for pointer parameters, the number of elements it points to. \\
\hline
\end{tabular}
\end{table}




The objective of this task is to validate that \SEMUS can properly generate test templates that can be processed by the underlying test generation tool (i.e., KLEE).

The files and variables to be configured for \SEMUS are specified in Tables~\ref{table:to_conf_semus} and~\ref{table:ttg_semus}. As detailed in the SUM document, these files (i.e., generate\_template\_config.json and running generate\_direct.py) support \SEMUS to correctly process the paths of the SUT, SUT compilation commands, the configuration of \SEMUS itself, and how to print the values of the functions under test, so \SEMUS can determine if a mutant has been killed.

After configuring the two files, the following command shall be executed for launching the test template generation. 

\begin{lstlisting}[language=bash]
 $ case_studies/$SUT/util_codes/generate_direct.py ../WORKSPACE/DOWNLOADED/casestudy/test.c direct \
                    " -I../WORKSPACE/DOWNLOADED/casestudy/" -c generate_template_config.json
\end{lstlisting}

Note that this command shall generate inside the directory \texttt{case\_studies/\$SUT/util\_codes} one folder for each source under analysis, and inside of these folders, one template for each function under test.

\subsubsection{Pass-Fail criteria}

\begin{itemize}
  \item The test templates shall be generated.
  \item The generate\_direct.py shall terminate with a 0 error code.
\end{itemize}

\subsection{Instrumenting the source code - \DAMA}
\label{subsec:instrumenting}

The variables for running the \DAMA pipeline on ESAIL must be set in the \emph{\DAMA\_configure.sh} file, as reported in Listing~\ref{lst:configure_esail}. The significance of these variables is described in the SUM.

Then the commands represented in Listing~\ref{lst:instrument_esail_cmds} must be run to generate the mutation probes.

\begin{lstlisting}[language=bash, label={lst:instrument_esail_cmds}]
bash DAMA_probe_generation.sh
\end{lstlisting}

The generated probes must be inserted in the target function, as reported in the SUM (Chapter 13).
The Software Under Test (LibParam or ESAIL) shall be compiled with the macro \texttt{-DMUTATIONOPT=-1} enabled.

\subsubsection{Pass-Fail criteria}

The following criteria must be respected in order to declare this task a pass:
\begin{itemize}
  \item The compilation shall be successful.
  \item The compilation log shall report no error.
\end{itemize}

\subsection{Configuring and running \emph{\DAMA\_compile.sh}}

A prerequisite for this task is having successfully performed \EMPH{Instrumenting the source code - \DAMA}, as described in Section~\ref{subsec:instrumenting}.

\subsubsection{ESAIL}

The commands for compiling the SVF, reported in Listing~\ref{lst:compile_esail} must be included in the \emph{\DAMA\_compile.sh} script in the appropriate section as described in the SUM (Chapter 13).


  \begin{lstlisting}[language=bash, label={lst:compile_esail}]

  compilation_folder="/home/svf/Svf"

  pushd $compilation_folder

  make install-debug

      if [ $? -eq 0 ]; then
          echo $x " compilation OK"
      else
          echo $x " compilation FAILED"
      fi

  popd

  \end{lstlisting}

Then the commands represented in Listing~\ref{lst:compile_esail_cmds} must be run.

  \begin{lstlisting}[language=bash, label={lst:compile_esail_cmds}]

  bash \DAMA_compile.sh "0" "TRUE"

  \end{lstlisting}

\subsubsection{LibParam}

The commands for configuring the LibParam test suite, reported in Listing~\ref{lst:compile_param} must be included in the \emph{\DAMA\_compile.sh} script in the appropriate section as described in the SUM (Chapter 13


\begin{lstlisting}[language=bash, label={lst:compile_param}]

TEST_FOLDER="/home/csp/libparam/tst"

pushd $TEST_FOLDER

for f in *; do
    if [ -d "$f" ] && [ "$f" != "include" ]; then

        pushd $f
        echo "cleaning..."
        ./waf clean

        echo "configuring..."
        if [ $singleton == "TRUE" ]; then
        ./waf configure --mutation-opt $mutant_id --singleton $singleton
        else
        ./waf configure --mutation-opt $mutant_id
        fi

        if [ $? -eq 0 ]; then
            echo $x " configuration OK"
        else
            echo $x " configuration FAILED"
        fi
        popd
    fi
done

popd

\end{lstlisting}

Then the commands represented in Listing~\ref{lst:compile_param_cmds} must be run.

  \begin{lstlisting}[language=bash, label={lst:compile_param_cmds}]

  bash \DAMA_compile.sh "0" "TRUE"

  \end{lstlisting}

\subsubsection{Pass-Fail criteria}

The following criteria must be respected in order to declare this task a pass:
\begin{itemize}
  \item The compilation log shall report no error.
  \item The compilation log shall report the phrase \texttt{compilation OK}.
\end{itemize}


\subsection{Configuring and running \emph{\DAMA\_run\_test.sh}}

A prerequisite for this task is having successfully performed \EMPH{Instrumenting the source code - \DAMA}, as described in Section~\ref{subsec:instrumenting}.

\subsubsection{ESAIL}

The commands for running ESAIL's test suite, reported in Listing~\ref{lst:run_esail} must be included in the \emph{\DAMA\_run\_test.sh} script in the appropriate section as described in the SUM (Chapter 13).


  \begin{lstlisting}[language=bash, label={lst:run_esail}]

  ESAIL=/home/svf/Obsw/Test/lib/esail.sh
  PARSE_RESULTS=/home/svf/Obsw/Test/lib/parse_results.sh
  echo -n "${mutant_id};COMPILED;${tst};" >> $results_file

  timeout $TIMEOUT $ESAIL --obsw /home/svf/Obsw/Source/_binaries/OBSW.exe --fast -n -c --source /home/svf/Obsw/Source --version 04010000 -t $tst &
  ESAIL_PID=$!

  wait $ESAIL_PID
  EXEC_RET_CODE=$?

  mutant_end_time=$(($(date +%s%N)/1000000))
  mutant_elapsed="$(($mutant_end_time-$mutant_start_time))"

  \end{lstlisting}

Then the commands represented in Listing~\ref{lst:run_esail_cmds} must be run.

  \begin{lstlisting}[language=bash, label={lst:run_esail_cmds}]

  bash \DAMA_compile.sh "-1" "TRUE" && bash \DAMA_run_tests.sh "-1" "./test_list.csv" "./"

  \end{lstlisting}

The file \texttt{test\_list.csv} must contain the list of all test cases, as exposed in Listings~\ref{lst:test_esail}.

  \begin{lstlisting}[label={lst:test_esail}]

  731;46287
  6343;6161251
  6000;535195
  612;50098
  791;27125
  767;67364
  743;71586
  683;66213
  1841;102287
  3045;60261
  2561;164595
  719;74033
  5702;280961
  2819;203036
  695;62058
  707;44825
  1222;94890
  2435;134397
  755;53084
  779;156284
  543;71224
  6616;35423
  524;32621
  561;27344
  497;31588
  943;25435
  3394;63879
  5176;30804
  5656;34471
  5218;38539
  4361;63238
  2733;127361
  6161;562902
  6262;100839
  3077;83077
  6117;61088
  2848;421200
  4497;156531
  3590;190508
  2985;265498
  5917;127059
  3004;101740
  3917;200336
  2890;522174
  3034;507824
  5374;276736
  2867;441252
  2753;187527
  2928;311267
  3023;79797
  3610;1283052
  2947;995408
  2909;290593
  2966;133739
  3651;293401
  3853;132605
  6138;760249
  3621;46376
  6600;51881
  5190;196980
  4294;46522
  4270;35237
  6583;144017
  6275;143658
  5393;47192
  2604;45503
  2411;83456
  3950;39675
  146;26984
  140;23420
  3969;39893
  113;28140
  74;26170
  154;22556
  126;24476
  1465;68582
  1377;633878
  1364;47647
  2327;254152
  2339;83202
  2395;284150
  6630;228215
  2254;36753
  6362;102271
  2277;53371
  5570;32536
  160;27916
  659;98164
  6315;25372
  1242;645102
  647;27085
  225;27339
  6401;58932
  1807;56712
  4931;46262
  6038;63676
  1505;2975291
  671;84703
  990;34764
  2007;38421
  1702;86591
  5791;90984
  3996;28060
  3358;113929
  5606;31639
  1545;28645
  1572;135209
  1742;26743
  3340;136863
  197;98851
  2375;50526
  4815;74190
  1293;27614
  3313;66150
  5645;26901
  3376;1068552
  3291;584423
  2093;189396
  953;63629
  6330;25552
  6736;2446525

  \end{lstlisting}

\subsubsection{LibParam}

The commands for running the LibParam test suite, reported in Listing~\ref{lst:run_param} must be included in the \emph{\DAMA\_run\_test.sh} script in the appropriate section as described in the SUM (Chapter 13).


  \begin{lstlisting}[language=bash, label={lst:run_param}]

  tmp_log="$results_dir"/tmp_log

    TEST_FOLDER="/home/csp/libparam/tst"

    pushd $TEST_FOLDER

    pushd $tst

    touch $tmp_log

    timeout $TIMEOUT ./waf --mutation-opt=$mutant_id  --singleton=TRUE 2>&1 | tee $tmp_log

    EXEC_RET_CODE=$?

    mutant_end_time=$(($(date +%s%N)/1000000))
    mutant_elapsed="$(($mutant_end_time-$mutant_start_time))"

    if [ $EXEC_RET_CODE -ne 124 ]; then
        if grep "successfully" $tmp_log
        then
            EXEC_RET_CODE=0
            echo "PASSED"
        else
            EXEC_RET_CODE=1
            echo "FAILED"
        fi
    fi
    popd
    rm $tmp_log
    popd

  \end{lstlisting}

Then the commands represented in Listing~\ref{lst:run_param_cmds} must be run.

% mutant_id=$1
% tests_list=$2
% \DAMA_FOLDER=$3

  \begin{lstlisting}[language=bash, label={lst:run_param_cmds}]

  bash \DAMA_compile.sh "-1" "TRUE" && bash \DAMA_run_tests.sh "-1" "./test_list.csv" "./"

  \end{lstlisting}

The file \texttt{test\_list.csv} must contain the list of all test cases, as exposed in Listings~\ref{lst:test_param}.

  \begin{lstlisting}[label={lst:test_param}]

  bindings,4427.200000
  example,4145.400000
  file_store,2954.000000
  i2c,2793.400000
  log,2881.200000
  param3,3007.800000
  param4,3273.600000
  rparam3,5099.400000
  rparam4,8003.200000
  serialize,2656.400000
  spi,2825.800000
  store,2774.400000
  store_load,3182.000000
  vmem_store_checksum_first,3005.800000
  vmem_store_checksum_last,2963.400000

  \end{lstlisting}

  \subsubsection{Pass-Fail criteria}

  The following criteria must be respected in order to declare this task a pass:

  \begin{itemize}
    \item All test cases in Listing~\ref{lst:test_param} (for LibParam) and Listing~\ref{lst:test_esail} (for ESAIL) shall be executed.
    \item All test cases listed in Listing~\ref{lst:test_param} (for LibParam) and Listing~\ref{lst:test_esail} (for ESAIL) shall pass.
  \end{itemize}

\subsection{Configuring and running \emph{\DAMA\_obtain\_coverage.sh}}

A prerequisite for this task is having successfully performed \EMPH{Instrumenting the source code - \DAMA}, as described in Section~\ref{subsec:instrumenting}.

\subsubsection{ESAIL}

The variables for running the \DAMA-pipeline on ESAIL must be set in the \emph{\DAMA\_configure.sh} file, as reported in Listing~\ref{lst:configure_esail}. The significance of these variables is described in the SUM (Chapter 13).

  \begin{lstlisting}[language=bash, label={lst:configure_esail}]

  # the location of the csv with all the test identifiers and the execution time
  tests_list=$\DAMA_FOLDER/tests.csv

  # the location of the csv containing the definitions of the mutation operators
  fault_model=$\DAMA_FOLDER/fault_model.csv

  # the datatype of the elements of the target buffer
  buffer_type="unsigned long long int"

  # padding: can be used to skip the first n bit of a buffer, normally set to 0
  padding=0

  # singleton: can set to true to load the fault model into a singleton   variable, normally set to "TRUE", can also  be set to "FALSE"
  singleton="TRUE"

  \end{lstlisting}

Then the commands represented in Listing~\ref{lst:coverage_esail_cmds} must be run.

  \begin{lstlisting}[language=bash, label={lst:coverage_esail_cmds}]

  bash \DAMA_obtain_coverage.sh

  \end{lstlisting}

\subsubsection{LibParam}

The variables for running the \DAMA-pipeline on ESAIL must be set in the \emph{\DAMA\_configure.sh} file, as reported in Listing~\ref{lst:configure_param}. The significance of these variables is described in the SUM (Chapter 13).

  \begin{lstlisting}[language=bash, label={lst:configure_param}]

  # the location of the csv with all the test identifiers and the execution time
  tests_list=$\DAMA_FOLDER/tests_param.csv

  # the location of the csv containing the definitions of the mutation operators
  fault_model=$\DAMA_FOLDER/LIBP-FM.csv

  # the datatype of the elements of the target buffer
  buffer_type="unsigned long long int"

  # padding: can be used to skip the first n bit of a buffer, normally set to 0
  padding=0

  # singleton: can set to true to load the fault model into a singleton variable, normally set to "TRUE", can also  be set to "FALSE"
  singleton="TRUE"

  \end{lstlisting}

Then the commands represented in Listing~\ref{lst:coverage_param_cmds} must be run.

  \begin{lstlisting}[language=bash, label={lst:coverage_param_cmds}]

  bash pipeline_scripts/\DAMA_obtain_coverage.sh ./

  \end{lstlisting}

\subsubsection{Pass-Fail criteria}

These criteria must be respected to consider this task a pass:
\begin{itemize}
  \item The compilation lof shall show no error.
  \item The compilation log shall report the phrase \texttt{compilation OK}.
  \item No test cases shall fail.
  \item A \texttt{testlist} folder shall be generated. Inside this folder there shall be a file called \texttt{test\_<mutant>} for every mutant.
  \item Every file shall contain a subset of the tests listed in Listing~\ref{lst:test_param} (for LibParam) or in Listing~\ref{lst:test_esail} (for ESAIL).
\end{itemize}




\subsection{Configuring and running \emph{\DAMA\_mutants\_launcher.sh}}

A prerequisite for this task is having successfully performed \EMPH{Instrumenting the source code - \DAMA}, as described in Section~\ref{subsec:instrumenting}.

\subsubsection{ESAIL}

The variables for running the \DAMA-pipeline on ESAIL must be set in the \emph{\DAMA\_configure.sh} file, as reported in Listing~\ref{lst:configure_esail}. The significance of these variables is described in the SUM (Chapter 13).

Then the commands represented in Listing~\ref{lst:launcher_esail_cmds} must be run.

  \begin{lstlisting}[language=bash, label={lst:launcher_esail_cmds}]

  bash \DAMA_mutants_launcher.sh ./

  \end{lstlisting}

  The compilation logs shall report the phrase \texttt{compilation OK}.
  A \texttt{testlist} folder shall be generated. Inside this folder there shall be a file called \texttt{test\_<mutationID>} for every mutant containing a subset of the tests listed in Listing~\ref{lst:test_esail}.
  A \texttt{results} folder shall be generated and it shall contain the files described in the SUM (Section 9.2).

\subsubsection{LibParam}

The variables for running the \DAMA-pipeline on ESAIL must be set in the \emph{\DAMA\_configure.sh} file, as reported in Listing~\ref{lst:configure_param}. The significance of these variables is described in the SUM (Chapter 13).

Then the commands represented in Listing~\ref{lst:launcher_param_cmds} must be run.

  \begin{lstlisting}[language=bash, label={lst:launcher_param_cmds}]

  bash \DAMA_mutants_launcher.sh ./

  \end{lstlisting}

\subsubsection{Pass-Fail criteria}

The following criteria shall be respected in order to consider this task a pass:

\begin{itemize}
  \item The compilation logs shall report the phrase \texttt{configuration OK}.
  \item A \texttt{testlist} folder shall be generated.
  \item Inside this folder there shall be a file called \texttt{test\_<mutationID>} for every mutant containing a subset of the tests listed in Listing~\ref{lst:test_param} for LibParam or in in Listing~\ref{lst:test_esail} for ESAIL.
  \item A \texttt{results} folder shall be generated and it shall contain the files described in the SUM (Section 9.2).
\end{itemize}

\ENDCHANGEDFINAL

\section{Features to be tested}
The scope of the SVS includes all baseline requirements expressed in the SSS.

% \section{Features not to be tested}
% The SVS w.r.t. TS or RB shall describe all the features and significant
% combinations not to be tested.

\section{Test pass and fail criteria}
The pass-fail criteria for the tasks belonging to the \emph{unit testing} validation approach are detailed in the SUTP.

Regarding the tasks belonging to the \emph{application to the case studies} validation approach, the pass criteria are described in Section~\ref{sec:case_studies}.


% \section{Items that cannot be validated by test}
% a. The SVS w.r.t. TS or RB shall list the tasks and items under tests that
% cannot be validated by a test.
% b. Each of them shall be properly justified
% c. For each of them, an analysis, inspection, or review of design shall be
% proposed.
%
% % \section{Manually and automatically generated code}
% % a. The SVS shall address separately the activities to be performed for
% % manually and automatically generated code, although they have the
% % same objective (ECSS‐Q‐ST‐80 clause 6.2.8.2 and 6.2.8.7).
%
% \chapter{Software validation testing specification design}
%
% \section{General}
% a. The SVS w.r.t. TS or RB shall define software
% validation testing specification design, giving the design grouping
% criteria such as function, component, or equipment management.
% b. For each identified test design, the SVS w.r.t. TS or RB shall provide the
% information given in <6.2>.
%
% \section{Organization of each identified test design}
%
% NOTE The SVS w.r.t. TS or RB defines each validation

\clearpage

\input{damat/related}
% !TEX root = MAIN.tex

\chapter{MASS - Software Unit Testing and Integration Testing Approach}


\section{Unit/Integration Testing Strategy}

Integration testing is out of scope because of the motivations discussed in Section~\ref{sec:SUTSIT:org}.

Unit testing aims to verify that the functional requirements of MASS units are correctly implemented; test inputs are identified through the category-partition method.

%\OSCAR{Possibly integration testing might be the use of \FAQAS with the case studies?}

\section{Tasks and Items under Test}

Testing concerns the source code mutation component (hereafter, \emph{SRCMutation}) of MASS.
SRCMutation is the component with the most complicate implementation logic and thus require detailed unit testing.
All the other components either filter or join data, their implementation is simpler than SRCMutation and thus their test automation is performed through system tests (described in SVS).

\section{Feature to be tested}

Testing concerns verifying the correct implementation of the mutation operators implemented by \emph{SRCMutation}.

\section{Feature not to be tested}

Testing does not concern the verification of the capability of \emph{SRCMutation} to parse source files valid according to C/C++ language grammar. Since \emph{SRCMutation} is implemented on top of CLANG, we assume parsing capabilities are inherited from CLANG.



\section{Test Pass - Fail Criteria}

Unit testing pass if all the following are true:
\begin{itemize}
	\item Every test case is executed
	\item All test cases pass
	\item Exceptions and unexpected messages do not appear on screen and logs.
\end{itemize}



\section{Manually and Automatically Generated Code}

The \FAQAS does not contain any automatically generated code.


\clearpage

\subsection{Data-driven mutation not based on buffers}



\MREVISION{C-P-31}{The communication between loosely coupled software components is performed by relying on APIs of a dedicated communication layer; which is typical in well designed software systems.
The functioning of such communication layer may vary from system to system. The \INDEX{communication layer} works by serializing and deserializing the data that should be transmitted on the communication channel. The goal of serialization/deserialization is to perform a data transformation, i.e, translate the representation of data from the format used inside the program (e.g., a data structure) into a low level format that can be transmitted on the channel (e.g., a stream of bytes or a memory buffer).}

\CHANGED{The differences that we may observe from system to system are related to the input interface of the communication layer. We may observe two cases:}

\CHANGED{\begin{itemize}
\item The communication layer provides serialization (deserialization) primitives that receive (produce) unstructured data (e.g, a memory buffer).
\item The communication layer provides serialization (deserialization) primitives that receive (produce) data structured according to a specific format. 
\end{itemize}}

\CHANGED{In both the two cases, the communication layer performs a \INDEX{data transformation}, i.e, translates the data from the format used inside the program (e.g., a data structure) into a low level format that can be transmitted on the channel (e.g., a memory buffer). However, this commonality does not enable the definition of a single solution to perform data mutation. More precisely, \EMPH{a solution that performs mutation on memory buffers may not be practical in both the two cases}. Indeed, to alter data that is already flattened on a low level representation it is necessary a (potentially complex) data model that describes how to load such data into a more structured representation that should drive the mutation. 
The data modelling effort would thus be redundant in case the data is structured according to a specific format.} 

\CHANGED{To minimize modelling costs, in the presence of a complex data structure, the data model should coincide with the data structure defined in the programming language used to implement the system (or the modelling language from which the program has been derived), while the fault model should be defined as an extension of such data structure (e.g., through annotations).
Modelling of data flattened into a low level representation is feasible only when this is already the input format of the communication layer (in such cases the transmitted data is not expected to follow a complex structure).
Finally, the definition of a generic data loading solution 
that loads data from a memory buffer into a more structured representation and works with any data structure, might be infeasible. 
%structures (e.g, fields of variable size and multiple dependencies among fields), 
%might be particularly expensive.
}

\CHANGED{In FAQAS industrial case studies, the \INDEX{communication layer} is implemented in-house by the company that produced the case study.
The system works by processing a flat structure stored in  a buffer array, which is the reason why the FAQAS \APPR solution focuses on buffers. However, space systems may rely on the ASN.1 compiler architecture; in this case, the data processed by the compiler is highly structured. The definition of the data structure is provided as an ASN.1 grammar that is then translated by the ASN.1 compiler into a C structure. For this reason, \EMPH{in FAQAS, we also design a preliminary solution for highly structured data defined with ASN.1}. Even if the ASN.1 data is translated into a C structure, we believe that a generic solution that rely on data models defined according to C structures might not be feasible. Indeed, data structures may contain elements with complex dependencies. For example, a tree data structure may define the tree depth  in a specific data field and programmers may assume that pointer to child nodes are not read when the max depth is not reached (i.e., pointer child pointers in leaf nodes are not NULL). Specifying such logic into a generic framework is particularly hard if not infeasible.}




\CHANGEDNOV{We believe that the technology implementing \INDEX{data mutation} should depend on the type of system under test. This is mostly due to the need for (1) implementing mutation operations that are fast and (2) reducing the amount of data-modelling to be manually performed (ideally engineers would like to reuse existing models and artefacts). Based on the case studies shared for WP2, we observe that data-driven mutation testing might be performed by modifying either data that is stored in an array or in a data structure defined through the ASN.1 grammar. In our vision, these two solutions differ for the strategy used to model the data and for the algorithms implemented to execute mutation. Despite a more general solution (e.g., based on UML models) that glue together these two strategies might be feasible, its implementation might be the target of an ESA activity. Indeed, when data modelling is based on generic high-level models it is necessary to implement a layer that translate high-level representation into low-level data that can be efficiently processed at runtime. The following subsections describe the two distinct cases; however, in FAQAS, we will focus on the implementation of a solution for buffer arrays. The main reasons are two (1) buffer arrays appear to be a common strategy for implementing data communication (2) in FAQAS we lack case studies based on the ASN.1 grammar (see Section~\ref{sec:caseStudies:ASN:data}). In the following, however, we provide the results of a preliminary study concerning the design of a data-driven method for the ASN.1 grammar.}

\ENDCHANGEDWPT

\clearpage

\subsubsection{Fault Model Specifications for ASN.1 grammar}
\label{subsub:asn1model}

The ASN.1 grammar enables engineers to specify data structures where the types of the items in the data structure are selected from a predefined set.

%When the data to be mutated is stored in a data structure defined through the ASN.1 grammar, the fault model is specified by indicating which operators to apply on the specific fields of the data structure. 

We have identified a set of feasible fault classes for each type supported by the ASN1SCC compiler.
The corresponding mutation operators are automatically configured based on the ASN.1 grammar (e.g., in the case of an attribute of type INTEGER, the min/max values of the VOR operator are derived from the boundaries of the INTEGER type).
Table~\ref{table:faultModel:FAQAS:ASN1} provides, for each of such types, the feasible fault classes and the configurations for the mutation operators.
In the configuration for the mutation operators, we refer to the variables (e.g., MIN and MAX) appearing in the ASN.1 xml file.

Figure~\ref{fig:ASN1ProbesGeneration} provides an overview of the process in place to generate probes including the fault model.
The engineer first export the ASN.1 grammar as XML, then he modifies the generated file by specifying, for each \emph{Asn1Type}, the mutation operator to be used (this is done by adding an xml attribute called \emph{MutationOperator} with a value specifying the name of the operator). 

An example is provided in Listings~\ref{asnXML} and \ref{asnXMLUpdated}. Listing~\ref{asnXML} provides the xml generated by the grammar, which includes two INTEGER types.
Listing~\ref{asnXMLUpdated} provides the xml updated by the engineer, who indicates that the two integers should be mutated with the VAR and the VOR operator. To tune the operators, the engineer updates the MIN and MAX values for those integers to capture only nominal values. 
In the case of the first integer (the one to be mutated with VAR), the engineer sets 5 as MAX.
In the case of the second integer (the one to be mutated with VOR), the engineer sets MIN and MAX to 0 and 50 respectively.
%The engineer can tune the mutation by changing the value ranges associated to the different types. For example, this could be done to restrict the valid range of an INTEGER from (MIN=-100, MAX=100) to a nominal range of (MIN=0,MAX=50).
In case a data type is defined through value range constraints, the FAQAS framework will configure one mutation operator instance for each range.

% !TEX root = ../MAIN.tex
\begin{table}[h]
\begin{center}
\small
\begin{tabular}{|p{2cm}|p{2cm}|p{4cm}|p{4cm}|}
\hline
\textbf{Types}&\textbf{Fault Classes}&\textbf{Parameters}&\textbf{Description}\\
\hline
INTEGER&
VAT&
\begin{minipage}{4cm}
T: MAX\\
D: 1\\
\end{minipage}
&
\begin{minipage}{4cm}
\end{minipage}
\\
\hline
INTEGER&
VBT&
\begin{minipage}{4cm}
T: MIN\\
D: 1\\
\end{minipage}
&
\begin{minipage}{4cm}
\end{minipage}
\\
\hline
INTEGER&
VOR&
\begin{minipage}{4cm}
MIN: MIN\\
MAX: MAX\\
D: 1\\
\end{minipage}
&
\begin{minipage}{4cm}
\end{minipage}
\\
\hline
REAL&
VAT&
\begin{minipage}{4cm}
T: MAX\\
D: 1\\
\end{minipage}
&
\begin{minipage}{4cm}
\end{minipage}
\\
\hline
REAL&
VBT&
\begin{minipage}{4cm}
T: MIN\\
D: 1\\
\end{minipage}
&
\begin{minipage}{4cm}
\end{minipage}
\\
\hline
REAL&
VOR&
\begin{minipage}{4cm}
MIN: MIN\\
MAX: MAX\\
D: 1\\
\end{minipage}
&
\begin{minipage}{4cm}
\end{minipage}
\\
\hline
ENUMERATED&
INV&
\begin{minipage}{4cm}
MIN: MIN\\
MAX: MAX\\
D: 1\\
\end{minipage}
&
\begin{minipage}{4cm}
\end{minipage}
\\
\hline
BOOLEAN&
BF&
\begin{minipage}{4cm}
MIN: 0\\
MAX: 0\\
\end{minipage}
&
\begin{minipage}{4cm}
\end{minipage}
\\
\hline
NULL&
BF&
\begin{minipage}{4cm}
MIN: 0\\
MAX: 0\\
\end{minipage}
&
\begin{minipage}{4cm}
\end{minipage}
\\
\hline
BIT STRING&
BF&
\begin{minipage}{4cm}
MIN: 0\\
MAX: 0\\
\end{minipage}
&
\begin{minipage}{4cm}
\end{minipage}
\\
\hline
OCTET STRING&
BF&
\begin{minipage}{4cm}
MIN: 0\\
MAX: 0\\
\end{minipage}
&
\begin{minipage}{4cm}
\end{minipage}
\\
\hline
IA5STRING&
BF&
\begin{minipage}{4cm}
MIN: 0\\
MAX: 0\\
\end{minipage}
&
\begin{minipage}{4cm}
\end{minipage}
\\
\hline
NUMERIC STRING&
BF&
\begin{minipage}{4cm}
MIN: 0\\
MAX: 0\\
\end{minipage}
&
\begin{minipage}{4cm}
\end{minipage}
\\
\hline
SEQUENCE&
-&
\begin{minipage}{4cm}
\end{minipage}
&
\begin{minipage}{4cm}
No mutation environed for this type.
\end{minipage}
\\
\hline
SET&
-&
\begin{minipage}{4cm}
\end{minipage}
&
\begin{minipage}{4cm}
No mutation environed for this type.
\end{minipage}
\\
\hline
CHOICE&
-&
\begin{minipage}{4cm}
\end{minipage}
&
\begin{minipage}{4cm}
No mutation environed for this type.
\end{minipage}
\\
\hline
SEQUENCE OF&
-&
\begin{minipage}{4cm}
\end{minipage}
&
\begin{minipage}{4cm}
No mutation environed for this type.
\end{minipage}
\\
\hline
SET OF&
-&
\begin{minipage}{4cm}
\end{minipage}
&
\begin{minipage}{4cm}
No mutation environed for this type.
\end{minipage}
\\



%Incorrect Identifier& Several transmission data fields have fixed values, for example fields identifying the transmitting satellite. Hardware/software errors may assign incorrect identifiers.\\
%%Incorrect Checksum& Hardware/software errors may result in an incorrect checksum for a Packet or VCDU.\\
%Incorrect Counter& Counters are used to track Packet or VCDU ordering. Hardware/software errors may assign incorrect counter values.\\
%Flipped Data Bits& Physical channel noise may flip one or more bits in the data transmission.\\
\hline
\end{tabular}
\end{center}
\caption{Data Fault Classes for ASN.1 data types.}
\label{table:faultModel:FAQAS:ASN1}
\end{table}%

Figure~\ref{fig:ASN1ProbesGeneration} shows that, finally, the FAQAS toolset generates a modified version of the ASN1 source code containing the serializer and deserializer functions. The generated source code contains the FAQAS API functions to mutate the data in ASN.1 data types. Examples are shown in the following sections (see Figure~\ref{ASN_mutations}).

\begin{figure}[h]
  \centering
    \includegraphics[width=12cm]{images/ASN1mutationProces}
      \caption{Data-driven probes generation process for ASN1.}
      \label{fig:ASN1ProbesGeneration}
\end{figure}


\input{listings/asnXML.tex}
% !TEX root =  ../MAIN.tex

\begin{minipage}{15cm}
\begin{lstlisting}[language=XML, caption=ASN1 grammar updated to reflect a fault model., label=asnXMLUpdated, mathescape=true]
<TypeAssignment Name="TypeNested" CName="TypeNested" AdaName="TypeNested" Line="16" CharPositionInLine="0">
  <Asn1Type id="MY-MODULE.TypeNested" Line="16" CharPositionInLine="15" ParameterizedTypeInstance="false">
    <SEQUENCE acnMaxSizeInBits="3087" acnMinSizeInBits="2688" uperMaxSizeInBits="3087" uperMinSizeInBits="528">
      <SEQUENCE_COMPONENT Name="intVal" Line="17" CharPositionInLine="4" AdaName="intVal" CName="intVal">
        <Asn1Type id="MY-MODULE.TypeNested.intVal" Line="17" CharPositionInLine="11" ParameterizedTypeInstance="false">
          <INTEGER MutationOperator="VAT" acnMaxSizeInBits="4" acnMinSizeInBits="4" uperMaxSizeInBits="4" uperMinSizeInBits="4">
            <Constraints>
              <Range>
                <Min>
                  <IntegerValue>0</IntegerValue>
                </Min>
                <Max>
                  <IntegerValue>5</IntegerValue>
                </Max>
              </Range>
            </Constraints>
            <WithComponentConstraints />
          </INTEGER>
        </Asn1Type>
      </SEQUENCE_COMPONENT>
      <SEQUENCE_COMPONENT Name="int2Val" Line="18" CharPositionInLine="4" AdaName="int2Val" CName="int2Val">
        <Asn1Type id="MY-MODULE.TypeNested.int2Val" Line="18" CharPositionInLine="12" ParameterizedTypeInstance="false">
          <INTEGER MutationOperator="VOR" acnMaxSizeInBits="5" acnMinSizeInBits="5" uperMaxSizeInBits="5" uperMinSizeInBits="5">
            <Constraints>
              <Range>
                <Min>
                  <IntegerValue>0</IntegerValue>
                </Min>
                <Max>
                  <IntegerValue>50</IntegerValue>
                </Max>
              </Range>
            </Constraints>
            <WithComponentConstraints />
          </INTEGER>
        </Asn1Type>
      </SEQUENCE_COMPONENT>
\end{lstlisting}
\end{minipage}






\clearpage
\subsection{FAQAS Data Mutation API and Probes}
\label{sec:FAQASDataMutationProbes}

In FAQAS, the data-driven mutation testing API is automatically generated from the fault model provided by engineers. \INDEX{Data mutation probes} are either manually implemented by software engineers (in the case data mutation should target an ad-hoc communication layer that works with data buffers) or automatically generated by the toolset (in the case data mutation should target an ASN.1-based communication layer).



\subsubsection{Data Mutation Probes for ASN.1}
\label{sec:FAQASDataMutationProbesASN}

%\DONE{This section still needs to be written. We may put a sequence diagram that show that at the beginning the probe loads the info about the mutation operation instance to execute and execute it if feasible.}

%Fabrizio: I removed the picture because it does not help the reader
%\begin{figure}[tb]
%  \centering
%    \includegraphics[width=\textwidth]{images/DataDrivenASNProcess}
%      \caption{Data-driven mutation process for ASN.1 grammars.}
%      \label{fig:DataDrivenASNProcess}
%\end{figure}


After the generation of the extended ASN.1 source code according to the 
the fault model definition process provided in Figure~\ref{fig:ASN1ProbesGeneration}, 
\MREVISION{C-P-37}{the FAQAS framework will automatically modify the extended ASN.1 deserializer (or serializer) code to insert calls to the FAQAS mutation API.
Particularly, the framework will insert one 
invocation of the automatically generated FAQAS mutation API for each data type to be mutated.}
Listings~\ref{ASN_encode} shows an example of a probe added at the beginning of function \emph{TypeNested\_encode} to mutated the TypeNested data to be encoded by function \emph{TypeNested\_encode}.
Listings~\ref{ASN_decode} shows an example of a probe added 
at the end of function \emph{TypeNested\_decode}
to mutated the TypeNested data decoded by function \emph{TypeNested\_decode}.

The insertion of the probe in the serializer code is useful when the fault model is not modified by the engineer but simply includes the boundary values automatically generated by the FAQAS framework. 
This is useful to generate invalid data to be serialized and thus test the capability of the ASN.1 serializer to detect illegal values.

The insertion of the probe in the deserializer code is useful to simulate the generation of invalid data from a faulty component. This is useful when the fault model had been modified by the engineer to reflect possible non-nominal cases with values belonging to the legal value domain.

%Figure~\ref{fig:DataDrivenASNProcess} provides an overview of the mutation process followed by the ASN.1 data mutation functions.
%At runtime, for each mutation operator instance, a single probe will be enabled. 

Similarly to the case of data mutation for data buffers, each mutant can implement a single mutation operation instance or work as a mutant schemata where the mutation operation instance is selected at runtime, based on a configuration parameter. Each mutation operator instance is identified by a unique identifier. 


%Fabrizio: You never introduced E, its not a good example!
%A MOI represents the mutation to be applied. More specifically, it contains the data type name to be mutated, and an ID that represents the mutation operator to be applied, and under what conditions applies.

%For example Listing~\ref{ASN_mutations} shows two possible MOI probes for the E data type. 
%\texttt{E\_1} exercises a mutation for the E data type when the value of \texttt{pVal} is less than or equal to 255. If the condition is true, then the value is modified by replacing it for the MAX value (e.g., 255).
%Similarly, \texttt{E\_2} exercises the E data type when the value of \texttt{pVal} is equal to 1299. If the condition applies, then the value is replaced by $1299 + 1$ (e.g., VAT mutation operator).
%The mutation is saved after its execution, so it is not performed twice. 

% !TEX root =  ../MAIN.tex
\begin{minipage}{14cm}
\begin{lstlisting}[style=CStyle, caption=Example of data-driven mutation probe for ASN.1 that has been added to the encoding function., label=ASN_encode]
flag TypeNested_Encode(const TypeNested* pVal, BitStream* pBitStrm, int* pErrCode, flag bCheckConstraints)
{
    TypeNested_mutate(pVal);
    flag ret = TRUE;
...
}
\end{lstlisting}
\end{minipage}

\begin{minipage}{14cm}
\begin{lstlisting}[style=CStyle, caption=Example of data-driven mutation probe for ASN.1 that has been added to the decoding function., label=ASN_decode]
flag TypeNested_Decode(TypeNested* pVal, BitStream* pBitStrm, int* pErrCode)
{
    flag ret = TRUE;
...
        // mutation
        TypeNested_mutate(pVal);

        return ret  && TypeNested_IsConstraintValid(pVal, pErrCode);
}
\end{lstlisting}
\end{minipage}

%flag E_Decode(E* pVal, BitStream* pBitStrm, int* pErrCode)
%{
%    flag ret = TRUE;
%    *pErrCode = 0;
%    (void)pVal;
%    (void)pBitStrm;
%
%
%    (*(pVal))=5; ret = TRUE; *pErrCode = 0;
%
%    // Manually added probe 
%    E_mutate(pVal);
%    // Manually added probe END
%    return ret  && E_IsConstraintValid(pVal, pErrCode);
%}



Listing~\ref{ASN_mutations} shows three possible mutation operation instances for the \emph{TypeNested} data type configured according to the fault model shown in Listing~\ref{asnXMLUpdated}.
\texttt{TypeNested\_1} applies the VAR operator,
 \texttt{TypeNested\_2} applies the VOR operator by setting the data value below the lower bound.
 \texttt{TypeNested\_3} applies the VOR operator by setting the data value above the upper bound.
% !TEX root =  ../MAIN.tex

\begin{lstlisting}[style=CStyle, caption=Example of automatically generated ASN.1 data-driven mutation operations., label=ASN_mutations]

// TypeNested type
void _FAQAS_TypeNested_mutate(TypeNested *pVal) {

	// ALREADY_MUTATED is a global variable 
	// that traces if in the current execution we already performed data mutation
	if ( ALREADY_MUTATED ){
		return;
	}
	
	// intVal,VAT,1
	if ( ! has_been_mutated("TypeNested_1") ){
		// check that the value is not already 
		// what we want to generate

		if (pVal->intVal != 6 ){
			pVal->intVal = 6;
			save_mutation("TypeNested_1");

			return;
		}
	}
	
	// int2Val,VOR,1
	if ( ! has_been_mutated("TypeNested_2") ){
		// check that the value is not already 
		// what we want to generate

		if (pVal->intVal != 0 ){
			pVal->intVal = -1;
			save_mutation("TypeNested_2");

			return;
		}
	}

	// int2Val,VOR,2
	if ( ! has_been_mutated("TypeNested_3") ){

        printf("%lu\n", pVal -> intVal);

        if (pVal->intVal != 10){
            pVal->intVal = 51;
            save_mutation("TypeNested_3");

            return;
        }
    }

...

\end{lstlisting}

%// max E OR 1st operand constraint
%if ( strcmp(buf,"E_1") ) {                                                  
%	if ((*pVal) <= 255UL) {                                                                                                               
%	    printf("%lu\n", *pVal);
%
%	    if (*pVal != 255UL) {
%	        *pVal = 255UL;
%	        save_mutation();
%
%	        return;
%	    }           
%	}       
%}
%
%// n+1 E 2nd operand constraint
%if ( strcmp(buf, "E_2") ) {
%    if ((*pVal) == 1299UL) {                                                                                                              
%        printf("%lu\n", *pVal);
%
%        *pVal = 1299UL + 1UL;
%        save_mutation();
%        return;
%    }       
%} 


\clearpage



\clearpage
\section{Test Suite Augmentation} % (fold)
\label{sec:data:test_suite_augmentation}

\STARTCHANGEDWPT

The \INDEX{test suite augmentation process} concerns the definition of additional test cases to increase the mutation score.
It consists of four activities \INDEX{Identify Test Inputs}, \INDEX{Generate Test Oracles}, \INDEX{Execute the SUT}, \INDEX{Fix the SUT}. 
Despite these activities match the ones performed in the case of code-driven mutation testing, they are triggered and implemented in a different manner, as described below.

In the presence of mutants not killed by test cases (i.e., when the  \INDEX{mutation score} is not equal to 100\%), engineers are expected to manually investigate the underlying problems. Indeed, as reported in Section~\ref{sec:mutationAnalysisResults}, two might be the reasons for a low MS: poor oracle quality and missing test input sequences (i.e., the software does not reach the state in which it could kill the mutant).
For the first case (poor oracle quality), manual work is needed because automated approaches to automatically generate test oracles in the presence of system or integration test suites are not available. For the second case, existing test generation approaches (e.g., KLEE) might suffer from scalability problem that prevent bringing the system into a desired state ; also, they cannot deal with systems whose components communicate through channels. For this reason, generating test oracles and fixing the SUT (in case a fault is discovered after test suite augmentation) shall be performed manually.

When mutation operators are not applied because of the lack of appropriate data to mutate (i.e., in the presence of fault model coverage and mutation operation coverage below 100\%), engineers are expected to generate new test inputs for the SUT that enable the application of all the mutation operators. 
However, the methodology to adopt may vary based on the test objective and the system architecture. 
We discuss the case of the producer-consumer and client-server architecture, two common software architectures. We leave the discussion of other architectures (e.g., broker architecture and event-bus architecture) to future work.

In Figures~\ref{fig:dataDrivenTestSuiteAugmentationC} to~\ref{fig:dataDrivenTestSuiteAugmentationE}, we exemplify the two architectures. In both the two cases, data-driven mutation may concern the generated data and occur either on the component that generates the data (Figure~\ref{fig:dataDrivenTestSuiteAugmentationC}), or on the component that receives the data (Figure~\ref{fig:dataDrivenTestSuiteAugmentationD}).
For the client-server case, instead, data mutation may concern also the request for data and be performed either on the client or the server (Figure~\ref{fig:dataDrivenTestSuiteAugmentationE}). For the producer-consumer case, static program analysis may be employed to automatically generate the missing data; to this end, we aim to rely on an \INDEX{extended data mutation probe}. For the client-server case, the \INDEX{extended data mutation probe} may still be used but only to generate message requests; therefore, it would be useful only when data-driven analysis is performed on the  request message. We exemplify the two cases below.

\begin{figure}[h]
  \centering
    \includegraphics[width=14cm]{images/dataDrivenTestSuiteAugmentationC}
      \caption{Data-driven mutation analysis for different architectures.}
      \label{fig:dataDrivenTestSuiteAugmentationC}
\end{figure}

\begin{figure}[h]
  \centering
    \includegraphics[width=14cm]{images/dataDrivenTestSuiteAugmentationD}
      \caption{Data-driven mutation analysis for different architectures.}
      \label{fig:dataDrivenTestSuiteAugmentationD}
\end{figure}

\begin{figure}[h]
  \centering
    \includegraphics[width=14cm]{images/dataDrivenTestSuiteAugmentationE}
      \caption{Data-driven mutation analysis for different architectures.}
      \label{fig:dataDrivenTestSuiteAugmentationE}
\end{figure}

\begin{figure}[h]
  \centering
    \includegraphics[width=14cm]{images/dataDrivenTestSuiteAugmentationB}
      \caption{Data-driven mutation analysis for different architectures.}
      \label{fig:dataDrivenTestSuiteAugmentationB}
\end{figure}

\ENDCHANGEDWPT

\clearpage
\subsection{Producer-consumer}


We assume to have a system that exchanges data of type TypeNested defined by relying on the ASN.1 grammar (see Listings~\ref{asnXMLUpdated} and ~\ref{asnXMLUpdated}). Also, we assume that the objective of data-driven mutation testing is to assess the quality of the test cases implemented to verify the consumer component. 
Such test cases may consist of sending predefined data through a producer component and verify that the consumer generates the expected output. 
To perform data-driven mutation, we may rely on a probe installed on the deserializer component. 


To enforce the generation of the required data types, we can augment the producer component with an extended mutation probe (called \emph{\_FAQAS\_TypeNested\_cover} in Listing~\ref{ASN_encode}). In this case the probe should not be used to mutate the data but it should include assertions that enable reachability analysis. Listing~\ref{ASN_encodeReachable} shows an example where, for each mutation operation implemented in the probe, we introduce an \emph{assert(false)} statement. Static analysis tools (e.g., KLEE)  can then be used to find inputs that enable reaching any of these assertions from the entry point of the producer component. For each assertion, the static analysis component will look for an input of the entry point (e.g., the main function) that enables reaching the assertion, i.e., generate data that can be mutated according to the provided mutation operation. The identified inputs can then be used to augment the test suite.

% !TEX root =  ../MAIN.tex
\begin{lstlisting}[style=CStyle, caption=Example of data-driven mutation probe for ASN.1 that has been added to the encoding function., label=ASN_encodeReachable]
flag TypeNested_Encode(const TypeNested* pVal, BitStream* pBitStrm, int* pErrCode, flag bCheckConstraints)
{
    TypeNested_mutate(pVal);
    flag ret = TRUE;
...
}


void _FAQAS_TypeNested_mutate(TypeNested *pVal) {

	// ALREADY_MUTATED is a global variable 
	// that traces if in the current execution we already performed data mutation
	if ( ALREADY_MUTATED ){
		return;
	}
	
	// intVal,VAT,1
	if ( ! has_been_mutated("TypeNested_1") ){
		// check that the value is not already 
		// what we want to generate

		if (pVal->intVal != 6 ){
			pVal->intVal = 6;
			assert(false);
			save_mutation("TypeNested_1");

			return;
		}
	}
	
	// int2Val,VOR,1
	if ( ! has_been_mutated("TypeNested_2") ){
		// check that the value is not already 
		// what we want to generate

		if (pVal->intVal != -1 ){
			pVal->intVal = -1;
			assert(false);
			save_mutation("TypeNested_2");

			return;
		}
	}

	// int2Val,VOR,2
	if ( ! has_been_mutated("TypeNested_3") ){

        printf("%lu\n", pVal -> intVal);

        if (pVal->intVal != 51){
        	    assert(false);	
            pVal->intVal = 51;
            save_mutation("TypeNested_3");

            return;
        }
    }


\end{lstlisting}



%flag E_Decode(E* pVal, BitStream* pBitStrm, int* pErrCode)
%{
%    flag ret = TRUE;
%    *pErrCode = 0;
%    (void)pVal;
%    (void)pBitStrm;
%
%
%    (*(pVal))=5; ret = TRUE; *pErrCode = 0;
%
%    // Manually added probe 
%    E_mutate(pVal);
%    // Manually added probe END
%    return ret  && E_IsConstraintValid(pVal, pErrCode);
%}


\clearpage
\subsection{Client-server}

\STARTCHANGEDWPT

For the client-server case, we rely on the libParam case study provided by GSL. Listing~\ref{GSLmutate} shows the mutation probe, which is inserted into function \emph{gs\_rparam\_process\_packet}, on the server side. The probe mutates the buffer \emph{v\_General}, which contains a copy of a message request (i.e., \emph{request}). In the case of GSL, the FVAT operator configured to mutate \emph{request-$\>$table\_id} cannot be applied (i.e., MOC is not equal to 100\%); this indicates that the test cases do not cover a scenario in which the client passes a \emph{table\_id} above the threshold. To generate such a test case we may rely on the extended probe combined with \INDEX{static program analysis}. 

Listing~\ref{GSLcover} shows how the INDEX{extended mutation probe} might be inserted into the code of libParam. In practice, it requires the engineer to know the portion of code that handles the generation of a request message. Unfortunately, injecting the mutation probe is not sufficient to enable test generation but engineers need also to prepare a test template to enable test generation with KLEE. Listing~\ref{GSLtest} shows an example of such template based on existing libParam test cases; such test case requires the initialization of a number of state variables, which limits the possibility to automate its definition. For this reason, within FAQAS we did not find it feasible to automate data-driven mutation analysis with a tool but we aim to evaluate its manual feasibility in WP4.

Finally, when data-driven mutation is applied to the data generated by the server, test automation is made unfeasible by the fact that KLEE cannot work in the presence of a communication channel within the code to be analyzed. Such shortcoming is not observed when we mutate request data because the extended mutation probe is installed only on the client; the producer-consumer case is not affected by such shortcoming because, in this case, the probe is installed on the producer. Alternative test generation tools or extensions of KLEE shall be considered to overcome such limitations.

% !TEX root =  ../MAIN.tex
\begin{lstlisting}[style=CStyle, caption=Example of data-driven mutation probe for libParam, label=GSLmutate]

static void gs_rparam_process_packet(csp_conn_t * conn, csp_packet_t * request_packet)
{
    csp_packet_t * reply_packet = NULL;
    gs_rparam_query_t * reply;


    /* Handle endian */
    gs_rparam_query_t * request = (gs_rparam_query_t *) request_packet->data;


    request->length = csp_ntoh16(request->length);
    request->checksum = csp_ntoh16(request->checksum);


    FaultModel *fm_General = _FAQAS_General_FM();
    unsigned long long int v_General[6];

    v_General[0] = (unsigned long long int) request->action;
    v_General[1] = (unsigned long long int) request->table_id;
    v_General[2] = (unsigned long long int) request->length;
    v_General[3] = (unsigned long long int) request->checksum;
    v_General[4] = (unsigned long long int) request->seq;
    v_General[5] = (unsigned long long int) request->total;


    _FAQAS_mutate(v_General,fm_General);
    
\end{lstlisting}



%flag E_Decode(E* pVal, BitStream* pBitStrm, int* pErrCode)
%{
%    flag ret = TRUE;
%    *pErrCode = 0;
%    (void)pVal;
%    (void)pBitStrm;
%
%
%    (*(pVal))=5; ret = TRUE; *pErrCode = 0;
%
%    // Manually added probe 
%    E_mutate(pVal);
%    // Manually added probe END
%    return ret  && E_IsConstraintValid(pVal, pErrCode);
%}


% !TEX root =  ../MAIN.tex
\begin{lstlisting}[style=CStyle, caption=Example of extended data-driven mutation probe for libParam, label=GSLcover]

/**
   Get string.
   @note If the returned string is max length, the value buffer will not be 0 terminated.
   @param[in] node CSP address
   @param[in] table_id remote table id.
   @param[in] addr parameter address (remote table).
   @param[in] checksum checksum
   @param[in] timeout_ms timeout
   @param[out] value returned value (user allocated)
   @param[in] value_size size of \a value, i.e. size of parameter type in bytes.
   @return_gs_error_t
*/
static inline gs_error_t gs_rparam_get_string(uint8_t node, gs_param_table_id_t table_id, uint16_t addr,
                                              uint16_t checksum, uint32_t timeout_ms, char * value, size_t value_size)
{
    return gs_rparam_get(node, table_id, addr, GS_PARAM_STRING, checksum, timeout_ms, value, value_size);
}


gs_error_t gs_rparam_get(uint8_t node,
                         gs_param_table_id_t table_id,
                         uint16_t addr,
                         gs_param_type_t type,
                         uint16_t checksum,
                         uint32_t timeout_ms,
                         void * value,
                         size_t value_element_size)
{
    return gs_rparam_get_array(node, table_id, addr, type, checksum, timeout_ms, value, value_element_size, 1);
}


gs_error_t gs_rparam_get_array(uint8_t node,
                               gs_param_table_id_t table_id,
                               uint16_t addr,
                               gs_param_type_t type,
                               uint16_t checksum,
                               uint32_t timeout_ms,
                               void * value,
                               size_t value_element_size,
                               size_t array_size)
{
    /* Calculate length */
    gs_rparam_query_t * query;
    const size_t query_payload_size = sizeof(query->payload.addr[0]) * array_size;
    const size_t query_size = RPARAM_QUERY_LENGTH(query, query_payload_size);
    const size_t reply_payload_element_size = value_element_size + sizeof(query->payload.addr[0]);
    const size_t reply_payload_size = reply_payload_element_size * array_size;
    const size_t reply_size = RPARAM_QUERY_LENGTH(query, reply_payload_size);

    query = alloca(reply_size);
    query->action = RPARAM_GET;
    query->table_id = table_id;
    query->checksum = csp_hton16(checksum);
    query->seq = 0;
    query->total = 0;
    for(unsigned int i = 0; i < array_size; i++) {
        query->payload.addr[i] = csp_hton16(addr + (value_element_size * i));
    }
    query->length = csp_hton16(query_payload_size);

    FaultModel *fm_General = _FAQAS_General_FM();
    unsigned long long int v_General[6];

    v_General[0] = (unsigned long long int) query->action;
    v_General[1] = (unsigned long long int) query->table_id;
    v_General[2] = (unsigned long long int) query->length;
    v_General[3] = (unsigned long long int) query->checksum;
    v_General[4] = (unsigned long long int) query->seq;
    v_General[5] = (unsigned long long int) query->total;


    _FAQAS_cover(v_General,fm_General);


    /* Run single packet transaction */
    if (csp_transaction2(CSP_PRIO_HIGH, node, GS_CSP_PORT_RPARAM, timeout_ms, query, query_size, query, reply_size, CSP_O_CRC32) <= 0) {
        return GS_ERROR_IO;
    }
 ... 
 
 }

\end{lstlisting}



%flag E_Decode(E* pVal, BitStream* pBitStrm, int* pErrCode)
%{
%    flag ret = TRUE;
%    *pErrCode = 0;
%    (void)pVal;
%    (void)pBitStrm;
%
%
%    (*(pVal))=5; ret = TRUE; *pErrCode = 0;
%
%    // Manually added probe 
%    E_mutate(pVal);
%    // Manually added probe END
%    return ret  && E_IsConstraintValid(pVal, pErrCode);
%}


% !TEX root =  ../MAIN.tex
\begin{lstlisting}[style=CStyle, caption=Test template to enable data-driven mutation testing for libParam, label=GSLtest]

    // a little hack - this is next element, we use it check for overwrite and missing 0 termiation
    memset(alltypes_mem.string_A, 'Z', sizeof(alltypes_mem.string_A));
    alltypes_mem.string_A[0][1] = 0;

    char buf[GS_TEST_ALLTYPES_STRING_LENGTH + 10];

    // get max size - no 0 termination
    memset(alltypes_mem.string, 'B', sizeof(alltypes_mem.string));
    memset(buf, 'A', sizeof(buf));
    buf[GS_TEST_ALLTYPES_STRING_LENGTH + 1] = 0;
    
    csp_node CSP_NODE;
    unsigned long long int tableID;
    klee_make_symbolic(&CSP_NODE, sizeof(CSP_NODE), ”CSP_NODE”);
    klee_make_symbolic(&tableID, sizeof(tableID), ”tableID”);
    gs_rparam_get_string(&CSP_NODE, tableID, GS_TEST_ALLTYPES_STRING, GS_RPARAM_MAGIC_CHECKSUM, 1000, buf, GS_TEST_ALLTYPES_STRING_LENGTH);

    
\end{lstlisting}



%flag E_Decode(E* pVal, BitStream* pBitStrm, int* pErrCode)
%{
%    flag ret = TRUE;
%    *pErrCode = 0;
%    (void)pVal;
%    (void)pBitStrm;
%
%
%    (*(pVal))=5; ret = TRUE; *pErrCode = 0;
%
%    // Manually added probe 
%    E_mutate(pVal);
%    // Manually added probe END
%    return ret  && E_IsConstraintValid(pVal, pErrCode);
%}


\ENDCHANGEDWPT

%For example, in the case of the example in Figure~\ref{fig:DataDrivenSimpleExample}, engineers would need to implement test cases that trigger the exchange of \emph{DataMessages}.
%Fully automated approaches to generate test cases for data-driven mutation testing are unavailable; however, techniques that generate input data from scratch~\cite{gligoric2010test} or augment input data~\cite{DiNardo:TOSEM:2017} can be adopted. 
%Also, when the data used by test cases is generated by simulators, meta-heuristic search can be used to drive the generation of input data~\cite{Abdessalem:ICSE:2018}. 
%
%The execution of the SUT and the repair of the SUT are performed manually as in the case of code-driven data mutation.
%
%
%\TODO{Clarify if we generate test cases or not}
%
%Section~\ref{sec:testGenerationData} provides details about the existing solutions to  \emph{Identify Test Inputs} and \emph{Generate Test Oracles}.


% !TEX root = MAIN.tex
\clearpage
\section{Evaluation of Data-driven Mutation Testing Toolsets}
\label{sec:toolsComparisonDataDriven}

This section describes an evaluation we conducted to identify a data-driven mutation testing tool applicable to space context. In particular, we assessed the Peach Fuzzer toolset.

% description of the toolset

% !TEX root = ../MAIN.tex

\begin{table}[h]
\begin{center}
\footnotesize
\begin{tabular}{|p{5cm}|p{9cm}|}
\hline
\textbf{Operator Name}&\textbf{Description}\\
\hline
ArrayVarianceMutator&Change the length of arrays. Given L the original length of the array, the length is changed in range L-N to L+N.\\
ArrayReverseOrderMutator&Reverse the order of an array.\\
ArrayRandomizeOrderMutator&Put array elements in random order.\\
DWORDSliderMutator&Slides a DWORD through the blob.\\
BitFlipperMutator&Flips a given \% of bits in blob. Default is 20\%.\\
BlobMutator&Randomly grows a Blob block or shrinks it.\\
DataTreeRemoveMutator&Remove nodes from data tree.\\
DataTreeDuplicateMutator&Duplicate a node's value starting at 2x through 50x.\\
DataTreeSwapNearNodesMutator&Swap the data of two nodes that are near each other in the data model.\\
NumericalVarianceMutator&Produce numbers that are defaultValue - N to defaultValue + N.\\
NumericalEdgeCaseMutator&Replace with random numbers of appropriate correct size.\\
FiniteRandomNumbersMutator&Produce a finite number of random numbers for each \emph{Number} element.\\
NumericalEvenDistributionMutator&Generate numbers evenly distributed through the total numerical space of the number range.\\
NullMutator&Does nothing, just test the data produced by the fuzzer.\\
PathValidationMutator&Does not mutate. Used to trace path of each test for path validation.\\
SizedVarianceMutator&Change the length of sizes to count - N to count + N.\\
SizedNumericalEdgeCasesMutator&Change the length of sizes to numerical edge cases.\\
SizedDataVarianceMutator& Change the length of sized data to count - N to count + N. Size indicator will stay the same.\\
SizedDataNumericalEdgeCasesMutator&Change the length of sizes to numerical edge cases.\\
StringCaseMutator&Change the case of a string.\\
UnicodeStringsMutator&Generate unicode strings.\\
ValidValuesMutator&Replace with random values other than the legal ones.\\
UnicodeBomMutator&Injects BOM markers into default value and longer strings.\\
UnicodeBadUtf8Mutator&Generate bad UTF-8 strings.\\
UnicodeUtf8ThreeCharMutator&Generate long UTF-8 three byte strings.\\
StringMutator&Generate a random unicode string, for each string node, one Node at a time.\\
XmlW3CMutator&Replace XML trees with invalid, non-well former, and valid (but random) XML trees.\\
PathMutator&Replace a path with an erroneous path generated according to 20 different rules.\\
HostnameMutator&Replace a hostname with an erroneous hostname generated according to 20 different rules.\\
IpAddressMutator&Replace an IP address with an erroneous IP address generated according to 20 different rules.\\
TimeMutator&Replace a time value with an erroneous value generated according to 3 different rules.\\
DateMutator&Replace a date with 60 predefined erroneous dates.\\ 
FilenameMutator&Replace a file name with an file name generated according to 10 different rules.\\
ArrayNumericalEdgeCasesMutator&This operator is not well documented in the source code of Peach.\\
BlobSpread&This operator is not well documented in the source code of Peach.\\
\hline
\end{tabular}
\end{center}
\caption{Mutation Operators for the opensource version of Peach~\cite{PeachMozilla}}
\label{table:PeachOperators}
\end{table}%

% !TEX root =  ../MAIN.tex

\begin{minipage}{15cm}
\begin{lstlisting}[language=XML, caption=Portion of a Peach data model., label=peach, mathescape=true]
<Number name="lfh_CompSize" size="32" endian="little" signed="false"/>
<Number name="lfh_DecompSize" size="32" endian="little" signed="false"/>
<Number name="lfh_FileNameLen" size="16" endian="little" signed="false">
    <Relation type="size" of="lfh_FileName"/>
</Number>
<Number name="lfh_ExtraFldLen" size="16" endian="little" signed="false">
    <Relation type="size" of="lfh_FldName"/>
</Number>
<String name="lfh_FileName"/>
<String name="lfh_ExtraField"/>
\end{lstlisting}
\end{minipage}



Peach~\cite{PeachMozilla,PeachFuzzer} is a fuzzing tool that relies on block models~\cite{pham2016model,spike} to perform data mutations. In other words, Peach perform mutations by altering the data of an input 􏰘according to a large, predefi􏰘ned set of rules. For example, Listing~\ref{peach} introduces a portion of a data model describing the properties of the Zip data format~\cite{zipformat}. 

Even though Peach is currently a proprietary software~\cite{PeachFuzzer}, the Mozilla Foundation maintains a community edition of the toolset~\cite{PeachMozilla}, the community edition implements basic features such as the fuzzing capabilities. The proprietary version of Peach instead provides features for automatic generation of test cases and detailed reports about the potential security threats of a software~\cite{PeachFuzzer}. The version we evaluated in this activity was the community edition provided by the Mozilla Foundation. We provide an overview of the mutation operators implemented by the Peach community edition in Table~\ref{table:PeachOperators}.

% what we did

In the assessment of Peach, we defined three criteria to understand its applicability to the space context software. The first criteria concerns assessing if the community edition of Peach does work and if it can be installed properly. The second criteria concerns its portability. Finally, the third criteria concerns assessing its compatibility with FAQAS case study systems.

Regarding the first criteria, we tested Peach by applying it to the \texttt{unzip} program, and zip file mutants with the fuzzing capabilities of Peach. For this objective we reproduced the steps indicated in~\cite{zipexample}. So, first, we generated a Peach Data Model for Zip data files, and then we specified a launcher that enables the complete mutation process.

Peach provides a monitoring infrastructure that enables the execution of the whole mutation process. The process consists of the following steps:
\begin{enumerate}
	\item Specifying the data model for the a data type into an XML file.
	\item Loading the data model into Peach.
	\item Generating a new mutant (i.e, a mutated input).
	\item Running the program taking as an input the generated mutant, and with the monitoring infrastructure enabled.
	\item If the program crashes the process is stopped.
	\item If the program does not crashes, the process goes back to step 3, performing a new mutation.
\end{enumerate}

In particular, we were able to generate mutants for the Zip data format, but we could not run the monitoring infrastructure since it had dependencies with graphical environments that prevent us to execute it properly.

Regarding the second criteria, and specifically its portability. We seek to integrate it into the case studies as a component of their software to mutate data once it is sent, basically the idea would be to intercept the methods that exchange data, and apply Peach directly to the variable containing the data.
During the evaluation, we discovered that Peach -mainly implemented in Python- can be used as a Python library, and that this library can be invoked to generate multiple mutants in a off-line mode.

% why it was discarded

Regarding the third criteria and its compatibility with our case studies, we conclude that its integration with embedded systems is unlikely to work, mainly because of the characteristics of our case study systems.
For example, the ESAIL case study system runs within a real-time operative system (i.e., RTEMS by Edisoft) that does not possess a filesystem. Therefore, integrating Peach into ESAIL is not feasible because it might affect the real-time performance of the application, and also because it will be necessary to implement a solution to port the Peach toolset into the ESAIL infrastructure.



