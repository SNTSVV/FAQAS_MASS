% !TEX root = MAIN.tex

\chapter{Data-driven Mutation Testing}
\label{chapter:datamutation}

% !TEX root = MutationTestingSurvey.tex

\section{Overview of the Data-driven Mutation Testing Process}
\label{sec:dataProcess}

	\begin{figure}
	\centering
		\includegraphics[width=\textwidth]{images/dataProcess}
		\caption{Data-driven Mutation Testing Process.}
		\label{fig:data:process}
	\end{figure}



This Chapter defines a test suite assessment process based on the injection of faults in the data processed by software components; we refer to this process as \INDEX{data-driven mutation testing}. 
%The definition of data-driven mutation testing is a unique contribution of this book; it has not been presented in previous literature work.

Data-driven mutation testing aims to assess test suites by simulating faults that affect the data produced, received, or exchanged by the software and its components.
It is based on a \INDEX{fault model} capturing the type of data faults that might affect the system. The fault model is produced by software engineers based on their domain knowledge and experience~\cite{di2015generating}. \MREVISION{C18}{The considered faults might be due to programming errors, hardware problems, or critical situations in the environment (e.g., noise in the channel).} The data is then automatically  mutated (i.e., modified) by a set of operators that aim to replicate the faults in the fault model. For example, the \INDEX{bit flip operator} flips a randomly selected bit of a field of the transmitted data (see Section~\ref{sec:data_operators}). 
%Mutation operators can be applied multiple times, on different data chunks or over repeated executions of a test case, ti


Figure \ref{fig:data:process} shows the reference data-driven mutation testing process that will be considered in FAQAS. The process is based on two main sub-processes, \EMPH{test suite evaluation} and \EMPH{test suite augmentation}, which are described in Sections~\ref{sec:data:test_suite_evaluation}~and~\ref{sec:data:test_suite_augmentation}, respectively. Differently from the code-driven mutation testing process introduced in Section~\ref{sec:process}, the data-driven mutation testing process has not been formalized by existing software testing literature. An in depth discussion of related work has been presented in deliverable D1.

Since data-driven mutation testing alters the data produced, received, or exchanged by the software or its components, it should be applied to evaluate test suites that trigger the execution and communication between multiple components (e.g., system or integration test cases). Data-driven mutation testing is not meant to be applied to assess unit test suites.

\clearpage
\section{Test Suite Evaluation} % (fold)
\label{sec:data:test_suite_evaluation}

The test suite evaluation process consists of three activities \EMPH{Execute the SUT}, \EMPH{Mutate Data},  and \EMPH{Analyze Results}.
The activity \INDEX{Execute the SUT} indicates that the SUT is executed against its automated test suite. 
The activity \INDEX{Mutate Data} concerns the automated modification of either the data received by the software, the data produced by the software, or the data exchanged by software components.
In Figure~\ref{fig:data:process}, the activity \EMPH{Mutate Data} is executed in parallel to the activity \EMPH{Execute the SUT} since data modification should occur at runtime during test cases execution, to simulate software faults affecting the data processed by the software.



%The fault model enables engineers to minimize the presence of equivalent mutants.
%The data model may capture the relation between inputs and outputs of the system

\begin{figure}
	\centering
		\includegraphics[width=10cm]{images/dataMutationExample}
		\caption{Example of implementation of a data mutation solution.}
		\label{fig:data:mutateData}
	\end{figure}
	
Figure~\ref{fig:data:mutateData} provides an example of how the Mutate Data activity can be implemented. Figure~\ref{fig:data:mutateData} shows that, to implement data mutation, it is necessary to integrate additional components that alter the data exchanged by the SUT and its components at runtime on a certain communication layer. We call such components \INDEX{mutation probes}.

An ideal target for data-driven mutation testing is the communication between loosely coupled software components; typically the ones that run on separated pieced for hardware (e.g., the on board controller and the ADCS).
In FAQAS case studies, the communication between such software components is performed by relying on APIs of a dedicated communication layer; this is typical of space systems. The main functionality of such communication layer is to serialize and deserialize data that should be transmitted on the channel. In FAQAS case studies, the communication layer is either implemented in-house or it is built by relying on the ASN.1 compiler architecture. In both the two cases, the communication layer implements distinct functions to serialize and deserialize data.
For this reason, in FAQAS, we expect mutation probes to be \EMPH{integrated into the software API functions that either serialize or deserialize the data being sent on the communication channel}.

Mutation probes are integrated into the software under test by engineers who \EMPH{manually} add specific function calls in the source code of the software under test. Since communication APIs vary from system to system, it is not possible to define a tool that automatically modifies the source code or the executable code of the SUT.  Instead, we provide a \INDEX{data-driven mutation testing API} to mutate the data specified by the engineer.
Figure~\ref{fig:data:mutationProbes} shows an example of mutation testing APIs integrated into the source code (API method names begin with the prefix \emph{\_FAQAS}).

To support generalizability, the FAQAS framework will enable to automatically mutate all the data types supported by the ASN.1 compiler. Also, it will provide methods that enable mutating data buffers provided as C arrays. In both the two cases a faulty model specifying how to alter the data (i.e., the attributes declared in an ASN.1 grammar or the array items).


\begin{figure}
	\centering
		\includegraphics[width=10cm]{images/dataMutationProbes}
		\caption{Example of integration of data mutation probes.}
		\label{fig:data:mutationProbes}
	\end{figure}

\TODO{ADD concrete example of integration}

	
	
In deliverable D1, we have clarified that, in a generic data-driven mutation testing process, the activity \EMPH{Mutate Data} may require a \INDEX{data model} that captures the characteristics and structure of the data to be mutated. 
The data model should be used to load a stream of bytes in structured form (e.g., an instance of a given data structure), which is necessary to drive data mutation. 
Also, the activity \EMPH{Mutate Data} should be driven by a \INDEX{fault model} that specifies the set of mutation operators to apply~\cite{di2015generating}. 
In FAQAS, the data model and the fault model are coupled into a same structured representation, i.e., the \INDEX{FAQAS fault model}, which is presented in Section~\ref{sec:faultModel}.


The activities \EMPH{Execute the SUT} and \EMPH{Mutate Data} are repeated till all the faults of the fault model had been applied. The possible stopping criteria are described in Section~\ref{sec:mutantsExecution}. 

The activity \INDEX{Process Outputs} processes all the outputs generated during the execution of test cases.
The collected outputs include the result of test cases execution (i.e., the list of test cases that either passed or failed) and the logs generated by the SUT during testing.
In the context of data-driven mutation testing both the \INDEX{test results} and the \INDEX{log files} are necessary to determine if a test suite kills a mutant.
Indeed, \emph{in the context of data-driven mutation testing a mutant is killed either if a test case fails, or if the software activates robustness features capable of handling the specific data fault.}
We need log files to determine if robustness features had been triggered.
For example, a system that implements a \INDEX{robust communication protocol} might simply request again the packets affected by errors thus avoiding failures. In this case, we need to inspect the log files to determine if the robustness feature had been triggered.
The fault model is expected to include only data fault classes that should either lead to failures or make the system generate an error entry in the log file.

Differently from code-driven mutation testing, data-driven mutation testing does not alter the software implementation but only the data processed by software components, for this reason it may help engineers to \INDEX{identify existing faults}, an objective that cannot be achieved by code-driven mutation testing. 
This is the case when \emph{Missing Error Reports} or \emph{Unexpected Failures }(e.g, crashes) are observed. In both the two cases, engineers should fix the system. In code-driven mutation testing, faults can be detected only after introducing new test cases that kill the generated mutants.

The activity \INDEX{Analyze Results} provides an assessment of the quality of the test suite for the SUT.
It is driven by two objectives:
\begin{itemize}
\item[(O1)] determine if the test suite is capable of detecting software faults that affect the data processed by the software components 
(e.g., we expect a test suite to fail in case the data exchanged by two components contains invalid values).
\item[(O2)] determine if the test suite exercises enough software behaviours to discover all the possible faults that may affect the data produced by the system
(i.e., it should be possible to alter the processed data to generate faulty data according to the fault model). 
\end{itemize}

\MREVISION{C19}{Objectives O1 and O2 are complementary, \REVTWO{C33}{they both should be addressed by data mutation.}
For example, a use case scenario for data-driven mutation testing could be the following: (i) data-driven mutation testing is applied to the data exchanged by \emph{component 1} and \emph{component 2} in Figure~\ref{fig:data:mutateData}, (ii) the data exchanged by the two components follow the data model in Figure~\ref{fig:DataDrivenSimpleExample}, and (iii) mutation testing is performed by applying the bit-flip mutation operator to every field of the messages being exchanged.
The data model in Figure~\ref{fig:DataDrivenSimpleExample} consists of a UML class diagram that indicates that the two components can send messages whose type can be either \emph{TimeMessage} or \emph{DataMessage}. A \emph{TimeMessage} contains only one field of type Long, which is the timestamp. 
A \emph{DataMessage} contains two fields, one field of type \emph{Integer} capturing the size of the payload, and one array of bytes containing the payload. 
Objective O1 is fulfilled when every mutant leads to the failure of least one test case.
Objective O2 is fulfilled when mutation testing generates at least (i) one \emph{TimeMessage} with field \emph{timestamp} being altered,
(ii) one \emph{DataMessage} with field \emph{size} being altered,
and (iii) one \emph{DataMessage} with field \emph{payload} being altered.
For a test suite consisting of two test cases that trigger the exchange of the messages as in the bottom-left part of Figure~\ref{fig:DataDrivenSimpleExample}, the execution of the bit flip mutation operator may lead to messages that lead to test failures. Since all the mutants are killed (i.e., the two test cases fail), objective O1 is achieved. However, the test suite does not lead to the exchange of any message of type \emph{DataMessage}, for this reason objective O2 is not achieved. Objective O2 enables us to determine that the test suite does not excercise the case in which the two components exchange messages of type \emph{DataMessage}. When data-driven mutation testing is performed against components that are expected to guarantee robustness against the exchange of erroneous data, as a by-product, objective O2 also ensures that components' robustness is properly tested.}



\REVTWO{C34}{Activity \emph{Analyze Results} concerns the automated computation of the \INDEX{mutation score} from execution data. It is 
computed as the weighted average of the percentage of mutants being killed and the percentage of mutation operators applied.}
The former enables data-driven mutation to achieve objective O1 in Section~\ref{sec:dataProcess}, the latter objective O2. 
%Details are provided in Section~\ref{sec:data:mutationscore}.

\REVTWO{C34}{Activity \emph{Analyze Results} takes as input the the data model, the fault model, the list of killed mutants, and the list of mutation operators applied.}
The list of applied mutation operators should enable engineers to determine if all the available mutation operators have been applied.


\begin{figure}[t!]
  \centering
    \includegraphics{images/DataDrivenSimpleExample}
      \caption{Simplified data mutation example.}
      \label{fig:DataDrivenSimpleExample}
\end{figure}

\clearpage



\clearpage
\subsection{FAQAS Fault Model}
\label{sec:dataModel}
\label{sec:faultModel}





A building block of the fault model are a set of \EMPH{data fault classes}.
A  \INDEX{data fault class} captures the type of an error that might affect the data. In turns, it specifies the mutation that should be applied in order to replace valid data with erroneous data. For each fault class we have defined a corresponding \INDEX{data mutation operator} having the same name. Each data mutation operator can be configured with a set of parameters. 

Table~\ref{table:faultModel:FAQAS} provides the list of data fault classes supported by FAQAS along with a description. For each fault class we indicate the data types to which it is expected to be applied, we identify four data types: 
\begin{itemize}
\item int, which indicates an integer
\item float, which indicates a floating point number
\item double, which indicates a double precision floating point number
\item bin, which indicate data that should be treated in its raw form
\end{itemize}


In FAQAS, data-driven mutation testing is performed by modifying either data that is stored in an array or in a data structure defined through the ASN.1 grammar. The following subsections describe the two distinct cases.







\begin{table}[h]
\begin{center}
\small
\begin{tabular}{|p{4cm}|p{10cm}|}
\hline
\textbf{Fault}&\textbf{Description}\\
\hline
Duplicate VCDU/Packet&The same VCDU/Packet appears twice in the transmission.\\
Missing VCDU/Packet& A VCDU/Packet is omitted during transmission.\\
Wrong Sequence& VCDUs/Packets are sent out of order. \\
Incorrect Identifier& Several transmission data fields have fixed values, for example fields identifying the transmitting satellite. Hardware/software errors may assign incorrect identifiers.\\
Incorrect Checksum& Hardware/software errors may result in an incorrect checksum for a Packet or VCDU.\\
Incorrect Counter& Counters are used to track Packet or VCDU ordering. Hardware/software errors may assign incorrect counter values.\\
Flipped Data Bits& Physical channel noise may flip one or more bits in the data transmission.\\
\hline
\end{tabular}
\end{center}
\caption{Fault Model of from \cite{di2015generating}}
\label{table:faultModel:SES}
\end{table}%

\subsubsection{Fault Model Specifications for Data Buffers}

When the data to be mutated is stored in an array, we require the definition of a fault model in the form of a block diagram since this format enables to represent the sequence of data items in the array. 
For each data block the FAQAS fault model captures its type and a list of data fault classes that might affect the block. 
The type of the data block indicates how the values of bits should be interpreted (e.g., as a floating point number). Since a data type may span over multiple items of the data buffer, for each block, we may indicate whether also the values of the following blocks should be used to load the data.

Table~\ref{table:faultModel:example} provides an example of two fault models described in tabular form. 
It resembles the CSV format supported by the FAQAS toolset. For each data item we report the span, type, and fault class. For each fault class we indicate the values of the configuration parameters for the corresponding mutation operator.


% !TEX root = ../MAIN.tex
\begin{table}[h]
\begin{center}
\small
\begin{tabular}{|p{1cm}|p{1.6cm}|p{1cm}|p{1cm}|p{1cm}|p{1cm}|p{1cm}|p{1.6cm}|p{1cm}|p{1cm}|}
\hline
\textbf{Fault Model Name}&\textbf{DataItem}&\textbf{Span}&\textbf{Type}&\textbf{Fault Class}&\textbf{Min}&\textbf{Max}&\textbf{Threshold}&\textbf{Delta}&\textbf{State}\\
\hline
IfHK&0&1&BIN&BF&0&0&-&-&-\\
IfHK&1&1&INT&VOR&0&5&-&1&-\\
IfHK&2&2&INT&VAT&-&-&300&10&-\\
IfHK&4&1&BIN&BF&0&6&-&-&-\\
\hline
IfStatus&0&1&BIN&BF&0&0&-&-&-\\
\hline
\end{tabular}
\end{center}
\caption{Example of a data-driven fault model specified in a (CSV) table.}
\label{table:faultModel:example}
\end{table}%

\clearpage
\subsubsection{Fault Model Specifications for ASN.1 grammar}

When the data to be mutated is stored in a data structure defined through the ASN.1 grammar, the fault model is specified by indicating which operators to apply on the specific fields of the data structure. Mutation operators are automatically configured based on the ASN.1 grammar (e.g., in the case of an attribute of type INTEGER, the min/max values of the VOR operator are derived from the boundaries of the INTEGER type).

\clearpage
\subsection{FAQAS Data Mutation Probes}
\label{sec:FAQASDataMutationProbes}

Data-driven mutation probes are automatically generated from the fault model provided by engineers.
Section~\ref{sec:FAQASDataMutationProbesBuffer} describe how data mutation probes are generated for data buffers.
Section~\ref{sec:FAQASDataMutationProbesASN} covers the case of ASN.1.


\subsubsection{Data Mutation Probes for Data Buffers}
\label{sec:FAQASDataMutationProbesBuffer}

In FAQAS we generate a different executable for each mutation operation to perform. However, the generated source code is the same for all the fault models working on a data buffer of the same type. A configuration option (i.e., a \emph{define directive}) passed to the compiler is what drives the configuration of the specific mutation. Figure~\ref{fig:DataDrivenBufferProcess} provides an overview of the envisioned process. The engineer prepares a single specification file for all the fault models that work with data buffers of a same time (e.g., int). The fault model specification is used by the FAQAS generator to automatically generate the mutation API. The engineer, then, modifies the source code of the SUT to add invocations to the mutation probes provided by the FAQAS API. Finally, the engineer executes the compiler in a for loop such that the compiler is executed with the directive \EMPH{-DMUTATIONOPT=i}, where \emph{i} is a value between 0 and \emph{max}. The value \emph{max} in this case coincides with the
overall number of instances of a mutation operation. An instance of a mutation operation is a mutation operation that belongs to a mutation operator defined for a specific data item of the fault model. For the fault model in Table in \ref{table:faultModel:example} we have 6 instances, one for each data item except for data item 2, whose VOR fault class includes two mutation operations.

\begin{figure}[tb]
  \centering
    \includegraphics{images/DataDrivenBufferProcess}
      \caption{Data-driven mutation process for buffered data.}
      \label{fig:DataDrivenBufferProcess}
\end{figure}

The code that invokes the automatically generated probe is manually inserted by the engineer as shown in Listing~\ref{probesExample}.
The logic of the mutation probes is predefined and shared by all the mutation probes. What determines the different behaviours is the fault model.
The definition of the fault model is automatically generated in a \EMPH{C struct} (see Listing~\ref{faultModelExample}). 
The backbone data structures are predefined an provided in Listing~\ref{faultModelStructure}.

The code of the mutation probe is shown in Listing~\ref{mutationProbe}. It works by identifying the identification of the data item to mutate, the mutation operator to apply, and the mutation operation to execute by invoking methods \EMPH{\_FAQAS\_selectItem},
\EMPH{\_FAQAS\_selectOperator}, and \EMPH{\_FAQAS\_selectOperation}, respectively. The implementation of these three methods is automatically generated based on the fault model definition file.
An example is shown in Listing~\ref{selectors}. The runtime behaviour depends on variable \EMPH{MUTATION}, whose value depends on the option passed at compile time. 
The variable \EMPH{MUTATION} univoquely identifies an instance of a mutation operation (i.e., a mutation operation that belongs to a mutation operator defined for a specific data item of the fault model).

% !TEX root =  ../MAIN.tex

\begin{minipage}{15cm}
\begin{lstlisting}[style=CStyle, caption=Example of a data-driven mutation probes., label=probesExample, mathescape=true]


int main()
{
    // Create a vector containing integers
    std::vector<int> v;

    v.push_back(1);
    v.push_back(2);
    v.push_back(3);
    v.push_back(4);
    v.push_back(5);

    //MANUALLY ADDED PROBE
    FaultModel *fm = _FAQAS_IfHK_FM();
    _FAQAS_mutate(v->data(),fm);
    //MANUALLY ADDED PROBE END

    for(std::vector<int>::iterator it = v.begin(); it != v.end(); ++it) {
        std::cout << *it << '\n';
    }
}
\end{lstlisting}
\end{minipage}



% !TEX root =  ../MAIN.tex

\begin{minipage}{15cm}
\begin{lstlisting}[style=CStyle, caption=Example of generated fault models in C., label=faultModelExample, mathescape=true]
#define SIZE_IfHK 4
#define SIZE_IfStatus 1


struct FaultModel* _FAQAS_IfHK_FM(){
FaultModel *fm = _FAQAS_create_FM(SIZE_IfHK);

fm->items[0].operators[0].type=BF;
fm->items[0].operators[0].min=0;
fm->items[0].operators[0].max=0;
fm->items[0].operators[0].state=-1;
fm->items[0].operatorsN=1;
fm->items[0].span=1;
fm->items[0].type=BIN;

fm->items[1].operators[0].type=VOR;
fm->items[1].operators[0].min=0;
fm->items[1].operators[0].max=5;
fm->items[1].operators[0].delta=1;
fm->items[1].operatorsN=1;
fm->items[1].span=1;
fm->items[1].type=INT;

fm->items[2].operators[0].type=BF;
fm->items[2].operators[0].min=0;
fm->items[2].operators[0].max=0;
fm->items[2].operators[0].state=-1;
fm->items[2].operatorsN=1;
fm->items[2].span=2;
fm->items[2].type=BIN;

fm->items[4].operators[0].type=BF;
fm->items[4].operators[0].min=0;
fm->items[4].operators[0].max=0;
fm->items[4].operators[0].state=-1;
fm->items[4].operatorsN=1;
fm->items[4].span=1;
fm->items[4].type=BIN;
return fm;
}
struct FaultModel* _FAQAS_IfStatus_FM(){
FaultModel *fm = _FAQAS_create_FM(SIZE_IfStatus);

fm->items[0].operators[0].type=BF;
fm->items[0].operators[0].min=0;
fm->items[0].operators[0].max=0;
fm->items[0].operators[0].state=-1;
fm->items[0].operatorsN=1;
fm->items[0].span=1;
fm->items[0].type=BIN;
return fm;
}
\end{lstlisting}
\end{minipage}



% !TEX root =  ../MAIN.tex

\begin{minipage}{15cm}
\begin{lstlisting}[style=CStyle, caption=Fault model data structures., label=faultModelStructure, mathescape=true]
#define MAX_OPS 10
#define ITEMS 10


int MUTATION=MUTATIONOPT;

enum DataType {
    INT,
    FLOAT,
    DOUBLE,
    BIN
};

enum MutationType{
    BF,
    IV,
    VOR,
    VAT,
    VBT,
    INV
};

int _FAQAS_mutated = 0;

struct MutationOperator {
    MutationType type;
    int min;
    int max;
    int delta;
    int state;
};

struct DataItem {
    DataType type;
    int span;
    int operatorsN;
    struct MutationOperator operators[MAX_OPS];
};

struct FaultModel {
    int itemsN;
    struct DataItem *items;
};

\end{lstlisting}
\end{minipage}



% !TEX root =  ../MAIN.tex

\begin{minipage}{15cm}
\begin{lstlisting}[style=CStyle, caption=Automatically generated selectors., label=selectors, mathescape=true]
int _FAQAS_selectItem(FaultModel *dm){
if ( MUTATION == 0 )
    return 0;
if ( MUTATION == 1 )
    return 1;
if ( MUTATION == 2 )
    return 1;
if ( MUTATION == 3 )
    return 2;
if ( MUTATION == 4 )
    return 4;
if ( MUTATION == 5 )
    return 0;
}
int _FAQAS_selectOperator(FaultModel *dm){
if ( MUTATION == 0 )
    return 0;
if ( MUTATION == 1 )
    return 0;
if ( MUTATION == 2 )
    return 0;
if ( MUTATION == 3 )
    return 0;
if ( MUTATION == 4 )
    return 0;
if ( MUTATION == 5 )
    return 0;
}
int _FAQAS_selectOperation(FaultModel *dm){
if ( MUTATION == 0 )
    return 0;
if ( MUTATION == 1 )
    return 0;
if ( MUTATION == 2 )
    return 1;
if ( MUTATION == 3 )
    return 0;
if ( MUTATION == 4 )
    return 0;
if ( MUTATION == 5 )
    return 0;
}
\end{lstlisting}
\end{minipage}



% !TEX root =  ../MAIN.tex

\begin{minipage}{15cm}
\begin{lstlisting}[style=CStyle, caption=Mutation API function., label=mutationProbe, mathescape=true]
int _FAQAS_mutate( int *data, FaultModel *fm ){
    if ( _FAQAS_mutated == 1 )
    return 0;

    if ( MUTATION == -1 )
    return 0;

    int pos = _FAQAS_selectItem(fm);
    int op = _FAQAS_selectOperator(fm);
    int opt = _FAQAS_selectOperation(fm);

    int valueInt;
    int valueBin;
    double valueDouble;


    //Load the data in the appripriate var
    if ( fm->items[pos].type == BIN ){
        valueBin = (int) data[pos];
    }
    if ( fm->items[pos].type == INT ){
        valueInt = (int) data[pos];
    }
    if ( fm->items[pos].type == DOUBLE ){
        valueDouble = (double) data[pos];
    }
...
    MutationOperator *OP = &(fm->items[pos].operators[op]);    
...     
        if ( OP->type == VOR ){
        if ( fm->items[pos].type == INT ){

            if ( opt == 0 ){
                valueInt = OP->min-OP->delta;
            } else if (opt == 1 ){
                valueInt = OP->max+OP->delta;
            } else {
                //ERROR
            }
            _FAQAS_mutated = 1;
        }

...

    }

...

    if ( _FAQAS_mutated != 1 ){
        return 0;
    }

    //
    //Store the data
    //
    //FIXME: handle span
    if ( fm->items[pos].type == INT ){
        data[pos] = valueInt;
    }
    if ( fm->items[pos].type == DOUBLE ){
        data[pos] = valueDouble;
    }
    if ( fm->items[pos].type == BIN ){
        data[pos] = valueBin;
    }
    
    ...

    
\end{lstlisting}
\end{minipage}




\subsubsection{Data Mutation Probes for ASN.1}
\label{sec:FAQASDataMutationProbesASN}


\clearpage
\subsection{Test suite execution}
\label{sec:mutantsExecution}

During data mutation the test suite is executed a number of times that depends on a stopping criterion chosen by the engineer. We foresee three possible stopping criteria (1) exercise each test case against a data fault class (hereafter, test coverage stopping criterion), (2) 
exercise each data fault class (hereafter, fault coverage stopping criterion), and (3) sample the mutations to apply against a certain test case (hereafter, sampling stopping criterion).


With the test coverage stopping criterion, each test case is executed multiple times till all the possible data fault classes had been injected.
A single data fault is injected for each run.
A data fault class is no longer injected after it has been determined that can it be killed by the test suite (i.e., a previously executed test case failed once the mutation was applied).
We indicate that a data item is mutable when (1) the fault model includes the definition of a data fault class \emph{c} for the data item and (2) the data fault class \emph{c} has not been already killed by the test suite.
The mutation algorithm can either mutate the first mutable data item observed or randomly decide whether to mutate the mutable data item. The second case enables the mutation of data items exchanged after long components interactions.
The repeated execution of a test case is terminated after it has been executed once without identifying any mutable data item.

With the fault coverage stopping criterion the full test suite is executed multiple times till all the possible data fault classes had been injected at least once (for the full test suite).
A data fault class is no longer injected after it has already been injected in a test case of the test suite.
The mutation algorithm can either mutate the first mutable data item observed or randomly decide whether to mutate the mutable data item.
The repeated execution of a test case is terminated after it has been executed once without identifying any mutable data item.

With the sampling stopping criterion each test case is executed multiple times as for the test coverage stopping criterion with the difference that (1) the injection of data faults follow a given distribution (e.g., only 5\% of the mutable data items are mutated) and (2) the repeated execution of a test case is terminated after no mutation has been performed.



\clearpage
\section{Test Suite Augmentation} % (fold)
\label{sec:data:test_suite_augmentation}

The test suite augmentation process concerns the definition of additional test cases to increase the mutation score.
It consists of four activities \emph{Identify Test Inputs}, \emph{Generate Test Oracles}, \emph{Execute the SUT}, \emph{Fix the SUT}. 
Despite these activities match the ones performed in the case of code-driven mutation testing, they are triggered and implemented in a different manner, as described below.

%The first two activities concern the definition of new test cases and the improvement of existing test cases.



In the presence of mutants not killed by test cases (i.e., when the \emph{\% of Mutants Killed} is not equal to 100\%), engineers are expected to improve the oracles of existing test cases. Indeed, the presence of mutants not killed by test cases indicates that the oracles of the test suite are not capable of detecting that the software is failing. 
Automated approaches for performing this activity in the presence of system or integration test suites are not available and thus it needs to be performed manually.

In the presence of operators not being applied (i.e., the \emph{\% Operators Applied} is not equal to 100\%), engineers are expected to generate new test inputs for the SUT that enable the application of all the mutation operators. 
For example, in the case of the example in Figure~\ref{fig:DataDrivenSimpleExample}, engineers would need to implement test cases that trigger the exchange of \emph{DataMessages}.
%Another example is that of components that exchange message sequences, in the case the test suite trigger only the exchange of message sequences containing only one message. 
%Indeed, in this situation, the mutation operator concerning the swapping of two packets cannot be applied. In this case, to achieve 100\% of Operators Applied, engineers should generate inputs that enable the application of the mutation operator (e.g., inputs that lead to a sequence of packets containing more than one packet). 
Fully automated approaches to generate test cases for data-driven mutation testing are unavailable; however, techniques that generate input data from scratch~\cite{gligoric2010test} or augment input data~\cite{DiNardo:TOSEM:2017} can be adopted. 
Also, when the data used by test cases is generated by simulators, meta-heuristic search can be used to drive the generation of input data~\cite{Abdessalem:ICSE:2018}. 

The execution of the SUT and the repair of the SUT are performed manually as in the case of code-driven data mutation.


\TODO{Clarify if we generate test cases or not}

Section~\ref{sec:testGenerationData} provides details about the existing solutions to  \emph{Identify Test Inputs} and \emph{Generate Test Oracles}.


