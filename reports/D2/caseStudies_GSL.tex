% !TEX root = MAIN.tex


\section{GSL - libgcsp}
\label{sec:caseStudies:GSL:libgcsp}

\subsection{Overview of the case study}

The GomSpace CSP library (libgscsp) is a GomSpace extension to the open source CubeSat Space Protocol library.
The GomSpace CSP library provides:
\begin{itemize}
\item convience wrapping of CSP functionality, primarily initialization.
\item  definition of standard CSP ports (used by other GomSpace products).
\item connecting low-level drivers (e.g. CAN, I2C from Embed library) with CSP interfaces 
\item generic CSP service dispatcher, forwards incoming connections to service handlers.
\end{itemize}

The libgscsp contains a GomSpace branch (https://github.com/GomSpace/libcsp) of the open source libcsp (https://github.com/libcsp/libcsp), located in the subfolder lib/libcsp. The two libcsp branches are kept as identical as possible, as features specific to GomSpace are placed in libgscsp.

Details about libgcsp are provided in the document \emph{gs-man-nanosoft-ms100-command-and-management-sdk-3.6.2-1-g67fe6e1.pdf} uploaded on Alfresco.

\TODO{Add details about size}

\TODO{Can we provide separate information about the code coverage for libgcsp (no libcsp branch) and for libcsp branch?}

\subsection{Code-driven mutation testing}

\TODO{Clarify which components we mutate}



\subsection{Data-driven mutation testing}

\TODO{We should indicate the functions we mutate.}



\section{GSL - libparam}
\label{sec:caseStudies:GSL:libparam}

\subsection{Overview of the case study}

The Parameter System (i.e., libparam) is a light-weight parameter system designed for GomSpace satellite subsystems. It is based around a logical memory architecture, where every parameter is referenced directly by its logical address. A backend system takes care of translating addresses into physical addresses.
The features of this system include:
\begin{itemize}
\item Direct memory access for quick parameter reads.
\item Simple data types: uint, int, float, double, string.
\item Arrays of simple data types.
\item Supports multiple stores per table, e.g. FRAM, MCU flash, file (binary or text).
\item Remote client with support for most features (rparam).
\item Packed GET, SET queries, supporting multiple parameter set/get in a single request. ? Data serialization and deserialization.
\item Supports both little and big-endian systems.
\item Commands for both local (param) and remote access (rparam).
\item Parameter server for remote access over CSP.
\item Compile-time configuration of parameter system
\end{itemize}

Details about libparam are provided in the document \emph{gs-man-nanosoft-ms100-command-and-management-sdk-3.6.2-1-g67fe6e1.pdf} uploaded on Alfresco.

\subsection{Code-driven mutation testing}

\TODO{We do it or not?}

\subsection{Data-driven mutation testing}

\TODO{We should indicate the functions we mutate}




\section{GSL - libutil}
\label{sec:caseStudies:GSL:libutil}

\subsection{Overview of the case study}

The Utility library provides cross-platform API?s for common functionality, for use in both embedded systems and standard PC?s running Linux. 

Details about libutil are provided in the document \emph{gs-man-nanosoft-ms100-command-and-management-sdk-3.6.2-1-g67fe6e1.pdf} uploaded on Alfresco.


\subsection{Code-driven mutation testing}

\TODO{Add some details about what we mutated}

\TODO{We can add some preliminary results}

\subsection{Data-driven mutation testing}

\TODO{We should indicate the functions we mutate}

