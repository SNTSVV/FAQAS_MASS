
% !TEX root = MutationTestingSurvey.tex

\chapter{Code improvement}

Despite the main objective of mutation testing is to assess the quality of existing test suites. The mutation testing approach has been used also in literature as a way to improve code quality,
in fact Arcaini et al.~\cite{arcaini2017novel,arcaini2015rehabilitating} and then Lopez et al.~\cite{lopez2018source} proposed the idea of taking advantage of \INDEX{equivalent mutants} for improving the quality of the artefact under test.
More precisely, they proposed that equivalent mutants can be used to improve \INDEX{code quality} properties, i.e., readability and efficiency, and for refactoring purposes.

% !TEX root =  ../MutationTestingSurvey.tex

\noindent\begin{minipage}{.45\textwidth}
\begin{lstlisting}[style=CStyle, caption=Original source code., label=equivalent1]
	x = 3;
	y = 6;
	z = x * 2;
\end{lstlisting}
\end{minipage}\hfill
\begin{minipage}{.45\textwidth}
\begin{lstlisting}[style=CStyle, caption=Equivalent mutant., label=equivalent2, mathescape=true]
	 x = 3;
$\Delta$ z = x * 2;
$\Delta$ y = 6;
\end{lstlisting}
\end{minipage}

In the case of code readability, for example, the mutation operator that swaps two statements in a program may produce equivalent mutants, given that both statements are independent, and casually improve qualities such as readability an efficiency of compiling. 
Listing~\ref{equivalent1} and~\ref{equivalent2} introduce this case. The mutant of Listing~\ref{equivalent2} has been produced through of the \textit{statement reordering operator} on the code excerpt from Listing~\ref{equivalent1}.
Particularly, the equivalent mutant shown in Listing~\ref{equivalent2} has better readability than the original code, because the distance between the definition and use of the variable \texttt{x} is shorter than in the original version.

