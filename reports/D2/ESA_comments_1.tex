% !TEX root = MAIN.tex

\section{Responses to ESA comments provided on 03.03.2020}
\label{sec:ESA:comments:1}


\setlength\LTleft{0pt}
\setlength\LTright{0pt}
\tiny 
%@{\extracolsep{\fill}}
\begin{longtable}{|p{1.5cm}|p{12cm}|@{}}
%\caption{\normalsize .}
\label{table:comments:responses} 
\textbf{Comment ID}&\textbf{Response}\\
\\
\hline
CSR-PABG-1&
\begin{minipage}{12cm}
\TODO{}
\end{minipage}\\
\\
\hline

%To generate the table:
%x=2;while [ $x -lt 54 ];do echo "CSR-PABG-$x&"; echo "\begin{minipage}{12cm}"; echo "\end{minipage}\\"; echo "\\"; echo "\hline"; x=$(($x+1)); done
%echo "CSR-PABG-$x&"; echo "\begin{minipage}{12cm}"; echo "\end{minipage}\\"; echo "\hline"

CSR-PABG-2&
\begin{minipage}{12cm}
\TODO{}
\end{minipage}\\
\hline
CSR-PABG-3&
\begin{minipage}{12cm}
Done.
\end{minipage}\\
\hline
CSR-PABG-4&
\begin{minipage}{12cm}
\TODO{}
\end{minipage}\\
\hline
CSR-PABG-5&
\begin{minipage}{12cm}
\TODO{}
\end{minipage}\\
\hline
CSR-PABG-6&
\begin{minipage}{12cm}
Sorry for that. We rephrased a couple of sentences in that section.
\end{minipage}\\
\hline
CSR-PABG-7&
\begin{minipage}{12cm}
The definition appears in D1 Section 1.2.5.
\end{minipage}\\
\hline
CSR-PABG-8&
\begin{minipage}{12cm}
\end{minipage}\\
\hline
CSR-PABG-9&
\begin{minipage}{12cm}
Yes, we have defined a RQ to evaluate SDL alone.
\end{minipage}\\
\hline
CSR-PABG-10&
\begin{minipage}{12cm}
Compile time optimizations appear not to be necessary if a smart compilation process is in place. Basicaly, the compilation routine that we propose (i.e., copy the mutated file inside a compiled source tree) is somehow an optmization.
\end{minipage}\\
\hline
CSR-PABG-11&
\begin{minipage}{12cm}
We addressed in the text. Section~\ref{sec:codemutation:evaluation}.
\end{minipage}\\
\hline
CSR-PABG-12&
\begin{minipage}{12cm}
\TODO{Oscar :)}
\end{minipage}\\
\hline
CSR-PABG-13&
\begin{minipage}{12cm}
We agree. For FAQAS, we are already addressing the problem of collecting fine-grained code coverage for ESAL (there is no filesystem thus gcov needs to be combined with gdb, according to guidelines from Thanassis Tsiodras at https://www.thanassis.space/coverage.html). To address the comment, we included the research questions RQ7 and RQ9.
\end{minipage}\\
\hline
CSR-PABG-14&
\begin{minipage}{12cm}
We did a mistake when copying files from an ongoing paper. Now Table~\ref{table:operators} is in line with the rest.
\end{minipage}\\
\hline
CSR-PABG-15&
\begin{minipage}{12cm}
We addressed the comment in Step 4 and in RQ1.
\end{minipage}\\
\hline
CSR-PABG-16&
\begin{minipage}{12cm}
RQ2 address this question.
\end{minipage}\\
\hline
CSR-PABG-17&
\begin{minipage}{12cm}
We rewrote line 1 to clarify where t comes from.
We clarified the rest in the text.
\end{minipage}\\
\hline
CSR-PABG-18&
\begin{minipage}{12cm}
Since these distances have nothing to do with mutation testing but it is part of our contribution, we prefer to not add to D1 (we would not have an appropriate section for them). If necessary we can provide additional description in D2.
\end{minipage}\\
\hline
CSR-PABG-19&
\begin{minipage}{12cm}
To clarify, we added the complete algorithm in Figure~\ref{alg:nonEquivalent:nonRedeundat}.
\end{minipage}\\
\hline
CSR-PABG-20&
\begin{minipage}{12cm}
KLEE appear more mature than CBMC, it is our primary choice at this stage.

The licence of both CBMC and KLEE appear to be compatible. The simplest solution is to not release CBMC/KLEE with the FAQAS framework but require their installation in a specific folder as a dependency. 

\textbf{CBMC LICENCE:}

\begin{verbatim}
(C) 2001-2017, Daniel Kroening, Edmund Clarke,
Computer Science Department, University of Oxford
Computer Science Department, Carnegie Mellon University

All rights reserved. Redistribution and use in source and binary forms, with
or without modification, are permitted provided that the following
conditions are met:

  1. Redistributions of source code must retain the above copyright
     notice, this list of conditions and the following disclaimer.

  2. Redistributions in binary form must reproduce the above copyright
     notice, this list of conditions and the following disclaimer in the
     documentation and/or other materials provided with the distribution.

  3. All advertising materials mentioning features or use of this software
     must display the following acknowledgement:

     This product includes software developed by Daniel Kroening,
     Edmund Clarke,
     Computer Science Department, University of Oxford
     Computer Science Department, Carnegie Mellon University

  4. Neither the name of the University nor the names of its contributors
     may be used to endorse or promote products derived from this software
     without specific prior written permission.

   
THIS SOFTWARE IS PROVIDED BY THE COPYRIGHT HOLDERS AND CONTRIBUTORS "AS IS"
AND ANY EXPRESS OR IMPLIED WARRANTIES, INCLUDING, BUT NOT LIMITED TO, THE
IMPLIED WARRANTIES OF MERCHANTABILITY AND FITNESS FOR A PARTICULAR PURPOSE
ARE DISCLAIMED. IN NO EVENT SHALL THE COPYRIGHT HOLDER OR CONTRIBUTORS BE
LIABLE FOR ANY DIRECT, INDIRECT, INCIDENTAL, SPECIAL, EXEMPLARY, OR
CONSEQUENTIAL DAMAGES (INCLUDING, BUT NOT LIMITED TO, PROCUREMENT OF
SUBSTITUTE GOODS OR SERVICES; LOSS OF USE, DATA, OR PROFITS; OR BUSINESS
INTERRUPTION) HOWEVER CAUSED AND ON ANY THEORY OF LIABILITY, WHETHER IN
CONTRACT, STRICT LIABILITY, OR TORT (INCLUDING NEGLIGENCE OR OTHERWISE)
ARISING IN ANY WAY OUT OF THE USE OF THIS SOFTWARE, EVEN IF ADVISED OF THE
POSSIBILITY OF SUCH DAMAGE.
\end{verbatim}
\end{minipage}\\


&
\begin{minipage}{12cm}
\textbf{KLEE LICENCE:}
\begin{verbatim}
==============================================================================
KLEE Release License
==============================================================================
University of Illinois/NCSA
Open Source License

http://klee.github.io/

Developed by:
    KLEE Team
    Stanford Checking Group

Copyright (c) 2007-2009 Stanford University.
All rights reserved.


Maintained since 2009 by:
    Software Reliability Group
    http://srg.doc.ic.ac.uk/
    Imperial College London


Improved and extended since 2009 by many developers.  For a full list
of contributors, refer to
         https://github.com/klee/klee/graphs/contributors
and the Git commit history.


Permission is hereby granted, free of charge, to any person obtaining a copy of
this software and associated documentation files (the "Software"), to deal with
the Software without restriction, including without limitation the rights to
use, copy, modify, merge, publish, distribute, sublicense, and/or sell copies
of the Software, and to permit persons to whom the Software is furnished to do
so, subject to the following conditions:

    * Redistributions of source code must retain the above copyright notice,
      this list of conditions and the following disclaimers.

    * Redistributions in binary form must reproduce the above copyright notice,
      this list of conditions and the following disclaimers in the
      documentation and/or other materials provided with the distribution.

    * Neither the names of the KLEE Team, Stanford University,
      Imperial College London, nor the names of its contributors may
      be used to endorse or promote products derived from this
      Software without specific prior written permission.

THE SOFTWARE IS PROVIDED "AS IS", WITHOUT WARRANTY OF ANY KIND, EXPRESS OR
IMPLIED, INCLUDING BUT NOT LIMITED TO THE WARRANTIES OF MERCHANTABILITY, FITNESS
FOR A PARTICULAR PURPOSE AND NONINFRINGEMENT.  IN NO EVENT SHALL THE
CONTRIBUTORS OR COPYRIGHT HOLDERS BE LIABLE FOR ANY CLAIM, DAMAGES OR OTHER
LIABILITY, WHETHER IN AN ACTION OF CONTRACT, TORT OR OTHERWISE, ARISING FROM,
OUT OF OR IN CONNECTION WITH THE SOFTWARE OR THE USE OR OTHER DEALINGS WITH THE
SOFTWARE.

==============================================================================
The KLEE software contains code written by third parties.  Such software will
have its own individual LICENSE.TXT file in the directory in which it appears.
This file will describe the copyrights, license, and restrictions which apply
to that code.

The disclaimer of warranty in the University of Illinois Open Source
License applies to all code in the KLEE distribution, and nothing in
any of the other licenses gives permission to use the names of the
KLEE team, Stanford University, or the names of its contributors to
endorse or promote products derived from this Software.

The following pieces of software have additional or alternate copyrights,
licenses, and/or restrictions:

Program             Directory
-------             ---------
klee-libc           runtime/klee-libc
\end{verbatim}

\end{minipage}\\
\hline
CSR-PABG-21&
\begin{minipage}{12cm}
SRCIror licence appear to be compatible. We provide it below.

\begin{verbatim}
University of Illinois/NCSA
Open Source License

Copyright (c) 2018 University of Illinois at Urbana-Champaign.
All rights reserved.

Developed by:       Darko Marinov Group
                    University of Illinois at Urbana-Champaign

Permission is hereby granted, free of charge, to any person obtaining a copy
of this software and associated documentation files (the "Software"), to
deal with the Software without restriction, including without limitation the
rights to use, copy, modify, merge, publish, distribute, sublicense, and/or
sell copies of the Software, and to permit persons to whom the Software is
furnished to do so, subject to the following conditions:
 1. Redistributions of source code must retain the above copyright notice,
     this list of conditions and the following disclaimers.
 2. Redistributions in binary form must reproduce the above copyright
    notice, this list of conditions and the following disclaimers in the
    documentation and/or other materials provided with the distribution.
 3. Neither the names of NCSA, University of Illinois, nor the names of its
    contributors may be used to endorse or promote products derived from this
    Software without specific prior written permission.

THE SOFTWARE IS PROVIDED "AS IS", WITHOUT WARRANTY OF ANY KIND, EXPRESS OR
IMPLIED, INCLUDING BUT NOT LIMITED TO THE WARRANTIES OF MERCHANTABILITY,
FITNESS FOR A PARTICULAR PURPOSE AND NONINFRINGEMENT.  IN NO EVENT SHALL THE
CONTRIBUTORS OR COPYRIGHT HOLDERS BE LIABLE FOR ANY CLAIM, DAMAGES OR OTHER
LIABILITY, WHETHER IN AN ACTION OF CONTRACT, TORT OR OTHERWISE, ARISING
FROM, OUT OF OR IN CONNECTION WITH THE SOFTWARE OR THE USE OR OTHER DEALINGS
WITH THE SOFTWARE.

Parts of this code has been adapted from the following two repositories:
https://github.com/eliben
https://github.com/eschulte/clang-mutate
\end{verbatim}
\end{minipage}\\
\hline
CSR-PABG-22&
\begin{minipage}{12cm}
\TODO{Oscar}
\end{minipage}\\
\hline
CSR-PABG-23&
\begin{minipage}{12cm}
The selected tool is largely extended and integrated into our framework. The input interface will be however, different than the original one.
\TODO{Oscar}
\end{minipage}\\
\hline
CSR-PABG-24&
\begin{minipage}{12cm}
We addressed the comment in Section~\ref{codeDriven:stepThree}.
\end{minipage}\\
\hline
CSR-PABG-25&
\begin{minipage}{12cm}
\TODO{Oscar}
\end{minipage}\\
\hline
CSR-PABG-26&
\begin{minipage}{12cm}
\TODO{Oscar}
\end{minipage}\\
\hline
CSR-PABG-27&
\begin{minipage}{12cm}
We suggest to provide such research question for the end of WP2.
\end{minipage}\\
\hline
CSR-PABG-28&
\begin{minipage}{12cm}
We added a new section in Chapter~\ref{chapter:datamutation} regarding the evaluation of the Peach data-driven mutation toolset.
\end{minipage}\\
\hline
CSR-PABG-29&
\begin{minipage}{12cm}
\end{minipage}\\
\hline
CSR-PABG-30&
\begin{minipage}{12cm}
\end{minipage}\\
\hline
CSR-PABG-31&
\begin{minipage}{12cm}
\end{minipage}\\
\hline
CSR-PABG-32&
\begin{minipage}{12cm}
\end{minipage}\\
\hline
CSR-PABG-33&
\begin{minipage}{12cm}
\end{minipage}\\
\hline
CSR-PABG-34&
\begin{minipage}{12cm}
\end{minipage}\\
\hline
CSR-PABG-35&
\begin{minipage}{12cm}
\end{minipage}\\
\hline
CSR-PABG-36&
\begin{minipage}{12cm}
\end{minipage}\\
\hline
CSR-PABG-37&
\begin{minipage}{12cm}
\end{minipage}\\
\hline
CSR-PABG-38&
\begin{minipage}{12cm}
\end{minipage}\\
\hline
CSR-PABG-39&
\begin{minipage}{12cm}
\end{minipage}\\
\hline
CSR-PABG-40&
\begin{minipage}{12cm}
\end{minipage}\\
\hline
CSR-PABG-41&
\begin{minipage}{12cm}
\end{minipage}\\
\hline
CSR-PABG-42&
\begin{minipage}{12cm}
\end{minipage}\\
\hline
CSR-PABG-43&
\begin{minipage}{12cm}
\end{minipage}\\
\hline
CSR-PABG-44&
\begin{minipage}{12cm}
We added a brief description in Section~\ref{sec:caseStudies:GSL:libgcsp} about the test framework, test suite and HPC configuration used in the GSL case studies.
\end{minipage}\\
\hline
CSR-PABG-45&
\begin{minipage}{12cm}
We clarified in Section~\ref{sec:caseStudies:GSL:libgcsp} that the use case libgscsp is covering integration testing.
\end{minipage}\\
\hline
CSR-PABG-46&
\begin{minipage}{12cm}

\begin{itemize}
	\item We should be assessing a test suite as a whole, and if some statements are not covered, this is already a sign that the test suite may not be complete enough.

	
	The point is that if a statement is not covered by the test suite, there is absolutely no chance that a mutant generated in the non-covered statement can be possibly detected by any test case (see \INDEX{killing conditions} of a mutant in D1 Section 1.3.1). For this reason, we make the assumption that test suites are of high quality standards, and therefore, we consider only the covered statements.

	\item When computing the mutation score, is it fare to not count those not-covered lines?

	Yes, is it fare because mutation testing assess the quality of existing test suites, and considering non-covered statements would be out of the scope of the technique. Also, including non-covered statements into the mutation score, would not let us to assess correctly the quality of the existing test suite.

  	\item Maybe the approach would be to run the test augmentation process to provide test cases for the missing statements??

	\TODO{I think the literature does not cover this issue}
	I think this is not possible, because we would be missing the test oracle. 
Also, in that case, we would be assessing through mutation testing the quality of the test augmentation process instead of the quality of the existing test suite.

\end{itemize}
\end{minipage}\\
\hline
CSR-PABG-47&
\begin{minipage}{12cm}
We clarified the scope of our analysis for the case study libgscsp in Section~\ref{sec:caseStudies:GSL:libgcsp}. 
\end{minipage}\\
\hline
CSR-PABG-48&
\begin{minipage}{12cm}
We clarified in Section~\ref{subsec:libgscsp_data} the mutations to be applied to the payload, and how they would be detected by the existing test suite.
\end{minipage}\\
\hline
CSR-PABG-49&
\begin{minipage}{12cm}
We added more details about the provided test suite and the scope of libparam in FAQAS.
\end{minipage}\\
\hline
CSR-PABG-50&
\begin{minipage}{12cm}
\TODO{I did not change the text for this comment, below there is my answer.}
The code coverage reported in Table~\ref{table:gslibutil_coverage} considers all the 201 test cases for all the source code of libutil.
\end{minipage}\\
\hline
CSR-PABG-51&
\begin{minipage}{12cm}
\end{minipage}\\
\hline
CSR-PABG-52&
\begin{minipage}{12cm}
\end{minipage}\\
\hline
CSR-PABG-53&
\begin{minipage}{12cm}
\end{minipage}\\
\hline


%Author: Pedro Barrios Subject: Sticky Note Date: 28/02/2020, 09:18:11
%Comment #4:
%It is perhaps interesting to add a chapter with some basic definitions?
%e.g. equivalent mutant, redundant mutant, mutation score, killed mutant, live mutant, weak mutation, ...
%Author: Pedro Barrios Subject: Sticky Note Date: 28/02/2020, 09:18:18
%Comment #5:
%Please, consider to add more examples.

% \bottomrule                                                             
\end{longtable}
\normalsize

\clearpage