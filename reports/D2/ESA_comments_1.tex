% !TEX root = MAIN.tex

\section{Responses to ESA comments provided on 03.03.2020}
\label{sec:ESA:comments:1}


\setlength\LTleft{0pt}
\setlength\LTright{0pt}
\tiny 
%@{\extracolsep{\fill}}
\begin{longtable}{|p{1.5cm}|p{12cm}|@{}}
%\caption{\normalsize .}
\label{table:comments:responses} 
\textbf{Comment ID}&\textbf{Response}\\
\\
\hline
CSR-PABG-1&
\begin{minipage}{12cm}
\TODO{}
\end{minipage}\\
\\
\hline

%To generate the table:
%x=2;while [ $x -lt 54 ];do echo "CSR-PABG-$x&"; echo "\begin{minipage}{12cm}"; echo "\end{minipage}\\"; echo "\\"; echo "\hline"; x=$(($x+1)); done
%echo "CSR-PABG-$x&"; echo "\begin{minipage}{12cm}"; echo "\end{minipage}\\"; echo "\hline"

CSR-PABG-2&
\begin{minipage}{12cm}
\TODO{}
\end{minipage}\\
\hline
CSR-PABG-3&
\begin{minipage}{12cm}
Done.
\end{minipage}\\
\hline
CSR-PABG-4&
\begin{minipage}{12cm}
\TODO{}
\end{minipage}\\
\hline
CSR-PABG-5&
\begin{minipage}{12cm}
\TODO{}
\end{minipage}\\
\hline
CSR-PABG-6&
\begin{minipage}{12cm}
\end{minipage}\\
\hline
CSR-PABG-7&
\begin{minipage}{12cm}
\end{minipage}\\
\hline
CSR-PABG-8&
\begin{minipage}{12cm}
\end{minipage}\\
\hline
CSR-PABG-9&
\begin{minipage}{12cm}
\end{minipage}\\
\hline
CSR-PABG-10&
\begin{minipage}{12cm}
\end{minipage}\\
\hline
CSR-PABG-11&
\begin{minipage}{12cm}
\end{minipage}\\
\hline
CSR-PABG-12&
\begin{minipage}{12cm}
\end{minipage}\\
\hline
CSR-PABG-13&
\begin{minipage}{12cm}
\end{minipage}\\
\hline
CSR-PABG-14&
\begin{minipage}{12cm}
\end{minipage}\\
\hline
CSR-PABG-15&
\begin{minipage}{12cm}
\end{minipage}\\
\hline
CSR-PABG-16&
\begin{minipage}{12cm}
\end{minipage}\\
\hline
CSR-PABG-17&
\begin{minipage}{12cm}
\end{minipage}\\
\hline
CSR-PABG-18&
\begin{minipage}{12cm}
\end{minipage}\\
\hline
CSR-PABG-19&
\begin{minipage}{12cm}
\end{minipage}\\
\hline
CSR-PABG-20&
\begin{minipage}{12cm}
\end{minipage}\\
\hline
CSR-PABG-21&
\begin{minipage}{12cm}
\end{minipage}\\
\hline
CSR-PABG-22&
\begin{minipage}{12cm}
\end{minipage}\\
\hline
CSR-PABG-23&
\begin{minipage}{12cm}
\end{minipage}\\
\hline
CSR-PABG-24&
\begin{minipage}{12cm}
\end{minipage}\\
\hline
CSR-PABG-25&
\begin{minipage}{12cm}
\end{minipage}\\
\hline
CSR-PABG-26&
\begin{minipage}{12cm}
\end{minipage}\\
\hline
CSR-PABG-27&
\begin{minipage}{12cm}
\end{minipage}\\
\hline
CSR-PABG-28&
\begin{minipage}{12cm}
\end{minipage}\\
\hline
CSR-PABG-29&
\begin{minipage}{12cm}
\end{minipage}\\
\hline
CSR-PABG-30&
\begin{minipage}{12cm}
\end{minipage}\\
\hline
CSR-PABG-31&
\begin{minipage}{12cm}
\end{minipage}\\
\hline
CSR-PABG-32&
\begin{minipage}{12cm}
\end{minipage}\\
\hline
CSR-PABG-33&
\begin{minipage}{12cm}
\end{minipage}\\
\hline
CSR-PABG-34&
\begin{minipage}{12cm}
\end{minipage}\\
\hline
CSR-PABG-35&
\begin{minipage}{12cm}
\end{minipage}\\
\hline
CSR-PABG-36&
\begin{minipage}{12cm}
\end{minipage}\\
\hline
CSR-PABG-37&
\begin{minipage}{12cm}
\end{minipage}\\
\hline
CSR-PABG-38&
\begin{minipage}{12cm}
\end{minipage}\\
\hline
CSR-PABG-39&
\begin{minipage}{12cm}
\end{minipage}\\
\hline
CSR-PABG-40&
\begin{minipage}{12cm}
\end{minipage}\\
\hline
CSR-PABG-41&
\begin{minipage}{12cm}
\end{minipage}\\
\hline
CSR-PABG-42&
\begin{minipage}{12cm}
\end{minipage}\\
\hline
CSR-PABG-43&
\begin{minipage}{12cm}
\end{minipage}\\
\hline
CSR-PABG-44&
\begin{minipage}{12cm}
We added a brief description in Section~\ref{sec:caseStudies:GSL:libgcsp} about the test framework, test suite and HPC configuration used in the GSL case studies.
\end{minipage}\\
\hline
CSR-PABG-45&
\begin{minipage}{12cm}
We clarified in Section~\ref{sec:caseStudies:GSL:libgcsp} that the use case libgscsp is covering integration testing.
\end{minipage}\\
\hline
CSR-PABG-46&
\begin{minipage}{12cm}

\begin{itemize}
	\item We should be assessing a test suite as a whole, and if some statements are not covered, this is already a sign that the test suite may not be complete enough.

	
	The point is that if a statement is not covered by the test suite, there is absolutely no chance that a mutant generated in the non-covered statement can be possibly detected by any test case (see \INDEX{killing conditions} of a mutant in D1 Section 1.3.1). For this reason, we make the assumption that test suites are of high quality standards, and therefore, we consider only the covered statements.

	\item When computing the mutation score, is it fare to not count those not-covered lines?

	Yes, is it fare because mutation testing assess the quality of existing test suites, and considering non-covered statements would be out of the scope of the technique. Also, including non-covered statements into the mutation score, would not let us to assess correctly the quality of the existing test suite.

  	\item Maybe the approach would be to run the test augmentation process to provide test cases for the missing statements??

	\TODO{I think the literature does not cover this issue}
	I think this is not possible, because we would be missing the test oracle. 
Also, in that case, we would be assessing through mutation testing the quality of the test augmentation process instead of the quality of the existing test suite.

\end{itemize}
\end{minipage}\\
\hline
CSR-PABG-47&
\begin{minipage}{12cm}
We clarified the scope of our analysis for the case study libgscsp in Section~\ref{sec:caseStudies:GSL:libgcsp}. 
\end{minipage}\\
\hline
CSR-PABG-48&
\begin{minipage}{12cm}
\TODO{How to mutate the payload data? Would those mutations be detectable by the test suite? (typically, the SW does not look into that data, apart perhaps of a checksum for the packet).}
\TODO{We should ask GSL if libgscsp test suite perform checks on the payload data.}
\end{minipage}\\
\hline
CSR-PABG-49&
\begin{minipage}{12cm}
\end{minipage}\\
\hline
CSR-PABG-50&
\begin{minipage}{12cm}
\end{minipage}\\
\hline
CSR-PABG-51&
\begin{minipage}{12cm}
\end{minipage}\\
\hline
CSR-PABG-52&
\begin{minipage}{12cm}
\end{minipage}\\
\hline
CSR-PABG-53&
\begin{minipage}{12cm}
\end{minipage}\\
\hline


%Author: Pedro Barrios Subject: Sticky Note Date: 28/02/2020, 09:18:11
%Comment #4:
%It is perhaps interesting to add a chapter with some basic definitions?
%e.g. equivalent mutant, redundant mutant, mutation score, killed mutant, live mutant, weak mutation, ...
%Author: Pedro Barrios Subject: Sticky Note Date: 28/02/2020, 09:18:18
%Comment #5:
%Please, consider to add more examples.

% \bottomrule                                                             
\end{longtable}
\normalsize

\clearpage