% !TEX root = MAIN.tex

\chapter{Baseline Documents}
\label{chapter:baseline}

Table~\ref{table:baseline} contains the documents baseline applicable to this CI.

\begin{table}[h]
\caption{Documents baseline.}
\label{table:baseline} 
\footnotesize
\centering
\begin{tabular}{|
@{\hspace{1pt}}p{20mm}@{\hspace{0pt}}|
@{\hspace{0pt}}p{45mm}@{\hspace{1pt}}|
@{\hspace{0pt}}p{10mm}@{\hspace{1pt}}|
@{\hspace{3pt}}p{55mm}@{\hspace{1pt}}|
p{4mm}|}
\hline
\textbf{Identifier}&\textbf{Reference}&\textbf{Issue}&\textbf{Description}\\
\hline
SSS & ITT-1-9873-ESA-FAQAS-SSS & 2.2 & Software System Specifications \\
IRD & ITT-1-9873-ESA-FAQAS-IRD & 1.0 & Interface Requirements Document \\
SCF & ITT-1-9873-ESA-FAQAS-SCF & 1.0 & Software Configuration File\\
SDD & ITT-1-9873-ESA-FAQAS-SDD & 3.1 & Software Design Document \\
SRF & ITT-1-9873-ESA-FAQAS-SRF & 1.0 & Software Reuse File \\
SUM & ITT-1-9873-ESA-FAQAS-SUM & 3.1 & Software User Manual \\
SUITP& ITT-1-9873-ESA-FAQAS-SUITP & 3.1 & Software Unit Test Plan \\
SRelD & ITT-1-9873-ESA-FAQAS-SRelD & 1.0 & Software Release Document \\
SPA & ITT-1-9873-ESA-FAQAS-SPA & 1.0 & Software Product Assurance\\
SUTR & ITT-1-9873-ESA-FAQAS-SUTR & 1.0 & Software Unit Test Report  \\
SVR  & ITT-1-9873-ESA-FAQAS-SVR & 1.0 & Software Validation Report \\
\hline
\end{tabular}
\end{table}